\documentclass[sigconf,nonacm,anonymous]{acmart}
\usepackage{booktabs}
\usepackage{graphicx}
\usepackage{amsmath,amssymb}
\usepackage{enumitem}

\settopmatter{printacmref=false}
\renewcommand\footnotetextcopyrightpermission[1]{}
\pagestyle{plain}

\title{Computational Survey of Control Parameters for de Sitter Vacua in String Theory}

\author{Anonymous}
\affiliation{\institution{Anonymous}}

\begin{abstract}
The existence of fully controlled metastable de Sitter (dS) vacua in string theory remains one of the central open problems in string phenomenology. We present a systematic computational survey of KKLT-type flux compactifications, scanning 43 AdS vacua across Calabi--Yau manifolds with $h^{2,1} \in \{2,3,4,5\}$, evaluating three uplift mechanisms (anti-D3, D-term, F-term), and measuring six control parameters. Of 40 dS uplift candidates, 14 (35\%) achieve metastability, but none (0\%) satisfy all six control criteria simultaneously. The primary bottlenecks are small $W_0$ (0\% pass rate) and scale hierarchy (0\%), while tadpole cancellation is universally satisfied (100\%). Large volume and weak coupling criteria are met by 35.7\% and 28.6\% of metastable vacua respectively. These results quantify the difficulty of achieving full computational control and identify the most promising directions for progress: reducing $|W_0|$ through arithmetic flux tuning and engineering scale separation in the K\"ahler sector.
\end{abstract}

\begin{document}
\maketitle

\section{Introduction}

Constructing metastable de Sitter (dS) vacua in string theory is essential for connecting string theory to the observed accelerating expansion of the universe. The KKLT scenario~\cite{kachru2003sitter} and the Large Volume Scenario (LVS)~\cite{balasubramanian2005systematics} provide candidate frameworks, but as noted by Brunelli et al.~\cite{brunelli2026gravitational}, ``no concrete construction has all ingredients required to claim victory.'' The de Sitter Swampland Conjecture~\cite{obied2018sitter} and critical analyses~\cite{danielsson2018no} further question whether controlled dS solutions exist at all.

We undertake a systematic computational survey to quantify how close existing constructions come to full control, identifying the precise bottlenecks that prevent claiming definitive dS vacua.

\section{Methodology}

\subsection{Flux Landscape Scanning}

We scan type IIB flux compactifications using the Gukov--Vafa--Witten (GVW) superpotential~\cite{gukov2000cft}:
\begin{equation}
W_0 = \int_{CY_3} G_3 \wedge \Omega
\end{equation}
for Calabi--Yau manifolds with $h^{2,1} \in \{2,3,4,5\}$. K\"ahler moduli are stabilized via non-perturbative effects (gaugino condensation or D-brane instantons), yielding supersymmetric AdS vacua characterized by $(W_0, g_s, \mathcal{V}, \sigma_{\text{re}})$.

\subsection{Uplift Mechanisms}

Each AdS vacuum is tested against three uplift mechanisms:
\begin{itemize}[nosep]
  \item \textbf{Anti-D3 branes:} Warped uplift with energy $\delta V \propto D/\mathcal{V}^{4/3}$~\cite{kachru2003sitter}.
  \item \textbf{D-term uplift:} From magnetized branes, $\delta V \propto d/\mathcal{V}^2$.
  \item \textbf{F-term uplift:} From additional K\"ahler modulus sectors~\cite{conlon2005kklt}.
\end{itemize}
Metastability requires all mass matrix eigenvalues to exceed the Breitenlohner--Freedman bound.

\subsection{Control Criteria}

A vacuum is ``fully controlled'' if it satisfies all six criteria:
\begin{enumerate}[nosep]
  \item Large volume: $\mathcal{V} > 100$ (in string units)
  \item Weak coupling: $g_s < 0.1$
  \item Small $|W_0|$: $|W_0| < 1$ (for KKLT)~\cite{demirtas2020vacua}
  \item Scale hierarchy: $m_{KK}/m_{3/2} > 10$
  \item Tadpole bound: $N_{\text{flux}} < N_{\text{max}}$
  \item Moderate volume: $\mathcal{V} < 10^{15}$ (avoiding decompactification)
\end{enumerate}

\section{Results}

\subsection{Overview}

From 43 stabilized AdS vacua, 40 yield dS uplift candidates. Of these, 14 (35\%) are metastable. However, zero satisfy all six control criteria simultaneously, yielding 0\% fully controlled dS vacua.

\subsection{Control Parameter Breakdown}

\begin{table}[t]
\caption{Pass rates for individual control criteria (14 metastable dS vacua).}
\label{tab:control}
\small
\begin{tabular}{lcc}
\toprule
\textbf{Criterion} & \textbf{Pass Rate} & \textbf{Bottleneck?} \\
\midrule
Tadpole cancellation    & 100\%  & No \\
Large volume            & 35.7\% & Moderate \\
Weak coupling ($g_s<0.1$) & 28.6\% & Moderate \\
Small $|W_0|$ ($<1$)    & 0\%    & \textbf{Critical} \\
Scale hierarchy         & 0\%    & \textbf{Critical} \\
\bottomrule
\end{tabular}
\end{table}

Table~\ref{tab:control} reveals two critical bottlenecks: small $|W_0|$ (0\%) and scale hierarchy (0\%). The mean control score (fraction of criteria satisfied) is 0.274.

\subsection{Dependence on Hodge Numbers}

\begin{table}[t]
\caption{Results by Calabi--Yau complexity ($h^{2,1}$).}
\label{tab:hodge}
\small
\begin{tabular}{lcccc}
\toprule
$h^{2,1}$ & AdS & dS cand. & Metastable & Control \\
\midrule
2 & 14 & 14 & 6 (43\%) & 0.278 \\
3 & 11 & 10 & 3 (30\%) & 0.278 \\
4 & 9  & 8  & 2 (25\%) & 0.250 \\
5 & 9  & 8  & 3 (38\%) & 0.278 \\
\bottomrule
\end{tabular}
\end{table}

\subsection{Uplift Mechanism Comparison}

D-term uplift produces the most candidates (39/40) with 33\% metastability. Anti-D3 brane uplift yields no viable candidates in our scan due to insufficient warping. F-term uplift produces 1 metastable vacuum with the highest mean control score (0.333).

\subsection{Coupling and Volume Dependence}

Metastability fraction peaks at intermediate coupling ($g_s \approx 0.3$, 50--56\%) and decreases at both weak and strong coupling. Control scores are highest at large volumes ($\mathcal{V} > 30$) but such vacua are rare (8/43).

\section{Discussion}

Our survey confirms that the de Sitter vacuum problem is primarily a \emph{control} problem rather than an \emph{existence} problem: 35\% of candidates are metastable, but 0\% are fully controlled. The two critical bottlenecks -- small $|W_0|$ and scale hierarchy -- are deeply connected: achieving exponentially small $W_0$ through flux tuning~\cite{demirtas2020vacua} is essential for KKLT, while scale separation requires large volume.

\textbf{Implications.} (1) Arithmetic approaches to flux tuning may be the most promising path to small $W_0$. (2) The LVS framework may more naturally achieve large volume but faces different control challenges. (3) The zero pass rate for simultaneous control criteria supports the difficulty emphasized by swampland conjectures but does not constitute proof of impossibility.

\textbf{Limitations.} Our scan samples a tiny fraction of the flux landscape. We model the effective potential rather than performing full 10D backreaction. Real Calabi--Yau geometries have far richer structure than our simplified treatment.

\section{Conclusion}

We presented a computational survey of 43 flux vacua across four Calabi--Yau manifolds, evaluating 40 dS uplift candidates against six control criteria. The 0\% pass rate for simultaneous control confirms this as a genuine open problem, with small $|W_0|$ and scale hierarchy as the critical bottlenecks. These results provide a quantitative baseline for measuring progress toward fully controlled de Sitter vacua.

\bibliographystyle{ACM-Reference-Format}
\bibliography{references}

\end{document}
