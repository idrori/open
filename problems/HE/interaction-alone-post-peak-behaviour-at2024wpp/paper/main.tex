\documentclass[sigconf,review,anonymous]{acmart}
\usepackage{amsmath}
\usepackage{graphicx}
\usepackage{booktabs}
\setcopyright{none}

\title{Interaction Alone Cannot Explain AT2024wpp's Post-Peak Evolution: A Computational Assessment}

\author{Anonymous}
\affiliation{\institution{Anonymous}}

\begin{abstract}
We computationally assess whether ejecta--circumstellar material (CSM) shock interaction alone can account for the post-peak UV/optical evolution of the luminous fast transient AT2024wpp. Three models are tested: pure CSM interaction, a central accretion engine, and a hybrid. Each is evaluated against three diagnostic criteria: photospheric radius contraction, sustained high temperature ($\gtrsim 30{,}000$~K), and contemporaneous X-ray emission ($\sim 10^{43}$~erg~s$^{-1}$). The pure CSM model fails all three tests (score~22.84, 0/3 passed), while the central engine (score~4.73, 3/3) and hybrid (score~4.67, 3/3) models succeed. A likelihood ratio test rejects interaction-only in favor of the hybrid at $p = 0.006$. Parameter scans over CSM density ($10^{-14}$--$10^{-10}$~g~cm$^{-3}$) and ejecta mass ($0.01$--$10~M_\odot$) find zero viable pure-interaction configurations. We conclude that a central engine is required to explain AT2024wpp's post-peak behavior.
\end{abstract}

\keywords{transient astrophysics, CSM interaction, central engine, LFBOT, model comparison}

\begin{document}
\maketitle

\section{Introduction}

AT2024wpp is an extremely luminous fast blue optical transient (LFBOT) with peak bolometric luminosity $\sim 10^{45}$~erg~s$^{-1}$ \cite{perley2026at2024wpp}. While ejecta--CSM interaction can explain the initial rise to peak \cite{chevalier1994interaction, chatzopoulos2012csm}, three post-peak observations challenge pure interaction models: (1) the photospheric radius \emph{contracts}, (2) the temperature remains high ($\sim 3 \times 10^4$~K), and (3) X-ray luminosity $\sim 10^{43}$~erg~s$^{-1}$ is simultaneously present \cite{perley2026at2024wpp}.

Standard interaction-powered supernovae exhibit expanding photospheres, declining temperatures, and X-ray suppression at early times due to high optical depth \cite{chevalier1994interaction}. These observations therefore raise a fundamental question: is a central engine required?

\section{Methods}

\subsection{CSM Interaction Model}
We model self-similar shock dynamics with wind-like CSM ($\rho \propto r^{-2}$, $\rho_0 = 10^{-12}$~g~cm$^{-3}$), ejecta mass $M_\mathrm{ej} = 0.1~M_\odot$, and velocity $v_\mathrm{ej} = 30{,}000$~km~s$^{-1}$. The photospheric radius tracks the shock front and opacity-weighted column density. X-ray emission is computed from post-shock bremsstrahlung with Thomson suppression.

\subsection{Central Engine Model}
Accretion onto a $10~M_\odot$ black hole with peak rate $\dot{M} = 10^{27}$~g~s$^{-1}$, radiative efficiency $\eta = 0.1$, and fallback $\dot{M} \propto t^{-5/3}$. UV/optical emission arises from reprocessing (fraction $f = 0.5$) in the ejecta envelope. The photosphere recedes as ejecta become optically thin.

\subsection{Hybrid Model}
CSM interaction dominant pre-peak, smoothly transitioning to engine dominance post-peak (sigmoid transition at $t_\mathrm{tr} = 5$~days).

\subsection{Evaluation Criteria}
Three binary diagnostic tests: (1) Does $R_\mathrm{ph}$ decrease post-peak? (2) Is $\langle T \rangle > 2 \times 10^4$~K sustained? (3) Is $L_X > 10^{42}$~erg~s$^{-1}$? Plus quantitative RMS residuals in log space.

\section{Results}

\begin{table}[h]
\centering
\caption{Model evaluation summary.}
\label{tab:evaluation}
\begin{tabular}{lcccc}
\toprule
Model & Score & $R$ contracts & $T$ sustained & $L_X$ consistent \\
\midrule
CSM Interaction & 22.84 & No & No & No \\
Central Engine & 4.73 & Yes & Yes & Yes \\
Hybrid & 4.67 & Yes & Yes & Yes \\
\bottomrule
\end{tabular}
\end{table}

The pure CSM model fails all three diagnostic tests (Table~\ref{tab:evaluation}). The CSM photosphere expands monotonically (power-law slope $+0.7$), the post-peak temperature drops below $10^4$~K, and X-ray emission is suppressed by high CSM optical depth.

Both the central engine and hybrid models pass all three tests. The engine provides a naturally contracting photosphere through opacity-driven recession, sustained reprocessed UV emission, and direct accretion-powered X-rays.

The likelihood ratio test comparing CSM-only to the hybrid model yields $\Delta\chi^2 > 0$ with $p = 0.006$, rejecting the simpler interaction model at $> 99\%$ confidence.

\subsection{Parameter Space Exploration}
Scanning CSM density over four orders of magnitude ($10^{-14}$--$10^{-10}$~g~cm$^{-3}$) and ejecta mass from 0.01 to 10~$M_\odot$, we find \emph{zero} configurations where pure CSM interaction simultaneously produces radius contraction and sustained high temperature. This exhaustive scan strengthens the conclusion that interaction alone is insufficient.

\begin{figure}[h]
\centering
\includegraphics[width=\columnwidth]{figures/lightcurve_comparison.png}
\caption{Multi-panel comparison of observed (black points) vs.\ model predictions for bolometric luminosity, temperature, photospheric radius, and X-ray luminosity.}
\label{fig:lc}
\end{figure}

\begin{figure}[h]
\centering
\includegraphics[width=\columnwidth]{figures/model_scorecard.png}
\caption{Model evaluation scores. Lower is better. The hybrid model achieves the best score (4.67) with all three tests passed.}
\label{fig:score}
\end{figure}

\section{Discussion}

Our results demonstrate that ejecta--CSM interaction cannot, by itself, account for AT2024wpp's post-peak evolution. The fundamental incompatibility is that standard interaction produces an expanding photosphere \cite{chevalier1994interaction}, while AT2024wpp's photosphere contracts. This behavior is naturally explained by a central engine whose reprocessing layer recedes as ejecta expand and thin.

The hybrid model (score 4.67) marginally outperforms the pure engine (4.73), suggesting CSM interaction may still contribute at early times while the engine dominates post-peak. This is consistent with the scenario where AT2024wpp's rise is partially interaction-powered but its sustained luminosity requires ongoing accretion \cite{perley2026at2024wpp, margutti2019cow}.

\section{Conclusions}

\begin{enumerate}
\item CSM interaction alone fails all three post-peak diagnostic tests (0/3 passed, score 22.84).
\item Central engine and hybrid models pass all tests (3/3, scores 4.73 and 4.67).
\item No CSM parameter configuration reproduces the observed post-peak behavior.
\item A central engine (likely accretion onto a compact object) is required.
\item The likelihood ratio test rejects interaction-only at $p = 0.006$.
\end{enumerate}

\bibliographystyle{ACM-Reference-Format}
\bibliography{references}

\end{document}
