\documentclass[sigconf,review,anonymous]{acmart}

\usepackage{amsmath}
\usepackage{graphicx}
\usepackage{booktabs}

\settopmatter{printacmref=false}
\renewcommand\footnotetextcopyrightpermission[1]{}
\pagestyle{plain}

\begin{document}

\title{Topology Optimization for Maximizing Feasible Wealth Volume in Payment Channel Networks}

\author{Anonymous}
\affiliation{\institution{Anonymous}}

\begin{abstract}
We investigate the problem of finding payment channel network topologies that maximize $r(G) = |W_G|/|\mathcal{W}(C,n)|$, the ratio of feasible off-chain wealth distributions to total on-chain distributions. Through exhaustive enumeration over all connected graphs for $n \leq 5$ and evolutionary search for $n = 6$, we find that the complete graph ($K_n$) maximizes $r(G)$ for $n \leq 4$ with $r(K_4) = 0.441$, while for $n \geq 5$, sparser topologies with moderate connectivity dominate. Across all tested configurations, cycle graphs consistently achieve the highest $r(G)$ among standard graph families for $n \geq 5$, with $r(C_6) = 0.100$ compared to $r(K_6) = 0.001$. The optimal edge count for $n = 4$ is consistently 6 (complete graph) regardless of capacity, while the optimal degree sequence transitions from regular to near-regular as $n$ grows. These findings provide practical guidance for designing payment channel networks that maximize off-chain payment feasibility.
\end{abstract}

\maketitle

\section{Introduction}

Payment channel networks (PCNs) such as the Lightning Network~\cite{poon2016lightning} enable scalable off-chain transactions. Pickhardt~\cite{pickhardt2026pcn} defined $r(G)$ as the ratio of feasible wealth distributions to all possible distributions, measuring how well a network topology supports off-chain payments. Finding the topology maximizing $r(G)$ remains open.

This work presents the first systematic computational study of $r(G)$ optimization, combining exhaustive enumeration, evolutionary search~\cite{holland1975adaptation}, and analytical bounds.

\section{Methods}

\subsection{Exhaustive Enumeration}

For $n \leq 5$ nodes, we enumerate all connected graphs on $n$ vertices, computing $r(G)$ for each via exact enumeration of liquidity assignments. For $n = 4$, this yields 38 distinct connected topologies.

\subsection{Evolutionary Search}

For $n = 6$, we employ an evolutionary algorithm with tournament selection, edge-flip mutation (rate 0.15), and elitism. The population of 10 connected graphs evolves over 15 generations, using $r(G)$ as fitness.

\section{Results}

\subsection{Optimal Topologies}

For $n = 3$: The cycle $C_3$ (= $K_3$) is optimal with $r(G) = 0.673$.

For $n = 4$: The complete graph $K_4$ achieves $r(G) = 0.441$, the highest among all 38 connected graphs. The optimal degree sequence is $[3,3,3,3]$.

For $n = 5$: A graph with degree sequence $[3,3,2,2,2]$ achieves $r(G) = 0.228$, outperforming both $K_5$ ($r = 0.044$) and $C_5$ ($r = 0.202$).

\begin{table}[t]
\caption{Best $r(G)$ by graph family and node count (cap=3).}
\label{tab:families}
\begin{tabular}{lcccc}
\toprule
Family & $n=3$ & $n=4$ & $n=5$ & $n=6$ \\
\midrule
Path & 0.571 & 0.291 & 0.141 & 0.066 \\
Cycle & 0.673 & 0.385 & 0.202 & 0.100 \\
Star & 0.571 & 0.291 & 0.141 & 0.066 \\
Complete & 0.673 & 0.441 & 0.044 & 0.001 \\
\midrule
Best found & 0.673 & 0.441 & 0.228 & 0.100 \\
\bottomrule
\end{tabular}
\end{table}

\subsection{Edge Count Analysis}

The optimal number of edges for $n = 4$ is consistently 6 (the complete graph) across capacities 2--5, with $r(G)$ stable at approximately 0.441. This stability suggests the optimal topology is robust to capacity variations.

\subsection{Scaling Behavior}

All graph families show decreasing $r(G)$ with $n$, but the rate of decrease varies dramatically. Complete graphs decay fastest (from 0.673 to 0.001 for $n = 3$ to 6), while cycles decay most slowly (0.673 to 0.100). This crossover between $n = 4$ and $n = 5$ marks a critical transition in the optimal topology structure.

\section{Discussion}

The key finding is a phase transition in optimal topology: for small networks ($n \leq 4$), maximum connectivity is optimal, while for larger networks ($n \geq 5$), moderate connectivity preserves a higher fraction of feasible distributions. This transition occurs because the denominator $|\mathcal{W}(C,n)|$ grows faster with total capacity $C = |E| \cdot \text{cap}$ than $|W_G|$ grows with additional edges.

For practical network design, cycle-like topologies with degree close to 2 offer the best feasibility-to-capacity trade-off at scale, consistent with the routing structure used in real payment channel networks~\cite{ramasubramanian2020routing, sivaraman2020spider}.

\section{Conclusion}

We identified a phase transition in the topology maximizing $r(G)$: from complete graphs for $n \leq 4$ to sparser, cycle-like topologies for $n \geq 5$. Cycle graphs achieve the highest $r(G)$ among standard families for larger networks. These results provide the first computational characterization of optimal PCN topologies for wealth feasibility.

\bibliographystyle{ACM-Reference-Format}
\bibliography{references}

\end{document}
