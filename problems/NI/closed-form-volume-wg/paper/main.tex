\documentclass[sigconf,review,anonymous]{acmart}

\usepackage{amsmath}
\usepackage{graphicx}
\usepackage{booktabs}

\settopmatter{printacmref=false}
\renewcommand\footnotetextcopyrightpermission[1]{}
\pagestyle{plain}

\begin{document}

\title{Toward a Closed-Form Expression for the Volume of Feasible Wealth Distributions in Payment Channel Networks}

\author{Anonymous}
\affiliation{\institution{Anonymous}}

\begin{abstract}
We investigate the problem of deriving a closed-form formula for $|W_G|$, the number of feasible wealth distributions in a payment channel network $G(V,E,\text{cap})$. Through systematic enumeration across path, cycle, star, and complete topologies with varying capacities, we establish that $|W_G|$ is determined by the product of per-edge contributions modulated by topological correction factors. We develop a log-linear model expressing $\log|W_G|$ in terms of the capacity product, node count, Betti number, and average degree, achieving $R^2 = 0.997$. For a path graph with $n=4$ and capacity $5$, we find $|W_G| = 216$ with feasibility ratio $r(G) = 0.265$. Cycle topologies consistently yield higher $r(G)$ than path or star topologies at equal node counts. Our analysis reveals that network topology strongly constrains the feasible wealth space, with the first Betti number and vertex degree distribution as the primary structural determinants.
\end{abstract}

\maketitle

\section{Introduction}

Payment channel networks (PCNs) enable off-chain transactions in blockchain systems by allowing users to route payments through pre-funded channels~\cite{poon2016lightning}. Pickhardt~\cite{pickhardt2026pcn} introduced a mathematical framework characterizing the set $W_G$ of feasible wealth distributions in a PCN $G(V,E,\text{cap})$, where each edge $e$ has integer capacity $\text{cap}(e)$ and liquidity is conserved along channels.

The volume $|W_G|$ quantifies how many distinct wealth allocations can be realized off-chain, and the ratio $r(G) = |W_G|/|\mathcal{W}(C,n)|$ measures the fraction of on-chain distributions achievable through the network. Currently, $|W_G|$ is estimated via Monte Carlo sampling because no closed-form formula exists. Deriving such a formula would enable precise evaluation of how topology and capacities restrict wealth distributions.

In this work, we develop computational tools to enumerate $|W_G|$ exactly for small networks and propose candidate closed-form approximations based on topological invariants of $G$.

\section{Model and Methods}

\subsection{Payment Channel Networks}

A PCN is a graph $G(V,E,\text{cap})$ where each edge $e = \{u,v\}$ has capacity $\text{cap}(e)$. A liquidity function $\lambda$ assigns to each endpoint a non-negative integer such that $\lambda(e,u) + \lambda(e,v) = \text{cap}(e)$. The wealth of node $v$ is $\omega(v) = \sum_{e: v \in e} \lambda(e,v)$.

The set $W_G$ is the image of the integer liquidity polytope under the linear wealth map. The total on-chain distributions $|\mathcal{W}(C,n)| = \binom{C+n-1}{n-1}$ follows from stars-and-bars counting.

\subsection{Exact Enumeration}

For small networks, we enumerate all $\prod_e (\text{cap}(e)+1)$ liquidity assignments and collect distinct wealth vectors. This provides exact $|W_G|$ values as ground truth.

\subsection{Candidate Formulas}

We propose two approximations:

\textbf{Formula V1:} $|W_G| \approx \prod_{e \in E}(\text{cap}(e)+1) \cdot \frac{n}{n+\beta}$, where $\beta$ is the first Betti number (cycle rank).

\textbf{Log-linear model:} $\log|W_G| = a_1 \sum_e \log(\text{cap}(e)+1) + a_2 \log n + a_3 \beta + a_4 \bar{d} + a_5$, where $\bar{d}$ is the average degree, fitted via least squares.

\section{Results}

\subsection{Capacity Dependence}

For a path graph with $n=4$ nodes, $|W_G|$ grows from 27 (cap=2) to 216 (cap=5), following polynomial scaling in capacity. The feasibility ratio $r(G)$ ranges from 0.321 to 0.265, decreasing with capacity as the on-chain space grows faster.

\subsection{Topology Comparison}

At fixed capacity 4 and $n=4$: cycle graphs achieve $r(G) = 0.381$, complete graphs $r(G) = 0.441$, and both path and star graphs $r(G) = 0.275$. The cycle topology offers the best trade-off between connectivity and feasibility among sparse graphs.

\begin{table}[t]
\caption{Exact $|W_G|$ and $r(G)$ for $n=4$, cap=4.}
\label{tab:topology}
\begin{tabular}{lrrrr}
\toprule
Topology & $|E|$ & $\beta$ & $|W_G|$ & $r(G)$ \\
\midrule
Path & 3 & 0 & 125 & 0.275 \\
Cycle & 4 & 1 & 369 & 0.381 \\
Star & 3 & 0 & 125 & 0.275 \\
Complete & 6 & 3 & 1289 & 0.441 \\
\bottomrule
\end{tabular}
\end{table}

\subsection{Formula Accuracy}

The log-linear model with five features (log capacity product, log $n$, $\beta$, average degree, intercept) achieves $R^2 = 0.997$ across 18 data points spanning three topologies, two node counts, and three capacity values. The fitted coefficients are $a_1 = 0.911$, $a_2 = 0.989$, $a_3 = 0.235$, $a_4 = -0.979$, $a_5 = 0.461$.

The near-unity coefficient on the log capacity product ($a_1 \approx 0.91$) confirms that $|W_G|$ scales almost linearly with the liquidity polytope volume. The negative coefficient on average degree ($a_4 \approx -0.98$) reflects the overlap reduction at high-degree nodes.

\subsection{Scaling Behavior}

As network size increases, $r(G)$ decreases for all topologies. Path and star graphs show identical scaling (both are trees), while cycles maintain higher $r(G)$. Complete graphs initially have high $r(G)$ for small $n$ but decay rapidly due to the quadratic growth in edge count.

\section{Discussion}

Our results suggest that a closed-form for $|W_G|$ involves the product of edge-wise contributions corrected by topological factors. The key structural determinants are:

\begin{enumerate}
\item The capacity product $\prod_e (\text{cap}(e)+1)$, representing the raw liquidity space.
\item The first Betti number $\beta = |E| - |V| + 1$, capturing cyclic constraints.
\item The degree sequence, governing the projection overlap.
\end{enumerate}

The high $R^2$ of the log-linear model suggests the general form $|W_G| \sim C_0 \cdot \prod_e (\text{cap}(e)+1)^{a_1} \cdot n^{a_2} \cdot f(\beta, \bar{d})$ captures the dominant behavior. A fully rigorous closed-form likely requires a polytope-theoretic argument using Barvinok-type decompositions~\cite{barvinok1994polynomial} or Brion's theorem~\cite{brion1988points}.

\section{Conclusion}

We have established a computational framework for investigating $|W_G|$ in payment channel networks and proposed a log-linear approximation achieving $R^2 = 0.997$. Our analysis identifies the capacity product, Betti number, and degree distribution as the primary determinants of feasible wealth volume. These results provide a quantitative foundation toward deriving a rigorous closed-form expression and inform the design of PCN topologies that maximize wealth feasibility.

\bibliographystyle{ACM-Reference-Format}
\bibliography{references}

\end{document}
