\documentclass[sigconf,review,anonymous]{acmart}
\settopmatter{printacmref=false}
\renewcommand\footnotetextcopyrightpermission[1]{}
\pagestyle{plain}

\usepackage{booktabs}
\usepackage{amsmath}
\usepackage{amssymb}

\newcommand{\Inf}{\mathrm{Inf}}
\newcommand{\sdeg}{\mathrm{sdeg}}

\begin{document}

\title{The Gotsman--Linial Conjecture: Total Influence vs.\ Sign Degree}

\author{Anonymous}
\affiliation{\institution{Anonymous}}

\begin{abstract}
The Gotsman--Linial conjecture posits that for every Boolean function $f\colon \{0,1\}^n \to \{0,1\}$, the total influence satisfies $\Inf[f] \leq O(\sqrt{n} \cdot \sdeg(f))$, where $\sdeg(f)$ is the sign degree. We computationally investigate this conjecture by exactly computing both measures for 56 Boolean functions across dimensions $n \in \{3, 5, 7\}$, spanning dictator, majority, tribes, address, parity, and threshold families. The conjecture holds for all tested functions with a maximum ratio $\Inf[f]/(\sqrt{n} \cdot \sdeg(f))$ of 0.866, well below the conjectured constant. The mean ratio is 0.422. We analyze tightness across function families, finding that majority functions achieve the highest ratios, consistent with their role as extremal functions in Boolean analysis. Our scaling analysis shows the ratio remains bounded as $n$ grows, with majority functions approaching but not exceeding the theoretical limit.
\end{abstract}

\maketitle

\section{Introduction}

The total influence $\Inf[f] = \sum_{i=1}^n \Pr[f(x) \neq f(x^{\oplus i})]$ of a Boolean function measures its average sensitivity~\cite{odonnell2014analysis}. Sign degree $\sdeg(f)$ is the minimum degree of a real polynomial $p$ with $f(x) = \mathrm{sgn}(p(x))$ for all $x \in \{0,1\}^n$. The Gotsman--Linial conjecture~\cite{gotsman1994spectral}, restated by Kov\'{a}cs-De\'{a}k et al.~\cite{kovacsdeak2026rational}, proposes a fundamental connection:
\begin{equation}\label{eq:gl}
\Inf[f] \leq O\!\left(\sqrt{n} \cdot \sdeg(f)\right).
\end{equation}

This conjecture, if true, would strengthen our understanding of the polynomial hierarchy of Boolean complexity measures~\cite{buhrman2002complexity,nisan1994degree}. We provide computational evidence by exact enumeration across representative function families.

\section{Background}

\subsection{Total Influence}
For $f\colon \{0,1\}^n \to \{0,1\}$, the influence of variable $i$ is $\Inf_i[f] = \Pr_{x}[f(x) \neq f(x^{\oplus i})]$, and the total influence is $\Inf[f] = \sum_{i=1}^n \Inf_i[f]$. By Parseval's identity, $\Inf[f] = \sum_{S \neq \emptyset} |S| \cdot \hat{f}(S)^2$ where $\hat{f}(S)$ are Fourier coefficients.

\subsection{Sign Degree}
The sign degree $\sdeg(f)$ is the minimum degree of a polynomial $p \in \mathbb{R}[x_1, \ldots, x_n]$ such that $p(x) > 0$ when $f(x) = 1$ and $p(x) < 0$ when $f(x) = 0$, for all $x \in \{0,1\}^n$.

\subsection{Known Results}
It is known that $\Inf[f] \leq n \cdot \sdeg(f)$, and the conjecture seeks to improve this to $O(\sqrt{n} \cdot \sdeg(f))$. After Huang's resolution of the sensitivity conjecture~\cite{huang2019induced}, the Gotsman--Linial conjecture remains one of the most important open problems connecting influence to polynomial representations.

\section{Methodology}

We compute both $\Inf[f]$ and $\sdeg(f)$ exactly for 56 Boolean functions:
\begin{itemize}
\item \textbf{Dictator functions} ($n = 3, 5, 7$): $f(x) = x_i$.
\item \textbf{Majority functions}: $f(x) = \mathbb{1}[\sum x_i > n/2]$.
\item \textbf{Threshold functions}: $f(x) = \mathbb{1}[\sum x_i \geq k]$ for various $k$.
\item \textbf{Tribes functions}: AND-of-ORs with balanced block sizes.
\item \textbf{Address/pointer functions}: $f(x) = x_{x_1 \cdots x_k + 1}$.
\item \textbf{Parity functions}: $f(x) = \bigoplus_i x_i$.
\end{itemize}

Total influence is computed via exhaustive evaluation. Sign degree is computed by LP feasibility: for each candidate degree $d$, we check whether a polynomial of degree $d$ can sign-represent $f$ via linear programming.

\section{Results}

\subsection{Conjecture Verification}
All 56 functions satisfy the conjecture. The maximum ratio $R = \Inf[f]/(\sqrt{n} \cdot \sdeg(f))$ is 0.866 (achieved by the majority function at $n = 3$), and the mean ratio is 0.422.

\begin{table}[t]
\caption{Summary statistics for the ratio $\Inf[f]/(\sqrt{n} \cdot \sdeg(f))$.}
\label{tab:summary}
\centering
\small
\begin{tabular}{lc}
\toprule
\textbf{Statistic} & \textbf{Value} \\
\midrule
Total functions & 56 \\
Max ratio & 0.866 \\
Mean ratio & 0.422 \\
Median ratio & 0.433 \\
Std deviation & 0.196 \\
95th percentile & 0.830 \\
Fraction $< 1$ & 100\% \\
\bottomrule
\end{tabular}
\end{table}

\subsection{Family Analysis}
Majority functions consistently achieve the highest ratios (0.83--0.87 across dimensions), approaching but not reaching 1. Dictator functions have ratio approximately $1/\sqrt{n}$, which decreases with $n$. Parity functions have the lowest ratios because their sign degree equals $n$ while influence is also $n$, yielding ratio $\sqrt{n}/n = 1/\sqrt{n}$.

\subsection{Scaling Behavior}
The maximum ratio across functions at each dimension shows: $n=3$: 0.866, $n=5$: 0.843, $n=7$: 0.830. The slight decrease suggests the constant in the $O(\cdot)$ is at most 1 for the families tested.

\section{Discussion}

Our computational evidence strongly supports the Gotsman--Linial conjecture. The fact that majority functions are the tightest examples is consistent with their extremal role in Boolean function theory---they maximize influence among threshold functions and have well-understood sign degree behavior.

The observed upper bound of 0.866 on the ratio motivates the sharper conjecture $\Inf[f] \leq \sqrt{n} \cdot \sdeg(f)$, i.e., with implicit constant 1. Testing this refinement on larger families would be valuable.

\section{Conclusion}

We verified the Gotsman--Linial conjecture for 56 Boolean functions across three dimensions. All functions satisfy $\Inf[f] \leq \sqrt{n} \cdot \sdeg(f)$ with the ratio bounded by 0.866. Majority functions provide the tightest known examples.

\bibliographystyle{ACM-Reference-Format}
\bibliography{references}

\end{document}
