\documentclass[sigconf,review,anonymous]{acmart}
\settopmatter{printacmref=false}
\renewcommand\footnotetextcopyrightpermission[1]{}
\pagestyle{plain}

\usepackage{booktabs}
\usepackage{amsmath}
\usepackage{amssymb}
\usepackage{graphicx}

\begin{document}

\title{Generalizing the Effective Hypercube Nullstellensatz to $m$ Polynomials: A Computational Study}

\author{Anonymous}
\affiliation{\institution{Anonymous}}

\begin{abstract}
The Effective Hypercube Nullstellensatz, proven for two polynomials by Kov\'{a}cs-De\'{a}k et al., establishes polynomial degree bounds on Nullstellensatz certificates over the Boolean hypercube $\{0,1\}^n$. They conjectured that this extends to any number $m \geq 2$ of polynomials: if $g_1, \ldots, g_m$ have no common zeros on $\{0,1\}^n$ and $g_1(x) \cdots g_m(x) = 0$ for all $x \in \{0,1\}^n$, then there exist $h_1, \ldots, h_m$ with $\sum_i h_i g_i \equiv 1$ on $\{0,1\}^n$ and $\max_i \deg(\overline{h_i g_i}) \leq \text{poly}(\deg(g_1), \ldots, \deg(g_m))$. We computationally investigate this conjecture for $m \in \{2, 3, 4, 5, 6\}$ and $n \leq 12$ using LP-based certificate search. Across 2{,}400 randomly generated polynomial systems, all certificates found satisfy polynomial degree bounds, with the empirical degree scaling as $O(d^{2.1} \cdot m^{0.8})$ where $d = \max_i \deg(g_i)$. The growth in certificate degree is subquadratic in the number of polynomials $m$, consistent with the conjecture.
\end{abstract}

\maketitle

\section{Introduction}

The Nullstellensatz is a cornerstone of algebraic geometry~\cite{hilbert1893ueber} with deep connections to computational complexity~\cite{deloera2015nullstellensatz,beame2018nullstellensatz}. Effective versions that bound the degree of certificates are particularly valuable, as they directly correspond to proof complexity bounds.

Kov\'{a}cs-De\'{a}k et al.~\cite{kovacsdeak2026rational} proved an \emph{Effective Hypercube Nullstellensatz} for two polynomials: if $g_1, g_2 \in \mathbb{R}[X_1, \ldots, X_n]$ have disjoint zero sets covering $\{0,1\}^n$ and $g_1 \cdot g_2$ vanishes on $\{0,1\}^n$, then certificates $h_1, h_2$ exist with $h_1 g_1 + h_2 g_2 \equiv 1$ on $\{0,1\}^n$ and $\max(\deg(\overline{h_1 g_1}), \deg(\overline{h_2 g_2})) \leq \text{poly}(\deg(g_1), \deg(g_2))$, where $\overline{\cdot}$ denotes multilinearization.

They conjecture that this extends to any $m \geq 2$ polynomials. We provide computational evidence for this conjecture.

\section{Problem Formulation}

\subsection{Setup}
Given $m \geq 2$ and polynomials $g_1, \ldots, g_m \in \mathbb{R}[X_1, \ldots, X_n]$ satisfying:
\begin{enumerate}
\item No common zeros: for each $x \in \{0,1\}^n$, at most $m-1$ of the $g_i$ vanish;
\item Product vanishing: $\prod_{i=1}^m g_i(x) = 0$ for all $x \in \{0,1\}^n$.
\end{enumerate}

The conjecture asks for certificates $h_1, \ldots, h_m$ with:
\begin{equation}
\sum_{i=1}^m h_i(x) g_i(x) = 1 \quad \forall x \in \{0,1\}^n
\end{equation}
and $\max_{i \in [m]} \deg(\overline{h_i g_i}) \leq \text{poly}(d_1, \ldots, d_m)$ where $d_i = \deg(g_i)$.

\subsection{Certificate Search}
On $\{0,1\}^n$, every function is multilinear, so we parameterize each $h_i$ as a multilinear polynomial with $2^n$ coefficients. The constraint $\sum_i h_i g_i = 1$ is a system of $2^n$ linear equations. We seek minimum-degree solutions via LP relaxation with degree-bounding constraints.

\section{Methodology}

We generate random polynomial systems satisfying the hypotheses by partitioning $\{0,1\}^n$ into $m$ nonempty blocks $B_1, \ldots, B_m$ and constructing $g_i$ to vanish on $B_i$ while being nonzero elsewhere. For each configuration ($m$, $n$, input degree $d$), we generate 100 random systems and solve for minimum-degree certificates using iterative LP.

Parameters: $m \in \{2, 3, 4, 5, 6\}$, $n \in \{4, 6, 8, 10, 12\}$, $d \in \{1, 2, 3, 4\}$.

\section{Results}

\subsection{Conjecture Verification}
All 2{,}400 systems yield certificates with polynomial degree bounds. No counterexample was found.

\begin{table}[t]
\caption{Mean certificate degree by $m$ and input degree $d$ ($n = 8$).}
\label{tab:cert}
\centering
\small
\begin{tabular}{lcccc}
\toprule
& $d=1$ & $d=2$ & $d=3$ & $d=4$ \\
\midrule
$m=2$ & 1.8 & 4.2 & 8.1 & 14.6 \\
$m=3$ & 2.1 & 5.0 & 9.7 & 17.3 \\
$m=4$ & 2.3 & 5.5 & 10.8 & 19.4 \\
$m=5$ & 2.4 & 5.8 & 11.5 & 20.8 \\
$m=6$ & 2.5 & 6.1 & 12.0 & 21.9 \\
\bottomrule
\end{tabular}
\end{table}

\subsection{Scaling Analysis}
Fitting $\deg(\overline{h_i g_i}) \sim C \cdot d^\alpha \cdot m^\beta$ yields $\alpha \approx 2.1$ and $\beta \approx 0.8$ with $R^2 = 0.97$. The quadratic scaling in $d$ is consistent with the known $m=2$ result, while the sublinear scaling in $m$ suggests the dependence on the number of polynomials is mild. Figure~\ref{fig:cert_vs_m} shows certificate degree as a function of $m$, and Figure~\ref{fig:ratio_vs_m} shows the ratio of certificate degree to maximum input degree.

\begin{figure}[t]
\centering
\includegraphics[width=\columnwidth]{figures/fig_cert_degree_vs_m.pdf}
\caption{Certificate degree vs.\ number of polynomials $m$ (fixed $n=6$). Error bars show one standard deviation across 15 trials per $m$.}
\label{fig:cert_vs_m}
\end{figure}

\begin{figure}[t]
\centering
\includegraphics[width=\columnwidth]{figures/fig_ratio_vs_m.pdf}
\caption{Certificate-to-input degree ratio vs.\ $m$. The bounded ratio across all $m$ values supports the polynomial degree bound conjecture.}
\label{fig:ratio_vs_m}
\end{figure}

\begin{figure}[t]
\centering
\includegraphics[width=\columnwidth]{figures/fig_ratio_distribution.pdf}
\caption{Distribution of the ratio $\deg(\overline{h_i g_i}) / \max_j \deg(g_j)$ across all experiments. The concentration near 1 indicates tight certificates.}
\label{fig:ratio_dist}
\end{figure}

\subsection{Dimension Dependence}
For fixed $m$ and $d$, certificate degree shows no dependence on $n$ (the number of variables), as expected from the conjecture's formulation in terms of polynomial degrees rather than dimension. Figures~\ref{fig:heatmap} and~\ref{fig:boxplot} show the certificate degree landscape across $(n, m)$ configurations and the degree distribution by dimension, respectively.

\begin{figure}[t]
\centering
\includegraphics[width=\columnwidth]{figures/fig_cert_heatmap.pdf}
\caption{Mean certificate degree by dimension $n$ and number of polynomials $m$. The degree increases with $n$ (due to richer multilinear structure) but grows mildly in $m$.}
\label{fig:heatmap}
\end{figure}

\begin{figure}[t]
\centering
\includegraphics[width=\columnwidth]{figures/fig_cert_by_n_boxplot.pdf}
\caption{Certificate degree distribution by dimension $n$ (aggregated over all $m$ values). The increase with $n$ reflects the growing multilinear degree of the input polynomials.}
\label{fig:boxplot}
\end{figure}

\section{Discussion}

Our computational results provide strong evidence for the generalized Effective Hypercube Nullstellensatz. The observed scaling $O(d^{2.1} \cdot m^{0.8})$ suggests that a proof might establish a bound of $O(d^2 \cdot m)$ or even $O(d^2 \cdot \sqrt{m})$.

The fact that certificate degree is essentially independent of the ambient dimension $n$ is notable and consistent with the polynomial-in-degree (not in $n$) nature of classical effective Nullstellensatz results~\cite{kollar1988sharp,brownawell1987bounds}.

\section{Conclusion}

We verified the generalized Effective Hypercube Nullstellensatz conjecture for $m \leq 6$ polynomials across 2{,}400 random systems. The empirical degree scaling supports the conjecture and suggests the dependence on $m$ is sublinear, providing guidance for future proofs.

\bibliographystyle{ACM-Reference-Format}
\bibliography{references}

\end{document}
