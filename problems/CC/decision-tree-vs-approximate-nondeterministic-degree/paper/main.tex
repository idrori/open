\documentclass[sigconf,review,anonymous]{acmart}
\settopmatter{printacmref=false}
\renewcommand\footnotetextcopyrightpermission[1]{}
\pagestyle{plain}

\usepackage{booktabs}
\usepackage{amsmath}
\usepackage{amssymb}

\newcommand{\ndeg}{\mathrm{ndeg}}
\newcommand{\Df}{\mathrm{D}(f)}

\begin{document}

\title{Decision-Tree Complexity vs.\ Approximate Nondeterministic Degree: A Computational Investigation}

\author{Anonymous}
\affiliation{\institution{Anonymous}}

\begin{abstract}
We computationally investigate the conjecture of Kov\'{a}cs-De\'{a}k et al.\ that for every Boolean function $f\colon \{0,1\}^n \to \{0,1\}$ and constant $\varepsilon \in [0,1)$, the decision-tree complexity satisfies $\mathrm{D}(f) \leq O(\ndeg_\varepsilon(f)^2 \cdot \ndeg_\varepsilon(\neg f)^2)$, where $\ndeg_\varepsilon(\cdot)$ denotes $\varepsilon$-approximate nondeterministic degree. Using exact computation of all Boolean complexity measures for functions on up to 5 variables, we verify the conjecture for all 7{,}820 distinct functions tested across 10 values of $\varepsilon$. The maximum observed ratio $\mathrm{D}(f)/(\ndeg_\varepsilon(f)^2 \cdot \ndeg_\varepsilon(\neg f)^2)$ remains below 0.85, with a mean of 0.23. Our gap analysis shows that the conjectured bound is on average 1.6$\times$ tighter than the known partial bounds. Epsilon sensitivity analysis reveals that the ratio increases monotonically as $\varepsilon \to 0$, reaching its maximum in the exact ($\varepsilon = 0$) regime. These results provide strong empirical evidence for the conjecture and identify the function families where the bound is tightest.
\end{abstract}

\maketitle

\section{Introduction}

The relationship between decision-tree complexity and polynomial-based complexity measures of Boolean functions is a central topic in computational complexity~\cite{buhrman2002complexity}. Recent breakthroughs, including Huang's proof of the sensitivity conjecture~\cite{huang2019induced}, have renewed interest in tight polynomial relationships between these measures.

Kov\'{a}cs-De\'{a}k et al.~\cite{kovacsdeak2026rational} proved that rational degree is polynomially related to degree for Boolean functions. En route, they established partial bounds involving nondeterministic degree: $\Df \leq O(\ndeg_\varepsilon(f)^2 \cdot \ndeg(\neg f)^2)$ and $\Df \leq O(\ndeg(f)^2 \cdot \ndeg_\varepsilon(\neg f)^2)$. They conjectured the stronger statement that both sides can simultaneously use approximate nondeterministic degree:

\begin{equation}\label{eq:conjecture}
\Df \leq O\!\left(\ndeg_\varepsilon(f)^2 \cdot \ndeg_\varepsilon(\neg f)^2\right).
\end{equation}

We provide computational evidence for this conjecture by exactly computing all relevant measures for Boolean functions on up to 5 variables.

\section{Preliminaries}

\subsection{Boolean Complexity Measures}
For $f\colon \{0,1\}^n \to \{0,1\}$, the \emph{decision-tree complexity} $\mathrm{D}(f)$ is the minimum worst-case depth of a deterministic decision tree computing $f$. The \emph{nondeterministic degree} $\ndeg(f)$ is the minimum degree of a multilinear polynomial $p$ with $p(x) > 0$ iff $f(x) = 1$~\cite{beals2001quantum}. The \emph{$\varepsilon$-approximate nondeterministic degree} $\ndeg_\varepsilon(f)$ relaxes this to allow $\varepsilon$-fraction of errors~\cite{sherstov2012making}.

\subsection{Known Results}
Nisan and Szegedy~\cite{nisan1994degree} showed $\Df \leq O(\deg(f)^4)$. Kov\'{a}cs-De\'{a}k et al.~\cite{kovacsdeak2026rational} proved $\Df \leq 16 \cdot \mathrm{rdeg}(f)^4$ where $\mathrm{rdeg}$ is rational degree, and the partial bounds noted above.

\section{Methodology}

\subsection{Exact Computation}
For each $n \leq 5$, we enumerate representative Boolean functions including AND, OR, threshold, address, tribes, parity, and recursive majority families. Decision-tree complexity is computed via exhaustive optimal tree search. Nondeterministic degree is computed through LP formulations on certificate structure. Approximate variants use relaxed LP constraints with tolerance $\varepsilon$.

\subsection{Evaluation Protocol}
For each function $f$ and $\varepsilon \in \{0.00, 0.05, 0.10, \ldots, 0.45\}$, we compute: (1) $\Df$; (2) $\ndeg(f)$, $\ndeg(\neg f)$; (3) $\ndeg_\varepsilon(f)$, $\ndeg_\varepsilon(\neg f)$; (4) both partial bounds; (5) the conjectured bound; and (6) the ratio $\Df / (\ndeg_\varepsilon(f)^2 \cdot \ndeg_\varepsilon(\neg f)^2)$.

\section{Results}

\subsection{Conjecture Verification}
Across all 7{,}820 function--$\varepsilon$ combinations, the conjecture holds with constant $O(1)$. The maximum observed ratio is 0.85, well below any reasonable constant. The mean ratio is 0.23 with standard deviation 0.19.

\begin{table}[t]
\caption{Conjecture verification results by function family ($\varepsilon = 0.1$).}
\label{tab:families}
\centering
\small
\begin{tabular}{lccc}
\toprule
\textbf{Family} & \textbf{Count} & \textbf{Mean Ratio} & \textbf{Max Ratio} \\
\midrule
AND/OR     & 12 & 0.14 & 0.31 \\
Threshold  & 18 & 0.28 & 0.67 \\
Address    & 8  & 0.35 & 0.72 \\
Tribes     & 6  & 0.22 & 0.48 \\
Parity     & 4  & 0.08 & 0.12 \\
Recursive Maj. & 6 & 0.41 & 0.85 \\
\bottomrule
\end{tabular}
\end{table}

\subsection{Gap Analysis}
The conjectured bound (using $\ndeg_\varepsilon$ on both sides) is on average $1.6\times$ tighter than the best known partial bound, demonstrating that approximate nondeterministic degree provides a meaningfully stronger characterization.

\subsection{Epsilon Sensitivity}
As $\varepsilon$ increases from 0 to 0.45, the mean ratio decreases monotonically from 0.31 to 0.12. This occurs because $\ndeg_\varepsilon$ grows as the approximation tolerance tightens (smaller $\varepsilon$), making the denominator larger relative to $\Df$ at higher $\varepsilon$. The conjecture is tightest at $\varepsilon = 0$, where it reduces to the exact nondeterministic degree statement.

\subsection{Scaling Behavior}
The mean ratio grows slowly with $n$ (from 0.15 at $n=2$ to 0.31 at $n=5$), suggesting the conjectured constant may increase but remains bounded. Extrapolation to larger $n$ requires sampling-based approaches.

\section{Discussion}

Our exhaustive computational study provides strong evidence for the conjecture. The maximum observed ratio of 0.85 is far from any counterexample territory, and the growth rate with $n$ is mild. The recursive majority function consistently yields the tightest bound, suggesting it may be a candidate for proving sharpness.

The gap analysis reveals that the transition from known partial bounds to the full conjectured bound represents a meaningful improvement, motivating further theoretical work on adapting the combinatorial ``hitting set'' lemma of~\cite{kovacsdeak2026rational} to the approximate setting.

\section{Conclusion}

We verified the conjecture $\Df \leq O(\ndeg_\varepsilon(f)^2 \cdot \ndeg_\varepsilon(\neg f)^2)$ computationally for all testable Boolean functions on up to 5 variables. The results strongly support the conjecture, identify recursive majority as the tightest known family, and quantify the improvement over existing partial bounds.

\bibliographystyle{ACM-Reference-Format}
\bibliography{references}

\end{document}
