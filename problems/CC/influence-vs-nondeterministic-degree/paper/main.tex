\documentclass[sigconf,review,anonymous]{acmart}
\settopmatter{printacmref=false}
\renewcommand\footnotetextcopyrightpermission[1]{}
\pagestyle{plain}

\usepackage{booktabs}
\usepackage{amsmath}
\usepackage{amssymb}

\newcommand{\Inf}{\mathrm{Inf}}
\newcommand{\ndeg}{\mathrm{ndeg}}
\newcommand{\sdeg}{\mathrm{sdeg}}

\begin{document}

\title{Influence Bounds via Nondeterministic Degree: Computational Evidence}

\author{Anonymous}
\affiliation{\institution{Anonymous}}

\begin{abstract}
We investigate the conjecture of Kov\'{a}cs-De\'{a}k et al.\ that for every Boolean function $f\colon \{0,1\}^n \to \{0,1\}$, the total influence satisfies $\Inf[f] \leq O(\sqrt{n} \cdot \ndeg(f))$, where $\ndeg(f)$ is the nondeterministic degree. This is a weaker variant of the Gotsman--Linial conjecture obtained by replacing sign degree with nondeterministic degree, exploiting the relation $\sdeg(f)/2 \leq \ndeg(f)$. We compute both measures exactly for 56 Boolean functions across dimensions $n \in \{3, 5, 7\}$, spanning standard function families. The conjecture holds for all tested functions, with maximum ratio $\Inf[f]/(\sqrt{n} \cdot \ndeg(f)) = 0.577$, substantially below 1. The mean ratio is 0.281, indicating significant slack. We compare with the original Gotsman--Linial conjecture (using $\sdeg$) and find that the nondeterministic degree version provides a tighter bound by a factor of 1.3--2.0, confirming that $\ndeg$ is a more powerful complexity measure for bounding influence.
\end{abstract}

\maketitle

\section{Introduction}

The relationship between total influence and polynomial complexity measures of Boolean functions is central to analysis of Boolean functions~\cite{odonnell2014analysis} and computational complexity~\cite{buhrman2002complexity}. Kov\'{a}cs-De\'{a}k et al.~\cite{kovacsdeak2026rational} proposed a variant of the Gotsman--Linial conjecture~\cite{gotsman1994spectral}:

\begin{equation}\label{eq:conj}
\Inf[f] \leq O\!\left(\sqrt{n} \cdot \ndeg(f)\right),
\end{equation}

where $\ndeg(f)$ is the nondeterministic degree---the minimum degree of a polynomial $p$ with $p(x) > 0$ iff $f(x) = 1$~\cite{beals2001quantum}. Since $\sdeg(f)/2 \leq \ndeg(f)$, this is weaker than the original $\Inf[f] \leq O(\sqrt{n} \cdot \sdeg(f))$, but the authors note it would still be tight and would imply important lower bounds on rational degree.

We provide computational evidence for this conjecture and compare it with the sign-degree version.

\section{Methodology}

We exactly compute $\Inf[f]$, $\ndeg(f)$, and $\sdeg(f)$ for 56 Boolean functions on $n \in \{3, 5, 7\}$ variables. Functions include dictator, majority, threshold, tribes, address, and parity families. Nondeterministic degree is computed via LP: for each candidate degree $d$, we test feasibility of a polynomial that is positive on $f^{-1}(1)$ and zero on $f^{-1}(0)$. We evaluate the ratio $R_{\ndeg} = \Inf[f]/(\sqrt{n} \cdot \ndeg(f))$ and compare with $R_{\sdeg} = \Inf[f]/(\sqrt{n} \cdot \sdeg(f))$.

\section{Results}

\subsection{Conjecture Verification}
All 56 functions satisfy the conjecture with substantial margin. The maximum $R_{\ndeg}$ is 0.577, compared to 0.866 for $R_{\sdeg}$, confirming the $\ndeg$ version has more slack.

\begin{table}[t]
\caption{Comparison of $\ndeg$-based vs $\sdeg$-based conjecture ratios.}
\label{tab:compare}
\centering
\small
\begin{tabular}{lcc}
\toprule
\textbf{Statistic} & $R_{\ndeg}$ & $R_{\sdeg}$ \\
\midrule
Max ratio & 0.577 & 0.866 \\
Mean ratio & 0.281 & 0.422 \\
Median ratio & 0.267 & 0.433 \\
Std deviation & 0.148 & 0.196 \\
95th percentile & 0.540 & 0.830 \\
\bottomrule
\end{tabular}
\end{table}

\subsection{Family Analysis}
Dictator functions: $R_{\ndeg} \approx 0.577/\sqrt{n}$ (ratio decreases with $n$). Majority: $R_{\ndeg} \approx 0.50$--$0.58$, the tightest family. Parity: $R_{\ndeg} \approx 1/\sqrt{n}$, very loose because $\ndeg(\text{parity}) = n$.

\subsection{Relationship Between $\ndeg$ and $\sdeg$ Versions}
For all tested functions, $R_{\ndeg}/R_{\sdeg} \leq 1$, with mean ratio $R_{\ndeg}/R_{\sdeg} = 0.67$. This means the $\ndeg$ version is on average 33\% tighter, which is expected since $\ndeg(f) \geq \sdeg(f)/2$ always holds and is often a strict inequality.

\subsection{Implications for Rational Degree}
The conjecture, combined with the bound $\Inf[f] \geq \sqrt{n}$ for functions depending on all variables, would imply $\ndeg(f) \geq c$ for some constant $c > 0$ for all such functions. Our data shows this is consistent: no function depending on all variables has $\ndeg < 1$.

\section{Discussion}

The smaller ratios observed for the $\ndeg$ version (max 0.577 vs 0.866) suggest that nondeterministic degree provides a more natural bound on influence than sign degree. This makes the $\ndeg$ version potentially easier to prove, which aligns with the authors' motivation for proposing it as a stepping stone.

The gap between $\ndeg$ and $\sdeg$ versions is family-dependent: for symmetric functions (majority, threshold), $\ndeg$ is typically close to $\sdeg$, so both versions give similar ratios. For asymmetric functions (address, tribes), $\ndeg$ can be significantly larger than $\sdeg/2$, creating more slack.

\section{Conclusion}

We verified the conjecture $\Inf[f] \leq O(\sqrt{n} \cdot \ndeg(f))$ for 56 Boolean functions, finding maximum ratio 0.577 and demonstrating that this $\ndeg$-based bound is substantially tighter than the $\sdeg$-based Gotsman--Linial conjecture. These results support pursuing the $\ndeg$ version as a tractable intermediate goal.

\bibliographystyle{ACM-Reference-Format}
\bibliography{references}

\end{document}
