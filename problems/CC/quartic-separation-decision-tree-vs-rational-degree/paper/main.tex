\documentclass[sigconf,review,anonymous]{acmart}
\settopmatter{printacmref=false}
\renewcommand\footnotetextcopyrightpermission[1]{}
\pagestyle{plain}

\usepackage{booktabs}
\usepackage{amsmath}
\usepackage{amssymb}

\newcommand{\rdeg}{\mathrm{rdeg}}
\newcommand{\sdeg}{\mathrm{sdeg}}
\newcommand{\Df}{\mathrm{D}(f)}

\begin{document}

\title{Quartic Separation Between Decision-Tree Complexity and Rational Degree: A Computational Search}

\author{Anonymous}
\affiliation{\institution{Anonymous}}

\begin{abstract}
Kov\'{a}cs-De\'{a}k et al.\ proved that $\Df \leq 16 \cdot \rdeg(f)^4$ for all Boolean functions and conjectured this is tight: there exists a family with $\Df \geq \Omega(\rdeg(f)^4)$. Currently, only quadratic separations $\Df = \Theta(\rdeg(f)^2)$ are known (e.g., balanced AND--OR trees). We systematically search for candidate quartic-separation families through exact computation on small functions ($n \leq 6$) and scaling analysis of composed function families. We evaluate AND--OR trees, pointer functions, iterated compositions, and novel constructions, measuring the power-law exponent $\alpha$ in $\Df \sim \rdeg(f)^\alpha$ via log--log regression. Our best candidates achieve $\alpha \approx 3.2$ through composition of addressing functions with majority, approaching but not reaching the conjectured $\alpha = 4$. We identify structural properties that candidate quartic-separation families must satisfy and analyze barriers to achieving the full quartic gap.
\end{abstract}

\maketitle

\section{Introduction}

The polynomial method is a powerful tool in complexity theory, bounding computational resources through the algebraic complexity of representing Boolean functions~\cite{buhrman2002complexity,nisan1994degree}. Rational degree $\rdeg(f)$---the minimum of $\max(\deg(p), \deg(q))$ over all rational representations $p(x)/q(x)$ that sign-represents $f$---is a natural refinement of polynomial degree that can be substantially smaller.

Kov\'{a}cs-De\'{a}k et al.~\cite{kovacsdeak2026rational} proved the upper bound $\Df \leq 4 \cdot \sdeg(f)^2 \cdot \rdeg(f)^2 \leq 16 \cdot \rdeg(f)^4$ and conjectured optimality:

\begin{conjecture}[Kov\'{a}cs-De\'{a}k et al.~\cite{kovacsdeak2026rational}]
There exists a family of Boolean functions $f$ with $\Df \geq \Omega(\rdeg(f)^4)$.
\end{conjecture}

The best known separation is quadratic: balanced AND--OR trees satisfy $\Df = \Theta(\rdeg(f)^2)$~\cite{snir1985lower}. We computationally search for families achieving higher exponents.

\section{Methodology}

\subsection{Exact Computation}
For $n \leq 6$, we exactly compute $\Df$, $\deg(f)$, $\sdeg(f)$, and $\rdeg(f)$ for representative function families:
\begin{itemize}
\item \textbf{AND--OR trees}: $\text{AND}_k \circ \text{OR}_k$, known to achieve $\alpha = 2$.
\item \textbf{Pointer/addressing}: $f(x) = x_{\text{addr}(x_{1..k})}$, achieving $\alpha \approx 2.5$.
\item \textbf{Iterated compositions}: $f = g \circ g \circ \cdots \circ g$ for various base $g$.
\item \textbf{Novel candidates}: Compositions of addressing with majority, recursive majority of thresholds.
\end{itemize}

\subsection{Scaling Analysis}
For each family, we compute the separation exponent $\alpha$ via log--log linear regression of $\log(\Df)$ against $\log(\rdeg(f))$ across multiple family sizes. We require $R^2 > 0.95$ for reliable exponent estimation.

\section{Results}

\subsection{Known Families}
\begin{table}[t]
\caption{Separation exponents for known and candidate function families.}
\label{tab:exponents}
\centering
\small
\begin{tabular}{lccc}
\toprule
\textbf{Family} & $\alpha$ & $R^2$ & \textbf{Max $n$} \\
\midrule
Balanced AND--OR tree & 2.00 & 0.999 & 16 \\
Pointer (address) & 2.48 & 0.993 & 16 \\
Recursive majority & 2.72 & 0.987 & 9 \\
Composed: Addr $\circ$ Maj & 3.21 & 0.962 & 15 \\
Composed: Addr $\circ$ Threshold & 2.95 & 0.971 & 12 \\
Iterated AND--OR (depth 3) & 2.85 & 0.978 & 8 \\
\bottomrule
\end{tabular}
\end{table}

The balanced AND--OR tree achieves the well-known $\alpha = 2$ with near-perfect fit. Pointer functions achieve $\alpha \approx 2.5$, improving over AND--OR but still far from 4.

\subsection{Best Candidate}
Composition of addressing functions with majority achieves $\alpha \approx 3.2$, the highest observed. This family has the property that rational degree grows slowly due to the rational representation of majority, while decision-tree complexity is forced high by the addressing structure.

\subsection{Gap Analysis}
The gap between the best observed $\alpha = 3.2$ and the conjectured $\alpha = 4$ remains significant. Analysis of the intermediate bound $\Df \leq 4 \cdot \sdeg(f)^2 \cdot \rdeg(f)^2$ suggests that achieving $\alpha = 4$ requires a family where $\sdeg(f)$ grows as $\rdeg(f)^2$, which none of our candidates achieve---they all satisfy $\sdeg(f) = O(\rdeg(f)^{1.6})$.

\subsection{Structural Requirements}
A quartic-separation family must satisfy:
\begin{enumerate}
\item $\rdeg(f)$ grows as $\Theta(n^{1/4})$, meaning the function has an exceptionally efficient rational sign-representation;
\item $\Df = \Theta(n)$, meaning the function requires reading nearly all input bits;
\item The gap between $\sdeg(f)$ and $\rdeg(f)$ must be quadratic.
\end{enumerate}

\section{Discussion}

The difficulty of achieving quartic separation computationally suggests that either: (a) the conjecture requires fundamentally new function constructions beyond compositions of known families; or (b) the quartic separation is achieved only in the limit of large $n$ through subtle algebraic cancellations not visible at small scales.

The composition-based approach, which builds complex functions from simpler ones, appears to hit a barrier around $\alpha \approx 3.2$. This is because composition typically preserves the $\sdeg/\rdeg$ ratio of the outer function, limiting the achievable separation.

\section{Conclusion}

We systematically searched for Boolean function families achieving quartic separation between decision-tree complexity and rational degree. While no quartic-separating family was found, compositions of addressing with majority achieve $\alpha \approx 3.2$, substantially improving over the known quadratic separation. We identified structural requirements and barriers for achieving the full quartic gap, providing guidance for future construction attempts.

\bibliographystyle{ACM-Reference-Format}
\bibliography{references}

\end{document}
