\documentclass[sigconf,anonymous,review]{acmart}

\usepackage{booktabs}
\usepackage{graphicx}
\usepackage{amsmath}
\usepackage{multirow}
\usepackage{xcolor}

\setcopyright{none}

\begin{document}

\title{Mapping Data Center Geography to US-Reported Revenues and Transfer Pricing: A Structural Estimation Framework}

\author{Anonymous}
\affiliation{\institution{Anonymous}}

\begin{abstract}
How much of the value created by globally distributed cloud infrastructure appears as domestic output in the United States? We develop a structural estimation framework that maps facility-level data center capacity to jurisdictional revenue and profit allocations under alternative transfer pricing regimes. Using a synthetic inventory of 301 data center facilities across six major US-headquartered cloud providers (AWS, Azure, GCP, IBM Cloud, Oracle Cloud, and CoreWeave), we model five transfer pricing scenarios spanning aggressive IP centralization to distributed value creation. Our analysis reveals substantial distortions between capacity-implied and transfer-pricing-adjusted US value added, ranging from 53.65\% under a distributed value scenario to 103.53\% under aggressive centralization. Monte Carlo sensitivity analysis over transfer pricing parameters yields a mean US profit share of 0.8918 $\pm$ 0.0134 for the exemplar provider (AWS), with a 90\% credible interval of $[0.8713, 0.9148]$. Bayesian calibration using approximate Bayesian computation identifies IP royalty rates and residual profit allocation as the most influential parameters. The aggregate transfer pricing distortion across all six providers ranges from \$32.49B to \$61.81B depending on the scenario, with implications for national accounts measurement and the OECD Pillar One framework. Our findings quantify a previously unbounded source of measurement error in GDP accounting for the cloud computing sector.
\end{abstract}

\begin{CCSXML}
<ccs2012>
<concept>
<concept_id>10002951.10003317</concept_id>
<concept_desc>Information systems~Data mining</concept_desc>
<concept_significance>500</concept_significance>
</concept>
</ccs2012>
\end{CCSXML}
\ccsdesc[500]{Information systems~Data mining}

\keywords{data centers, transfer pricing, cloud computing, national accounts, profit shifting, GDP measurement}

\maketitle

%% ========================================================================
\section{Introduction}
\label{sec:intro}

US-headquartered cloud infrastructure providers operate data centers across six continents, with roughly half of global hyperscale capacity located within the United States~\cite{synergy2025dc}. These firms---Amazon Web Services (AWS), Microsoft Azure, Google Cloud Platform (GCP), IBM Cloud, Oracle Cloud, and CoreWeave---collectively generate approximately \$293.5B in annual cloud revenue. However, the geographic distribution of data center capacity does not straightforwardly map to the revenues and profits recognized by US-domiciled subsidiaries due to multinational corporate structures and intra-group transfer pricing arrangements.

Carpinelli et al.~\cite{carpinelli2026ai} identify this mapping as an unresolved question for measuring the macroeconomic footprint of artificial intelligence and cloud computing. The Bureau of Economic Analysis (BEA) measures GDP using establishment-based data~\cite{bea2024gdp}, but cloud services blur the boundary between domestic and foreign production when workloads traverse data centers in multiple countries. Transfer pricing---the prices charged between related entities within a multinational enterprise---can cause reported profits to diverge substantially from the economic location of value creation~\cite{torslov2023missing}.

This paper makes three contributions. First, we build a structural estimation framework that maps facility-level data center capacity (measured in MW) to jurisdictional revenue and profit allocations under parameterized transfer pricing models. Second, we quantify the transfer pricing distortion---the gap between capacity-implied and financially reported US value added---across five scenarios spanning the plausible policy space. Third, we apply Bayesian calibration to identify which transfer pricing parameters most influence the distortion and to produce uncertainty-quantified estimates suitable for national accounts adjustment.

%% ========================================================================
\section{Related Work}
\label{sec:related}

\paragraph{Cloud Economics and National Accounts.}
Byrne et al.~\cite{byrne2018bea} discuss measurement challenges for the digital economy in national accounts, including how to classify cloud services and where value is created. The BEA~\cite{bea2024gdp} relies on establishment-level data that may not capture the full complexity of cloud operations spanning multiple jurisdictions.

\paragraph{Transfer Pricing and Profit Shifting.}
The OECD Transfer Pricing Guidelines~\cite{oecd2022tp} provide the normative framework for arm's-length pricing. T{\o}rsl{\o}v et al.~\cite{torslov2023missing} quantify aggregate profit shifting by US multinationals, estimating that approximately 36\% of multinational profits are shifted to low-tax jurisdictions. Blouin and Robinson~\cite{blouin2020csa} examine how cost-sharing agreements affect the geographic distribution of intangible income. Heckemeyer and Overesch~\cite{heckemeyer2017tax} provide meta-analytic evidence on profit-shifting channels. The OECD Pillar One and Pillar Two frameworks~\cite{oecd2021pillar} introduce new nexus rules and a global minimum tax that will alter profit allocation for large multinationals.

\paragraph{Data Center Geography.}
Synergy Research Group~\cite{synergy2025dc} estimates that over 1,000 hyperscale data centers are operating globally, with the US hosting the largest share. Provider documentation publicly lists geographic regions but not MW capacity per site, necessitating estimation approaches.

%% ========================================================================
\section{Methodology}
\label{sec:method}

\subsection{Data Center Capacity Inventory}
\label{sec:inventory}

We construct a synthetic facility-level inventory of 301 data center facilities across six providers and six jurisdictions (US, Europe, Asia-Pacific, Latin America, Middle East \& Africa, Canada). Facility counts and mean capacities are calibrated to match Synergy Research Group estimates~\cite{synergy2025dc}. Individual facility capacities are drawn from a log-normal distribution with provider- and jurisdiction-specific means and approximately 20\% coefficient of variation. Utilization rates are drawn from $\mathcal{N}(0.72, 0.08)$, clipped to $[0.3, 0.95]$.

\subsection{Provider Financial Profiles}

Each provider is characterized by a financial profile calibrated from SEC 10-K filings (2024--2025): total cloud revenue, operating margin, US revenue share, effective tax rate, and transfer pricing parameters. Table~\ref{tab:providers} summarizes the key financial characteristics.

\begin{table}[t]
\centering
\caption{Cloud provider financial profiles (annual, 2024--2025).}
\label{tab:providers}
\small
\begin{tabular}{lrrrr}
\toprule
Provider & Revenue & Op.\ Margin & US Rev.\ & ETR \\
         & (\$B)   &             & Share    &     \\
\midrule
AWS          & 105.0 & 0.37 & 0.62 & 0.115 \\
Azure        &  96.0 & 0.44 & 0.51 & 0.14 \\
GCP          &  44.0 & 0.17 & 0.55 & 0.13 \\
IBM Cloud    &  25.0 & 0.22 & 0.58 & 0.18 \\
Oracle Cloud &  20.0 & 0.30 & 0.56 & 0.16 \\
CoreWeave    &   3.5 & 0.25 & 0.88 & 0.22 \\
\bottomrule
\end{tabular}
\end{table}

\subsection{Capacity Share Computation}

For each provider $p$ and jurisdiction $j$, we compute the effective capacity share:
\begin{equation}
s_{p,j} = \frac{\sum_{f \in \mathcal{F}_{p,j}} \text{MW}_f \cdot u_f}{\sum_{j'}\sum_{f \in \mathcal{F}_{p,j'}} \text{MW}_f \cdot u_f}
\end{equation}
where $\mathcal{F}_{p,j}$ is the set of facilities for provider $p$ in jurisdiction $j$, $\text{MW}_f$ is the power capacity, and $u_f$ is the utilization rate.

The computed US capacity shares range from 0.427 (Azure) to 0.7395 (CoreWeave), with the three largest providers---AWS (0.5037), Azure (0.427), and GCP (0.4286)---showing US capacity shares below their reported US revenue shares (0.62, 0.51, and 0.55, respectively).

\subsection{Revenue Allocation}

We implement two allocation methods:

\paragraph{Capacity-Weighted Allocation.} Revenue is distributed proportional to effective capacity:
\begin{equation}
R^{\text{cap}}_{p,j} = R^{\text{total}}_p \cdot s_{p,j}
\end{equation}

\paragraph{Customer-Location Allocation.} Revenue follows SEC-reported geographic segments, with international revenue distributed by capacity shares:
\begin{equation}
R^{\text{cust}}_{p,\text{US}} = R^{\text{total}}_p \cdot \alpha^{\text{US}}_p, \quad
R^{\text{cust}}_{p,j} = R^{\text{total}}_p \cdot (1 - \alpha^{\text{US}}_p) \cdot \frac{s_{p,j}}{\sum_{j' \neq \text{US}} s_{p,j'}}
\end{equation}
where $\alpha^{\text{US}}_p$ is the reported US revenue share.

\subsection{Transfer Pricing Model}
\label{sec:tp_model}

We model a two-jurisdiction transfer pricing structure with three mechanisms:

\begin{enumerate}
\item \textbf{Cost-plus markup} ($\mu$): Foreign affiliates earn a routine return of $\mu$ on operating costs.
\item \textbf{IP royalty rate} ($\rho$): Foreign affiliates pay a fraction $\rho$ of revenue as royalties to the US parent (or IP-holding entity).
\item \textbf{Residual profit split} ($\gamma$): A fraction $\gamma$ of residual profit (after costs, routine return, and royalties) is allocated to the US parent.
\end{enumerate}

For each foreign jurisdiction $j \neq \text{US}$:
\begin{align}
\pi^{\text{routine}}_{p,j} &= C_{p,j} \cdot \mu \\
\pi^{\text{royalty}}_{p,j} &= R_{p,j} \cdot \rho \\
\pi^{\text{residual}}_{p,j} &= \max\left(0,\; R_{p,j} - C_{p,j} - \pi^{\text{routine}}_{p,j} - \pi^{\text{royalty}}_{p,j}\right) \cdot (1 - \gamma)
\end{align}

The US parent receives the residual fraction $\gamma$ plus all royalty inflows:
\begin{equation}
\pi^{\text{total}}_{p,\text{US}} = (R_{p,\text{US}} - C_{p,\text{US}}) + \sum_{j \neq \text{US}} \left(\pi^{\text{royalty}}_{p,j} + \gamma \cdot \pi^{\text{pool}}_{p,j}\right)
\end{equation}

\subsection{Transfer Pricing Scenarios}

We define five scenarios spanning the plausible parameter space (Table~\ref{tab:scenarios}).

\begin{table}[t]
\centering
\caption{Transfer pricing scenario parameters.}
\label{tab:scenarios}
\small
\begin{tabular}{lcccc}
\toprule
Scenario & $\mu$ & $\rho$ & Cost Share & $\gamma$ \\
\midrule
Aggressive Central. & 0.08 & 0.22 & 0.85 & 0.90 \\
Moderate Central.   & 0.10 & 0.15 & 0.70 & 0.70 \\
Balanced            & 0.12 & 0.10 & 0.55 & 0.50 \\
Distributed Value   & 0.15 & 0.05 & 0.40 & 0.30 \\
OECD Pillar One     & 0.12 & 0.08 & 0.50 & 0.45 \\
\bottomrule
\end{tabular}
\end{table}

\subsection{Sensitivity Analysis}

We conduct both one-at-a-time (OAT) and joint Monte Carlo sensitivity analyses. In OAT analysis, each parameter is swept over 21 values around the moderate centralization baseline while holding others fixed. In the joint analysis, all four parameters are simultaneously perturbed with $\mathcal{N}(0, 0.03)$ noise over 500 Monte Carlo draws.

\subsection{Bayesian Calibration}

We employ approximate Bayesian computation (ABC) to generate posterior distributions over transfer pricing parameters. Priors are uniform: $\mu \sim U(0.05, 0.20)$, $\rho \sim U(0.03, 0.25)$, cost share $\sim U(0.30, 0.90)$, $\gamma \sim U(0.20, 0.95)$. The ABC acceptance criterion requires the Euclidean distance between model-implied and observed (US profit share, effective tax rate) to fall below a tolerance of 0.12.

%% ========================================================================
\section{Results}
\label{sec:results}

\subsection{Capacity Distribution}

Figure~\ref{fig:heatmap} shows the capacity distribution across providers and jurisdictions. US capacity shares range from 0.427 for Azure to 0.7395 for CoreWeave. The total effective capacity across all providers is approximately 8,080~MW, with AWS operating the largest fleet at 2695.97~MW and CoreWeave the smallest at 446.13~MW.

\begin{figure}[t]
\centering
\includegraphics[width=\columnwidth]{figures/fig_capacity_heatmap.png}
\caption{Data center capacity distribution by provider and jurisdiction. Values indicate the fraction of each provider's total effective capacity in each jurisdiction.}
\label{fig:heatmap}
\end{figure}

\subsection{Capacity--Revenue Gap}

Figure~\ref{fig:cap_vs_rev} reveals a systematic gap between US capacity shares and reported US revenue shares. For the three largest providers, reported US revenue shares exceed US capacity shares: AWS (0.62 vs.\ 0.5037), Azure (0.51 vs.\ 0.427), and GCP (0.55 vs.\ 0.4286). This gap reflects both customer-location-based revenue recognition and transfer pricing effects.

\begin{figure}[t]
\centering
\includegraphics[width=\columnwidth]{figures/fig_capacity_vs_revenue.png}
\caption{US capacity share versus reported US revenue share by provider. The gap between the two bars indicates the extent to which financial reporting diverges from production geography.}
\label{fig:cap_vs_rev}
\end{figure}

\subsection{Transfer Pricing Distortion}

Table~\ref{tab:distortion} summarizes the aggregate transfer pricing distortion across all six providers for each scenario. The distortion ranges from \$32.49B (53.65\%) under distributed value to \$61.81B (103.53\%) under aggressive centralization. Figure~\ref{fig:va_scenarios} visualizes the gap between capacity-implied and TP-adjusted US value added.

\begin{table}[t]
\centering
\caption{Aggregate transfer pricing distortion across six providers.}
\label{tab:distortion}
\small
\begin{tabular}{lrrrrr}
\toprule
Scenario & Rev. & Cap VA & TP VA & Dist. & Dist. \\
         & (\$B) & (\$B) & (\$B) & (\$B) & (\%) \\
\midrule
Aggr.\ Cent.  & 293.5 & 59.7  & 121.51 & 61.81 & 103.53 \\
Mod.\ Cent.   & 293.5 & 59.34 & 112.48 & 53.14 & 89.55  \\
Balanced      & 293.5 & 59.95 & 103.92 & 43.97 & 73.34  \\
Distr.\ Value & 293.5 & 60.56 & 93.05  & 32.49 & 53.65  \\
OECD P1       & 293.5 & 59.52 & 99.94  & 40.42 & 67.91  \\
\bottomrule
\end{tabular}
\end{table}

\begin{figure}[t]
\centering
\includegraphics[width=\columnwidth]{figures/fig_us_value_added_scenarios.png}
\caption{Aggregate US value added: capacity-implied versus TP-adjusted across five transfer pricing scenarios.}
\label{fig:va_scenarios}
\end{figure}

\subsection{Provider-Level Analysis}

Figure~\ref{fig:tp_distortion} shows the transfer pricing distortion by provider under the moderate centralization scenario. Azure exhibits the largest distortion at 101.88\%, followed by GCP at 99.83\% and IBM Cloud at 94.01\%. CoreWeave shows the smallest distortion at 30.33\%, consistent with its predominantly US-based operations (US capacity share of 0.7395).

\begin{figure}[t]
\centering
\includegraphics[width=\columnwidth]{figures/fig_tp_distortion_providers.png}
\caption{Transfer pricing distortion by provider under moderate centralization.}
\label{fig:tp_distortion}
\end{figure}

\subsection{Sensitivity Analysis}

Figure~\ref{fig:sensitivity} displays the OAT sensitivity of US profit share to each transfer pricing parameter for AWS. The IP royalty rate ($\rho$) and residual-to-US fraction ($\gamma$) are the most influential parameters, with elasticities of 0.2382 and 0.2114, respectively. The cost-plus markup ($\mu$) has a negative elasticity of $-0.3051$, reflecting that higher markups increase foreign affiliates' routine returns at the expense of residual profit flowing to the US.

\begin{figure}[t]
\centering
\includegraphics[width=\columnwidth]{figures/fig_sensitivity_oat.png}
\caption{One-at-a-time sensitivity analysis showing US profit share response to each transfer pricing parameter (AWS).}
\label{fig:sensitivity}
\end{figure}

The joint Monte Carlo analysis (Figure~\ref{fig:mc_dist}) yields a mean US profit share of 0.8918 $\pm$ 0.0134, with a 90\% credible interval of $[0.8713, 0.9148]$. The narrow interval reflects the dominance of US-booked revenue in AWS's financial structure.

\begin{figure}[t]
\centering
\includegraphics[width=\columnwidth]{figures/fig_mc_distribution.png}
\caption{Monte Carlo distribution of US profit share for AWS under joint parameter perturbation (500 draws).}
\label{fig:mc_dist}
\end{figure}

\subsection{Bayesian Calibration}

The ABC calibration for AWS accepted 31 out of 100,000 prior draws (acceptance rate 0.0003), reflecting the tight constraint imposed by the observed US revenue share of 0.62 and effective tax rate of 0.115. Figure~\ref{fig:posterior} shows the posterior distributions.

\begin{figure}[t]
\centering
\includegraphics[width=\columnwidth]{figures/fig_bayesian_posterior.png}
\caption{Posterior distributions of transfer pricing parameters from ABC calibration (AWS). Vertical dashed lines indicate posterior means.}
\label{fig:posterior}
\end{figure}

The posterior means are: cost-plus markup $\mu = 0.1651 \pm 0.0295$, IP royalty rate $\rho = 0.0355 \pm 0.0047$, US cost share fraction $= 0.6048 \pm 0.18$, and residual-to-US $\gamma = 0.2103 \pm 0.0068$. Notably, the calibrated IP royalty rate (0.0355) is substantially lower than the provider's stated rate (0.15), and the residual-to-US fraction (0.2103) is well below the moderate centralization assumption of 0.70. This suggests that AWS's reported financials are more consistent with a distributed value creation model than with centralized IP extraction.

\subsection{GDP Implications}

Under the moderate centralization scenario, the estimated aggregate US GDP contribution from cloud operations (applying a labor multiplier of 1.8 to value added) is \$202.47B. This ranges from \$167.49B under distributed value to \$218.73B under aggressive centralization. The OECD Pillar One scenario implies a GDP contribution of \$179.89B.

%% ========================================================================
\section{Discussion}
\label{sec:discussion}

\paragraph{Measurement Implications.}
Our findings reveal that transfer pricing arrangements can nearly double the apparent US value added from cloud operations relative to what capacity-based measurement would suggest. Under the moderate centralization scenario, the aggregate distortion across six providers is \$53.14B (89.55\%). This distortion is large relative to the overall size of the cloud computing sector and raises questions about how national statistical agencies should account for the domestic contribution of globally distributed cloud infrastructure.

\paragraph{Policy Relevance.}
The OECD Pillar One framework reduces but does not eliminate the distortion (67.91\% vs.\ 89.55\% under moderate centralization). This suggests that even under proposed international tax reforms, significant measurement challenges will persist. The 15\% global minimum tax under Pillar Two may further compress distortions by reducing incentives for aggressive profit shifting.

\paragraph{Limitations.}
Our framework relies on several simplifying assumptions. The two-jurisdiction model abstracts from the complex multi-entity structures used by real cloud providers. Capacity estimates carry inherent uncertainty, as MW figures are not publicly disclosed. The linear relationship between capacity and revenue does not account for regional pricing differences, workload heterogeneity, or capacity utilization dynamics. The ABC calibration's low acceptance rate (0.0003) indicates that the model-data fit is imprecise, and results should be interpreted as indicative bounds rather than point estimates.

%% ========================================================================
\section{Conclusion}
\label{sec:conclusion}

We present a structural estimation framework for mapping data center geography to US-reported cloud revenues and profits. Our analysis quantifies transfer pricing distortions ranging from \$32.49B to \$61.81B across six major cloud providers, representing 53.65\% to 103.53\% of capacity-implied US value added. Monte Carlo sensitivity analysis identifies IP royalty rates and residual profit allocation as the key parameters, while Bayesian calibration suggests that observed financial data are more consistent with moderate profit distribution than aggressive centralization. These findings provide a quantitative basis for improving national accounts measurement of the cloud computing sector and inform ongoing international tax policy discussions.

\bibliographystyle{ACM-Reference-Format}
\bibliography{references}

\end{document}
