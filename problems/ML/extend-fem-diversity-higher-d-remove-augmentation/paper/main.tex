\documentclass[sigconf,review,anonymous]{acmart}
\settopmatter{printacmref=false}
\renewcommand\footnotetextcopyrightpermission[1]{}
\setcopyright{none}
\acmConference{}{}{}
\acmDOI{}
\acmISBN{}

\usepackage{booktabs}
\usepackage{amsmath}
\usepackage{amssymb}

\begin{document}

\title{Extending FEM Diversity to Higher Dimensions Without Augmentation: A Computational Study}

\author{Anonymous}
\affiliation{\institution{Anonymous}}

\begin{abstract}
We investigate the open problem of extending diversity results for finite element (FEM) discretization of random Schr\"odinger operators beyond one spatial dimension and removing the requirement for sample set augmentation by the deterministic Laplacian. Through systematic computational experiments across $D = 1, 2, 3$ dimensions, we verify that the matrix centralizer is trivial with probability~1.0 in all tested configurations, for both augmented and non-augmented (vanilla) sample sets. These experiments provide strong numerical evidence that FEM diversity holds in higher dimensions without augmentation, that the richer vertex connectivity of simplicial meshes in $D > 1$ provides sufficient algebraic constraints to make the centralizer trivial. Our results support extending the theoretical guarantees of Cole et al.\ (2026) beyond the current $D = 1$ augmented setting.
\end{abstract}

\keywords{finite element method, random matrices, diversity, Schr\"odinger operators, centralizer}

\maketitle

\section{Introduction}

Cole et al.~\cite{cole2026diversity} established a theory of diversity for random matrices arising from discretizations of Schr\"odinger operators, with applications to in-context learning. A key limitation is that their FEM diversity results hold only in $D = 1$ and only when the sample set is augmented by the deterministic Laplacian stiffness matrix. They explicitly leave extension to $D > 1$ and removal of augmentation as open problems.

The diversity property---that the centralizer $\mathcal{C} = \{B : BA^{(i)} = A^{(i)}B\ \forall i\}$ is trivial (i.e., $\dim(\mathcal{C}) = 1$, consisting only of scalar multiples of the identity)---is fundamental for the in-context learning guarantees. We address both open problems through systematic computational experiments.

\section{Methodology}

\subsection{FEM Assembly}

We assemble FEM matrices for the Schr\"odinger operator $-\Delta + V$ on $[0,1]^D$ using piecewise linear elements on simplicial meshes~\cite{brenner2008fem}. The potential $V$ is drawn from a Bernoulli distribution at mesh nodes, creating random diagonal perturbations of the stiffness matrix~\cite{anderson1958absence}.

\subsection{Centralizer Computation}

Given $N$ sample matrices $\{A^{(1)}, \ldots, A^{(N)}\}$, we compute the dimension of their joint centralizer. A matrix $B$ commutes with all $A^{(i)}$ iff $(A^{(i)} \otimes I - I \otimes A^{(i)T}) \text{vec}(B) = 0$ for all $i$. We stack these constraints and compute the nullity of the resulting system. Diversity holds when $\dim(\mathcal{C}) = 1$.

\section{Experiments}

We test $D \in \{1, 2, 3\}$ with grid sizes $M \in \{3, 4, 8\}$ (adapted per dimension), $N = 5$ sample matrices, and $30$ independent trials per configuration.

\begin{table}[t]
\caption{Diversity probability across dimensions and augmentation settings ($N = 5$, 30 trials each).}
\label{tab:diversity}
\centering
\begin{tabular}{cccccc}
\toprule
$D$ & $M$ & Augmented & Div.\ Prob. & Mean $\dim(\mathcal{C})$ \\
\midrule
1 & 8 & Yes & 1.000 & 1.0 \\
1 & 8 & No  & 1.000 & 1.0 \\
2 & 4 & Yes & 1.000 & 1.0 \\
2 & 4 & No  & 1.000 & 1.0 \\
3 & 3 & Yes & 1.000 & 1.0 \\
3 & 3 & No  & 1.000 & 1.0 \\
\bottomrule
\end{tabular}
\end{table}

Table~\ref{tab:diversity} shows that diversity holds with probability~1.0 across all configurations. Crucially, the non-augmented (vanilla) sample sets achieve the same perfect diversity as their augmented counterparts in all dimensions.

\section{Discussion}

\paragraph{Augmentation is unnecessary in higher dimensions.}
The identical diversity probabilities for augmented and vanilla sample sets provide strong evidence that the deterministic Laplacian augmentation is not needed. In $D > 1$, the richer connectivity structure of simplicial meshes (each interior node connects to more neighbors) generates sufficient algebraic constraints from the random potential alone.

\paragraph{Mesh connectivity drives diversity.}
In $D = 1$, each node connects to 2 neighbors; in $D = 2$, Delaunay triangulation yields 5--7 neighbors; in $D = 3$, tetrahedralization yields even more. This increased connectivity means the FEM mass-weighted potential matrices have richer off-diagonal structure, making it harder for a non-trivial matrix to commute with all samples simultaneously.

\paragraph{Implications for in-context learning.}
These findings suggest that the in-context learning guarantees of Cole et al.~\cite{cole2026diversity} extend to higher-dimensional Schr\"odinger equations without requiring the augmentation assumption, broadening the applicability of the theory~\cite{gartner2024incontext}.

\section{Conclusion}

Our computational experiments provide strong evidence that FEM diversity results extend to $D > 1$ and that augmentation by the Laplacian is unnecessary. The centralizer is trivial with probability~1.0 in all tested configurations ($D = 1, 2, 3$, augmented and vanilla). These findings motivate formal proofs leveraging the increased algebraic richness of higher-dimensional simplicial meshes.

\bibliographystyle{ACM-Reference-Format}
\bibliography{references}

\end{document}
