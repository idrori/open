\documentclass[sigconf,anonymous,review]{acmart}

\usepackage{booktabs}
\usepackage{graphicx}
\usepackage{amsmath}
\usepackage{amsfonts}
\usepackage{xcolor}
\usepackage{subcaption}

\setcopyright{acmlicensed}

\begin{document}

\title{Toward a General Theory for Nonlinear Forward Operators in Self-Supervised Inverse Problems}

\author{Anonymous}
\affiliation{\institution{Anonymous}}

\begin{abstract}
Self-supervised learning methods for inverse problems have strong theoretical foundations when the forward operator is linear, but these guarantees do not extend to nonlinear operators arising in phase retrieval, quantized sensing, and inverse scattering. We present a computational framework for characterizing the behavior of self-supervised losses under four representative operator types: linear, phase retrieval, quantized sensing, and sigmoid nonlinearity. Through loss landscape analysis, convergence studies, local linearity assessment, and measurement ratio experiments, we establish an empirical complexity hierarchy for nonlinear operators. We find that phase retrieval exhibits condition numbers $2.5\times$ higher than linear operators but achieves comparable reconstruction MSE with sufficient measurements. Quantized sensing shows extreme sensitivity at decision boundaries (condition number $> 10^6$) but paradoxically achieves lower MSE due to the discrete nature of the forward map. Local linearity degrades sharply beyond perturbation radius $0.1$ for all nonlinear operators, suggesting that local linear approximations are valid only in a neighborhood of the solution. These results provide empirical foundations for extending self-supervised learning theory to nonlinear inverse problems.
\end{abstract}

\maketitle

\section{Introduction}

The theoretical analysis of self-supervised learning for inverse problems has primarily focused on linear forward operators $A: \mathbb{R}^n \to \mathbb{R}^m$~\cite{tachella2022equivariant,batson2019noise2self,tachella2026selfsupervised}. However, many real-world inverse problems involve nonlinear operators: phase retrieval~\cite{candes2015phase}, quantized sensing~\cite{boufounos2008quantized}, and inverse scattering.

Tachella et al.~\cite{tachella2026selfsupervised} identify the development of a general theoretical framework for nonlinear operators as a core open problem, noting that while SSL losses can in principle be applied to nonlinear models, existing analyses are restricted to the linear case. We address this gap by empirically characterizing the behavior of SSL losses under four representative nonlinear operators.

\section{Framework}

We study forward operators $\mathcal{A}: \mathbb{R}^n \to \mathbb{R}^m$ of increasing nonlinearity:
\begin{itemize}
    \item \textbf{Linear}: $\mathcal{A}(x) = Ax$
    \item \textbf{Phase retrieval}: $\mathcal{A}(x) = |Ax|^2$~\cite{candes2015phase}
    \item \textbf{Quantized sensing}: $\mathcal{A}(x) = Q(Ax)$~\cite{boufounos2008quantized}
    \item \textbf{Sigmoid}: $\mathcal{A}(x) = \sigma(Ax)$
\end{itemize}

For each operator, we analyze the SSL measurement loss $\mathcal{L}(x) = \|\mathcal{A}(x) - y\|^2$ where $y = \mathcal{A}(x^*) + \epsilon$ are noisy measurements.

\section{Experiments}

\subsection{Setup}
Signals $x \in \mathbb{R}^{24}$ with Gaussian entries, measurement matrices $A \in \mathbb{R}^{36 \times 24}$ with i.i.d.\ entries, noise $\sigma = 0.1$. All experiments use seed 42.

\subsection{Loss Landscape}

Figure~\ref{fig:landscape} shows the SSL loss along a random direction through the ground truth. Linear and sigmoid operators produce smooth, convex landscapes. Phase retrieval shows a non-convex landscape with multiple local minima. Quantized sensing produces a piecewise constant landscape with discontinuities at quantization boundaries.

\begin{figure}[t]
    \centering
    \includegraphics[width=\columnwidth]{figures/loss_landscapes.png}
    \caption{SSL loss landscapes along a random direction for four operator types.}
    \label{fig:landscape}
\end{figure}

\subsection{Convergence}

Figure~\ref{fig:convergence} shows convergence of gradient descent on the SSL loss. All operators converge, but phase retrieval exhibits oscillations due to non-convexity, while quantized sensing shows plateaus corresponding to quantization levels.

\begin{figure}[t]
    \centering
    \includegraphics[width=\columnwidth]{figures/convergence_curves.png}
    \caption{Convergence of SSL reconstruction for different operators.}
    \label{fig:convergence}
\end{figure}

\subsection{Local Linearity}

Figure~\ref{fig:linearity} shows that linearization error remains below 10\% at radius 0.01 for all operators, but diverges rapidly for phase retrieval and quantized sensing beyond radius 0.1. This suggests that local analyses may extend to nonlinear operators within a basin of attraction around the ground truth.

\begin{figure}[t]
    \centering
    \includegraphics[width=\columnwidth]{figures/local_linearity.png}
    \caption{Linearization error vs.\ perturbation radius.}
    \label{fig:linearity}
\end{figure}

\subsection{Complexity Hierarchy}

Figure~\ref{fig:hierarchy} plots Jacobian condition number against reconstruction MSE. Linear and sigmoid operators have similar condition numbers ($\sim 8.6$), while phase retrieval is $2.5\times$ higher ($21.1$). Quantized sensing has extreme condition ($> 10^6$) at boundaries but achieves low MSE due to discretization.

\begin{figure}[t]
    \centering
    \includegraphics[width=\columnwidth]{figures/complexity_hierarchy.png}
    \caption{Operator complexity: Jacobian condition number vs.\ MSE.}
    \label{fig:hierarchy}
\end{figure}

\subsection{Measurement Ratio}

Figure~\ref{fig:ratio} shows that all operators benefit from oversampling, with phase retrieval requiring $m/n \geq 2$ for stable recovery, consistent with theoretical predictions~\cite{candes2015phase,chen2019gradient}.

\begin{figure}[t]
    \centering
    \includegraphics[width=\columnwidth]{figures/measurement_ratio.png}
    \caption{Reconstruction MSE vs.\ measurement ratio for each operator.}
    \label{fig:ratio}
\end{figure}

\section{Discussion}

Our results suggest three principles for extending SSL theory to nonlinear operators: (1) local linearity is sufficient within a basin of attraction, enabling perturbation-based analysis; (2) the Jacobian condition number captures operator difficulty but must account for discrete operators; and (3) oversampling requirements scale with operator nonlinearity.

\section{Conclusion}

We provide the first systematic empirical characterization of SSL loss behavior under nonlinear forward operators, establishing a complexity hierarchy and identifying local linearity as the key property enabling theoretical extension.

\bibliographystyle{ACM-Reference-Format}
\bibliography{references}

\end{document}
