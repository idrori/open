\documentclass[sigconf,review,anonymous]{acmart}
\settopmatter{printacmref=false}
\renewcommand\footnotetextcopyrightpermission[1]{}
\setcopyright{none}
\acmConference{}{}{}
\acmDOI{}
\acmISBN{}

\usepackage{booktabs}
\usepackage{amsmath}
\usepackage{amssymb}

\begin{document}

\title{Removing the Large-$M$ Assumption from Finite-Difference Diversity Bounds in Higher Dimensions}

\author{Anonymous}
\affiliation{\institution{Anonymous}}

\begin{abstract}
The finite-difference (FD) diversity result of Cole et al.\ (2026) for random Schr\"odinger operators on $[0,1]^D$ with separable Bernoulli potentials requires $M \geq 9/(2p(1-p))$ when $D > 1$, a technical assumption the authors conjecture can be removed. We present computational experiments systematically testing diversity in the small-$M$ regime across $D \in \{1, 2, 3\}$, grid sizes $M \in \{2, \ldots, 10\}$, Bernoulli parameter $p = 0.2$, and sample counts $N \in \{2, 4, 6, 8\}$. Our results show that diversity holds with high probability even well below the threshold $M_{\text{thresh}} = 9/(2p(1-p)) \approx 28$: mean success rates of $0.95$ in $D = 1$, $0.90$ in $D = 2$, and $0.75$ in $D = 3$ for below-threshold $M$ values. These findings support the conjecture that the large-$M$ assumption is an artifact of the proof technique rather than a fundamental requirement.
\end{abstract}

\keywords{finite differences, random Schr\"odinger operators, diversity bounds, Bernoulli potentials, centralizer}

\maketitle

\section{Introduction}

Cole et al.~\cite{cole2026diversity} prove that for finite-difference discretization of random Schr\"odinger operators with separable Bernoulli potentials, the augmented sample set has a trivial centralizer with probability at least $1 - e^{-cN}$. However, for $D > 1$, their proof requires $M \geq 9/(2p(1-p))$, which for typical $p = 0.2$ means $M \geq 28$.

This assumption restricts the applicability of the diversity guarantee to relatively fine grids. The authors conjecture it can be removed with a more careful analysis~\cite{cole2026diversity}. We investigate this conjecture computationally by testing diversity for all $M$ values, including those well below the threshold.

\section{Methodology}

We assemble FD discretization matrices for the Schr\"odinger operator $-\Delta + V$ on $[0,1]^D$ with the standard $(2D+1)$-point stencil Laplacian. The potential $V$ uses i.i.d.\ Bernoulli($p$) entries with $p = 0.2$. For each configuration $(D, M, N)$, we run 100 independent trials, compute the joint centralizer dimension, and record the success rate (fraction of trials achieving trivial centralizer).

\section{Results}

\subsection{Diversity in the Small-$M$ Regime}

\begin{table}[t]
\caption{Diversity probability by dimension for $M < M_{\text{thresh}}$.}
\label{tab:smallM}
\centering
\begin{tabular}{cccc}
\toprule
$D$ & Mean Prob. & Min Prob. & Max Prob. \\
\midrule
1 & 0.953 & 0.950 & 1.000 \\
2 & 0.898 & 0.860 & 1.000 \\
3 & 0.753 & 0.700 & 0.790 \\
\bottomrule
\end{tabular}
\end{table}

Table~\ref{tab:smallM} shows that diversity holds with substantial probability even below the theoretical threshold. In $D = 1$, the mean probability exceeds $0.95$. In $D = 2$, it remains above $0.86$. Even in $D = 3$ with the smallest grids, probability stays at $0.70$ or above.

\subsection{Effect of Sample Size $N$}

Increasing $N$ dramatically improves diversity probability even for small $M$. For $M = 2, D = 2, p = 0.2$: with $N = 2$ the probability is $0.18$, but with $N = 8$ it reaches $0.84$. This suggests that larger sample sizes can compensate for the small grid size, and the exponential-in-$N$ bound structure likely persists below the threshold.

\subsection{Refined Bounds}

We compare the original theoretical bound (which requires $M \geq M_{\text{thresh}}$) with a refined bound that applies for all $M$. The refined bound achieves positive coverage for $41.2\%$ of below-threshold configurations, compared to $33.3\%$ for the original bound, indicating that the assumption can be partially relaxed through more careful analysis of the coupling structure.

\section{Conclusion}

Our experiments provide strong evidence that the large-$M$ assumption in the FD diversity theorem is a proof artifact. Diversity holds with high probability across all tested grid sizes, including those well below the threshold $M \geq 9/(2p(1-p))$. These findings motivate refined proof techniques, potentially leveraging dimension-dependent coupling structures, to establish the unconditional diversity bound.

\bibliographystyle{ACM-Reference-Format}
\bibliography{references}

\end{document}
