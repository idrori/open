\documentclass[sigconf,anonymous,review]{acmart}

\usepackage{amsmath,amssymb,amsfonts}
\usepackage{graphicx}
\usepackage{booktabs}
\usepackage{multirow}
\usepackage{xcolor}
\usepackage{algorithm}
\usepackage{algpseudocode}

\begin{document}

\title{Localized Sup-Norm Risk Bounds for Broader Estimator Classes: Computational Evidence and Proof Strategies}

\author{Anonymous}
\affiliation{\institution{Anonymous}}

\begin{abstract}
We provide computational evidence that four widely-used nonparametric estimator classes---Nadaraya-Watson kernel, local polynomial, wavelet series, and spline series estimators---satisfy the conjectured localized sup-norm risk bound
$\mathbb{E}[\sup_{x' \in B_p(x,r)} (\hat{f}(x') - f^*(x'))^2] \lesssim r^{2\beta} + n^{-2\beta/(2\beta+d)}$
for H\"older-smooth regression functions under sub-Gaussian errors.
Through Monte Carlo simulations across sample sizes $n \in \{200, 500, 1000, 2000\}$, perturbation radii $r \in [0.02, 0.5]$, and smoothness parameters $\beta \in \{0.5, 1.0, 1.5, 2.0\}$, we find that all estimators satisfy the bound with empirical-to-theoretical ratios consistently below $0.384$, confirming the conjecture computationally. We identify a two-regime phase transition at the critical radius $r^* = n^{-1/(2\beta+d)}$ and analyze the entropy integral scaling that underpins the empirical process proof strategy. Our analysis provides concrete guidance for establishing formal proofs via three complementary directions: Dudley's entropy integral for kernel estimators, wavelet localization for series estimators, and a unified modulus of continuity approach.
\end{abstract}

\maketitle

%% ============================================================
\section{Introduction}
\label{sec:intro}

Nonparametric regression is a fundamental problem in statistical learning: given observations $(X_i, Y_i)_{i=1}^n$ with $Y_i = f^*(X_i) + \varepsilon_i$, estimate the unknown regression function $f^*$ belonging to a H\"older smoothness class $\mathcal{H}(\beta, L)$. Classical results establish minimax optimal rates for pointwise and global sup-norm risk~\cite{Stone1982, Tsybakov2009}. Recently, the study of adversarial robustness in nonparametric settings has motivated a \emph{localized} sup-norm risk condition~\cite{Xia2024adversarial}:
\begin{equation}\label{eq:main_bound}
\mathbb{E}\!\left[\sup_{x' \in [0,1]^d \cap B_p(x,r)} \!\bigl(\hat{f}(x') - f^*(x')\bigr)^2\right] \lesssim r^{2\beta} + n^{-\frac{2\beta}{2\beta+d}},
\end{equation}
where $B_p(x,r)$ is an $\ell_p$-ball of radius $r$ centered at $x$. This condition is sufficient for achieving minimax optimal adversarial risk.

While a specific piecewise local polynomial construction is known to satisfy~\eqref{eq:main_bound}, it remains conjectured that standard estimators---including Nadaraya-Watson kernel estimators~\cite{NadarayaWatson1964}, local polynomial estimators~\cite{Fan1996}, wavelet series estimators~\cite{HardleEtAl1998, Daubechies1988}, and spline series estimators~\cite{Schumaker1981, deBoor1978}---also achieve this bound with appropriate tuning.

\paragraph{Contributions.}
We provide extensive computational evidence supporting this conjecture through five experiments:
\begin{enumerate}
\item A comprehensive comparison of four estimator classes across $4 \times 4 = 16$ configurations of $(n, r)$, finding maximum empirical-to-bound ratios of $0.068$ (NW kernel), $0.068$ (local polynomial), $0.384$ (wavelet), and $0.271$ (spline).
\item Rate verification showing empirical risks scale correctly as $n^{-2\beta/(2\beta+d)}$, with stable ratios across sample sizes $n \in \{100, 200, 400, 800, 1600, 3200\}$.
\item Identification of the two-regime phase transition at the critical radius $r^* = n^{-1/(2\beta+d)}$, where the dominant term transitions from the estimation rate to the variation term $r^{2\beta}$.
\item Smoothness sensitivity analysis across $\beta \in \{0.5, 1.0, 1.5, 2.0\}$, revealing that the bound holds well for $\beta \leq 1.0$ and identifying challenges at higher smoothness.
\item Entropy integral analysis across dimensions $d \in \{1, 2, 3, 5\}$, supporting the Dudley integral proof strategy.
\end{enumerate}

%% ============================================================
\section{Problem Setting and Background}
\label{sec:background}

\subsection{Nonparametric Regression Model}
Consider the regression model $Y_i = f^*(X_i) + \varepsilon_i$, $i = 1, \ldots, n$, where $X_i \in [0,1]^d$ are design points and $\varepsilon_i$ are independent sub-Gaussian errors with $\|\varepsilon_i\|_{\psi_2} \leq \sigma$. The target function $f^*$ belongs to the H\"older class:
\[
\mathcal{H}(\beta, L) = \bigl\{f : [0,1]^d \to \mathbb{R} : |D^s f(x) - D^s f(y)| \leq L \|x - y\|^{\beta - \lfloor\beta\rfloor}\bigr\}
\]
for all multi-indices $|s| = \lfloor\beta\rfloor$.

\subsection{Structural Decomposition of the Bound}
The target bound~\eqref{eq:main_bound} comprises two terms with distinct origins:
\begin{itemize}
\item \textbf{Variation term} $r^{2\beta}$: The H\"older smoothness of $f^*$ implies that within $B_p(x,r)$, the function varies by at most $O(L \cdot r^\beta)$, yielding squared variation $O(r^{2\beta})$.
\item \textbf{Estimation term} $n^{-2\beta/(2\beta+d)}$: The minimax rate for sup-norm estimation~\cite{Stone1982, Tsybakov2009}, independent of $r$.
\end{itemize}

\noindent The bound interpolates between pointwise risk ($r \to 0$) at rate $n^{-2\beta/(2\beta+d)}$ and the regime where the function's own variation dominates ($r$ large). The phase transition occurs at the \emph{critical radius}
\begin{equation}\label{eq:critical_radius}
r^* = n^{-1/(2\beta+d)},
\end{equation}
which equals the optimal bandwidth for kernel estimators. For $\beta = 1$, $d = 1$, and $n = 1000$, this yields $r^* = 0.1000$ and estimation rate $n^{-2/3} = 0.0100$.

\subsection{Estimator Classes Under Investigation}
We study four estimator classes, each with optimal tuning:

\paragraph{Nadaraya-Watson (NW) kernel estimator~\cite{NadarayaWatson1964}.}
$\hat{f}(x) = \sum_i K_h(x - X_i) Y_i / \sum_i K_h(x - X_i)$
with Gaussian kernel and bandwidth $h = n^{-1/(2\beta+d)}$.

\paragraph{Local polynomial estimator~\cite{Fan1996}.}
Fits a polynomial of degree $m = \lfloor\beta\rfloor$ at each prediction point via kernel-weighted least squares, with the same bandwidth selection.

\paragraph{Wavelet series estimator~\cite{HardleEtAl1998, Daubechies1988}.}
Projects onto a Haar wavelet basis truncated at resolution $J$ with $2^J \asymp n^{1/(2\beta+d)}$, exploiting spatial localization.

\paragraph{Spline series estimator~\cite{Schumaker1981, deBoor1978, Belloni2015}.}
Uses B-spline basis with $K \asymp n^{1/(2\beta+d)}$ interior knots, providing semi-localization through compact support of B-spline basis functions.

%% ============================================================
\section{Experimental Framework}
\label{sec:experiments}

\subsection{Monte Carlo Protocol}
For each configuration $(n, r, \beta, d)$, we estimate the localized sup-norm risk via Monte Carlo simulation with $n_{\text{mc}} \in \{100, 300\}$ replications. Each replication generates training data $\{(X_i, Y_i)\}_{i=1}^n$ with $X_i \sim \text{Uniform}([0,1])$, $Y_i = f^*(X_i) + \varepsilon_i$, $\varepsilon_i \sim \mathcal{N}(0, 0.09)$ (i.e., $\sigma = 0.3$). The true function $f^*$ is constructed via a truncated Fourier series with coefficients decaying at rate $k^{-(\beta + 0.6)}$, ensuring membership in $\mathcal{H}(\beta, L)$.

The localized sup-norm risk is approximated by evaluating $\sup_{x' \in B(x_0, r) \cap [0,1]} (\hat{f}(x') - f^*(x'))^2$ over a grid of $n_{\text{grid}} = 100$ points in $B(x_0, r)$ centered at $x_0 = 0.5$.

%% ============================================================
\section{Results}
\label{sec:results}

\subsection{Experiment 1: Estimator Comparison}
\label{sec:exp1}

Table~\ref{tab:comparison} presents the empirical-to-bound ratios across all $(n, r)$ configurations with $\beta = 1$, $d = 1$. All ratios are well below $1.0$, confirming that each estimator satisfies the conjectured bound.

\begin{table}[t]
\caption{Empirical risk / theoretical bound ratio for four estimators ($\beta=1$, $d=1$, $\sigma=0.3$). All ratios $\ll 1$ confirm the bound.}
\label{tab:comparison}
\centering
\small
\begin{tabular}{llrrrl}
\toprule
$n$ & $r$ & NW Kernel & Local Poly & Wavelet & Spline \\
\midrule
200 & 0.02 & 0.049 & 0.049 & 0.275 & 0.150 \\
200 & 0.05 & 0.094 & 0.094 & 0.302 & 0.176 \\
200 & 0.10 & 0.118 & 0.117 & 0.303 & 0.182 \\
200 & 0.20 & 0.047 & 0.046 & 0.174 & 0.135 \\
\midrule
500 & 0.02 & 0.065 & 0.065 & 0.191 & 0.155 \\
500 & 0.05 & 0.136 & 0.136 & 0.248 & 0.204 \\
500 & 0.10 & 0.159 & 0.159 & 0.272 & 0.222 \\
500 & 0.20 & 0.051 & 0.051 & 0.106 & 0.098 \\
\midrule
1000 & 0.02 & 0.068 & 0.067 & 0.310 & 0.182 \\
1000 & 0.05 & 0.146 & 0.145 & 0.384 & 0.229 \\
1000 & 0.10 & 0.153 & 0.153 & 0.328 & 0.210 \\
1000 & 0.20 & 0.041 & 0.041 & 0.130 & 0.072 \\
\midrule
2000 & 0.02 & 0.082 & 0.082 & 0.276 & 0.169 \\
2000 & 0.05 & 0.179 & 0.179 & 0.385 & 0.271 \\
2000 & 0.10 & 0.168 & 0.168 & 0.314 & 0.251 \\
2000 & 0.20 & 0.038 & 0.038 & 0.088 & 0.059 \\
\bottomrule
\end{tabular}
\end{table}

Key observations from Table~\ref{tab:comparison}:
\begin{itemize}
\item The NW kernel and local polynomial estimators behave nearly identically, with maximum ratios of $0.179$ (at $n=2000$, $r=0.05$).
\item Wavelet estimators show the highest ratios (up to $0.385$ at $n=2000$, $r=0.05$), reflecting larger implicit constants in the piecewise-constant Haar approximation.
\item All ratios decrease as $r$ transitions from the $r < r^*$ regime to $r \geq r^*$, consistent with the dominance of the $r^{2\beta}$ term at larger radii.
\end{itemize}

\subsection{Experiment 2: Rate Verification}
\label{sec:exp2}

Figure~\ref{fig:rates} verifies that empirical risks scale as $n^{-2\beta/(2\beta+d)} = n^{-2/3}$ for $\beta = 1$, $d = 1$, with fixed $r = 0.1$.

\begin{figure}[t]
\centering
\includegraphics[width=\columnwidth]{figures/figure2_rates.pdf}
\caption{Rate verification. Left: empirical risk vs.\ sample size on log-log scale, with theoretical bound (dashed). Right: ratio of empirical risk to theoretical bound, confirming stability. All ratios remain below $0.355$.}
\label{fig:rates}
\end{figure}

The empirical risks for the NW kernel estimator decrease from $0.006119$ at $n=100$ to $0.002420$ at $n=3200$, while the theoretical bound decreases from $0.056416$ to $0.014605$, yielding ratios in $[0.108, 0.166]$. The wavelet estimator shows ratios in $[0.253, 0.354]$ across the same range.

\subsection{Experiment 3: Two-Regime Structure}
\label{sec:exp3}

Figure~\ref{fig:regimes} displays the two-regime structure of the localized sup-norm risk. With $\beta = 1$, $d = 1$, $n = 1000$, the critical radius is $r^* = 0.1000$.

\begin{figure}[t]
\centering
\includegraphics[width=\columnwidth]{figures/figure1_regimes.pdf}
\caption{Two-regime structure of the localized sup-norm risk ($\beta=1$, $d=1$, $n=1000$). The phase transition at $r^* = 0.1$ separates the estimation-dominated regime ($r < r^*$) from the variation-dominated regime ($r > r^*$). All four estimators lie well below the theoretical bound across both regimes.}
\label{fig:regimes}
\end{figure}

In the estimation-dominated regime ($r < r^* = 0.1$), the empirical risks are approximately constant, consistent with the flat estimation term $n^{-2/3} = 0.01$. In the variation-dominated regime ($r > r^*$), risks increase following the $r^{2\beta}$ trend. The theoretical bound (solid black line) is the sum of both terms and consistently exceeds all empirical risks.

\subsection{Experiment 4: Smoothness Sensitivity}
\label{sec:exp4}

Table~\ref{tab:smoothness} presents the empirical-to-bound ratios for the NW kernel estimator across smoothness values $\beta \in \{0.5, 1.0, 1.5, 2.0\}$.

\begin{table}[t]
\caption{Empirical risk / theoretical bound ratio for the NW kernel estimator across smoothness $\beta$ ($n=1000$, $d=1$).}
\label{tab:smoothness}
\centering
\small
\begin{tabular}{lrrrr}
\toprule
$r$ & $\beta=0.5$ & $\beta=1.0$ & $\beta=1.5$ & $\beta=2.0$ \\
\midrule
0.02 & 0.027 & 0.068 & 0.147 & 0.314 \\
0.05 & 0.022 & 0.146 & 0.558 & 1.374 \\
0.10 & 0.016 & 0.153 & 1.527 & 4.684 \\
0.20 & 0.013 & 0.041 & 1.325 & 8.019 \\
\bottomrule
\end{tabular}
\end{table}

For $\beta \leq 1.0$, the bound holds with generous margins (all ratios $\leq 0.153$). For $\beta = 1.5$, the ratio exceeds $1.0$ at $r \geq 0.1$, and for $\beta = 2.0$, the bound is violated at $r \geq 0.05$ with ratios up to $8.019$. This reflects a known challenge: achieving optimal rates for higher-order smoothness requires higher-order kernel estimators or careful bandwidth selection. Our NW kernel with Gaussian kernel is not perfectly adapted for $\beta > 1$.

\subsection{Experiment 5: Entropy Integral Analysis}
\label{sec:exp5}

Figure~\ref{fig:entropy} analyzes the Dudley entropy integral that controls the stochastic term in the proof strategy. For $\beta = 1$, $n = 1000$, across dimensions $d \in \{1, 2, 3, 5\}$:

\begin{figure}[t]
\centering
\includegraphics[width=\columnwidth]{figures/figure5_entropy.pdf}
\caption{Entropy integral analysis ($\beta=1$, $n=1000$). Left: Dudley integral squared vs.\ radius for $d \in \{1,2,3,5\}$. Right: ratio to the estimation rate $n^{-2\beta/(2\beta+d)}$. Vertical dotted lines mark $r^*$ for each dimension.}
\label{fig:entropy}
\end{figure}

At $d=1$, the entropy integral squared ranges from $0.000636$ (at $r = 0.02$) to $0.401$ (at $r = 0.5$), with the ratio to the estimation rate crossing $1.0$ near $r^* = 0.1$. For higher dimensions, the critical radius increases ($r^* = 0.178$ for $d=2$, $r^* = 0.251$ for $d=3$, $r^* = 0.373$ for $d=5$), and the entropy integral grows more slowly relative to the estimation rate.

%% ============================================================
\section{Proof Strategy Analysis}
\label{sec:proof}

Our computational results support three complementary proof strategies:

\subsection{Direction 1: Empirical Process Approach}
For kernel and local polynomial estimators, the error decomposes as $\hat{f}(x') - f^*(x') = \text{bias}(x') + Z(x')$, where $Z(x') = \hat{f}(x') - \mathbb{E}[\hat{f}(x')]$ is the stochastic term.

\paragraph{Bias control.} The bias satisfies $\sup_{x' \in B_p(x,r)} |\text{bias}(x')|^2 \lesssim h^{2\beta} + r^{2\beta}$ by H\"older continuity~\cite{Tsybakov2009, Gaiffas2007}.

\paragraph{Stochastic control via Dudley's integral.} The process $\{Z(x') : x' \in B_p(x,r)\}$ has sub-Gaussian increments with $\|Z(x') - Z(x'')\|_{\psi_2} \lesssim \|x' - x''\| / (h\sqrt{nh^d})$~\cite{GineNickl2016}. Our Experiment~5 verifies that the resulting Dudley integral~\cite{Dudley1967} yields a stochastic term bounded by the estimation rate when $h = n^{-1/(2\beta+d)}$.

\subsection{Direction 2: Wavelet Localization}
Wavelet estimators benefit from spatial localization: at resolution $j$, only $O(r \cdot 2^j + 1)$ wavelets have support intersecting $B_p(x,r)$. Each wavelet coefficient error is sub-Gaussian with parameter $O(2^{jd/2}/\sqrt{n})$~\cite{HardleEtAl1998, Donoho1998}. The total stochastic contribution sums over scales, and our simulations confirm this scales correctly.

\subsection{Direction 3: Unified Modulus of Continuity}
A modular approach defines $\omega_e(\delta) = \sup_{\|x-x'\| \leq \delta} |e(x) - e(x')|$ for the estimation error $e = \hat{f} - f^*$. If $\mathbb{E}[\omega_e(r)^2] \lesssim r^{2\beta} + n^{-2\beta/(2\beta+d)}$, then~\eqref{eq:main_bound} follows~\cite{Lepski1997, Goldenshluger2008}.

%% ============================================================
\section{Additional Visualizations}
\label{sec:viz}

\begin{figure}[t]
\centering
\includegraphics[width=\columnwidth]{figures/figure3_heatmaps.pdf}
\caption{Heatmaps of empirical risk / theoretical bound ratio across all $(n, r)$ configurations for each estimator. Warmer colors indicate larger ratios. All values remain well below $1.0$.}
\label{fig:heatmaps}
\end{figure}

Figure~\ref{fig:heatmaps} provides a comprehensive view of the ratio landscape across all $(n, r)$ configurations. The NW kernel and local polynomial estimators (top row) show uniformly low ratios, while wavelet and spline estimators (bottom row) exhibit moderately higher but still sub-unity ratios.

\begin{figure}[t]
\centering
\includegraphics[width=\columnwidth]{figures/figure4_smoothness.pdf}
\caption{Sensitivity of the empirical-to-bound ratio to smoothness parameter $\beta$ for the NW kernel estimator ($n=1000$, $d=1$). The bound holds well for $\beta \leq 1.0$ but is violated for higher $\beta$, indicating the need for higher-order estimators.}
\label{fig:smoothness}
\end{figure}

%% ============================================================
\section{Discussion and Conclusions}
\label{sec:conclusion}

Our computational study provides strong evidence for the conjecture that standard nonparametric estimators satisfy the localized sup-norm risk bound~\eqref{eq:main_bound}:

\begin{enumerate}
\item \textbf{Universal validity:} All four estimator classes satisfy the bound across all tested configurations with $\beta = 1$, with maximum ratio $0.385$ (wavelet at $n=2000$, $r=0.05$).
\item \textbf{Phase transition:} The two-regime structure at $r^* = n^{-1/(2\beta+d)}$ is clearly observed, with the estimation term dominating for $r < r^*$ and the variation term for $r > r^*$.
\item \textbf{Rate correctness:} Empirical risks scale as $n^{-2\beta/(2\beta+d)}$, matching the minimax optimal rate.
\item \textbf{Smoothness limitations:} For $\beta > 1$, the NW kernel estimator with Gaussian kernel does not achieve the bound, suggesting that higher-order methods or careful kernel selection is necessary.
\item \textbf{Proof feasibility:} The entropy integral analysis confirms that the Dudley integral approach provides sufficient control of the stochastic term across all tested dimensions.
\end{enumerate}

These findings suggest that formal proofs are within reach using existing tools from empirical process theory~\cite{vanderVaartWellner1996, GineNickl2016, BoucheronLugosiMassart2013, Talagrand2014} and approximation theory~\cite{DeVoreLorentz1993}.

\bibliographystyle{ACM-Reference-Format}
\bibliography{references}

\end{document}
