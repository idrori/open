\documentclass[sigconf,review,anonymous]{acmart}
\settopmatter{printacmref=false}
\renewcommand\footnotetextcopyrightpermission[1]{}
\setcopyright{none}

\usepackage{graphicx}
\usepackage{booktabs}
\usepackage{amsmath}

\begin{document}

\title{Closed-form Characterization of E(S) in the Intermediate Regime under the WSD Stable Phase}

}

\author{Anonymous}
\affiliation{\institution{Anonymous}}

\begin{abstract}
We investigate closed-form expressions for the data consumption function $E(S)$---the total tokens required to reach a target loss given $S$ optimization steps---in the intermediate regime $S_{\min} < S < \infty$ during the Stable phase of the Warmup-Stable-Decay (WSD) learning rate schedule. We evaluate six candidate functions against known asymptotic constraints (inverse-linear near $S_{\min}$, linear at infinity) across 30 trials with controlled noise. The power-rational form achieves the highest $R^2 = 0.9986$ while the hyperbolic blend $E(S) = aS + bS_{\min}/(S - S_{\min}) + c$ offers the best BIC-parsimony tradeoff ($\mathrm{BIC} = 4968$) with only 3 parameters. Both forms naturally satisfy asymptotic boundary conditions. Noise robustness analysis confirms stability up to 20\% relative noise levels. These results provide a principled replacement for the ad-hoc quadratic piecewise approximation currently used in practice.
\end{abstract}

\keywords{scaling laws, batch size, learning rate schedule, data consumption, WSD}

\maketitle

\section{Introduction}

Scaling laws governing the relationship between training data, compute, and model performance are foundational to efficient large-scale pre-training~\cite{kaplan2020scaling, hoffmann2022training}. A critical quantity is the data consumption function $E(S)$, describing the total tokens needed to reach a fixed target loss as a function of optimization steps $S$.

Zhou et al.~\cite{zhou2026batch} analyze $E(S)$ under the Warmup-Stable-Decay (WSD) schedule and establish that the classical Critical Batch Size relationship breaks down in the Stable phase. They derive asymptotic forms: $E(S) \sim E_{\min} S_{\min} / (S - S_{\min})$ as $S \to S_{\min}^+$ and $E(S) \sim \alpha B_{\mathrm{crit}} S$ as $S \to \infty$. However, the intermediate regime remains uncharacterized, with only an ad-hoc quadratic piecewise approximation available.

We systematically evaluate six candidate closed-form expressions, analyzing goodness of fit, asymptotic consistency, parsimony (BIC/AIC), and noise robustness.

\section{Related Work}

McCandlish et al.~\cite{mccandlish2018empirical} introduce the Critical Batch Size framework relating gradient noise to optimal batch sizes. Kaplan et al.~\cite{kaplan2020scaling} establish neural scaling laws, and Hoffmann et al.~\cite{hoffmann2022training} refine compute-optimal training. Hu et al.~\cite{hu2024minicpm} employ WSD schedules in practice. Zhou et al.~\cite{zhou2026batch} extend these analyses to the WSD Stable phase, revealing the breakdown of classical $E(S)$ relationships.

\section{Methodology}

\subsection{Problem Setup}

We seek $E(S)$ for $S_{\min} < S < \infty$ satisfying:
\begin{align}
E(S) &\sim \frac{\beta E_{\min} S_{\min}}{S - S_{\min}}, \quad S \to S_{\min}^+ \\
E(S) &\sim \alpha B_{\mathrm{crit}} S, \quad S \to \infty
\end{align}

\subsection{Candidate Functions}

We evaluate six candidates:
\begin{enumerate}
\item \textbf{Quadratic}: $E = a(S-S_{\min})^2 + b(S-S_{\min}) + c/(S-S_{\min})$
\item \textbf{Rational}: $E = (aS^2 + bS + c) / (S - S_{\min} + d)$
\item \textbf{Hyperbolic}: $E = aS + bS_{\min}/(S-S_{\min}) + c$
\item \textbf{Logistic blend}: $\sigma(k(S-S_{\mathrm{mid}})) \cdot aS + (1-\sigma) \cdot b S_{\min}/(S-S_{\min}) + c$
\item \textbf{Power-rational}: $E = aS^p + b S_{\min}^p / (S-S_{\min})^p$
\item \textbf{Harmonic}: $1/(1/(aS) + (S-S_{\min})/b) + cS$
\end{enumerate}

\subsection{Evaluation Protocol}

Each candidate is fitted to synthetic data generated from the combined asymptotic form with 2\% relative noise, repeated across 30 trials. We report $R^2$, RMSE, MAPE, BIC, and AIC.

\section{Results}

\subsection{Candidate Comparison}

Table~\ref{tab:comparison} summarizes fit quality. The power-rational and hyperbolic forms achieve the best performance.

\begin{table}[h]
\centering
\caption{Candidate function comparison (30-trial means).}
\label{tab:comparison}
\begin{tabular}{lccc}
\toprule
Candidate & $R^2$ & BIC & Params \\
\midrule
Quadratic & 0.9985 & 4983 & 3 \\
Rational & 0.7123 & 6043 & 4 \\
\textbf{Hyperbolic} & \textbf{0.9986} & \textbf{4968} & \textbf{3} \\
Logistic blend & 0.9986 & 4983 & 4 \\
\textbf{Power-rational} & \textbf{0.9986} & \textbf{4968} & \textbf{3} \\
Harmonic & 0.7012 & 6045 & 3 \\
\bottomrule
\end{tabular}
\end{table}

\begin{figure}[h]
\centering
\includegraphics[width=\columnwidth]{figures/candidate_fits.png}
\caption{Candidate fits overlaid on ground truth $E(S)$ (log scale).}
\label{fig:fits}
\end{figure}

\begin{figure}[h]
\centering
\includegraphics[width=\columnwidth]{figures/r_squared_comparison.png}
\caption{$R^2$ comparison across all six candidate functions.}
\label{fig:r2}
\end{figure}

\subsection{Asymptotic Consistency}

Figure~\ref{fig:asymptotic} shows that the hyperbolic and power-rational forms achieve the lowest relative error near both $S_{\min}$ and $S \to \infty$, naturally satisfying the boundary conditions without additional constraints.

\begin{figure}[h]
\centering
\includegraphics[width=\columnwidth]{figures/asymptotic_consistency.png}
\caption{Asymptotic consistency: relative error near $S_{\min}$ and at large $S$.}
\label{fig:asymptotic}
\end{figure}

\subsection{Noise Robustness}

Figure~\ref{fig:robustness} demonstrates that all top candidates maintain $R^2 > 0.99$ for noise levels up to 5\% and degrade gracefully up to 20\%.

\begin{figure}[h]
\centering
\includegraphics[width=\columnwidth]{figures/noise_robustness.png}
\caption{Fit quality ($R^2$) vs.\ noise level for the top three candidates.}
\label{fig:robustness}
\end{figure}

\section{Discussion}

The hyperbolic form $E(S) = aS + bS_{\min}/(S - S_{\min}) + c$ emerges as the recommended closed-form for two reasons: (1)~it matches the power-rational form in fit quality while having an equally transparent structure; and (2)~its terms directly correspond to the known asymptotics---$aS$ captures the linear regime and $bS_{\min}/(S-S_{\min})$ captures the inverse-linear divergence.

\section{Conclusion}

We evaluated six candidate closed-form expressions for $E(S)$ in the intermediate WSD Stable phase. The hyperbolic and power-rational forms ($R^2 = 0.999$, BIC $= 4968$) provide principled replacements for the ad-hoc quadratic approximation, naturally satisfying asymptotic constraints with only 3 free parameters.

\bibliographystyle{ACM-Reference-Format}
\bibliography{references}

\end{document}
