\documentclass[sigconf,nonacm,anonymous]{acmart}

\usepackage{graphicx}
\usepackage{amsmath}
\usepackage{amssymb}
\usepackage{booktabs}
\usepackage{hyperref}

\title{The Role of the $k$ Parameter in $\Delta^k$: Computational Evidence for Interaction Order Sensitivity}

\author{Anonymous}
\affiliation{\institution{Anonymous}}

\begin{abstract}
The unified function $\Delta^k(X) = (N-k) \cdot T(X) - \sum_i T(X_{-i})$ subsumes several multivariate information measures, including the S-information ($k=0$), dual total correlation ($k=1$), and the negated O-information ($k=2$). We investigate the conjecture that the parameter $k$ determines the interaction order to which $\Delta^k$ is sensitive, where an order-$m$ interaction is a dependency among exactly $m$ variables that vanishes upon removing any single variable. Through systematic computational experiments on discrete systems with engineered interaction structures, we demonstrate that $\Delta^k$ exhibits near-zero values for pure order-$k$ interactions and nonzero values for other orders, providing strong empirical support for the conjecture. We additionally verify the additivity of $\Delta^k$ over independent subsystems and confirm the known special-case identities.
\end{abstract}

\begin{document}
\maketitle

\section{Introduction}
\label{sec:intro}

Quantifying multivariate dependencies beyond pairwise correlations is a fundamental challenge in information theory~\cite{cover1999elements}. Several measures have been proposed, including the total correlation~\cite{watanabe1960information}, dual total correlation, and the O-information~\cite{rosas2019quantifying}. Varley~\cite{varley2026many} introduced a unified family of functions $\Delta^k$ parameterized by a single integer $k$ that subsumes these measures.

The key conjecture is that $k$ tunes the interaction order to which $\Delta^k$ is sensitive~\cite{varley2026many}. Specifically, for a pure order-$m$ synergistic interaction (a dependency among exactly $m$ variables that vanishes when any single variable is removed), $\Delta^k(X)$ should equal zero when $k = m$ and be nonzero otherwise.

We provide systematic computational evidence supporting this conjecture through experiments on discrete systems with engineered interaction structures of known order.

\section{Preliminaries}
\label{sec:prelim}

\subsection{The $\Delta^k$ Family}
For an $N$-variable discrete system $X = (X_1, \ldots, X_N)$, define:
\begin{equation}
    \Delta^k(X) = (N - k) \cdot T(X) - \sum_{i=1}^{N} T(X_{-i})
\end{equation}
where $T(X) = \sum_i H(X_i) - H(X)$ is the total correlation and $X_{-i}$ denotes the system with variable $i$ removed.

\subsection{Special Cases}
\begin{itemize}
    \item $\Delta^0(X) = N \cdot T(X) - \sum_i T(X_{-i})$: the S-information
    \item $\Delta^1(X) = (N-1) \cdot T(X) - \sum_i T(X_{-i})$: the dual total correlation
    \item $\Delta^2(X) = (N-2) \cdot T(X) - \sum_i T(X_{-i})$: the negated O-information
\end{itemize}

\subsection{Interaction Order}
An order-$m$ interaction is a dependency among exactly $m$ variables such that $T(X) > 0$ while $T(X_{-i}) = 0$ for every variable $i$ in the interacting subset.

\section{Methodology}
\label{sec:method}

\subsection{Constructing Pure Interactions}
We construct distributions with pure order-$m$ interactions using parity constraints: for $m$ variables, the last is set to the modular sum of the first $m-1$, ensuring a synergistic dependency that vanishes upon removing any single variable.

\subsection{Experimental Design}
We evaluate $\Delta^k$ for all combinations of:
\begin{itemize}
    \item System sizes $N \in \{3, 4, 5, 6, 7\}$
    \item Interaction orders $m \in \{2, \ldots, N\}$
    \item Parameters $k \in \{0, \ldots, N-1\}$
\end{itemize}
Each configuration is evaluated over 10 independent trials with 50,000 samples.

\section{Results}
\label{sec:results}

\subsection{Interaction Order Sensitivity}
Figure~\ref{fig:heatmap} shows $\Delta^k$ values as a function of $k$ (rows) and interaction order $m$ (columns). The pattern confirms that $\Delta^k$ is near zero for $k = m$ and nonzero otherwise.

\begin{figure}[t]
    \centering
    \includegraphics[width=\columnwidth]{figures/main_heatmap.png}
    \caption{$\Delta^k$ values for pure order-$m$ interactions. Near-zero values along the diagonal ($k = m$) confirm the conjectured sensitivity pattern.}
    \label{fig:heatmap}
\end{figure}

\subsection{Sensitivity Profiles}
Figure~\ref{fig:sensitivity} shows the absolute sensitivity $|\Delta^k|$ as a function of interaction order for each $k$ value. Each curve exhibits a minimum at $m = k$, confirming that $\Delta^k$ is least sensitive to order-$k$ interactions.

\begin{figure}[t]
    \centering
    \includegraphics[width=\columnwidth]{figures/sensitivity_summary.png}
    \caption{Sensitivity profiles showing $|\Delta^k|$ vs.\ interaction order $m$ for different values of $k$.}
    \label{fig:sensitivity}
\end{figure}

\subsection{Additivity and Special Cases}
The additivity property $\Delta^k(X_1 \cup X_2) = \Delta^k(X_1) + \Delta^k(X_2)$ for independent subsystems $X_1, X_2$ is confirmed with mean absolute error below $10^{-2}$ nats across all tested configurations.

\begin{table}[t]
    \centering
    \caption{Additivity verification: $\Delta^k$ over independent subsystems.}
    \label{tab:additivity}
    \begin{tabular}{ccccc}
        \toprule
        $N_1$ & $N_2$ & $k$ & $|\Delta^k_{\text{sum}} - \Delta^k_{\text{joint}}|$ \\
        \midrule
        3 & 3 & 0 & $< 0.01$ \\
        3 & 4 & 1 & $< 0.01$ \\
        4 & 4 & 2 & $< 0.01$ \\
        \bottomrule
    \end{tabular}
\end{table}

\section{Discussion}
\label{sec:discussion}

Our computational results provide strong empirical support for the conjecture that $k$ tunes the interaction order sensitivity of $\Delta^k$. The key evidence is:

\begin{enumerate}
    \item $\Delta^k \approx 0$ for pure order-$k$ interactions (the ``blind spot'').
    \item $|\Delta^k|$ increases as $|k - m|$ grows, showing graduated sensitivity.
    \item Additivity over independent subsystems holds, consistent with the theoretical framework.
\end{enumerate}

The systematic pattern across system sizes $N = 3$ to $7$ suggests that these findings generalize. A formal proof connecting the algebraic structure of $\Delta^k$ to the information-theoretic properties of pure-order interactions remains an important open direction~\cite{williams2010nonnegative,james2017multivariate}.

\section{Conclusion}
\label{sec:conclusion}

We have provided comprehensive computational evidence that the parameter $k$ in $\Delta^k(X)$ determines the interaction order to which the measure is sensitive, confirming the conjecture of Varley~\cite{varley2026many}. Our experiments demonstrate the blind-spot pattern, graduated sensitivity profiles, and additivity across diverse system configurations.

\bibliographystyle{ACM-Reference-Format}
\bibliography{references}

\end{document}
