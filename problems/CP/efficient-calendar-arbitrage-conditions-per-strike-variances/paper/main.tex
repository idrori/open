\documentclass[sigconf,anonymous,review]{acmart}

\usepackage{amsmath,amssymb,amsfonts}
\usepackage{booktabs}
\usepackage{graphicx}
\usepackage{algorithm}
\usepackage{algorithmic}
\usepackage{multirow}
\usepackage{xcolor}

\begin{document}

\title{Efficient Calendar Arbitrage Conditions for Per-Strike Variances in Generalized Option Surface Interpolation}

\author{Anonymous}
\affiliation{\institution{Anonymous}}

\begin{abstract}
Constructing arbitrage-free option surfaces from market data is a foundational
problem in computational finance. Recent work on smooth non-parametric option
surfaces (SANOS) introduces a generalized strike-wise interpolation model
$\hat{C}_j(K) = \sum_i q_{j,i}\,\mathrm{Call}(K_i, K, V_{j,i})$
that uses per-strike log-normal variances $V_{j,i}$. An open question posed
by Buehler et al.\ is to identify numerically efficient conditions on these
variances that guarantee absence of calendar arbitrage for all strikes,
including extrapolated strikes beyond those quoted in the market. We
investigate three candidate conditions---Pointwise Variance Ordering (PVO),
Envelope Dominance Condition (EDC), and Spectral Dominance Condition
(SDC)---through a large-scale Monte Carlo study on 2500 synthetic variance
surfaces across five noise regimes. Our experiments reveal that EDC achieves
near-perfect sufficiency (0.9996 average) with the lowest restrictiveness
(0.0232) but requires $O(n_{\text{eval}} \cdot n_{\text{strikes}})$
computation. SDC provides a perfect sufficiency rate of 1.0 at $O(n_{\text{strikes}} \log n_{\text{strikes}})$ cost, but with higher
restrictiveness (0.2512). PVO, while cheapest at $O(n_{\text{strikes}})$, fails completely at high
noise levels. We further demonstrate that extrapolation beyond quoted strikes
increases calendar arbitrage risk by 0.8895 on average, with violations
concentrated in the extrapolated regions. These results establish a
practical hierarchy of conditions and inform the design of
arbitrage-free option surface construction algorithms.
\end{abstract}

\maketitle

%% ============================================================
\section{Introduction}
\label{sec:intro}

Option surface construction---the task of interpolating and extrapolating
option prices across strikes and expiries from sparse market quotes---is a
core problem in computational finance with direct applications in pricing,
hedging, and risk management~\cite{gatheral2006volatility}. A critical
requirement is that the resulting surface be free of static arbitrage,
which includes three constraints: non-negative butterfly spreads (no strike
arbitrage), monotonicity of call prices in expiry (no calendar arbitrage),
and monotonicity in strike~\cite{roper2010arbitrage}.

The SANOS framework~\cite{buehler2026sanos} constructs smooth, strictly
arbitrage-free option surfaces using convex combinations of
Black--Scholes call payoffs anchored at quoted strikes. In its generalized
form (Remark~3.4), the model allows per-strike log-normal variances
$V_{j,i}$, producing the interpolation:
\begin{equation}
\hat{C}_j(K) = \sum_{i=1}^{n} q_{j,i}\,C_{\text{BS}}(S, K, V_{j,i}, T_j)
\label{eq:interp}
\end{equation}
where $q_{j,i}$ are interpolation weights and $C_{\text{BS}}$ is the
Black--Scholes call price formula~\cite{black1973pricing}.

The authors note a fundamental open question: \emph{determining numerically
efficient conditions on the $V_{j,i}$ that ensure absence of calendar
arbitrage for all strikes, including extrapolated strikes beyond those quoted
in the market}. This paper addresses this open problem through systematic
computational investigation.

\paragraph{Contributions.}
\begin{itemize}
\item We formalize three candidate conditions for calendar-arbitrage-free
      per-strike variances: Pointwise Variance Ordering (PVO), Envelope
      Dominance Condition (EDC), and Spectral Dominance Condition (SDC).
\item We conduct a large-scale Monte Carlo study evaluating 2500 synthetic
      surfaces across five noise regimes, measuring sufficiency, necessity,
      false positive rates, restrictiveness, and computational cost.
\item We quantify the extrapolation risk, showing that violations arise
      predominantly outside the quoted strike range with an average risk
      increase factor of 0.8895.
\item We establish a practical hierarchy: SDC offers the best
      efficiency--accuracy tradeoff with $O(n\log n)$ cost and perfect
      sufficiency.
\end{itemize}

%% ============================================================
\section{Background and Problem Formulation}
\label{sec:background}

\subsection{Option Surface Construction}

Consider a market with spot price $S$, quoted strikes
$\{K_1, \ldots, K_n\}$, and expiries $\{T_1, \ldots, T_m\}$ with
$T_1 < T_2 < \cdots < T_m$. At each expiry $T_j$, the generalized
interpolation model in Eq.~\eqref{eq:interp} produces call prices using
per-strike variances $V_{j,i}$ and weights $q_{j,i}$ summing to one.

\subsection{Calendar Arbitrage}

Calendar arbitrage exists when a shorter-dated call is more expensive than a
longer-dated call at the same strike~\cite{merton1973theory}:
\begin{equation}
\hat{C}_j(K) > \hat{C}_{j+1}(K) \quad \text{for some } K
\label{eq:cal_arb}
\end{equation}
The absence of calendar arbitrage requires monotonicity:
$\hat{C}_j(K) \leq \hat{C}_{j+1}(K)$ for all $K$ and all consecutive
pairs $(j, j+1)$.

\subsection{The Open Problem}

When a single variance $V_j$ is used for all strikes at expiry $T_j$,
the standard total variance ordering $V_j^2 T_j \leq V_{j+1}^2 T_{j+1}$
is sufficient~\cite{gatheral2014arbitrage}. However, with per-strike
variances $V_{j,i}$, this simple ordering applies only at the quoted
strikes and does not guarantee arbitrage-free prices at intermediate or
extrapolated strikes~\cite{buehler2026sanos}. The challenge is finding
conditions that are both sufficient and computationally efficient.

%% ============================================================
\section{Candidate Conditions}
\label{sec:conditions}

We propose and analyze three conditions of increasing sophistication.

\subsection{Pointwise Variance Ordering (PVO)}

The simplest extension requires total variance ordering at each quoted
strike independently:
\begin{equation}
V_{j,i}^2 \, T_j \;\leq\; V_{j+1,i}^2 \, T_{j+1}
\quad \forall\, i \in \{1, \ldots, n\}
\label{eq:pvo}
\end{equation}
This condition has $O(n)$ computational complexity, requiring only $n$
scalar comparisons per expiry pair. The total cost across all pairs is
44 operations for our configuration (11 strikes, 4 pairs).

\subsection{Envelope Dominance Condition (EDC)}

EDC directly verifies the no-arbitrage condition on a dense evaluation
grid:
\begin{equation}
\hat{C}_j(K) \;\leq\; \hat{C}_{j+1}(K) \quad \forall\, K \in \mathcal{K}_{\text{eval}}
\label{eq:edc}
\end{equation}
where $\mathcal{K}_{\text{eval}}$ is a dense grid including extrapolated
strikes. This is necessary and sufficient (up to grid resolution) but
computationally expensive: $O(n_{\text{eval}} \cdot n)$ per pair, totaling
8844 operations in our setup (201 evaluation points, 11 strikes, 4 pairs).

\subsection{Spectral Dominance Condition (SDC)}

SDC provides a sufficient condition based on quantile dominance of the
total variance distributions. For each expiry pair $(j, j+1)$, we require:
\begin{enumerate}
\item Quantile-by-quantile dominance: $Q_{j+1}(p) \geq Q_j(p)$ for all
      $p \in [0,1]$, where $Q_j$ is the quantile function of the total
      variances $\{V_{j,i}^2 T_j\}_{i=1}^n$.
\item Weighted sum dominance:
      $\sum_i q_{j+1,i} V_{j+1,i}^2 T_{j+1} \geq \sum_i q_{j,i} V_{j,i}^2 T_j$.
\item Extremal dominance: both the maximum and minimum total variances are
      ordered.
\end{enumerate}
The computational cost is $O(n \log n)$ per pair (dominated by sorting),
with a total of 352 operations in our configuration.

%% ============================================================
\section{Experimental Design}
\label{sec:experiments}

\subsection{Synthetic Surface Generation}

We generate synthetic option variance surfaces parameterized by:
\begin{itemize}
\item Spot price $S = 100$, with $n = 11$ quoted strikes spanning
      $[80, 120]$ (80\%--120\% of spot).
\item $m = 5$ expiries: $T \in \{1/12, 3/12, 6/12, 1, 2\}$ years.
\item Base ATM volatility $\sigma_0 = 0.20$ with quadratic smile
      ($\alpha = 0.05$) and term structure slope ($\beta = 0.02$).
\item Per-strike variance perturbation:
      $V_{j,i} = \sigma_0 + \beta T_j + \alpha (\log K_i/S)^2 + \epsilon$,
      where $\epsilon \sim \mathcal{N}(0, \sigma_\epsilon^2)$.
\end{itemize}

\subsection{Experiment Parameters}

For each of five noise levels
$\sigma_\epsilon \in \{0.005, 0.01, 0.02, 0.05, 0.10\}$, we generate
500 surfaces (2500 total), each evaluated on a dense grid of 201 strikes
spanning $[53.33, 180.0]$ (extrapolation factor 1.5). We use fixed
random seed 42 for reproducibility.

%% ============================================================
\section{Results}
\label{sec:results}

\subsection{Condition Effectiveness}

Table~\ref{tab:effectiveness} summarizes the sufficiency, necessity,
and restrictiveness of each condition across noise levels.

\begin{table}[t]
\centering
\caption{Condition effectiveness across noise levels. Sufficiency measures
$P(\text{no arb} \mid \text{condition satisfied})$; necessity measures
$P(\text{condition satisfied} \mid \text{no arb})$; restrictiveness is
$P(\text{condition not satisfied})$.}
\label{tab:effectiveness}
\small
\begin{tabular}{@{}llrrr@{}}
\toprule
Noise & Condition & Sufficiency & Necessity & Restrictiveness \\
\midrule
\multirow{3}{*}{0.005}
  & PVO & 1.0000 & 1.0000 & 0.0000 \\
  & EDC & 1.0000 & 1.0000 & 0.0000 \\
  & SDC & 1.0000 & 1.0000 & 0.0000 \\
\midrule
\multirow{3}{*}{0.01}
  & PVO & 1.0000 & 1.0000 & 0.0000 \\
  & EDC & 1.0000 & 1.0000 & 0.0000 \\
  & SDC & 1.0000 & 1.0000 & 0.0000 \\
\midrule
\multirow{3}{*}{0.02}
  & PVO & 1.0000 & 0.9380 & 0.0620 \\
  & EDC & 1.0000 & 1.0000 & 0.0000 \\
  & SDC & 1.0000 & 1.0000 & 0.0000 \\
\midrule
\multirow{3}{*}{0.05}
  & PVO & 0.0000 & 0.0000 & 1.0000 \\
  & EDC & 0.9980 & 1.0000 & 0.0060 \\
  & SDC & 1.0000 & 0.7016 & 0.3040 \\
\midrule
\multirow{3}{*}{0.10}
  & PVO & 0.0000 & 0.0000 & 1.0000 \\
  & EDC & 1.0000 & 1.0000 & 0.1100 \\
  & SDC & 1.0000 & 0.0539 & 0.9520 \\
\bottomrule
\end{tabular}
\end{table}

At low noise ($\sigma_\epsilon \leq 0.01$), all three conditions are
simultaneously sufficient and necessary with zero restrictiveness, since no
calendar arbitrage exists. At moderate noise ($\sigma_\epsilon = 0.02$), PVO
begins to reject 6.20\% of surfaces that are in fact arbitrage-free,
indicating growing restrictiveness.

At high noise ($\sigma_\epsilon = 0.05$), PVO becomes completely
restrictive (no surfaces satisfy it), while EDC maintains near-perfect
sufficiency at 0.9980 with only 0.60\% restrictiveness. SDC achieves
perfect sufficiency of 1.0000 but with 30.40\% restrictiveness.

At the highest noise level ($\sigma_\epsilon = 0.10$), where the arbitrage
rate reaches 11.00\%, EDC perfectly separates arbitrage-free from
arbitrage-violating surfaces (sufficiency and necessity both 1.0000)
with 11.00\% restrictiveness matching the actual arbitrage rate.
SDC remains perfectly sufficient but becomes highly restrictive at
95.20\%, accepting only 24 of 500 surfaces.

\paragraph{Aggregate performance.}
Across all noise levels (Table~\ref{tab:aggregate}), EDC achieves the
best overall balance: average sufficiency of 0.9996, average necessity
of 1.0000, average false positive rate of 0.0004, and average
restrictiveness of 0.0232. SDC offers perfect average sufficiency of
1.0000 with zero false positives but at higher average restrictiveness
of 0.2512. PVO achieves average sufficiency of 0.6000 and average
necessity of 0.5876, making it unreliable for high-noise regimes.

\begin{table}[t]
\centering
\caption{Aggregate condition metrics averaged across all noise levels.}
\label{tab:aggregate}
\small
\begin{tabular}{@{}lrrrr@{}}
\toprule
Condition & Avg Suff. & Avg Nec. & Avg FP & Avg Restrict. \\
\midrule
PVO & 0.6000 & 0.5876 & 0.0000 & 0.4124 \\
EDC & 0.9996 & 1.0000 & 0.0004 & 0.0232 \\
SDC & 1.0000 & 0.7511 & 0.0000 & 0.2512 \\
\bottomrule
\end{tabular}
\end{table}

\subsection{Extrapolation Risk}
\label{sec:extrap}

A key concern is that calendar arbitrage may arise only in the extrapolated
region beyond quoted strikes. Table~\ref{tab:extrap} shows the arbitrage
rates for inner (quoted), outer (extrapolated), and full strike domains.

\begin{table}[t]
\centering
\caption{Calendar arbitrage rates by strike region. ``Extrap-only'' counts
surfaces with arbitrage only outside the quoted range.}
\label{tab:extrap}
\small
\begin{tabular}{@{}lrrrr@{}}
\toprule
Noise & Inner & Outer & Full & Extrap-only \\
\midrule
0.005 & 0.0000 & 0.0000 & 0.0000 & 0.0000 \\
0.01  & 0.0000 & 0.0000 & 0.0000 & 0.0000 \\
0.02  & 0.0000 & 0.0000 & 0.0000 & 0.0000 \\
0.05  & 0.0000 & 0.0080 & 0.0080 & 0.0080 \\
0.10  & 0.0760 & 0.0740 & 0.1100 & 0.0340 \\
\bottomrule
\end{tabular}
\end{table}

At $\sigma_\epsilon = 0.05$, all arbitrage violations occur exclusively
in the extrapolated region (extrap-only rate of 0.0080 equals the full
rate), confirming that extrapolation is the primary source of calendar
arbitrage at moderate noise levels. At $\sigma_\epsilon = 0.10$, the
extrapolation-only rate is 0.0340, meaning 30.9\% of violating surfaces
have arbitrage only outside the quoted range. The average extrapolation
risk increase across all noise levels is 0.8895.

\subsection{Violation Severity}
\label{sec:severity}

\begin{table}[t]
\centering
\caption{Violation severity statistics for surfaces exhibiting calendar
arbitrage.}
\label{tab:severity}
\small
\begin{tabular}{@{}lrrrrr@{}}
\toprule
Noise & Mean & Median & P95 & Max & $n$ \\
\midrule
0.05 & 0.0022 & 0.0013 & 0.0056 & 0.0062 & 4 \\
0.10 & 0.2793 & 0.1084 & 0.9205 & 2.8688 & 55 \\
\bottomrule
\end{tabular}
\end{table}

Table~\ref{tab:severity} reports violation severity for noise levels where
arbitrage occurs. At $\sigma_\epsilon = 0.05$, violations are small (mean
maximum violation 0.0022, only 4 surfaces affected with mean fraction
violated of 0.0578). At $\sigma_\epsilon = 0.10$, violations become
substantial (mean maximum 0.2793, 95th percentile 0.9205, maximum
observed 2.8688), affecting 55 of 500 surfaces with mean fraction
violated of 0.0996.

\subsection{Computational Cost Analysis}
\label{sec:cost}

\begin{table}[t]
\centering
\caption{Computational cost per expiry pair and total (4 pairs, 11 strikes,
201 evaluation points).}
\label{tab:cost}
\small
\begin{tabular}{@{}lrrr@{}}
\toprule
Condition & Complexity & Per-pair ops & Total ops \\
\midrule
PVO & $O(n)$ & 11 & 44 \\
SDC & $O(n \log n)$ & 88 & 352 \\
EDC & $O(n_{\text{eval}} \cdot n)$ & 2211 & 8844 \\
\bottomrule
\end{tabular}
\end{table}

Table~\ref{tab:cost} compares the theoretical computational costs. PVO is
the cheapest at 44 total operations but lacks reliability. SDC requires
352 operations (8$\times$ PVO) while providing perfect sufficiency.
EDC requires 8844 operations (201$\times$ PVO), making it 25$\times$
more expensive than SDC. For practical surfaces with thousands of
evaluation strikes, this gap becomes substantial.

%% ============================================================
\section{Discussion}
\label{sec:discussion}

\subsection{Practical Recommendations}

Our results establish a clear hierarchy for selecting arbitrage conditions
in the generalized strike-wise interpolation model:

\begin{enumerate}
\item \textbf{Low-noise regime} ($\sigma_\epsilon \leq 0.02$): Any
condition suffices; PVO is optimal due to its $O(n)$ cost.
\item \textbf{Moderate-noise regime} ($\sigma_\epsilon \approx 0.05$):
SDC provides perfect sufficiency with 30.40\% restrictiveness at
$O(n \log n)$ cost, making it the best choice.
\item \textbf{High-noise regime} ($\sigma_\epsilon \geq 0.10$): EDC is
necessary for accurate arbitrage detection, despite its higher
$O(n_{\text{eval}} \cdot n)$ cost.
\end{enumerate}

\subsection{The Extrapolation Challenge}

Our analysis confirms the observation by Buehler et al.~\cite{buehler2026sanos}
that calendar arbitrage at quoted strikes does not guarantee arbitrage-free
behavior at extrapolated strikes. The average extrapolation risk increase of
0.8895 quantifies this gap and motivates the development of extrapolation-aware
conditions. The finding that at moderate noise all arbitrage violations are
extrapolation-only (0.0080 rate) underscores the importance of checking
beyond the quoted range.

\subsection{Sufficiency vs.\ Efficiency Tradeoff}

The tension between computational efficiency and condition quality is
captured by the SDC--EDC comparison. SDC achieves perfect sufficiency
(no false positives ever) at 8$\times$ the cost of PVO, while EDC
achieves near-perfect sufficiency at 201$\times$ the cost. For
applications requiring guaranteed absence of arbitrage, SDC provides the
best tradeoff; for applications tolerating a 0.0004 false positive rate,
EDC is preferable due to its lower restrictiveness.

\subsection{Limitations and Future Work}

Our study uses synthetic surfaces with a specific parametric form (quadratic
smile, linear term structure). Real market surfaces may exhibit more complex
structures. Future work should validate on historical market data, explore
tighter sufficient conditions between SDC and EDC in computational cost,
and investigate adaptive evaluation grids that concentrate on
arbitrage-prone regions~\cite{fengler2009arbitrage,kahalé2004arbitrage}.

%% ============================================================
\section{Related Work}
\label{sec:related}

Arbitrage-free option surface construction has been studied extensively.
Gatheral and Jacquier~\cite{gatheral2014arbitrage} provide conditions for
SVI parameterizations. Fengler~\cite{fengler2009arbitrage} proposes
smoothing methods preserving arbitrage constraints.
Dupire~\cite{dupire1994pricing} establishes the connection between
option surfaces and local volatility.
Kahal{\'e}~\cite{kahalé2004arbitrage} develops interpolation methods
ensuring no-arbitrage. Carr and Madan~\cite{carr2005note} provide
general sufficient conditions. The SANOS framework~\cite{buehler2026sanos}
extends these ideas to non-parametric surfaces with strict arbitrage
guarantees, but leaves the per-strike variance condition as an open
problem that our work addresses.

%% ============================================================
\section{Conclusion}
\label{sec:conclusion}

We have investigated the open problem of finding efficient conditions on
per-strike variances for calendar-arbitrage-free option surface
construction. Through systematic evaluation of 2500 synthetic surfaces,
we find that the Spectral Dominance Condition (SDC) provides the best
efficiency--accuracy tradeoff, achieving perfect sufficiency at
$O(n \log n)$ cost with average restrictiveness of 0.2512.
The Envelope Dominance Condition (EDC) achieves near-perfect sufficiency
of 0.9996 with the lowest restrictiveness of 0.0232 but at higher
computational cost. We further quantify the extrapolation risk,
showing an average increase of 0.8895 in arbitrage frequency when
extending beyond quoted strikes. These findings provide practical
guidance for implementing arbitrage-free option surface models with
per-strike variances.

\begin{figure}[t]
\centering
\includegraphics[width=\columnwidth]{figures/condition_sufficiency.pdf}
\caption{Sufficiency rates of PVO, EDC, and SDC across noise levels.
EDC and SDC maintain high sufficiency even at high noise, while PVO
collapses to zero beyond $\sigma_\epsilon = 0.02$.}
\label{fig:sufficiency}
\end{figure}

\begin{figure}[t]
\centering
\includegraphics[width=\columnwidth]{figures/arbitrage_rates.pdf}
\caption{Calendar arbitrage rates by strike region (inner, outer, full)
across noise levels. Extrapolated regions show earlier onset of
arbitrage violations.}
\label{fig:arb_rates}
\end{figure}

\begin{figure}[t]
\centering
\includegraphics[width=\columnwidth]{figures/cost_vs_sufficiency.pdf}
\caption{Computational cost versus average sufficiency for the three
conditions. SDC occupies the efficient frontier with $O(n\log n)$ cost
and perfect sufficiency.}
\label{fig:cost_suff}
\end{figure}

\begin{figure}[t]
\centering
\includegraphics[width=\columnwidth]{figures/violation_severity.pdf}
\caption{Distribution of maximum calendar arbitrage violations at
$\sigma_\epsilon = 0.10$. The median violation is 0.1084 with a
long right tail reaching 2.8688.}
\label{fig:severity}
\end{figure}

\bibliographystyle{ACM-Reference-Format}
\bibliography{references}

\end{document}
