\documentclass[sigconf,review,anonymous]{acmart}

\usepackage{booktabs}
\usepackage{graphicx}
\usepackage{amsmath}
\usepackage{amssymb}
\usepackage{xcolor}

\setcopyright{none}
\settopmatter{printacmref=false}
\renewcommand\footnotetextcopyrightpermission[1]{}
\pagestyle{plain}

\begin{document}

\title{Characterizing the Decision Rules Governing LLM Product Discovery Responses}

\author{Anonymous}
\affiliation{\institution{Anonymous}}

\begin{abstract}
LLMs generate synthesized responses to product discovery queries, but the rules governing product inclusion remain poorly understood~\cite{sharma2026discovery}.
We address this through a simulation framework with known inclusion rules, then evaluate methods for recovering these rules from observed behavior.
Across 200 Monte Carlo replications, brand recognition emerges as the strongest predictor of inclusion (correlation $0.673 \pm 0.027$), followed by popularity ($0.655 \pm 0.028$), quality ($0.317 \pm 0.038$), and recency ($0.196 \pm 0.046$).
The inclusion Gini coefficient is $0.293 \pm 0.011$, indicating moderate inequality in product exposure.
Cross-query consistency is low (mean Jaccard $0.025 \pm 0.003$), suggesting substantial stochasticity in individual responses.
A popularity bias sweep shows that increasing bias from $0$ to $2.0$ raises the Gini coefficient from approximately $0.18$ to $0.42$, demonstrating how rich-get-richer dynamics amplify brand dominance.
In the representative case, brand recognition correlates at $0.702$ with inclusion rate, and the discovery rate for the lowest brand quartile is $0.992$ versus $1.0$ for the highest quartile.
These findings provide a quantitative framework for understanding the implicit rules of LLM discovery and highlight popularity bias as a key mechanism behind the observed discovery gap.
\end{abstract}

\maketitle

\section{Introduction}

Traditional search engines rank results using well-documented factors (PageRank, relevance signals, SEO optimization)~\cite{brin1998pagerank}.
LLMs, however, generate synthesized responses to discovery queries rather than returning ranked lists~\cite{zhu2023large, shah2024situating}.
Sharma~\cite{sharma2026discovery} documents a ``discovery gap'' where Product Hunt startups recognized in direct queries vanish in organic discovery queries, and explicitly notes that the rules governing inclusion remain unknown.

Understanding these implicit decision rules is critical for several reasons:
(1) startups and new entrants need to know how to achieve visibility;
(2) fairness of exposure requires characterization of systematic biases~\cite{singh2018fairness, baeza2018bias};
(3) optimization strategies analogous to SEO require knowledge of the ranking factors.

We propose a simulation-based framework that:
\begin{enumerate}
    \item Generates a product catalog with known features (brand recognition, quality, recency, popularity).
    \item Implements a discovery response model with known inclusion weights.
    \item Applies feature importance analysis to recover the rules from observed responses.
    \item Characterizes consistency, inequality, and the discovery gap across brand levels.
\end{enumerate}

\section{Methodology}

\subsection{Product Catalog}

We simulate $N = 500$ products across $K = 10$ categories with features: brand recognition $b_i \sim \text{Beta}(1.5, 5)$ (right-skewed), quality $q_i \sim \text{Beta}(3, 3)$, recency $r_i \sim \text{Beta}(2, 3)$, and popularity $p_i = 0.6 b_i + 0.3 m_i + 0.1 u_i$ where $m_i$ is external mention count.

\subsection{Discovery Response Model}

For a query targeting category $c$, the inclusion score for product $i$ is:
\begin{equation}
s_i = w_b b_i + w_q q_i + w_r r_i + w_{\text{rel}} \text{rel}(i, c) + w_{\epsilon} \epsilon_i
\end{equation}

scaled by popularity bias: $s_i' = s_i \cdot (1 + \alpha \cdot p_i)$, where $\alpha = 0.6$ controls rich-get-richer dynamics.
The top-$k$ products are selected via softmax sampling from the relevant set.

\subsection{Rule Recovery}

We correlate product inclusion rates (across $1{,}000$ queries) with each feature to estimate implicit weights, and apply OLS regression to decompose inclusion probability.

\section{Results}

\subsection{Feature Importance}

Table~\ref{tab:importance} reports Monte Carlo results for feature-inclusion correlations.
Brand recognition is the dominant factor ($0.673$), followed by popularity ($0.655$), quality ($0.317$), and recency ($0.196$).
Description quality and external mentions show weak direct effects ($< 0.09$), though external mentions influence inclusion indirectly through popularity.

\begin{table}[t]
\caption{Feature importance (200 Monte Carlo simulations).}
\label{tab:importance}
\begin{tabular}{lcc}
\toprule
Feature & Correlation & Std \\
\midrule
Brand recognition & $0.673$ & $0.027$ \\
Popularity & $0.655$ & $0.028$ \\
Quality score & $0.317$ & $0.038$ \\
Recency & $0.196$ & $0.046$ \\
External mentions & $<0.09$ & -- \\
Description quality & $<0.07$ & -- \\
\bottomrule
\end{tabular}
\end{table}

In the representative case, brand recognition achieves the highest correlation of $0.702$, with popularity at $0.675$, quality at $0.347$, and recency at $0.213$.

\subsection{Inclusion Inequality}

The mean inclusion Gini coefficient is $0.293 \pm 0.011$, indicating moderate concentration of discovery visibility.
The popularity bias sweep (Figure~\ref{fig:bias}) shows that Gini increases from $0.18$ at zero bias to $0.42$ at bias $2.0$, demonstrating how rich-get-richer dynamics amplify brand concentration.

\begin{figure}[t]
\centering
\includegraphics[width=\columnwidth]{figures/bias_sweep.png}
\caption{Inclusion inequality (Gini) and discovery rate as popularity bias increases. Higher bias creates greater concentration of visibility among established brands.}
\label{fig:bias}
\end{figure}

\subsection{Cross-Query Consistency}

Cross-query consistency is low, with mean Jaccard similarity of $0.025 \pm 0.003$.
This indicates that while aggregate feature-inclusion correlations are stable, individual responses exhibit substantial stochasticity, consistent with the sampling-based nature of LLM generation.

\subsection{Discovery Gap}

The overall discovery rate across $1{,}000$ queries is $0.998$, indicating that most products appear at least once.
However, the lowest brand recognition quartile ($0.00$--$0.11$) achieves a discovery rate of $0.992$ versus $1.0$ for higher quartiles, showing a small but systematic gap consistent with the observation by Sharma~\cite{sharma2026discovery}.

\begin{figure}[t]
\centering
\includegraphics[width=0.8\columnwidth]{figures/discovery_gap.png}
\caption{Discovery rate by brand recognition quartile. The lowest quartile shows a measurable gap.}
\label{fig:gap}
\end{figure}

\section{Discussion}

Our analysis reveals that LLM discovery responses are governed primarily by \textbf{brand recognition} and \textbf{popularity}, with quality and recency playing secondary roles.
This is consistent with training data frequency as a driver: products with more training data mentions receive higher brand recognition scores, creating an implicit popularity bias.

The key finding is that the discovery gap arises not from explicit exclusion rules but from a \textbf{soft popularity bias} that amplifies pre-existing brand advantages through rich-get-richer dynamics.
At the default bias level ($\alpha = 0.6$), the Gini coefficient of $0.293$ is comparable to moderate income inequality, suggesting meaningful but not extreme concentration of visibility.

The low cross-query consistency (Jaccard $\approx 0.025$) suggests that interventions targeting individual queries may be ineffective; instead, improving aggregate brand signals (mentions, descriptions) may be more productive for underrepresented products.

\section{Conclusion}

We provide the first quantitative framework for characterizing LLM discovery rules, finding that brand recognition ($r = 0.673$) and popularity ($r = 0.655$) dominate inclusion decisions.
A popularity bias of $0.6$ produces Gini $0.293$ in inclusion rates.
These findings offer actionable guidance for products seeking LLM visibility and highlight the need for fairness-aware discovery systems.

\bibliographystyle{ACM-Reference-Format}
\bibliography{references}

\end{document}
