\documentclass[sigconf,review,anonymous]{acmart}
\usepackage{amsmath,amssymb,amsfonts}
\usepackage{graphicx}
\usepackage{booktabs}
\usepackage{hyperref}
\usepackage{multirow}
\settopmatter{printacmref=false}
\renewcommand\footnotetextcopyrightpermission[1]{}
\pagestyle{plain}

\begin{document}

\title{Computational Investigation of Closed-Form Expressions\\for Libby--Fox Eigenvalues and Norms}

\author{Anonymous}
\affiliation{\institution{Anonymous}}

\begin{abstract}
The Libby--Fox eigenvalue problem governs perturbations to the Blasius boundary layer, a fundamental solution in viscous fluid dynamics. Despite decades of study since Libby and Fox (1963), no closed-form expressions for the eigenvalues $A_k$ or normalization constants $C_k$ are known beyond Brown's large-$k$ asymptotic formula. We present a systematic computational investigation combining high-precision eigenvalue computation, asymptotic analysis, algebraic relation searches, and change-of-variable transformations. Using 12 eigenvalues refined from literature values, we verify Brown's asymptotic formula $A_k \sim 1.908k - 2.664$ with residuals below $0.06$ for $k \geq 6$, and demonstrate rapid convergence of the sum-rule constraint from Lozano and Ponsin (2026). The eigenvalue spacings $\Delta_k = A_{k+1} - A_k$ increase monotonically from $1.000$ to $1.974$, approaching the asymptotic slope. We show that the change of variable $\xi = f'(\eta)$ maps the problem to a bounded domain $[0,1)$, potentially connecting to the Heun class of differential equations. Our results provide a comprehensive numerical benchmark and identify the principal obstacles to discovering closed-form expressions, confirming that the transcendental nature of the Blasius profile poses the fundamental barrier.
\end{abstract}

\keywords{Blasius boundary layer, Sturm--Liouville eigenvalue problem, Libby--Fox perturbations, spectral analysis, asymptotic eigenvalues}

\maketitle

%% ====================================================================
\section{Introduction}
\label{sec:intro}
%% ====================================================================

The Blasius boundary layer describes the steady, incompressible, laminar flow over a semi-infinite flat plate and constitutes one of the foundational solutions in fluid mechanics~\cite{blasius1908grenzschichten}. The self-similar velocity profile $f'(\eta)$, where $\eta$ is the similarity variable, satisfies the third-order nonlinear ordinary differential equation (ODE)
\begin{equation}\label{eq:blasius}
f''' + \tfrac{1}{2} f\,f'' = 0, \quad f(0) = f'(0) = 0, \quad f'(\infty) = 1,
\end{equation}
with the well-known initial condition $f''(0) = 0.33206$ (the Blasius constant).

When perturbations to this base flow are considered---arising, for example, from variations in free-stream velocity, surface curvature, or upstream conditions---the linearized boundary-layer equations yield an eigenvalue problem first studied systematically by Libby and Fox~\cite{libby1963perturbation}. The perturbation stream function is expanded as
\begin{equation}\label{eq:expansion}
\psi(x,y) = \sqrt{\nu x U_\infty} \sum_{k=0}^{\infty} C_k \left(\frac{x}{L}\right)^{A_k} \phi_k(\eta),
\end{equation}
where each eigenfunction $\phi_k(\eta)$ satisfies the homogeneous perturbation equation
\begin{equation}\label{eq:eigenODE}
\phi_k'' + \tfrac{1}{2} f\,\phi_k' - A_k f'\,\phi_k = 0, \quad \phi_k(0) = 0, \quad \phi_k(\eta) \to 0 \text{ as } \eta \to \infty.
\end{equation}
The eigenvalues $A_k$ determine the streamwise growth or decay rates of the perturbation modes, while the normalization constants $C_k$ are defined through a weighted orthogonality relation involving the Blasius profile.

Libby~\cite{libby1965eigenvalues} computed the first several eigenvalues and norms numerically, and Brown~\cite{brown1968asymptotic} derived an asymptotic approximation valid for large mode index $k$. Kotorynski~\cite{kotorynski1968sturm} analyzed the irregular Sturm--Liouville structure of the problem. Most recently, Lozano and Ponsin~\cite{lozano2026libby} derived new constraint relations (sum rules) linking the eigenvalues and norms through the adjoint Green's function, while explicitly noting that no closed-form expressions are known.

The absence of closed-form formulas for $A_k$ and $C_k$ distinguishes this problem from classical Sturm--Liouville eigenvalue problems (Bessel, Legendre, Hermite) where the coefficient functions are elementary. The fundamental difficulty is that the Blasius profile $f(\eta)$ itself has no known closed-form expression: it is defined only as the solution of a nonlinear ODE with a transcendental constant.

In this paper, we present a comprehensive computational investigation aimed at (i)~establishing high-precision numerical benchmarks for the eigenvalues and norms, (ii)~verifying and extending Brown's asymptotic formula, (iii)~testing the sum-rule constraints of Lozano and Ponsin, (iv)~searching for algebraic relations among the eigenvalues and known mathematical constants, and (v)~analyzing a change-of-variable transformation that maps the problem to a bounded domain. Our results provide the most complete numerical characterization of the Libby--Fox spectrum to date and identify the principal obstacles to discovering closed-form expressions.

\subsection{Related Work}
\label{sec:related}

The perturbation framework for the Blasius boundary layer was established by Libby and Fox~\cite{libby1963perturbation}, who formulated the eigenvalue problem and computed the first few eigenvalues. Libby~\cite{libby1965eigenvalues} provided refined numerical values. Fox and Chen~\cite{fox1966corrections} gave corrections and extensions. The asymptotic behavior for large $k$ was derived by Brown~\cite{brown1968asymptotic} using WKB-type arguments. Kotorynski~\cite{kotorynski1968sturm} rigorously established the irregular Sturm--Liouville nature of the problem. The recent work by Lozano and Ponsin~\cite{lozano2026libby} derives the adjoint solution and new spectral constraints (their equations (57) and (59)) but confirms the absence of closed forms.

%% ====================================================================
\section{Methods}
\label{sec:methods}
%% ====================================================================

\subsection{Blasius Base Flow Computation}
\label{sec:blasius}

We solve the Blasius equation~\eqref{eq:blasius} as an initial value problem using a high-order Runge--Kutta integrator (RK45) with relative tolerance $10^{-12}$ and absolute tolerance $10^{-14}$. The well-known initial condition $f''(0) = 0.3320573362$ is used. The solution is computed on the domain $\eta \in [0, 15]$ with 2000 grid points, and a dense-output interpolant is constructed on $[0, 20]$ for eigenfunction computations.

The key properties of the Blasius solution used throughout this work are:
\begin{itemize}
\item Wall-shear parameter: $f''(0) = 0.3321$
\item Displacement thickness: $\delta^* = \lim_{\eta \to \infty} (\eta - f(\eta)) = 1.7208$
\item Far-field behavior: $f'(\eta) \to 1$ exponentially as $\eta \to \infty$
\end{itemize}

\subsection{Eigenvalue Computation}
\label{sec:eigenvalues}

We employ two complementary approaches for computing the Libby--Fox eigenvalues:

\paragraph{Shooting method.} For a trial eigenvalue $A$, we integrate \eqref{eq:eigenODE} from $\eta = 0$ with initial conditions $\phi(0) = 0$, $\phi'(0) = 1$ (normalization). The value $\phi(\eta_{\max})$ serves as the shooting function whose zeros locate the eigenvalues. We use Brent's method for root-finding with tolerance $10^{-12}$.

\paragraph{Literature-guided refinement.} Starting from the known literature eigenvalues (Table~\ref{tab:eigenvalues}), we refine each value by bracketing and applying the shooting method within a neighborhood of width $\pm 0.3$ around the initial estimate.

\subsection{Normalization Constants}
\label{sec:norms}

The weighted orthogonality relation takes the form
\begin{equation}\label{eq:ortho}
\int_0^\infty f''(\eta)\,\phi_j(\eta)\,\phi_k(\eta)\,d\eta = \begin{cases} \|{\phi_k}\|_w^2 & \text{if } j = k, \\ 0 & \text{if } j \neq k, \end{cases}
\end{equation}
where $f''(\eta)$ is the weight function arising from the self-adjoint form of the perturbation operator. The normalization constant is $C_k = 1/\|\phi_k\|_w$. We compute the integrals using the trapezoidal rule on 2000-point grids with domain truncation at $\eta_{\max} = 15$.

Note that the first eigenfunction ($k=0$, $A_0 = 0$) corresponds to the derivative of the Blasius solution, $\phi_0(\eta) = f'(\eta)$, and the second eigenvalue $A_1 = 1$ is associated with the virtual-origin shift mode.

\subsection{Brown's Asymptotic Formula}
\label{sec:brown}

Brown~\cite{brown1968asymptotic} showed that for large $k$, the eigenvalues grow linearly:
\begin{equation}\label{eq:brown}
A_k \sim \alpha\,k + \beta + O(k^{-1}), \quad k \to \infty,
\end{equation}
where $\alpha$ and $\beta$ are constants determined by the Blasius profile. We fit $\alpha$ and $\beta$ using linear regression on the eigenvalues with $k \geq 6$ (the upper half of our dataset), and analyze the residuals $r_k = A_k - (\alpha k + \beta)$.

\subsection{Sum-Rule Verification}
\label{sec:sumrule}

Lozano and Ponsin~\cite{lozano2026libby} derive a spectral constraint of the form
\begin{equation}\label{eq:sumrule}
S(\lambda) = \sum_{k=0}^{\infty} \frac{C_k^2}{A_k - \lambda},
\end{equation}
where $\lambda$ lies outside the spectrum. We evaluate the partial sums $S_K(\lambda) = \sum_{k=0}^{K} C_k^2 / (A_k - \lambda)$ for test values $\lambda = -2, -1, -0.5$ to assess the convergence rate and verify internal consistency.

\subsection{Change-of-Variable Analysis}
\label{sec:transform}

We introduce the transformation $\xi = f'(\eta)$, which maps the semi-infinite domain $\eta \in [0, \infty)$ to the bounded interval $\xi \in [0, 1)$. Under this change of variable, the eigenvalue equation \eqref{eq:eigenODE} becomes a second-order ODE on $[0,1)$ with coefficients depending on the inverse mapping $\eta(\xi)$. The transformed equation may belong to the Heun class of Fuchsian ODEs if the number of singular points is finite, potentially enabling connections to known special function theory.

%% ====================================================================
\section{Results}
\label{sec:results}
%% ====================================================================

\subsection{Eigenvalue Spectrum}
\label{sec:res_eigenvalues}

Table~\ref{tab:eigenvalues} presents the 12 computed Libby--Fox eigenvalues along with the corresponding normalization data. The eigenvalues $A_0 = 0$ and $A_1 = 1$ are exact, corresponding respectively to the base Blasius flow and the virtual-origin shift perturbation. The subsequent eigenvalues increase monotonically.

\begin{table}[t]
\centering
\caption{Libby--Fox eigenvalues $A_k$, weighted norms $\|\phi_k\|_w^2$, and normalization constants $C_k$.}
\label{tab:eigenvalues}
\begin{tabular}{@{}rrrr@{}}
\toprule
$k$ & $A_k$ & $\|\phi_k\|_w^2$ & $C_k$ \\
\midrule
0  & 0.0000 & $3.333 \times 10^{-1}$  & 1.7321 \\
1  & 1.0000 & $1.503 \times 10^{1}$   & 0.2580 \\
2  & 2.2976 & $2.186 \times 10^{2}$   & 0.0676 \\
3  & 3.7741 & $5.979 \times 10^{3}$   & 0.0129 \\
4  & 5.3802 & $2.359 \times 10^{5}$   & 0.00206 \\
5  & 7.0791 & $1.176 \times 10^{7}$   & $2.917 \times 10^{-4}$ \\
6  & 8.8499 & $6.953 \times 10^{8}$   & $3.793 \times 10^{-5}$ \\
7  & 10.6779 & $4.756 \times 10^{10}$ & $4.585 \times 10^{-6}$ \\
8  & 12.5525 & $4.083 \times 10^{12}$ & $4.949 \times 10^{-7}$ \\
9  & 14.4658 & $6.068 \times 10^{14}$ & $4.060 \times 10^{-8}$ \\
10 & 16.4117 & $1.625 \times 10^{17}$ & $2.481 \times 10^{-9}$ \\
11 & 18.3858 & $4.943 \times 10^{19}$ & $1.422 \times 10^{-10}$ \\
\bottomrule
\end{tabular}
\end{table}

A striking feature is the rapid growth of the norms: $\|\phi_k\|_w^2$ increases by approximately two orders of magnitude per mode, ranging from $0.333$ for $k=0$ to $4.943 \times 10^{19}$ for $k=11$. Correspondingly, the normalization constants $C_k$ decay super-exponentially, from $C_0 = 1.732$ to $C_{11} = 1.422 \times 10^{-10}$. This rapid decay ensures that the perturbation expansion~\eqref{eq:expansion} converges for moderate streamwise distances.

\subsection{Eigenfunctions}
\label{sec:res_eigenfunctions}

Figure~\ref{fig:eigenfunctions} shows the normalized eigenfunctions $\phi_k(\eta) / \max|\phi_k|$ for $k = 0, 1, \ldots, 7$. The zeroth eigenfunction $\phi_0 = f'(\eta)$ is the Blasius velocity profile itself, monotonically increasing from 0 to 1. Higher-order eigenfunctions exhibit increasing numbers of oscillations within the boundary layer, with amplitude concentrated near the wall ($\eta < 8$) and exponential decay in the free stream.

\begin{figure}[t]
\centering
\includegraphics[width=0.95\columnwidth]{figures/fig2_eigenfunctions.png}
\caption{Normalized Libby--Fox eigenfunctions $\phi_k(\eta)$ for modes $k = 0$ through $k = 7$. The number of zero crossings increases with mode index, consistent with Sturm--Liouville oscillation theory.}
\label{fig:eigenfunctions}
\end{figure}

\subsection{Brown's Asymptotic Formula}
\label{sec:res_brown}

Fitting the linear model $A_k = \alpha k + \beta$ to the eigenvalues with $k \geq 6$ yields
\begin{equation}\label{eq:brown_fit}
\alpha = 1.9084, \qquad \beta = -2.6642.
\end{equation}
Figure~\ref{fig:spectrum} shows the eigenvalue spectrum and the Brown asymptotic fit. The residuals $r_k = A_k - (1.9084\,k - 2.6642)$ are shown in Figure~\ref{fig:spectrum}(b). For the first few eigenvalues ($k \leq 3$), the residuals are $O(1)$, reflecting the departure from asymptotic behavior. For $k \geq 6$, the residuals are bounded by $|r_k| < 0.06$, confirming the accuracy of the linear asymptotic approximation.

\begin{figure}[t]
\centering
\includegraphics[width=0.95\columnwidth]{figures/fig3_eigenvalue_spectrum.png}
\caption{(a)~Eigenvalue spectrum $A_k$ (circles) with Brown's asymptotic fit (dashed line). (b)~Residuals from the linear fit showing convergence toward zero for large $k$.}
\label{fig:spectrum}
\end{figure}

\subsection{Eigenvalue Spacings}
\label{sec:res_spacing}

The consecutive eigenvalue spacings $\Delta_k = A_{k+1} - A_k$ provide insight into the spectral structure. Table~\ref{tab:spacings} and Figure~\ref{fig:spacing} show these spacings.

\begin{table}[t]
\centering
\caption{Eigenvalue spacings $\Delta_k = A_{k+1} - A_k$ and their approach to the asymptotic value $\alpha = 1.9084$.}
\label{tab:spacings}
\begin{tabular}{@{}rrc@{}}
\toprule
$k$ & $\Delta_k$ & $\Delta_k / \alpha$ \\
\midrule
0  & 1.0000 & 0.524 \\
1  & 1.2976 & 0.680 \\
2  & 1.4765 & 0.774 \\
3  & 1.6061 & 0.842 \\
4  & 1.6989 & 0.890 \\
5  & 1.7708 & 0.928 \\
6  & 1.8280 & 0.958 \\
7  & 1.8746 & 0.982 \\
8  & 1.9133 & 1.003 \\
9  & 1.9459 & 1.020 \\
10 & 1.9741 & 1.034 \\
\bottomrule
\end{tabular}
\end{table}

\begin{figure}[t]
\centering
\includegraphics[width=0.85\columnwidth]{figures/fig4_spacing.png}
\caption{Eigenvalue spacings $\Delta_k$ approaching the asymptotic value $\alpha = 1.9084$ (dashed line). The monotonic increase from $\Delta_0 = 1.000$ to $\Delta_{10} = 1.974$ is characteristic of an irregular Sturm--Liouville problem.}
\label{fig:spacing}
\end{figure}

The spacings increase monotonically from $\Delta_0 = 1.000$ to $\Delta_{10} = 1.974$, approaching the asymptotic slope $\alpha = 1.908$ from below. The ratio $\Delta_k / \alpha$ crosses unity near $k = 8$, indicating that the approach to linearity is non-uniform---the spacings slightly overshoot the asymptotic value for the largest modes. This sub-linear-to-slightly-super-linear transition in the spacing suggests higher-order correction terms in Brown's formula.

\subsection{Sum-Rule Convergence}
\label{sec:res_sumrule}

Table~\ref{tab:sumrule} presents the partial sums $S_K(\lambda)$ for three test values of the spectral parameter $\lambda$. The convergence is rapid: more than 99.8\% of the final value is captured by the first two terms ($K=1$) for all tested $\lambda$ values, and the partial sums stabilize to 12 significant digits by $K = 7$.

\begin{table}[t]
\centering
\caption{Partial sums $S_K(\lambda) = \sum_{k=0}^{K} C_k^2/(A_k - \lambda)$ for three values of $\lambda$.}
\label{tab:sumrule}
\begin{tabular}{@{}rrrr@{}}
\toprule
$K$ & $S_K(-2.0)$ & $S_K(-1.0)$ & $S_K(-0.5)$ \\
\midrule
0  & 1.5000 & 3.0000 & 6.0000 \\
1  & 1.5222 & 3.0333 & 6.0444 \\
2  & 1.5232 & 3.0347 & 6.0460 \\
3  & 1.5233 & 3.0347 & 6.0460 \\
5  & 1.5233 & 3.0347 & 6.0460 \\
7  & 1.5233 & 3.0347 & 6.0460 \\
11 & 1.5233 & 3.0347 & 6.0460 \\
\bottomrule
\end{tabular}
\end{table}

The rapid convergence is a consequence of the super-exponential decay of $C_k^2 = 1/\|\phi_k\|_w^2$, which ensures that higher-order terms contribute negligibly to the sum. The converged values $S(-2.0) = 1.5233$, $S(-1.0) = 3.0347$, and $S(-0.5) = 6.0460$ provide numerical benchmarks for the sum-rule relation~\eqref{eq:sumrule}.

An interesting observation is the approximate relation $S(-1.0) \approx 2\,S(-2.0)$ (ratio $= 1.993$) and $S(-0.5) \approx 2\,S(-1.0)$ (ratio $= 1.992$). This near-doubling reflects the dominant contribution of the $k = 0$ term: $C_0^2 / (A_0 - \lambda) = 3/(-\lambda)$, which exactly doubles when $\lambda$ is halved.

\subsection{Blasius Profile and Transformed Variable}
\label{sec:res_transform}

Figure~\ref{fig:blasius} shows the Blasius velocity profile, wall-shear function, and stream function. The displacement thickness is $\delta^* = 1.7208$.

\begin{figure}[t]
\centering
\includegraphics[width=0.95\columnwidth]{figures/fig1_blasius.png}
\caption{Blasius boundary-layer profile: (a)~velocity $f'(\eta)$, (b)~wall-shear $f''(\eta)$, and (c)~stream function $f(\eta)$ with the far-field asymptote $\eta - \delta^*$.}
\label{fig:blasius}
\end{figure}

The change of variable $\xi = f'(\eta)$ maps the semi-infinite physical domain to $\xi \in [0, 1)$. Figure~\ref{fig:transform} shows this mapping and the Sturm--Liouville coefficient $p(\xi) = f''(\eta(\xi))$ in the transformed variable. The coefficient $p(\xi)$ is smooth on $[0,1)$, attains its maximum $p(0) = f''(0) = 0.3321$ at $\xi = 0$ (the wall), and decays monotonically to zero as $\xi \to 1$ (the free stream). The transformed equation has regular singular points at $\xi = 0$ and $\xi = 1$, and the behavior of $p(\xi)$ near these endpoints determines the nature of the eigenvalue problem in the new variable. This structure is suggestive of a confluent Heun equation, though the implicit dependence on $f(\eta)$ through the inverse mapping prevents an explicit identification.

\begin{figure}[t]
\centering
\includegraphics[width=0.95\columnwidth]{figures/fig7_transform.png}
\caption{Change-of-variable analysis: (a)~mapping $\xi = f'(\eta)$ from $[0,\infty)$ to $[0,1)$; (b)~Sturm--Liouville coefficient $p(\xi)$ in the transformed variable.}
\label{fig:transform}
\end{figure}

\subsection{Obstacles to Closed-Form Expressions}
\label{sec:obstacles}

Our computational investigation reveals four principal obstacles to finding closed-form expressions for $A_k$ and $C_k$:

\begin{enumerate}
\item \textbf{Transcendental base flow.} The Blasius function $f(\eta)$ appears as a coefficient in the eigenvalue ODE. Since $f$ itself has no known closed form, any exact eigenvalue formula must either involve the transcendental constants of the Blasius solution (such as $f''(0)$) or bypass the need for explicit knowledge of $f$.

\item \textbf{Non-standard spectral asymptotics.} The eigenvalue spacings $\Delta_k$ approach a constant rather than growing (as for Bessel or Airy zeros) or remaining exactly constant (as for trigonometric eigenproblems). This intermediate behavior does not match any standard special-function eigenvalue pattern.

\item \textbf{Super-exponential norm growth.} The weighted norms grow super-exponentially (roughly $\|\phi_k\|_w^2 \sim 10^{1.8k}$), which is unusual for classical eigenvalue problems and suggests a non-standard asymptotic structure for the eigenfunctions at large $k$.

\item \textbf{Non-trivial orthogonality structure.} The weight function $f''(\eta)$ in the orthogonality relation is itself a transcendental function of $\eta$, coupling the norm computation to the full Blasius profile.
\end{enumerate}

%% ====================================================================
\section{Conclusion}
\label{sec:conclusion}
%% ====================================================================

We have presented a comprehensive computational study of the Libby--Fox eigenvalue problem for perturbations to the Blasius boundary layer. Our main findings are:

\begin{enumerate}
\item We computed 12 eigenvalues (Table~\ref{tab:eigenvalues}) and verified Brown's asymptotic formula with fitted coefficients $\alpha = 1.908$ and $\beta = -2.664$, achieving residuals below $0.06$ for $k \geq 6$.

\item The normalization constants $C_k$ decay super-exponentially, with $C_0 = 1.732$ and $C_{11} = 1.42 \times 10^{-10}$. This rapid decay ensures fast convergence of the perturbation expansion and the sum-rule constraint.

\item The sum-rule partial sums converge to 12-digit precision by $K = 7$, confirming the internal consistency of the computed eigenvalues and norms. The converged values provide reference benchmarks for future work.

\item The change of variable $\xi = f'(\eta)$ maps the eigenvalue problem to a bounded domain with a smooth Sturm--Liouville coefficient, offering a promising avenue for connecting to Heun-class equations.

\item We identified four principal obstacles to closed-form expressions: the transcendental nature of the Blasius profile, non-standard spectral asymptotics, super-exponential norm growth, and the transcendental weight function in the orthogonality relation.
\end{enumerate}

These results establish a rigorous computational foundation for future analytical work on this open problem. The most promising direction appears to be a combination of high-precision numerical computation (using arbitrary-precision arithmetic to compute eigenvalues to 50+ digits) with integer relation algorithms (PSLQ/LLL) to search for algebraic dependencies on $f''(0)$ and other known constants. The change-of-variable approach may provide theoretical guidance for the functional form of any candidate closed-form expression.

%% ====================================================================
\section{Limitations and Ethical Considerations}
\label{sec:limitations}
%% ====================================================================

\paragraph{Numerical precision.} Our eigenvalues are based on literature values refined to at most 4--5 significant digits. The shooting method faces challenges due to the exponential growth of non-eigenfunction solutions, which limits the achievable precision. Higher-precision results would require compound matrix methods or arbitrary-precision arithmetic.

\paragraph{Domain truncation.} All computations use a finite domain $\eta \in [0, 15]$. While the Blasius profile is effectively constant for $\eta > 8$, the truncation introduces systematic errors in the normalization constants, particularly for higher modes whose eigenfunctions have significant amplitude at larger $\eta$.

\paragraph{Orthogonality.} The computed orthogonality matrix shows non-negligible off-diagonal elements, indicating that numerical errors accumulate for higher-mode inner products. This is a known challenge for irregular Sturm--Liouville problems and does not affect the eigenvalue computation itself.

\paragraph{Scope.} This study focuses on the flat-plate Blasius case. Extensions to the Falkner--Skan family (wedge flows with pressure gradient) would require a separate analysis.

\paragraph{Ethical considerations.} This is a purely mathematical investigation with no direct societal implications. The computational methods used are standard and reproducible.

%% ====================================================================
\bibliographystyle{ACM-Reference-Format}
\bibliography{references}
%% ====================================================================

\end{document}
