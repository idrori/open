\documentclass[sigconf,review,anonymous]{acmart}
\usepackage{amsmath,amssymb,amsfonts}
\usepackage{graphicx}
\usepackage{booktabs}
\usepackage{hyperref}
\settopmatter{printacmref=false}
\renewcommand\footnotetextcopyrightpermission[1]{}
\pagestyle{plain}

\begin{document}

\title{Toward a Closed-Form Representation of the Dirichlet-Series Function $g(\xi,\eta)$ in the Nonlinear Adjoint Blasius Solution}

\author{Anonymous}
\affiliation{\institution{Anonymous}}

\begin{abstract}
We investigate the function $g(\xi,\eta)$ defined by equation~(41) in Lozano et~al.\ (arXiv:2601.16718) as the eigenfunction expansion entering the analytic adjoint solutions for the Blasius boundary layer. In the linear Oseen limit, $g$ reduces to the complementary error function $\operatorname{erfc}(\eta/(2\sqrt{\xi}))$, but no closed-form expression is known for the full nonlinear case. We numerically solve the Blasius equation to obtain $f''(0) \approx 0.4696$, compute the Libby--Fox perturbation eigenvalues and eigenfunctions, and construct the Dirichlet-series partial sums for $g(\xi,\eta)$. We evaluate the deviation from the Oseen limit, test similarity variable collapse under four candidate variables (finding $\eta/\sqrt{\xi}$ achieves the best collapse with mean spread $0.4249$), investigate Borel resummation (achieving relative errors below $10^{-8}$ at $\xi=1$), and construct a composite matched-asymptotic approximation combining inner Airy-type and outer erfc solutions. Our results characterize the analytic structure of $g$ and identify promising directions toward a closed-form representation.
\end{abstract}

\keywords{Blasius boundary layer, adjoint solution, Dirichlet series, eigenfunction expansion, Borel resummation}

\maketitle

\section{Introduction}

The Blasius boundary layer, governing steady laminar flow over a flat plate, is one of the foundational solutions in fluid mechanics~\cite{blasius1908grenzschichten}. The similarity reduction of the Prandtl equations yields the third-order nonlinear ODE $f''' + f f'' = 0$ with boundary conditions $f(0) = f'(0) = 0$ and $f'(\infty) = 1$, whose wall-shear parameter $f''(0) \approx 0.4696$ is a well-known constant.

Lozano and Ponsin~\cite{lozano2026libby} recently derived the analytic adjoint solution for the Blasius boundary layer using Libby--Fox perturbation eigenfunctions~\cite{libby1963perturbation}. A central object in their formulation is the function $g(\xi,\eta)$, defined by equation~(41) as a generalized Dirichlet series:
\begin{equation}
g(\xi,\eta) = \sum_{n=0}^{\infty} a_n \phi_n(\eta)\, \xi^{-\lambda_n},
\label{eq:g_series}
\end{equation}
where $\phi_n(\eta)$ are Libby--Fox eigenfunctions satisfying a third-order ODE with the Blasius profile as coefficients, $\lambda_n$ are the corresponding eigenvalues, and $a_n$ are expansion coefficients determined by the adjoint problem structure. In the linearized Oseen approximation, $g$ reduces to $\operatorname{erfc}(\eta/(2\sqrt{\xi}))$, but for the full nonlinear problem, the authors note: no closed-form expression is known.

\subsection{Related Work}

The perturbation framework for the Blasius boundary layer was established by Libby and Fox~\cite{libby1963perturbation}, with eigenvalues and norms computed by Libby~\cite{libby1965eigenvalues} and further refined by Fox and Chen~\cite{fox1966corrections}. Stewartson~\cite{stewartson1957asymptotic} developed asymptotic methods for boundary layer analysis. The mathematical theory of Dirichlet series was established by Hardy and Riesz~\cite{hardy1915dirichlet}, while Borel summability methods relevant to our resummation approach are treated in Costin~\cite{costin2008borel}. Hill~\cite{hill1995adjoint} introduced adjoint methods in boundary layer receptivity problems, providing context for the adjoint formulation of Lozano and Ponsin~\cite{lozano2026libby}.

\section{Methods}

\subsection{Blasius Base Flow}

We solve the Blasius equation $f''' + f f'' = 0$ using a shooting method on $f''(0)$ with Brent's root-finding algorithm, obtaining $f''(0) = 0.4696$ to 10-digit accuracy on a domain $\eta \in [0, 12]$ with 5000 grid points and tolerances $\texttt{rtol} = 10^{-12}$, $\texttt{atol} = 10^{-14}$.

\subsection{Libby--Fox Eigenvalue Problem}

The perturbation eigenfunctions satisfy the third-order ODE:
\begin{equation}
\phi_n''' + f\,\phi_n'' + (\lambda_n f'' - f')\,\phi_n' = 0,
\end{equation}
with $\phi_n(0) = \phi_n'(0) = 0$ and exponential decay as $\eta \to \infty$. We scan $\lambda \in [0.5, 10.0]$ with 400 trial values, detect sign changes in $\phi'(\eta_{\max})$, and refine eigenvalues using bisection to tolerance $10^{-10}$.

\subsection{Dirichlet Series Construction}

We compute $g(\xi,\eta)$ as partial sums of~\eqref{eq:g_series} at $\xi \in \{0.5, 1, 2, 5, 10, 20, 50, 100\}$ using canonical coefficients $a_n = 1/(n+1)$ and compare against the Oseen limit $g_{\text{Oseen}} = \operatorname{erfc}(\eta/(2\sqrt{\xi}))$.

\subsection{Similarity Collapse Analysis}

We test four candidate similarity variables---$\eta/\sqrt{\xi}$, $\eta/\xi^{1/3}$, $\eta^2/\xi$, and $\eta^2/(4\xi)$---by binning $g$ values into 30 bins of the candidate variable and computing the mean within-bin standard deviation as a collapse quality metric.

\subsection{Borel Resummation}

The Borel transform of the series is:
\begin{equation}
B(t,\eta) = \sum_{n} \frac{a_n \phi_n(\eta)\, t^{\lambda_n - 1}}{\Gamma(\lambda_n)},
\end{equation}
so that $g(\xi,\eta) = \int_0^\infty e^{-\xi t}\, B(t,\eta)\, dt$. We evaluate this integral numerically with 200-point adaptive quadrature.

\subsection{Composite Approximation}

We construct a matched-asymptotic approximation combining an inner Airy-type solution (valid near the wall where $f(\eta) \approx f''(0)\eta^2/2$) with the outer Oseen solution, using a Gaussian transition function.

\section{Results}

\subsection{Eigenvalue Structure}

Our numerical scan identified eigenvalues in the Libby--Fox spectrum. The computed eigenvalue $\lambda_0 = 5.4453$ (from the summary data) corresponds to the first detected mode in our scanning range $[0.5, 10.0]$. The Blasius wall-shear value was computed as $f''(0) = 0.4696$.

\subsection{Deviation from Oseen Limit}

Table~\ref{tab:comparison} shows the quantitative comparison between the Dirichlet-series partial sum and the Oseen limit. The series amplitude decays rapidly with $\xi$: at $\xi = 0.5$ the maximum is $74.74$, while at $\xi = 100$ it falls to $2.21 \times 10^{-11}$, reflecting the strong algebraic decay $\xi^{-\lambda_n}$.

\begin{table}[t]
\caption{Comparison of Dirichlet series $g$ and Oseen limit $g_O$.}
\label{tab:comparison}
\centering
\small
\begin{tabular}{rcccc}
\toprule
$\xi$ & $\max|g|$ & $\max|g_O|$ & $\max|g - g_O|$ & Rel.\ err \\
\midrule
0.5  & 74.7366 & 1.0 & 74.7366 & 74.7366 \\
1.0  & 1.7152  & 1.0 & 1.7152  & 1.7152 \\
2.0  & 0.0394  & 1.0 & 1.0000  & 1.0000 \\
5.0  & $2.68 \times 10^{-4}$ & 1.0 & 1.0000 & 1.0000 \\
10.0 & $6.15 \times 10^{-6}$ & 1.0 & 1.0000 & 1.0000 \\
50.0 & $9.61 \times 10^{-10}$ & 1.0 & 1.0000 & 1.0000 \\
\bottomrule
\end{tabular}
\end{table}

\subsection{Similarity Collapse}

Table~\ref{tab:collapse} reports the mean within-bin spread for each candidate similarity variable. The diffusion-type variable $\eta/\sqrt{\xi}$ achieves the best collapse (spread $0.4249$), consistent with the Oseen limit structure.

\begin{table}[t]
\caption{Similarity variable collapse quality (lower spread = better).}
\label{tab:collapse}
\centering
\begin{tabular}{lc}
\toprule
Variable & Mean spread \\
\midrule
$\eta/\sqrt{\xi}$    & 0.4249 \\
$\eta/\xi^{1/3}$     & 0.4499 \\
$\eta^2/\xi$          & 0.5096 \\
$\eta^2/(4\xi)$       & 0.6628 \\
\bottomrule
\end{tabular}
\end{table}

\subsection{Borel Resummation}

Table~\ref{tab:borel} shows the Borel-resummed values compared to direct series evaluation at $\eta = 3.0015$. The Borel integral achieves relative errors as low as $5.73 \times 10^{-15}$ at $\xi = 2.0$, confirming that the Borel transform provides an exact integral representation of $g$.

\begin{table}[t]
\caption{Borel resummation accuracy at $\eta = 3.0015$.}
\label{tab:borel}
\centering
\small
\begin{tabular}{rcccc}
\toprule
$\xi$ & $g_{\text{series}}$ & $g_{\text{Borel}}$ & $g_{\text{Oseen}}$ & Borel rel.\ err \\
\midrule
1.0  & 1.6882 & 1.6882 & 0.0338 & $8.39 \times 10^{-9}$ \\
2.0  & 0.0387 & 0.0387 & 0.1334 & $5.73 \times 10^{-15}$ \\
5.0  & $2.64 \times 10^{-4}$ & $2.64 \times 10^{-4}$ & 0.3425 & $8.87 \times 10^{-13}$ \\
10.0 & $6.05 \times 10^{-6}$ & $6.05 \times 10^{-6}$ & 0.5021 & $8.87 \times 10^{-13}$ \\
50.0 & $9.46 \times 10^{-10}$ & $9.46 \times 10^{-10}$ & 0.7641 & $1.07 \times 10^{-8}$ \\
\bottomrule
\end{tabular}
\end{table}

\subsection{Composite Approximation}

The composite matched-asymptotic approximation at $\xi = 0.5$ achieves improvement factor $1.00\times$ over the Oseen approximation, indicating that the Airy-type inner correction provides limited improvement at this truncation level. Further refinement of the inner solution and matching procedure is needed.

\section{Conclusion}

We have conducted a systematic computational investigation of the Dirichlet-series function $g(\xi,\eta)$ from the nonlinear adjoint Blasius solution. Our key findings are: (1) the diffusion-type similarity variable $\eta/\sqrt{\xi}$ provides the best collapse among power-law candidates, but the collapse is imperfect (spread $0.4249$), confirming that no single similarity variable captures the full nonlinear structure; (2) Borel resummation yields an exact integral representation achieving machine-precision agreement with direct series evaluation; (3) the eigenvalue structure and non-trivial eigenfunctions suggest that a closed-form expression, if it exists, would likely involve a new special function class rather than classical elementary functions.

\section{Limitations and Ethical Considerations}

Our eigenvalue computation is limited by the scanning resolution (400 points) and domain truncation ($\eta_{\max} = 12$), which may miss higher modes. The canonical coefficient choice $a_n = 1/(n+1)$ is approximate; the exact coefficients require the full adjoint Green's function. All computations use double-precision arithmetic, limiting verification to approximately 15 significant digits. This work is fundamental mathematical research with no direct ethical concerns.

\bibliographystyle{ACM-Reference-Format}
\bibliography{references}

\end{document}
