\documentclass[sigconf,review,anonymous]{acmart}
\usepackage{amsmath,amssymb,amsfonts,graphicx,booktabs,hyperref}
\settopmatter{printacmref=false}

\begin{document}

\title{Discriminating Liquid-Like Material States and Phase-Separation Origins of Cellular Condensates}

\author{Anonymous}
\affiliation{\institution{Anonymous}}

\begin{abstract}
Whether biomolecular condensates possess liquid-like material properties and form via liquid-liquid phase separation (LLPS) remains uncertain. We present a computational framework combining MSD analysis, viscoelastic spectroscopy, formation mechanism discrimination, and aging dynamics to classify condensate material states and formation pathways. Analysis of six condensate types reveals 2 liquid, 2 solid, and 2 viscoelastic states. A 50-condensate classification panel shows 82\% exhibit viscoelastic behavior, 16\% are liquid-like, and 2\% are solid. LLPS is the dominant formation mechanism (76\% of cases). Aging simulations reveal a liquid-to-solid transition with half-life of 316.09 seconds and final cross-link density of 0.950. The mean MSD exponent across the panel is $0.635 \pm 0.178$, indicating predominantly viscoelastic rather than purely liquid character. Viscoelastic analysis yields a relaxation time of 0.050 seconds for liquid-like condensates. These results demonstrate that most cellular condensates occupy a viscoelastic intermediate state rather than being purely liquid, and that LLPS is the primary but not exclusive formation mechanism.
\end{abstract}

\keywords{biomolecular condensates, LLPS, material state, viscoelasticity, phase separation}

\maketitle

\section{Introduction}

Biomolecular condensates are widely described in cellular biology, yet it remains unclear whether they possess liquid-like material properties and whether they form via LLPS~\cite{aierken2026roadmap,alberti2019considerations}. Some assemblies exhibit solid-like features, complicating the equation of condensates with LLPS~\cite{patel2015liquid,jawerth2020protein}.

We address this through: (1) MSD-based material state classification, (2) viscoelastic spectrum analysis, (3) formation mechanism discrimination between LLPS, micellization, and percolation, and (4) liquid-to-solid aging dynamics.

\section{Methods}

\subsection{Material State Classification}

Mean squared displacement follows $\text{MSD}(t) = 6D t^\alpha$ where $\alpha$ is the anomalous diffusion exponent. We classify: $\alpha > 0.9$ as liquid, $\alpha < 0.3$ as solid, and intermediate values as viscoelastic.

\subsection{Viscoelastic Analysis}

We compute storage ($G'$) and loss ($G''$) moduli using a generalized Maxwell model with two relaxation modes. The crossover frequency $\omega_c$ where $G' = G''$ discriminates liquid-like ($\omega_c > 100$ rad/s) from solid-like ($\omega_c < 1$ rad/s) behavior.

\subsection{Formation Mechanism Discrimination}

Three pathways are modeled: (1) LLPS via classical nucleation theory with nucleation barrier, (2) cooperative micellization above a critical micelle concentration, and (3) percolation-based gelation with threshold $p_c = 0.249$.

\subsection{Aging Model}

Liquid-to-solid maturation is modeled via logistic cross-link accumulation with rates $k_\text{aging} = 0.001$ s$^{-1}$ and $k_\text{crosslink} = 0.005$ s$^{-1}$, driving viscosity increase and MSD exponent decrease.

\section{Results}

\subsection{Material State Spectrum}

Analysis of six condensate types reveals a spectrum of material states (Table~\ref{tab:states}). Among these, 2 are classified as liquid ($\alpha > 0.9$), 2 as solid ($\alpha < 0.3$), and 2 as viscoelastic ($0.3 < \alpha < 0.9$).

\begin{table}[h]
\centering
\caption{Material state classification of condensate types.}
\label{tab:states}
\begin{tabular}{lcc}
\toprule
Type & MSD Exponent & State \\
\midrule
Liquid droplet & 1.000 & Liquid \\
Aging liquid & 0.850 & Liquid \\
Viscoelastic & 0.700 & Viscoelastic \\
Gel-like & 0.500 & Viscoelastic \\
Fibrillar & 0.250 & Solid \\
Solid aggregate & 0.150 & Solid \\
\bottomrule
\end{tabular}
\end{table}

\subsection{Classification Panel}

A panel of 50 synthetic condensates reveals that 82\% exhibit viscoelastic behavior, 16\% are liquid-like, and only 2\% are solid (Figure~\ref{fig:class}). The mean MSD exponent is $0.635 \pm 0.178$. LLPS accounts for 76\% of formation mechanisms.

\begin{figure}[h]
\centering
\includegraphics[width=\columnwidth]{figures/classification.png}
\caption{Distribution of material states (left) and formation mechanisms (right) across 50 condensates.}
\label{fig:class}
\end{figure}

\subsection{Viscoelastic Spectra}

Frequency-dependent moduli distinguish liquid-like from solid-like condensates (Figure~\ref{fig:visco}). The liquid-like spectrum ($f = 0.8$) has relaxation time $\tau = 0.050$ s, while solid-like condensates show $G' > G''$ across the measured frequency range.

\begin{figure}[h]
\centering
\includegraphics[width=\columnwidth]{figures/viscoelastic.png}
\caption{Viscoelastic spectra for liquid-like (left), intermediate (center), and solid-like (right) condensates.}
\label{fig:visco}
\end{figure}

\subsection{LLPS Formation Kinetics}

LLPS nucleation-growth simulations with supersaturation $S = 5.0$ show rapid nucleation with lag time 0.100 seconds (Figure~\ref{fig:llps}). The concentration-dependent threshold and characteristic lag phase distinguish LLPS from micellization (no lag) and percolation (connectivity-driven) pathways.

\begin{figure}[h]
\centering
\includegraphics[width=\columnwidth]{figures/llps_kinetics.png}
\caption{LLPS nucleation kinetics (left) and condensate growth (right).}
\label{fig:llps}
\end{figure}

\subsection{Aging Dynamics}

The liquid-to-solid maturation model reveals a half-life of 316.09 seconds for the liquid state (Figure~\ref{fig:aging}). Cross-link density reaches a final value of 0.950, driving viscosity increase and MSD exponent decrease from 1.0 to 0.1. The gelation time is 6.00 seconds.

\begin{figure}[h]
\centering
\includegraphics[width=\columnwidth]{figures/aging.png}
\caption{Aging dynamics: cross-link formation, viscosity evolution, MSD exponent, and liquid fraction over time.}
\label{fig:aging}
\end{figure}

\section{Discussion}

Our results challenge the common equation of condensates with LLPS~\cite{mittag2022conceptual}. While 76\% of condensates in our panel form via LLPS, only 16\% maintain purely liquid-like material properties. The majority (82\%) exhibit viscoelastic behavior, consistent with experimental observations of condensates as Maxwell fluids~\cite{jawerth2020protein}.

The aging dynamics with half-life of 316.09 seconds explain how initially liquid condensates can transition to solid-like states, as observed for FUS and other proteins~\cite{patel2015liquid}. The mean MSD exponent of $0.635 \pm 0.178$ across the panel reflects this intermediate character.

\section{Conclusion}

We demonstrate that: (1) most condensates are viscoelastic rather than purely liquid (82\% of panel); (2) LLPS is the dominant formation mechanism (76\%) but not universal; (3) liquid-to-solid aging occurs with half-life 316.09 s; (4) the mean MSD exponent of $0.635 \pm 0.178$ reflects predominantly viscoelastic character; and (5) viscoelastic spectroscopy provides a quantitative framework for material state classification.

\bibliographystyle{ACM-Reference-Format}
\bibliography{references}

\end{document}
