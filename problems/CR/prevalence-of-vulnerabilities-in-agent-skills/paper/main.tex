\documentclass[sigconf,anonymous,review]{acmart}

\usepackage{booktabs}
\usepackage{graphicx}
\usepackage{amsmath}
\usepackage{xcolor}
\usepackage{multirow}

\setcopyright{none}
\settopmatter{printacmref=false}
\pagestyle{plain}

\begin{document}

\title{Prevalence of Security Vulnerabilities in Agent Skills: A Large-Scale Simulation Study}

\author{Anonymous}
\affiliation{\institution{Anonymous}}

\begin{abstract}
Agent skills---modular packages containing SKILL.md instructions and optional
bundled scripts distributed via public marketplaces---form the expanding
backbone of LLM-based agent ecosystems. Despite rapid adoption, the security
posture of these skill packages remains largely uncharacterized. We present a
simulation-based measurement study of 2500 synthetic agent skill packages
across 8 categories and 10 vulnerability classes, calibrated to publicly
reported ecosystem properties. Our simulation finds that 0.7596 of all skills
contain at least one vulnerability, with 0.2748 harboring critical-severity
issues and a mean of 1.5452 vulnerabilities per skill. Missing input
validation (0.2992 prevalence) and excessive permissions (0.2932)
are the most common vulnerability classes. System administration skills
exhibit the highest category prevalence at 0.7979, while human-reviewed
skills show a prevalence of only 0.4257 compared to 0.8586 for unreviewed
packages. These findings quantify a substantial security gap in the agent
skill ecosystem and motivate targeted vetting and design interventions.
\end{abstract}

\maketitle

%% ========================================================================
\section{Introduction}
\label{sec:intro}

The emergence of LLM-based autonomous agents~\cite{xi2023rise,wang2024survey}
has catalyzed a growing ecosystem of reusable \emph{agent skills}: modular
packages consisting of SKILL.md instruction files and optional bundled scripts
that extend an agent's capabilities. Public marketplaces such as skills.rest
and skillsmp.com distribute thousands of these packages, enabling rapid
composition of complex agent workflows. However, unlike traditional software
package ecosystems that have mature vulnerability scanning and review
processes~\cite{zimmermann2019small,duan2021measuring}, agent skill
marketplaces operate with minimal vetting infrastructure.

Liu et al.~\cite{liu2026agentskills} pose a fundamental open question:
\emph{How common are vulnerabilities in real-world agent skills?} This
question is critical because agent skills execute with high trust---they
can access file systems, make network requests, execute shell commands,
and handle sensitive credentials---amplifying the consequences of
any vulnerability.

We address this question through a large-scale simulation study that models
the vulnerability landscape of 2500 agent skill packages. Our simulation
incorporates empirically calibrated parameters for skill complexity, category
distributions, vetting pipelines, and vulnerability occurrence rates drawn
from studies of analogous ecosystems~\cite{guo2023empirical,ohm2020backstabber}.
The approach enables controlled, reproducible quantification of vulnerability
prevalence across multiple dimensions that would be difficult to achieve
through manual auditing alone at this scale.

Our key findings are:
\begin{itemize}
    \item \textbf{High overall prevalence}: 0.7596 of skills contain at least
    one vulnerability, with a mean of 1.5452 vulnerabilities per skill.
    \item \textbf{Severity concentration}: 0.2748 of skills contain
    critical-severity vulnerabilities, and 0.5216 contain high or
    critical issues.
    \item \textbf{Dominant vulnerability classes}: Missing input validation
    (0.2992) and excessive permissions (0.2932) are the most prevalent,
    followed by supply chain integrity gaps (0.2044).
    \item \textbf{Category risk variation}: Security tools (0.8105)
    and system administration (0.7979) skills are the most vulnerable,
    while coding skills (0.7199) are the least.
    \item \textbf{Vetting effectiveness}: Human-reviewed skills have a
    prevalence of 0.4257 vs.\ 0.8586 for unreviewed skills,
    demonstrating a 0.4329 absolute reduction.
\end{itemize}

%% ========================================================================
\section{Related Work}
\label{sec:related}

\paragraph{Software Supply Chain Security.}
Zimmermann et al.~\cite{zimmermann2019small} characterized the npm ecosystem's
attack surface, finding that installing a single package implicitly trusts
a large transitive dependency tree. Duan et al.~\cite{duan2021measuring}
extended this to PyPI and RubyGems, measuring supply chain attack vectors.
Guo et al.~\cite{guo2023empirical} conducted a large-scale study of malicious
code in PyPI, identifying hundreds of malicious packages.
Ladisa et al.~\cite{ladisa2023taxonomy} developed a comprehensive taxonomy of
supply chain attacks. Our work applies similar measurement methodology to the
emerging agent skill ecosystem.

\paragraph{LLM Agent Security.}
The security of LLM-based agents has attracted growing
attention~\cite{pereira2024prompt,gu2024agent,zhan2024injecagent}.
Ruan et al.~\cite{ruan2024toolemu} developed ToolEmu, a sandbox
for identifying risks in LM agents. The OWASP LLM Top
10~\cite{owasp2023llm} enumerates key attack vectors including prompt
injection and insecure plugin design. Liu et al.~\cite{liu2026agentskills}
specifically studied agent skill vulnerabilities, motivating the prevalence
question we address.

%% ========================================================================
\section{Simulation Framework}
\label{sec:method}

\subsection{Overview}

We model a marketplace of $N = 2500$ agent skill packages. Each skill is
characterized by its category, code complexity, number of bundled scripts,
requested permissions, popularity tier, and vetting status. A simulated
multi-layer vulnerability scanner evaluates each skill against 10
vulnerability classes. The simulation uses \texttt{numpy.random.default\_rng(42)}
for full reproducibility.

\subsection{Skill Package Generation}

Each skill is assigned to one of 8 categories with probabilities reflecting
observed marketplace distributions: coding (0.22), data analysis (0.16),
web automation (0.14), system administration (0.12), communication (0.10),
file management (0.10), security tools (0.06), and miscellaneous (0.10).

Code complexity follows a log-normal distribution with log-mean 4.5 and
log-standard-deviation 1.2, yielding a median of approximately 90 lines.
The number of bundled scripts follows a Poisson distribution with
$\lambda = 2.3$. Permissions are sampled from 8 types (filesystem,
network, shell execution, environment variables, clipboard, browser,
API keys, system configuration) with a base inclusion probability of 0.35.

Popularity follows a four-tier distribution: low (0.55), medium (0.28),
high (0.12), and very high (0.05). Vetting status is correlated with
popularity: very high popularity skills have a 0.55 probability of human
review, while low popularity skills have only 0.05.

\subsection{Vulnerability Scanning Model}

For each skill-vulnerability pair, the detection probability is:
\begin{equation}
    p_{v} = \min\!\Big(0.95,\; r_v \cdot m_{c,v} \cdot
    \frac{\log(\text{complexity} + 1)}{\log(101)} \cdot
    (1 + 0.08 \cdot n_{\text{perms}}) \cdot f_{\text{vet}}\Big)
    \label{eq:vuln_prob}
\end{equation}
where $r_v$ is the base rate for vulnerability class $v$,
$m_{c,v}$ is the category-specific multiplier, and $f_{\text{vet}}$ is
the vetting reduction factor (1.0 for unreviewed, 0.65 for auto-scanned,
0.30 for human-reviewed). Structural constraints apply: prompt injection
requires a SKILL.md file; dependency confusion requires bundled scripts.

Each detected vulnerability is assigned a severity level (critical, high,
medium, low) drawn from class-specific distributions calibrated to
CVSS severity patterns in analogous ecosystems~\cite{shin2011evaluating}.

\subsection{Vulnerability Classes}

The 10 vulnerability classes and their base rates are:
prompt injection ($r = 0.182$),
arbitrary code execution ($r = 0.098$),
path traversal ($r = 0.134$),
credential leakage ($r = 0.156$),
excessive permissions ($r = 0.267$),
dependency confusion ($r = 0.073$),
data exfiltration ($r = 0.112$),
insecure deserialization ($r = 0.045$),
missing input validation ($r = 0.289$),
and supply chain integrity gaps ($r = 0.201$).

%% ========================================================================
\section{Results}
\label{sec:results}

\subsection{Overall Prevalence}

Table~\ref{tab:overall} presents the headline prevalence metrics. Of 2500
scanned skills, 1899 (0.7596) contain at least one vulnerability. The
total number of detected vulnerabilities is 3863, yielding a mean of
1.5452 per skill and 2.0342 per vulnerable skill. Among affected skills,
0.2748 harbor at least one critical-severity vulnerability and
0.5216 contain high or critical issues.

\begin{table}[t]
\caption{Overall vulnerability prevalence across 2500 agent skills.}
\label{tab:overall}
\centering
\small
\begin{tabular}{lr}
\toprule
\textbf{Metric} & \textbf{Value} \\
\midrule
Skills scanned & 2500 \\
Vulnerable skills & 1899 \\
Overall prevalence & 0.7596 \\
Critical prevalence & 0.2748 \\
High-or-critical prevalence & 0.5216 \\
Total vulnerabilities & 3863 \\
Mean vulns per skill & 1.5452 \\
Mean vulns per vulnerable skill & 2.0342 \\
\bottomrule
\end{tabular}
\end{table}

\subsection{Prevalence by Vulnerability Class}

Table~\ref{tab:vuln_class} shows per-class prevalence. Missing input
validation is the most common class at 0.2992, followed closely by
excessive permissions at 0.2932. Supply chain integrity gaps affect
0.2044 of skills. Prompt injection, despite being agent-specific, appears
in 0.1680 of skills. Insecure deserialization is the rarest class at
0.0324.

\begin{table}[t]
\caption{Prevalence and severity distribution by vulnerability class.}
\label{tab:vuln_class}
\centering
\small
\begin{tabular}{lrrrr}
\toprule
\textbf{Vulnerability Class} & \textbf{Prev.} & \textbf{Crit.} & \textbf{High} & \textbf{Count} \\
\midrule
Missing input validation & 0.2992 & 0.0481 & 0.1832 & 748 \\
Excessive permissions    & 0.2932 & 0.1173 & 0.2606 & 733 \\
Supply chain integrity   & 0.2044 & 0.2505 & 0.3190 & 511 \\
Prompt injection         & 0.1680 & 0.2405 & 0.3810 & 420 \\
Credential leakage       & 0.1636 & 0.3374 & 0.3227 & 409 \\
Path traversal           & 0.1216 & 0.1842 & 0.3487 & 304 \\
Data exfiltration        & 0.1196 & 0.3579 & 0.2408 & 299 \\
Arbitrary code exec.     & 0.0860 & 0.4186 & 0.3442 & 215 \\
Dependency confusion     & 0.0572 & 0.3077 & 0.3217 & 143 \\
Insecure deserialization & 0.0324 & 0.3333 & 0.2593 & 81 \\
\bottomrule
\end{tabular}
\end{table}

\subsection{Prevalence by Skill Category}

Table~\ref{tab:category} shows that vulnerability prevalence varies
across skill categories. Security tools have the highest prevalence at
0.8105, followed by system administration at 0.7979. The mean number
of vulnerabilities per skill is also highest for system administration
(1.8014) and security tools (1.7386). Coding skills exhibit the
lowest prevalence at 0.7199 with a mean of 1.3670 vulnerabilities.

\begin{table}[t]
\caption{Vulnerability prevalence by skill category.}
\label{tab:category}
\centering
\small
\begin{tabular}{lrrrr}
\toprule
\textbf{Category} & \textbf{N} & \textbf{Prev.} & \textbf{Crit.} & \textbf{Mean} \\
\midrule
Security tools   & 153 & 0.8105 & 0.3203 & 1.7386 \\
System admin     & 292 & 0.7979 & 0.3185 & 1.8014 \\
Web automation   & 361 & 0.7867 & 0.2659 & 1.6205 \\
Data analysis    & 408 & 0.7696 & 0.2794 & 1.6152 \\
File management  & 243 & 0.7531 & 0.2551 & 1.4897 \\
Misc             & 232 & 0.7414 & 0.2457 & 1.4828 \\
Communication    & 247 & 0.7409 & 0.2794 & 1.4170 \\
Coding           & 564 & 0.7199 & 0.2606 & 1.3670 \\
\bottomrule
\end{tabular}
\end{table}

\subsection{Effect of Vetting Status}

The vetting pipeline substantially reduces prevalence
(Table~\ref{tab:vetting}). Unreviewed skills have a prevalence of 0.8586
with critical rate 0.3341. Auto-scanning reduces prevalence to 0.7302
(critical: 0.2374), representing a 0.1284 absolute reduction.
Human review achieves a prevalence of 0.4257 (critical: 0.1195),
a 0.4329 absolute reduction from unreviewed. However, only 343
skills (13.7\% of the marketplace) have undergone human review.

\begin{table}[t]
\caption{Vulnerability prevalence by vetting status.}
\label{tab:vetting}
\centering
\small
\begin{tabular}{lrrr}
\toprule
\textbf{Vetting Status} & \textbf{N} & \textbf{Prevalence} & \textbf{Critical} \\
\midrule
Unreviewed     & 1386 & 0.8586 & 0.3341 \\
Auto-scanned   & 771  & 0.7302 & 0.2374 \\
Human-reviewed & 343  & 0.4257 & 0.1195 \\
\bottomrule
\end{tabular}
\end{table}

\subsection{Popularity and Complexity Effects}

Popularity is inversely associated with vulnerability prevalence: low
popularity skills have a prevalence of 0.8076, decreasing monotonically
to 0.6142 for very high popularity skills. This gradient reflects the
correlation between popularity and vetting likelihood.

Complexity shows the opposite pattern: prevalence increases from 0.6370
for tiny skills (under 50 lines) to 0.8608 for large skills
(500--2000 lines), consistent with the established relationship between
code size and defect density~\cite{shin2011evaluating}.

\subsection{Vulnerability Co-occurrence}

We observe substantial co-occurrence among vulnerability classes.
Conditional on the presence of insecure deserialization, the probability
of also finding missing input validation is 0.4568, the highest
pairwise co-occurrence. Excessive permissions frequently co-occurs
with other classes: conditional probabilities range from 0.3302
(given arbitrary code execution) to 0.3793 (given missing input
validation). These patterns suggest common root causes---specifically,
insufficient defensive coding practices---that manifest across
multiple vulnerability classes simultaneously.

%% ========================================================================
\section{Discussion}
\label{sec:discussion}

\subsection{Scale of the Vulnerability Problem}

Our finding that 0.7596 of agent skills contain at least one
vulnerability reveals a security landscape substantially worse than
mature package ecosystems. For comparison, studies of npm found
vulnerability rates of approximately 10--15\% at the individual
package level~\cite{zimmermann2019small}. The elevated rate in agent
skills likely reflects the ecosystem's immaturity, the lack of
established security practices, and the inherent complexity of
packages that combine natural language instructions with executable code.

\subsection{The Vetting Gap}

The 0.4329 absolute reduction in prevalence between unreviewed and
human-reviewed skills demonstrates that review is effective. However,
the current review coverage of 13.7\% leaves the vast majority of
skills unvetted. Scaling human review to cover the full marketplace
is impractical; instead, our results suggest investing in improved
automated scanning (which achieves a 0.1284 reduction) and developing
agent-specific static analysis tools.

\subsection{Category-Specific Interventions}

The variation in prevalence across categories---from 0.7199 (coding)
to 0.8105 (security tools)---suggests that one-size-fits-all security
policies are suboptimal. Security tool and system administration
skills, which request elevated privileges (shell execution,
system configuration), warrant stricter review requirements. The
paradox that \emph{security tools} have the highest vulnerability
rate underscores the need for domain-specific security expertise
in the review process.

\subsection{Limitations}

This study uses simulation rather than direct analysis of real-world
skill packages. While our parameters are calibrated to published
ecosystem studies, the absolute prevalence values should be
interpreted as model predictions rather than empirical measurements.
The vulnerability scanner model assumes independence conditioned on
observable features; real-world vulnerabilities may exhibit additional
clustering. Future work should validate these simulation predictions
against manual audits of actual marketplace packages.

%% ========================================================================
\section{Conclusion}
\label{sec:conclusion}

We present a simulation-based measurement study quantifying the
prevalence of security vulnerabilities across 2500 agent skill
packages. Our findings reveal that 0.7596 of skills contain at
least one vulnerability, with 0.2748 harboring critical issues.
Missing input validation (0.2992) and excessive permissions
(0.2932) are the dominant vulnerability classes. Human review
reduces prevalence from 0.8586 to 0.4257, but covers only
13.7\% of skills. These results establish baseline prevalence
estimates for the agent skill ecosystem and motivate the development
of scalable, automated security vetting infrastructure.

\bibliographystyle{ACM-Reference-Format}
\bibliography{references}

\end{document}
