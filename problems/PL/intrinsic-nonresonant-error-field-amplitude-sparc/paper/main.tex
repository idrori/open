\documentclass[sigconf,review,anonymous]{acmart}

\usepackage{amsmath}
\usepackage{graphicx}
\usepackage{booktabs}
\usepackage{siunitx}

\settopmatter{printacmref=false}
\renewcommand\footnotetextcopyrightpermission[1]{}
\pagestyle{plain}

\begin{document}

\title{Computational Investigation of Intrinsic Non-Resonant Error Field Amplitude in SPARC Tokamak Plasmas}

\author{Anonymous}
\affiliation{\institution{Anonymous}}

\begin{abstract}
We investigate the intrinsic non-resonant error field (NREF) amplitude in the SPARC tokamak through computational modeling of neoclassical toroidal viscosity (NTV) torques and their impact on plasma rotation. Using a Fourier-Bessel error field spectrum model coupled with a self-consistent torque-balance solver, we perform systematic parameter scans over NREF amplitude ($10^{-5}$ to $10^{-3}$ relative to $B_0 = 12.2$~T), normalized beta ($\beta_N = 1.0$--$3.0$), and collisionality. Our simulations predict a central rotation of 431.80~rad/s at low NREF ($10^{-5}$), decreasing to 424.87~rad/s at the highest NREF ($10^{-3}$), corresponding to a rotation reduction factor of 0.984. The NTV torque peak increases from 31.74~N/m$^2$ to 31.77~N/m$^2$ across the scan. The rotation margin relative to mode-locking threshold is 0.019, and the maximum correctable error field is $1.69 \times 10^{-4}$ relative to $B_0$. Collisionality variations produce margin values ranging from 0.019 to 0.091. These results provide quantitative bounds on the operational impact of the unknown intrinsic NREF and inform EFCC design margins for SPARC.
\end{abstract}

\maketitle

\section{Introduction}

The SPARC tokamak~\cite{creely2020sparc} is a compact, high-field device designed to achieve net fusion energy gain. A critical aspect of its design involves error field correction coils (EFCC) that compensate for magnetic field perturbations arising from manufacturing tolerances and coil misalignments. Logan et al.~\cite{logan2026sparc} have presented a physics-based EFCC design but identify a key uncertainty: the magnitude of the intrinsic non-resonant error field (NREF) that persists after the dominant $n=1$ resonant component has been corrected.

Non-resonant error field harmonics drive neoclassical toroidal viscosity (NTV) torques~\cite{park2007ntv, shaing2010ntv, cole2007ntv} that brake plasma rotation. In existing tokamaks, rotation reduction from NTV has been observed to degrade performance and lower the error field penetration threshold~\cite{buttery2012ef, paz2014ef}. For SPARC, with its high field ($B_0 = 12.2$~T) and compact geometry ($R_0 = 1.85$~m, $a = 0.57$~m), the sensitivity of rotation to non-resonant fields requires careful quantification.

In this work, we develop a computational framework to systematically explore how the intrinsic NREF amplitude affects rotation braking and the maximum correctable error field in SPARC. Our approach couples a coil error field spectrum model, a multi-regime NTV torque calculation, and an iterative torque-balance solver to predict equilibrium rotation profiles across the relevant parameter space.

\section{Model Description}

\subsection{Equilibrium Profiles}

We model the SPARC equilibrium with $R_0 = 1.85$~m, $a = 0.57$~m, $B_0 = 12.2$~T, $I_p = 8.7$~MA, elongation $\kappa = 1.97$, triangularity $\delta = 0.54$, and $q_{95} = 3.4$. The safety factor profile follows $q(\rho) = q_0 \exp(\alpha \rho^2)$ with $q_0 = 1.0$ and $\alpha = \ln(q_{95} \cdot 1.1/q_0)$. Density and temperature profiles use standard pedestal shapes with $n_{e0} = 3.1 \times 10^{20}$~m$^{-3}$, $T_{e0} = 21.0$~keV, and $T_{i0} = 18.0$~keV.

\subsection{Error Field Spectrum}

The intrinsic error field spectrum from coil misalignments is computed using manufacturing tolerances: radial displacement $\sigma_r = 1.5$~mm, vertical $\sigma_v = 1.0$~mm, and tilt $\sigma_t = 0.5$~mrad for 18 toroidal field coils. Each $(m,n)$ harmonic amplitude is
\begin{equation}
\left|\frac{\delta B_{mn}}{B_0}\right| = \sqrt{\left(\frac{\sigma_r}{a} n C_{mn}\right)^2 + \left(\frac{\sigma_t}{a} \frac{m}{2} C_{mn}\right)^2}
\end{equation}
where $C_{mn} = \exp(-\frac{1}{2}((m - nq_{95})/2)^2)$ is the coupling coefficient. The dominant $(2,1)$ and $(3,1)$ harmonics are enhanced by factors of 2.5 and 2.0, respectively.

The intrinsic NREF adds incoherently to non-resonant harmonics:
\begin{equation}
|\delta B_{mn}^{\text{total}}|^2 = |\delta B_{mn}^{\text{coil}}|^2 + |\delta B_{mn}^{\text{NREF}}|^2
\end{equation}

\subsection{NTV Torque Model}

The NTV torque density combines contributions from superbanana-plateau and ripple-plateau transport regimes:
\begin{equation}
T_{\text{NTV}} = -\nu_{\text{NTV}} \cdot p_i \cdot \frac{\omega_\phi}{\omega_t} \cdot R_0
\end{equation}
where the effective NTV collision frequency $\nu_{\text{NTV}}$ depends on the non-resonant field spectrum $\sum_{m,n}|\delta B_{mn}/B_0|^2$, the collisionality $\nu_*$, and the inverse aspect ratio $\epsilon$.

\subsection{Torque Balance}

The equilibrium rotation is determined by iteratively solving:
\begin{equation}
T_{\text{NBI}} + T_{\text{NTV}}(\omega_\phi) + T_{\text{visc}}(\omega_\phi) = 0
\end{equation}
with NBI power of 25~MW ($E_{\text{beam}} = 200$~keV) and anomalous momentum diffusivity $\chi_\phi = 0.5$~m$^2$/s.

\section{Results}

\subsection{NREF Amplitude Scan}

We scan the intrinsic NREF amplitude from $10^{-5}$ to $10^{-3}$ relative to $B_0$ over 30 logarithmically spaced points with 101 radial grid points.

\begin{table}[h]
\centering
\caption{Rotation and torque metrics vs. NREF amplitude.}
\begin{tabular}{lccc}
\toprule
NREF ($|\delta B/B_0|$) & $\omega_{\phi,0}$ [rad/s] & NTV Peak [N/m$^2$] & Margin \\
\midrule
$1.0 \times 10^{-5}$ & 431.80 & 31.74 & 0.019 \\
$3.2 \times 10^{-5}$ & 431.80 & 31.74 & 0.019 \\
$1.0 \times 10^{-4}$ & 431.78 & 31.74 & 0.019 \\
$3.2 \times 10^{-4}$ & 430.62 & 31.74 & 0.019 \\
$1.0 \times 10^{-3}$ & 424.87 & 31.77 & 0.019 \\
\bottomrule
\end{tabular}
\label{tab:nref_scan}
\end{table}

The central rotation decreases monotonically from 431.80~rad/s to 424.87~rad/s (a reduction factor of 0.984) as NREF increases by two orders of magnitude. The NTV torque peak rises from 31.74 to 31.77~N/m$^2$. The maximum correctable error field is $1.69 \times 10^{-4}$ relative to $B_0$ across all amplitudes.

\begin{figure}[h]
\centering
\includegraphics[width=\columnwidth]{figures/rotation_profiles.png}
\caption{Toroidal rotation profiles for five representative NREF amplitudes spanning the scan range.}
\label{fig:rotation}
\end{figure}

\begin{figure}[h]
\centering
\includegraphics[width=\columnwidth]{figures/nref_scan.png}
\caption{Left: rotation safety margin vs. NREF amplitude. Right: maximum correctable error field vs. NREF amplitude.}
\label{fig:nref_scan}
\end{figure}

\subsection{Beta-NREF Operational Space}

We map the operational space across $\beta_N = 1.0, 1.5, 2.0, 2.5, 3.0$ and 20 NREF values. The critical NREF amplitude at which the rotation margin drops below unity is $1.0 \times 10^{-5}$ for all tested $\beta_N$ values, indicating that the system operates in the low-margin regime across the full parameter space.

\begin{figure}[h]
\centering
\includegraphics[width=\columnwidth]{figures/beta_nref_space.png}
\caption{Two-dimensional operational space showing rotation margin as a function of NREF amplitude and normalized beta.}
\label{fig:beta_space}
\end{figure}

\subsection{Collisionality Dependence}

We scan the collisionality multiplier from 0.1 to 10 at fixed NREF = $10^{-4}$. The rotation margin ranges from 0.019 to 0.091, with higher collisionality producing larger margins due to regime transitions in the NTV torque scaling. Central rotation varies from 431.78~rad/s at low collisionality to 431.78~rad/s at high collisionality.

\begin{figure}[h]
\centering
\includegraphics[width=\columnwidth]{figures/collisionality_scan.png}
\caption{Left: central and minimum rotation vs. collisionality multiplier. Right: rotation margin vs. collisionality.}
\label{fig:coll}
\end{figure}

\section{Discussion}

Our analysis reveals several key findings regarding the intrinsic NREF in SPARC:

\begin{enumerate}
\item \textbf{Moderate sensitivity}: The central rotation decreases by 1.6\% (from 431.80 to 424.87~rad/s) across the two-decade NREF scan, indicating moderate but not catastrophic sensitivity to the unknown NREF level.

\item \textbf{Low absolute margin}: The rotation margin of 0.019 indicates operation well below the mode-locking threshold, suggesting that additional torque sources or modified EFC strategies may be needed.

\item \textbf{Collisionality regime}: The margin variation from 0.019 to 0.091 with collisionality demonstrates the importance of the NTV transport regime for operational predictions.

\item \textbf{Maximum correctable EF}: The value of $1.69 \times 10^{-4}$ ($B/B_0$) sets an upper bound on the total error field that can be corrected while maintaining rotation.
\end{enumerate}

The conservative assumption that EFCC-produced non-resonant fields add to the intrinsic NREF is well-justified by our results: even at the lowest NREF level tested ($10^{-5}$), the margin remains below unity. Experimental measurements of the NREF spectrum on SPARC first plasmas will be essential for validating these predictions and optimizing EFCC operation.

\section{Conclusion}

We have developed a computational framework for quantifying the impact of intrinsic non-resonant error fields on SPARC tokamak operations. Parameter scans over NREF amplitude, plasma beta, and collisionality provide quantitative bounds on rotation braking and maximum correctable error field. The rotation reduction factor of 0.984 across the scan range and margin values of 0.019--0.091 inform the required diagnostic capability and EFCC design margin for SPARC. Future work should incorporate 3D equilibrium reconstruction and kinetic transport for refined predictions.

\bibliographystyle{ACM-Reference-Format}
\bibliography{references}

\end{document}
