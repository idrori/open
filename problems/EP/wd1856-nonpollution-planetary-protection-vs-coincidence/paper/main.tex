\documentclass[sigconf,review,anonymous]{acmart}
\settopmatter{printacmref=false}
\renewcommand\footnotetextcopyrightpermission[1]{}
\pagestyle{plain}

\usepackage{graphicx}
\usepackage{amsmath}
\usepackage{booktabs}

\begin{document}

\title{Statistical Framework for Distinguishing Planetary Protection from Coincidence in Unpolluted White Dwarf Systems}

\author{Anonymous}
\affiliation{\institution{Anonymous}}

\begin{abstract}
Approximately 25--50\% of white dwarfs exhibit photospheric metal pollution from accreting planetary debris, yet WD~1856+534---despite hosting a close-in giant planet---shows no detectable pollution. We develop a statistical framework combining population synthesis, Bayesian hypothesis testing, and Monte Carlo power analysis to determine whether this absence reflects dynamical shielding by the planet or observational coincidence. Our simulations of 10{,}000 synthetic white dwarf systems show that under a planetary protection model, planet-hosting white dwarfs exhibit a pollution rate of 10.2\%, significantly below the 19.7\% background rate ($p < 10^{-8}$, binomial test). Bayesian model comparison yields decisive evidence for the protection hypothesis (Bayes factor $> 10^6$). Power analysis indicates that a sample of $\sim$50 white dwarf systems with close-in giant planets would provide 80\% power to detect protection at the 5\% significance level, assuming 75\% shielding efficiency. Our debris trajectory simulations for the WD~1856+534b system (13.8~$M_J$, 0.02~AU) predict an 18.5\% total shielding fraction. These results provide a quantitative roadmap for resolving this open question with forthcoming survey data.
\end{abstract}

\maketitle

\section{Introduction}

White dwarf stars frequently exhibit photospheric metal pollution, with observational surveys indicating that 25--50\% of hydrogen-atmosphere (DA) white dwarfs show detectable metal absorption lines \cite{zuckerman2003, koester2014}. This pollution is attributed to the ongoing accretion of disrupted planetesimals and asteroids that venture too close to the white dwarf \cite{farihi2016, jura2014}.

WD~1856+534 presents a remarkable exception: despite hosting one of the few confirmed giant planets orbiting a white dwarf---WD~1856+534b, a $\sim$13.8~$M_J$ body in a $\sim$1.4-day orbit at $\sim$0.02~AU \cite{vanderburg2020}---no photospheric metal pollution has been detected. Zhang et al.\ \cite{zhang2026protection} used N-body simulations to demonstrate that such close-in planets can dynamically eject or intercept debris, providing a ``protective'' mechanism. However, they noted that current samples are too limited to statistically distinguish protection from coincidence.

In this work, we develop and apply a comprehensive statistical framework to address this open question. Our approach combines: (1) population synthesis of white dwarf systems, (2) frequentist and Bayesian hypothesis tests, (3) Monte Carlo power analysis to determine required sample sizes, and (4) debris trajectory simulations parameterized by planet properties.

\section{Methods}

\subsection{Population Synthesis}

We generate a synthetic population of $N = 10{,}000$ white dwarfs, of which 500 host close-in giant planets. Each system is assigned:
\begin{itemize}
    \item \textbf{Stellar properties}: effective temperature $T_\mathrm{eff} \sim \mathcal{N}(10{,}000, 3000)$~K, cooling age from $\mathrm{Exp}(1.5)$~Gyr, atmosphere type (80\% DA, 20\% DB).
    \item \textbf{Planet properties}: mass from a power-law distribution $[1, 13]~M_J$, orbital separation log-uniform in $[0.01, 0.1]$~AU.
    \item \textbf{Debris activity}: 30\% of systems have active debris delivery with accretion rates $\dot{M} \sim \mathcal{N}(10^8, 5\times10^7)$~g/s.
\end{itemize}

\subsection{Protection Model}

The planetary shielding efficiency $\eta$ is modeled as:
\begin{equation}
    \eta(M_p, a, \dot{M}) = \tanh\!\left(\frac{M_p}{3\,M_J}\right) \exp\!\left(-\frac{a}{0.05\,\mathrm{AU}}\right) \frac{1}{1 + (\dot{M}/5\times10^8)^2}
\end{equation}
Under the protection hypothesis, intrinsically polluted planet-hosting systems have their pollution blocked with probability $\eta$. Under the null hypothesis, planets have no effect on pollution.

\subsection{Statistical Tests}

We employ three complementary approaches:
\begin{enumerate}
    \item \textbf{Binomial test}: One-sided test for reduced pollution rate among planet-hosting WDs relative to the background rate.
    \item \textbf{Fisher's exact test}: Contingency table analysis comparing pollution incidence between planet and non-planet subsamples.
    \item \textbf{Bayesian model comparison}: We compute the Bayes factor $B_{10}$ by integrating the binomial likelihood over a uniform prior on the protection-reduced rate.
\end{enumerate}

\subsection{Power Analysis}

Monte Carlo simulations (5{,}000 iterations) estimate the statistical power to detect protection at $\alpha = 0.05$ and $\alpha = 0.01$ for sample sizes ranging from 5 to 200 systems, across protection efficiencies of 50\%, 75\%, and 95\%.

\subsection{Debris Trajectory Simulations}

We simulate 5{,}000 test particles on highly eccentric orbits ($e > 0.8$) interacting with a planet of specified mass and separation. Particles are classified as ejected (gravitational scattering), intercepted (captured by planet), or accreted (reaching the white dwarf).

\section{Results}

\subsection{Population-Level Analysis}

Under the protection model, planet-hosting white dwarfs exhibit a detected pollution rate of 10.2\%, compared to 19.7\% for the general population (Table~\ref{tab:rates}). The null model predicts 18.4\% for planet-hosting systems.

\begin{table}[h]
\centering
\caption{Pollution rates under protection and null models.}
\label{tab:rates}
\begin{tabular}{lcc}
\toprule
\textbf{Subsample} & \textbf{Protection} & \textbf{Null} \\
\midrule
Planet-hosting & 0.102 & 0.184 \\
No planet & 0.197 & 0.200 \\
Overall & 0.192 & 0.199 \\
\bottomrule
\end{tabular}
\end{table}

\subsection{Statistical Tests}

The binomial test rejects the null hypothesis with $p < 10^{-8}$ (Table~\ref{tab:tests}). Fisher's exact test yields an odds ratio of 0.464 ($p < 10^{-8}$). Bayesian model comparison provides decisive evidence for the protection hypothesis with $\log_{10} B_{10} = 6.40$.

\begin{table}[h]
\centering
\caption{Statistical test results ($N_{\text{planet}} = 500$).}
\label{tab:tests}
\begin{tabular}{lcc}
\toprule
\textbf{Test} & \textbf{Statistic} & \textbf{$p$-value} \\
\midrule
Binomial & Effect size: 0.095 & $7.4 \times 10^{-9}$ \\
Fisher exact & OR = 0.464 & $1.5 \times 10^{-8}$ \\
Bayes factor & $\log_{10} B_{10}$ = 6.40 & --- \\
\bottomrule
\end{tabular}
\end{table}

\subsection{Power Analysis}

Figure~\ref{fig:power} shows the statistical power as a function of sample size. At 75\% protection efficiency, approximately 50 systems are needed for 80\% power at $\alpha = 0.05$. At 50\% efficiency, the required sample exceeds 100 systems.

\begin{figure}[h]
\centering
\includegraphics[width=\columnwidth]{figures/fig2_power_analysis.png}
\caption{Statistical power versus sample size for three protection efficiencies.}
\label{fig:power}
\end{figure}

\subsection{WD 1856+534b Debris Simulations}

For the WD~1856+534b system (13.8~$M_J$, 0.02~AU), our trajectory simulations predict a total shielding fraction of 18.5\% (15.2\% ejected, 3.1\% intercepted) from 5{,}000 test particles. The computed protection efficiency is $\eta = 0.644$.

\begin{figure}[h]
\centering
\includegraphics[width=\columnwidth]{figures/fig3_trajectory_shielding.png}
\caption{(a) Shielding fraction versus planet mass. (b) Debris trajectory outcomes for WD~1856+534b.}
\label{fig:traj}
\end{figure}

\subsection{Future Survey Predictions}

With 50 analogous systems, the median Bayes factor exceeds 10 (strong evidence). With 200 systems, the probability of achieving decisive evidence ($B_{10} > 100$) exceeds 95\% (Figure~\ref{fig:bayes}).

\begin{figure}[h]
\centering
\includegraphics[width=\columnwidth]{figures/fig4_bayesian_evidence.png}
\caption{(a) Median Bayes factor versus sample size. (b) Frequentist power and probability of decisive Bayesian evidence.}
\label{fig:bayes}
\end{figure}

\section{Conclusion}

We present a statistical framework for resolving whether the absence of pollution in WD~1856+534 reflects planetary protection or observational coincidence. Our population synthesis demonstrates that the protection signal is statistically detectable with current methods, provided sufficient sample sizes. With 75\% protection efficiency and $\sim$50 planet-hosting WD systems, both frequentist and Bayesian approaches achieve robust discrimination. Upcoming surveys from Gaia, LSST, and JWST follow-up programs are expected to assemble such samples within the next decade, making this framework directly applicable to forthcoming observational data.

\section{Limitations and Ethical Considerations}

Our model makes simplifying assumptions about debris orbit distributions and detection efficiencies. The protection efficiency parameterization, while inspired by N-body results, requires calibration against detailed dynamical simulations. Selection effects in planet detection surveys may bias comparisons between planet-hosting and field white dwarfs. Future work should incorporate realistic survey selection functions. This work involves computational modeling of astrophysical phenomena and raises no direct ethical concerns.

\bibliographystyle{ACM-Reference-Format}
\bibliography{references}

\end{document}
