\documentclass[sigconf,review,anonymous]{acmart}
\settopmatter{printacmref=false}
\renewcommand\footnotetextcopyrightpermission[1]{}
\pagestyle{plain}

\usepackage{graphicx}
\usepackage{amsmath}
\usepackage{booktabs}

\begin{document}

\title{Statistical and Mechanistic Analysis of the Relationship Between Traumatic Brain Injury and Creutzfeldt-Jakob Disease}

\author{Anonymous}
\affiliation{\institution{Anonymous}}

\begin{abstract}
The relationship between traumatic brain injury (TBI) and subsequent development of Creutzfeldt-Jakob disease (CJD) remains uncertain. We develop a computational framework combining epidemiological simulation, prion kinetics modeling, statistical power analysis, and Bayesian causal inference to evaluate this relationship. Our population simulation of 100{,}000 individuals over 20 years demonstrates that the extremely low incidence of sCJD ($\sim$1.5 per million/year) fundamentally limits observational study power. Even with a true risk ratio of 3.0, detecting the association requires cohorts exceeding 100{,}000 individuals. Prion kinetics modeling shows TBI-induced neuroinflammation could accelerate PrP$^{\mathrm{Sc}}$ accumulation by disrupting the blood-brain barrier and upregulating PrP$^{\mathrm{C}}$ expression. Bayesian model comparison across three hypotheses---coincidence, causal link, and detection bias---finds that with 23 reported literature cases, the posterior probability of a causal relationship is 0.316 versus 0.382 for coincidence. These results indicate that current evidence is insufficient to confirm or refute causality, and identify the statistical requirements for definitive studies.
\end{abstract}

\maketitle

\section{Introduction}

Sporadic Creutzfeldt-Jakob disease (sCJD) is a fatal neurodegenerative prion disease with an incidence of approximately 1.5 cases per million per year \cite{ladogana2005mortality}. Multiple case reports have described CJD onset following traumatic brain injury (TBI) \cite{zeng2026scjd}, raising questions about whether TBI can trigger or accelerate prion disease. TBI affects approximately 69 million individuals globally each year \cite{dewan2018tbi} and has been linked to increased risk of other neurodegenerative conditions \cite{brett2019tbi_risk}.

The prion protein PrP$^{\mathrm{C}}$ is normally expressed in neural tissue and can undergo spontaneous misfolding into the pathogenic PrP$^{\mathrm{Sc}}$ isoform \cite{prusiner1998prions, collinge2001prion}. TBI causes blood-brain barrier disruption, neuroinflammation, and potential upregulation of PrP$^{\mathrm{C}}$ expression \cite{mackenzie2017tbi}, providing a plausible biological mechanism for accelerated prion conversion. However, Zeng et al.\ \cite{zeng2026scjd} explicitly state that the relationship is ``not definite.''

\section{Methods}

\subsection{Epidemiological Simulation}

We simulate a cohort of $N = 100{,}000$ individuals (5\% with TBI history) followed for 20 years. CJD outcomes are modeled as Bernoulli trials with annual probability modified by genotype (PRNP codon 129), age, TBI exposure, and TBI severity (mild/moderate/severe).

\subsection{Prion Kinetics Model}

PrP$^{\mathrm{Sc}}$ accumulation follows autocatalytic kinetics with doubling time of 90 days. Under TBI, the conversion rate is transiently enhanced by a factor of 2.0 (BBB disruption) and PrP$^{\mathrm{C}}$ substrate is upregulated by 1.5$\times$ for 180 days post-injury.

\subsection{Bayesian Causal Inference}

We compare three hypotheses with priors $P(H_1) = 0.5$ (coincidence), $P(H_2) = 0.25$ (causal), $P(H_3) = 0.25$ (detection bias). Likelihoods are computed using Poisson models calibrated to 23 reported literature cases.

\section{Results}

\subsection{Epidemiological Analysis}

At the base CJD incidence of 1.5 per million/year, only 5 CJD cases occurred in the non-TBI group and 0 in the TBI group over 20 years in our 100{,}000-person cohort---reflecting the fundamental challenge of studying this rare disease (Table~\ref{tab:epi}).

\begin{table}[h]
\centering
\caption{CJD occurrence over 20-year follow-up ($N = 100{,}000$).}
\label{tab:epi}
\begin{tabular}{lccc}
\toprule
\textbf{Group} & \textbf{N} & \textbf{CJD Cases} & \textbf{Rate/M} \\
\midrule
TBI exposed & 5{,}000 & 0 & 0 \\
No TBI & 95{,}000 & 5 & 52.6 \\
\bottomrule
\end{tabular}
\end{table}

\subsection{Power Analysis}

Even with a risk ratio of 3.0, detecting TBI-CJD association at 80\% power requires cohorts exceeding 500{,}000 individuals (Figure~\ref{fig:power}). For RR = 5.0, approximately 100{,}000 individuals suffice.

\begin{figure}[h]
\centering
\includegraphics[width=\columnwidth]{figures/fig3_power.png}
\caption{Statistical power to detect TBI-CJD association by sample size and assumed risk ratio.}
\label{fig:power}
\end{figure}

\subsection{Prion Kinetics}

TBI-induced acceleration of prion conversion advances the time to clinical threshold by approximately 6--12 months in our kinetics model (Figure~\ref{fig:kinetics}), suggesting TBI may accelerate rather than initiate prion disease.

\begin{figure}[h]
\centering
\includegraphics[width=\columnwidth]{figures/fig2_prion_kinetics.png}
\caption{PrP$^{\mathrm{Sc}}$ accumulation kinetics with and without TBI-induced acceleration.}
\label{fig:kinetics}
\end{figure}

\subsection{Bayesian Model Comparison}

Given 23 reported literature cases, Bayesian analysis yields posterior probabilities of 0.382 for coincidence, 0.316 for a causal link, and 0.302 for detection bias (Figure~\ref{fig:bayes}). No hypothesis is strongly favored.

\begin{figure}[h]
\centering
\includegraphics[width=\columnwidth]{figures/fig4_bayesian.png}
\caption{(a) Prior and posterior probabilities for three causal hypotheses. (b) Bayes factors relative to coincidence.}
\label{fig:bayes}
\end{figure}

\section{Conclusion}

Our analysis demonstrates that the TBI-CJD relationship cannot be resolved with current data and conventional study designs. The extremely low incidence of sCJD requires population-scale studies ($>$500{,}000 individuals) to achieve adequate statistical power. Prion kinetics modeling provides a plausible mechanistic pathway through TBI-induced acceleration of ongoing prion conversion, but this alone does not establish causation. Multi-national prion disease registries with systematic TBI history collection represent the most promising path forward.

\section{Limitations and Ethical Considerations}

Our simulation uses simplified prion kinetics that do not capture the full complexity of strain-dependent conversion. The epidemiological model assumes independence of risk factors. Ethical considerations include the sensitivity of linking brain injury to fatal prion disease in populations such as athletes and military personnel. Any clinical implications should be communicated carefully to avoid undue alarm.

\bibliographystyle{ACM-Reference-Format}
\bibliography{references}

\end{document}
