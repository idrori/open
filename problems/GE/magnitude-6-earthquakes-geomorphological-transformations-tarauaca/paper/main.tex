\documentclass[sigconf,review,anonymous]{acmart}
\usepackage{amsmath}
\usepackage{graphicx}
\usepackage{booktabs}
\setcopyright{none}

\title{Can Magnitude 6 Deep Earthquakes Drive Geomorphological Change in the Tarauac\'{a} Fault Region?}

\author{Anonymous}
\affiliation{\institution{Anonymous}}

\begin{abstract}
We assess whether earthquakes with magnitude $\sim$6.0~Mb at depths of $\sim$580~km can explain the rapid geomorphological transformations observed in the Tarauac\'{a} fault region of Acre, Brazil. Using ground motion prediction equations adapted for deep intraslab events, Newmark sliding-block analysis, simplified liquefaction assessment, and Monte Carlo simulation (5{,}000 realizations), we find that single M6 events at this depth produce PGA of only 0.012~g at 50~km epicentral distance---well below standard thresholds for landslides (0.05~g) and liquefaction (0.1~g). The total per-event surface displacement is 0.62~cm. Monte Carlo assessment yields zero probability of liquefaction or significant ($>$10~cm) single-event impact. Cumulative loading over 50 events gives 30.3~cm of displacement, but the recurrence interval ($\sim$100~yr for M6) implies $\sim$16{,}000~yr to accumulate 1~m. A minimum magnitude of $\sim$7.4 is required for $>$5~cm single-event displacement at this depth. We conclude that individual M$\sim$6 deep earthquakes are insufficient to explain the observed transformations; alternative mechanisms (shallow seismicity, fluvial processes) likely dominate.
\end{abstract}

\keywords{earthquake hazard, geomorphology, seismic wave propagation, Tarauac\'{a} fault, deep earthquakes}

\begin{document}
\maketitle

\section{Introduction}

The Tarauac\'{a} fault region in Acre, Brazil, has experienced notable geomorphological changes over recent decades. Cris\'{o}stomo (2023) attributed these transformations to neotectonic activity, and Moreira et al.\ \cite{moreira2026tarauaca} posed the open question of whether M$\sim$6 deep earthquakes could account for the observed landscape changes.

Deep earthquakes in the Acre region typically occur at depths of 500--620~km within the subducting Nazca slab \cite{assumpacao2004intraplate}. At such depths, seismic energy undergoes significant attenuation before reaching the surface, making the relationship between deep seismicity and surface geomorphology highly non-trivial.

\section{Methods}

\subsection{Ground Motion Prediction}
We employ a ground motion prediction equation (GMPE) adapted for deep intraslab events:
\begin{equation}
\ln(\mathrm{PGA}) = a + b \cdot M_b + c \cdot \ln(R_{\mathrm{hyp}}) + d \cdot z
\end{equation}
with coefficients $a=-2.5$, $b=1.2$, $c=-1.7$, $d=0.003$, where $R_{\mathrm{hyp}}$ is hypocentral distance and $z$ is depth \cite{abrahamson2014gmpe}.

\subsection{Geomorphological Response}
We evaluate three mechanisms: (1)~Newmark sliding-block analysis for landslide displacement \cite{newmark1965sliding, jibson2007newmark}; (2)~simplified liquefaction potential using CSR/CRR methodology \cite{seed1985liquefaction}; (3)~ground settlement from dynamic densification.

\subsection{Monte Carlo Assessment}
We run 5{,}000 simulations sampling magnitude ($M_b \in [5.5, 6.5]$), depth ($z \in [500, 620]$~km), and epicentral distance ($\Delta \in [20, 100]$~km) uniformly to derive probabilistic impact estimates.

\section{Results}

\subsection{Single-Event Ground Motion}
For a reference M6.0 earthquake at 580~km depth, we compute PGA~$= 0.012$~g at 50~km epicentral distance, corresponding to MMI~$\approx$~3.5 (``weak'' shaking). PGV is $\sim$1.0~cm/s.

\begin{table}[h]
\centering
\caption{Ground motion and geomorphological impact for M6.0 at 580~km depth.}
\label{tab:results}
\begin{tabular}{lc}
\toprule
Parameter & Value \\
\midrule
PGA at 50~km [g] & 0.012 \\
MMI at 50~km & 3.5 \\
Landslide displacement [cm] & 0.00 \\
Settlement [cm] & 0.62 \\
Total displacement [cm] & 0.62 \\
Liquefaction likely & No \\
Impact score (0--1) & 0.006 \\
\bottomrule
\end{tabular}
\end{table}

\subsection{Monte Carlo Results}
Across 5{,}000 simulations: mean PGA~$= 0.013$~g, mean displacement~$= 0.69$~cm, liquefaction probability~$= 0\%$, probability of significant ($>$10~cm) impact~$= 0\%$.

\subsection{Cumulative and Threshold Analysis}
Cumulative displacement over 50~events is 30.3~cm. Given a recurrence rate of 0.01/yr, achieving 1~m of cumulative displacement requires $\sim$16{,}000~years. The minimum magnitude for $>$5~cm single-event displacement at 580~km depth is $\sim$7.4.

\begin{figure}[h]
\centering
\includegraphics[width=\columnwidth]{figures/pga_vs_distance.png}
\caption{PGA vs.\ epicentral distance for M6.0 at 580~km depth. Horizontal lines mark liquefaction and landslide thresholds.}
\label{fig:pga}
\end{figure}

\begin{figure}[h]
\centering
\includegraphics[width=\columnwidth]{figures/magnitude_sensitivity.png}
\caption{Left: PGA vs.\ magnitude at 50~km distance. Right: surface displacement vs.\ magnitude. The M6 event falls below geomorphological significance thresholds.}
\label{fig:sensitivity}
\end{figure}

\section{Discussion}

Our analysis demonstrates that M$\sim$6 deep earthquakes at 580~km depth produce surface ground motions far below thresholds for significant geomorphological modification. The PGA of 0.012~g is an order of magnitude below the Keefer (1984) threshold for earthquake-triggered landslides \cite{keefer1984landslides}.

The cumulative loading pathway requires $>$16{,}000 years for meter-scale changes, which is too slow to explain ``rapid'' transformations observed over recent decades. Alternative mechanisms must be considered: (1)~shallow seismicity within the fault zone itself, (2)~fluvial erosion and deposition, (3)~anthropogenic landscape modification, or (4)~rare larger-magnitude events.

\section{Conclusions}
\begin{enumerate}
\item M$\sim$6 earthquakes at 580~km depth produce PGA~$= 0.012$~g (MMI~3.5) at 50~km.
\item Single-event surface displacement is only 0.62~cm; liquefaction probability is zero.
\item Monte Carlo assessment (5{,}000 runs) confirms zero probability of significant single-event impact.
\item A minimum magnitude of $\sim$7.4 is needed for $>$5~cm single-event displacement.
\item Individual M$\sim$6 deep earthquakes cannot explain the observed rapid transformations.
\end{enumerate}

\bibliographystyle{ACM-Reference-Format}
\bibliography{references}

\end{document}
