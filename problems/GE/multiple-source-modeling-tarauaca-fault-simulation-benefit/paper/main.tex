\documentclass[sigconf,review,anonymous]{acmart}
\usepackage{amsmath}
\usepackage{graphicx}
\usepackage{booktabs}
\setcopyright{none}

\title{Multiple-Source Modeling Improves P--SV Wave Simulation for Deep Tarauac\'{a} Fault Earthquakes}

\author{Anonymous}
\affiliation{\institution{Anonymous}}

\begin{abstract}
We investigate whether representing earthquake sources as multiple Gaussian pulse sub-sources aligned with the Tarauac\'{a} fault improves simulation fidelity for deep P--SV wavefields compared to a single-source model. Using layered-earth Green's functions for the Acre, Brazil region (source depth 580~km), we compare synthetic seismograms across 20 surface receivers at 25--500~km. The single and five-source models yield a mean cross-correlation of 0.525, envelope misfit of 0.101, and PGA difference of 51.3\%, demonstrating that source representation significantly affects waveform predictions. Directivity analysis shows up to 3.64$\times$ amplification in the forward-rupture direction ($135^\circ$ azimuth). An N-source scaling study (N = 1--20) reveals progressive decorrelation, with cross-correlation decreasing from 1.0 (N = 1) to 0.26 (N = 20). We conclude that multiple-source modeling is recommended for accurate simulation of deep Tarauac\'{a} earthquakes.
\end{abstract}

\keywords{seismic simulation, P-SV waves, earthquake source, directivity, multiple sources}

\begin{document}
\maketitle

\section{Introduction}

Moreira et al.\ \cite{moreira2026tarauaca} modeled P--SV seismic wave propagation from deep earthquakes in Acre, Brazil, using a single Gaussian pulse source. They posed the open question of whether multiple sub-sources aligned with the Tarauac\'{a} fault would better simulate observed wavefields from deep, intense earthquakes associated with Nazca plate subduction.

Finite-fault source models are standard in seismology for moderate-to-large earthquakes, capturing rupture propagation, directivity, and extended-source effects \cite{aki2002quantitative, somerville1997directivity}. However, for the specific geometry of the Tarauac\'{a} fault at depths of 500--620~km, the benefit of multi-source representations had not been quantified.

\section{Methods}

\subsection{Source Models}
We define single and multiple Gaussian pulse sources with seismic moment $M_0 = 1.12 \times 10^{18}$~N~m (M$_b$ 6.0), dominant frequency $f_0 = 1$~Hz, and pulse width $\sigma = 1$~s. The multi-source model distributes $N = 5$ sub-sources along 80~km of fault at 16~km spacing, with rupture propagation at 2.8~km/s \cite{brune1970source}.

\subsection{Wave Propagation}
We use analytical Green's functions for a five-layer earth model (sediment, upper/lower crust, upper mantle, transition zone) with appropriate P/S velocities, densities, and Q factors. Seismograms are computed at 20 receivers from 25 to 500~km \cite{hartzell1978sources}.

\subsection{Comparison Metrics}
We evaluate: (1)~normalized cross-correlation, (2)~envelope misfit, (3)~spectral misfit, and (4)~PGA difference.

\section{Results}

\begin{table}[h]
\centering
\caption{Average waveform comparison metrics across 20 receivers.}
\label{tab:metrics}
\begin{tabular}{lc}
\toprule
Metric & Value \\
\midrule
Mean cross-correlation & 0.525 \\
Mean envelope misfit & 0.101 \\
Mean PGA difference & 51.3\% \\
Max directivity amplification & 3.64$\times$ \\
Forward-rupture azimuth & $135^\circ$ \\
\bottomrule
\end{tabular}
\end{table}

The waveforms from single and multi-source models differ substantially (Table~\ref{tab:metrics}). The cross-correlation of 0.525 indicates that the multi-source model produces fundamentally different waveforms, primarily due to rupture directivity and source duration effects.

\begin{figure}[h]
\centering
\includegraphics[width=\columnwidth]{figures/source_comparison.png}
\caption{Left: Source time functions for single (blue) and multi-source (red, $N=5$) models. Right: Cross-correlation with the single-source model as a function of $N$.}
\label{fig:stf}
\end{figure}

\begin{figure}[h]
\centering
\includegraphics[width=\columnwidth]{figures/metrics_vs_distance.png}
\caption{Waveform similarity (left), envelope misfit (center), and PGA difference (right) as functions of epicentral distance.}
\label{fig:metrics}
\end{figure}

\subsection{Directivity}
The forward-rupture direction (azimuth $135^\circ$, along strike) shows amplification up to 3.64$\times$, while the backward direction shows de-amplification. This azimuthal dependence is absent from single-source simulations.

\subsection{N-Source Scaling}
Cross-correlation decreases from 1.0 ($N=1$) to 0.26 ($N=20$), while source duration increases. The optimal range $N = 3$--5 balances physical realism with computational tractability.

\section{Discussion}

The 51.3\% mean PGA difference and cross-correlation of only 0.525 demonstrate that source representation is a first-order effect for Tarauac\'{a} fault simulations. Single-source models systematically miss directivity effects that redistribute energy azimuthally, which is critical for hazard assessment \cite{somerville1997directivity}.

\section{Conclusions}
\begin{enumerate}
\item Multi-source modeling produces significantly different waveforms (CC = 0.525, PGA diff = 51.3\%).
\item Directivity amplification reaches 3.64$\times$ in the forward-rupture direction.
\item Cross-correlation decreases monotonically with increasing $N$ (0.26 at $N=20$).
\item Multiple-source modeling ($N = 3$--5) is recommended for deep Tarauac\'{a} fault earthquakes.
\end{enumerate}

\bibliographystyle{ACM-Reference-Format}
\bibliography{references}

\end{document}
