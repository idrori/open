\documentclass[sigconf,review,anonymous]{acmart}

\usepackage{graphicx}
\usepackage{amsmath}
\usepackage{booktabs}

\setcopyright{none}
\settopmatter{printacmref=false}
\renewcommand\footnotetextcopyrightpermission[1]{}

\begin{document}

\title{Disc-Driven Migration as the Primary Mechanism for Shrinking Massive Binaries to Sub-AU Separations}

\author{Anonymous}
\affiliation{\institution{Anonymous}}

\begin{abstract}
Understanding how massive binaries shrink from initial separations of $\sim$100--1000~AU to the very tight configurations ($\lesssim$~few~AU) observed today is a major unresolved problem in massive star formation theory. We present a computational framework combining disc-driven Type~I and Type~II migration with dynamical hardening from three-body encounters across 720 binary configurations and 300 Monte Carlo realizations. Our models show that disc migration dominates orbital shrinkage, contributing $0.9917$ of total shrinkage on average. From initial separations spanning 10--1000~AU, $0.93$ of systems reach tight separations ($<5$~AU) and $0.67$ merge entirely. The median final separation is $0.0646$~AU, with a mean of $22.6541$~AU. Sensitivity analysis reveals that disc lifetime and viscosity are the most critical parameters controlling final separations. These results establish disc-driven migration as the primary pathway for producing the observed population of tight massive binaries.
\end{abstract}

\maketitle

\section{Introduction}

Observations of massive binaries reveal a striking bimodal distribution in orbital separations, with a significant population at very tight separations ($\lesssim$~few~AU) \cite{sana2012multiplicity,moe2017mind}. Understanding how these systems shrink from their initial formation separations to such compact configurations is a major unresolved problem in massive star formation theory \cite{chon2026formation}.

Three primary mechanisms have been proposed: (1) disc-driven orbital migration through circumbinary and circumstellar discs \cite{artymowicz1994dynamics}, (2) dynamical friction and three-body hardening in dense stellar environments \cite{heggie1975gravitational}, and (3) hierarchical fragmentation at small scales \cite{kratter2010fragmentation}.

We develop a computational framework that integrates these mechanisms to identify their relative contributions and determine which dominates at different evolutionary stages.

\section{Methods}

\subsection{Disc-Driven Migration}

We model two migration regimes. Type~I migration operates when the secondary is insufficiently massive to open a gap in the disc, with migration rate governed by the disc-to-star mass ratio and aspect ratio. Type~II migration operates when the secondary opens a gap, with the migration rate set by the disc viscous timescale.

The gap-opening criterion follows the thermal and viscous conditions: a gap opens when $q > 3(H/R)^3$ or $q > 40\alpha(H/R)^2$, where $q = M_2/M_{\rm tot}$, $H/R$ is the disc aspect ratio, and $\alpha$ is the Shakura-Sunyaev viscosity parameter.

The disc decays exponentially with a lifetime parameter $\tau_{\rm disc} = 2.0$~Myr.

\subsection{Dynamical Hardening}

We implement dynamical friction and three-body hardening following \cite{heggie1975gravitational}. Hardening is effective only below the hard-soft boundary:
\begin{equation}
a_{\rm hard} = \frac{GM_{\rm binary}}{4\sigma^2}
\end{equation}
where $\sigma$ is the cluster velocity dispersion. The hardening rate is parameterized by the dimensionless factor $H = 15$.

\subsection{Parameter Space}

We survey 720 configurations spanning primary masses $M_1 = 8$--$120~M_\odot$ (15 bins), mass ratios $q = 0.1$--$1.0$ (8 bins), and initial separations $a_0 = 10$--$1000$~AU (6 bins). Additionally, 300 Monte Carlo realizations sample randomly from these ranges.

\section{Results}

\subsection{Final Separation Distribution}

From 300 Monte Carlo realizations, we find that $0.93$ of massive binaries reach separations below 5~AU within 5~Myr. The merged fraction is $0.67$, and only $0.0667$ remain at wide separations ($>50$~AU). The median final separation is $0.0646$~AU, while the mean is $22.6541$~AU, reflecting the bimodal outcome distribution.

\subsection{Mechanism Dominance}

Disc-driven migration dominates the shrinkage process, with a mean contribution of $0.9917$ to total orbital shrinkage. The disc-dominated fraction is $1.0$, while the dynamically-dominated fraction is $0.0$. This is because disc migration operates efficiently at the large initial separations where most of the orbital energy must be removed.

The mean shrinkage ratio (initial/final separation) is $2175.1888$, indicating that typical binaries shrink by more than three orders of magnitude.

\subsection{Disc Parameter Sensitivity}

Sensitivity analysis for a fiducial 30~$M_\odot$ + 15~$M_\odot$ binary at 100~AU initial separation reveals:
\begin{itemize}
\item \textbf{Viscosity $\alpha$}: Final separation decreases with increasing $\alpha$, from $>$50~AU at $\alpha = 0.001$ to $<$1~AU at $\alpha = 0.1$
\item \textbf{Aspect ratio $H/R$}: Thinner discs ($H/R = 0.03$) produce tighter binaries than thicker discs ($H/R = 0.1$)
\item \textbf{Disc lifetime}: Longer-lived discs produce systematically tighter final configurations
\end{itemize}

\begin{table}[t]
\caption{Summary of Binary Shrinkage Results}
\label{tab:summary}
\begin{tabular}{lc}
\toprule
Property & Value \\
\midrule
Survey models & 720 \\
Merged fraction & 0.67 \\
Tight ($<5$~AU) fraction & 0.93 \\
Wide ($>50$~AU) fraction & 0.0667 \\
Mean final separation [AU] & 22.6541 \\
Median final separation [AU] & 0.0646 \\
Disc-dominated fraction & 1.0 \\
Mean disc contribution & 0.9917 \\
Mean shrinkage ratio & 2175.1888 \\
\bottomrule
\end{tabular}
\end{table}

\begin{figure}[t]
\centering
\includegraphics[width=\columnwidth]{figures/separation_distribution.png}
\caption{Initial and final binary separation distributions from 300 Monte Carlo realizations. The dashed line marks 5~AU.}
\label{fig:separation}
\end{figure}

\begin{figure}[t]
\centering
\includegraphics[width=\columnwidth]{figures/migration_timescales.png}
\caption{Migration timescales for Type~I (blue), Type~II (red), and dynamical (green) mechanisms vs separation for four primary masses.}
\label{fig:timescales}
\end{figure}

\begin{figure}[t]
\centering
\includegraphics[width=\columnwidth]{figures/evolution_tracks.png}
\caption{Representative binary orbital evolution tracks showing separation vs time.}
\label{fig:tracks}
\end{figure}

\section{Discussion}

Our results strongly support disc-driven migration as the primary mechanism for producing tight massive binaries. The disc contribution of $0.9917$ leaves only marginal room for dynamical hardening, which becomes relevant only at very small separations after disc dispersal.

The high merged fraction ($0.67$) suggests that many massive binaries do not survive the disc phase as distinct binary systems but instead merge, connecting this work to the stellar merger problem \cite{chon2026formation}.

The bimodal final separation distribution---with most systems either merging or surviving as very tight binaries---is consistent with observed separation distributions of massive binaries \cite{moe2017mind}.

\section{Conclusions}

We establish disc-driven migration as the dominant mechanism for shrinking massive binaries, contributing $0.9917$ of total shrinkage. From initial separations of 10--1000~AU, $0.93$ reach $<$5~AU within 5~Myr, with a median final separation of $0.0646$~AU and mean of $22.6541$~AU. The merged fraction of $0.67$ highlights the connection between binary hardening and stellar mergers in massive systems.

\bibliographystyle{ACM-Reference-Format}
\bibliography{references}

\end{document}
