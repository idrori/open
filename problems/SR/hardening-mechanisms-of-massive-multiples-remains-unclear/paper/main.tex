\documentclass[sigconf,review,anonymous]{acmart}

\usepackage{graphicx}
\usepackage{amsmath}
\usepackage{booktabs}

\setcopyright{none}
\settopmatter{printacmref=false}
\renewcommand\footnotetextcopyrightpermission[1]{}

\begin{document}

\title{Disc-Driven Torques as the Primary Hardening Mechanism for Massive Multiple-Star Systems}

\author{Anonymous}
\affiliation{\institution{Anonymous}}

\begin{abstract}
The dynamical processes that harden massive multiple-star systems to tight separations remain unclear. We present a computational framework that quantifies the relative contributions of disc-driven torques, turbulent gas dynamics, and few-body (Kozai-Lidov) interactions across 400 Monte Carlo realizations of hierarchical triple systems with primary masses 10--100~$M_\odot$. Disc torques contribute a mean fraction of $0.9994 \pm 0.0017$ of total hardening, with dynamical interactions contributing $0.0006 \pm 0.0017$ and turbulent gas effects negligible. Of the simulated systems, $0.405$ merge entirely and $0.425$ reach tight separations ($<5$~AU) within 3~Myr. The mean hardening ratio is $75.1168$ with a median of $1.6568$. Kozai-Lidov oscillations from outer companions can drive eccentricities above $0.9$ at mutual inclinations exceeding $40^\circ$, but their net energy extraction is secondary. These results establish circumbinary disc torques as the dominant hardening pathway during the embedded formation phase.
\end{abstract}

\maketitle

\section{Introduction}

Massive stars predominantly exist in binary and higher-order multiple systems \cite{sana2012multiplicity,moe2017mind}. Observations indicate that these systems can undergo significant hardening shortly after formation, yet the specific processes responsible remain unclear \cite{chon2026formation}.

Three mechanisms have been proposed: (1) resonant torques from circumbinary/circumstellar discs \cite{munoz2019circumbinary}, (2) dynamical friction from turbulent natal gas, and (3) few-body gravitational interactions including Kozai-Lidov oscillations in hierarchical triples \cite{kozai1962secular}. We develop a computational framework to quantify their relative contributions.

\section{Methods}

\subsection{Disc Torque Model}

We model resonant torques from circumbinary discs following the viscous evolution framework. The disc mass decays exponentially with timescale $\tau_{\rm disc} = 2.0$~Myr, and the torque strength depends on the binary mass ratio and the disc-to-star mass ratio.

\subsection{Turbulent Gas Model}

We compute gas dynamical friction from the natal molecular cloud with density $n = 10^5$~cm$^{-3}$, temperature $T = 30$~K, and turbulent Mach number $\mathcal{M} = 5$. Both laminar dynamical friction and stochastic torques from density fluctuations are included.

\subsection{Kozai-Lidov Model}

For hierarchical triples, we compute the secular Kozai-Lidov eccentricity oscillation with tidal dissipation at periastron. Hardening occurs when high-eccentricity excursions bring the inner binary within the tidal radius.

\subsection{Population Survey}

We simulate 400 hierarchical triple systems with primary masses $M_1 = 10$--$100~M_\odot$, inner separations $a_{\rm in} = 5$--$200$~AU, and outer separations $a_{\rm out} = 100$--$5000$~AU, evolving each for 3~Myr.

\section{Results}

\subsection{Mechanism Dominance}

Disc torques overwhelmingly dominate the hardening process, contributing a mean fraction of $0.9994 \pm 0.0017$ across all 400 realizations. The dynamical (Kozai-Lidov) contribution is $0.0006 \pm 0.0017$, and the turbulent gas contribution is negligible at $< 0.001$.

\subsection{Hardening Outcomes}

Of the 400 simulated triple systems:
\begin{itemize}
\item $0.405$ merge entirely within 3~Myr
\item $0.425$ reach tight separations ($< 5$~AU)
\item Mean hardening ratio: $75.1168$
\item Median hardening ratio: $1.6568$
\end{itemize}

The large difference between mean and median hardening ratios reflects a bimodal outcome: systems with sufficiently massive discs undergo dramatic hardening, while those with less favorable initial conditions experience modest shrinkage.

\subsection{Kozai-Lidov Effects}

Kozai-Lidov oscillations can drive inner binary eccentricities above $0.9$ at mutual inclinations exceeding $40^\circ$, but the net orbital energy extraction through tidal dissipation at periastron is modest compared to disc torques during the embedded phase.

\begin{table}[t]
\caption{Summary of Hardening Mechanism Contributions}
\label{tab:summary}
\begin{tabular}{lcc}
\toprule
Mechanism & Mean Fraction & Std \\
\midrule
Disc torques & 0.9994 & 0.0017 \\
Gas turbulence & 0.0 & 0.0 \\
Dynamical (KL) & 0.0006 & 0.0017 \\
\midrule
Merged fraction & 0.405 & -- \\
Tight fraction & 0.425 & -- \\
Mean hardening ratio & 75.1168 & -- \\
Median hardening ratio & 1.6568 & -- \\
\bottomrule
\end{tabular}
\end{table}

\begin{figure}[t]
\centering
\includegraphics[width=\columnwidth]{figures/mechanism_fractions.png}
\caption{Relative contributions of the three hardening mechanisms across 400 triple system realizations.}
\label{fig:fractions}
\end{figure}

\begin{figure}[t]
\centering
\includegraphics[width=\columnwidth]{figures/kozai_survey.png}
\caption{Kozai-Lidov effects: maximum eccentricity (left) and hardening ratio (right) vs mutual inclination for different outer-to-inner separation ratios.}
\label{fig:kozai}
\end{figure}

\begin{figure}[t]
\centering
\includegraphics[width=\columnwidth]{figures/mass_dependence.png}
\caption{Hardening ratio (left) and final separation (right) vs primary mass for a fixed geometry.}
\label{fig:mass}
\end{figure}

\section{Discussion}

Our results demonstrate that disc-driven torques are the primary hardening mechanism for massive multiple-star systems during the embedded formation phase. The dominance of disc torques ($0.9994$) over dynamical processes ($0.0006$) is robust across the parameter space explored.

The negligible contribution of turbulent gas dynamical friction is explained by the relatively low gas densities at the relevant orbital separations compared to the disc surface density at the inner disc edge.

Kozai-Lidov oscillations become important after disc dispersal, providing a secondary hardening channel for hierarchical systems with favorable geometries (high mutual inclination, moderate separation ratios).

\section{Conclusions}

We establish disc-driven torques as the dominant hardening mechanism for massive multiple-star systems, contributing $0.9994 \pm 0.0017$ of total hardening. From 400 realizations, $0.405$ merge and $0.425$ reach tight separations within 3~Myr, with a mean hardening ratio of $75.1168$.

\bibliographystyle{ACM-Reference-Format}
\bibliography{references}

\end{document}
