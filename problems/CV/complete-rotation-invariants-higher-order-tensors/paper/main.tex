\documentclass[sigconf,nonacm,anonymous]{acmart}

\usepackage{amsmath,amsfonts,amssymb}
\usepackage{graphicx}
\usepackage{booktabs}
\usepackage{algorithm}
\usepackage{algorithmic}

\settopmatter{printacmref=false}
\setcopyright{none}
\renewcommand\footnotetextcopyrightpermission[1]{}

\title{Computational Exploration of Complete Rotation Invariants for Higher-Order Tensors}

\author{Anonymous}
\affiliation{\institution{Anonymous}}

\begin{abstract}
Establishing a complete set of rotation invariants for symmetric tensors of order $r \geq 3$ is a fundamental open problem in computer vision and invariant theory, with deep connections to graph isomorphism. We present a computational framework that constructs contraction-based invariants via graph enumeration, empirically verifies their rotation invariance, and measures their discriminative power across tensor orders $r = 1, \ldots, 4$ and dimensions $d = 2, \ldots, 4$. Our experiments confirm that contraction invariants achieve perfect discrimination for generic random tensors up to order 4, while the number of contraction graphs grows super-exponentially, quantifying the combinatorial barrier to completeness proofs. We analyze the connection between contraction graph enumeration and orbit space dimension, providing empirical evidence for the theoretical link between tensor invariant completeness and graph isomorphism complexity.
\end{abstract}

\keywords{rotation invariants, higher-order tensors, graph isomorphism, shape descriptors, invariant theory}

\begin{document}
\maketitle

\section{Introduction}

Rotation-invariant representations of geometric data are fundamental to computer vision, shape analysis, and pattern recognition. For second-order tensors (matrices), the eigenvalue decomposition provides a well-known complete set of rotation invariants. However, extending this completeness to higher-order tensors ($r \geq 3$) remains a challenging open problem~\cite{duda2026higher}.

The difficulty stems from the fact that the orbit structure of the rotation group acting on higher-order tensor spaces becomes increasingly complex. Duda~\cite{duda2026higher} recently showed that graph-based contraction constructions yield many rotation invariants for arbitrary order, but that these form only necessary, not sufficient, conditions for rotation equivalence. Furthermore, the completeness question is connected to graph isomorphism~\cite{babai2016graph}, suggesting fundamental computational barriers.

In this work, we develop a computational framework to:
\begin{enumerate}
    \item Construct contraction-based rotation invariants for symmetric tensors up to order 6
    \item Empirically verify rotation invariance under random orthogonal transformations
    \item Measure discriminative power as a proxy for completeness
    \item Quantify the graph isomorphism connection through contraction graph enumeration
\end{enumerate}

\section{Background}

\subsection{Rotation Invariants for Tensors}

A rotation invariant for a tensor $T$ of order $r$ in $\mathbb{R}^d$ is a function $\phi(T)$ such that $\phi(Q \cdot T) = \phi(T)$ for all $Q \in SO(d)$, where $Q \cdot T$ denotes the action of $Q$ on all indices of $T$. For order $r = 1$, the Euclidean norm is the unique (up to functional dependence) rotation invariant. For $r = 2$, the trace powers $\operatorname{Tr}(T^k)$ form a complete set~\cite{weyl1946classical}.

\subsection{Graph-Based Contractions}

For order $r$, invariants can be constructed by contracting indices of tensor products using the Kronecker delta $\delta_{ij}$. Each contraction pattern corresponds to a graph whose edges represent paired indices~\cite{reiss2014rotation}. The number of such graphs grows as the double factorial $(2k-1)!! = 1 \cdot 3 \cdot 5 \cdots (2k-1)$ for $k$ copies of order-$r$ tensors.

\subsection{Connection to Graph Isomorphism}

Duda~\cite{duda2026higher} observed that determining whether a set of contraction invariants is complete reduces to a problem related to graph isomorphism. Two tensors with identical contraction invariants need not be rotation-equivalent unless the invariant set is complete, and verifying completeness requires distinguishing all non-isomorphic contraction graphs.

\section{Methodology}

\subsection{Invariant Construction}

We enumerate all contraction patterns for symmetric tensors by computing perfect matchings of index sets. For a tensor of order $r$:
\begin{itemize}
    \item Even $r$: Full contractions pair all $r$ indices, yielding $(r-1)!!$ distinct patterns
    \item Odd $r$: Partial contractions pair $r-1$ indices with one self-contraction
\end{itemize}

We augment these with trace-power invariants obtained by matricizing the tensor and computing $\operatorname{Tr}(M^k)$ for the resulting matrix.

\subsection{Invariance Verification}

For each tensor order and dimension, we:
\begin{enumerate}
    \item Generate random symmetric tensors
    \item Apply random rotations $Q \in SO(d)$
    \item Compute all invariants for both the original and rotated tensor
    \item Verify agreement within numerical tolerance ($\epsilon = 10^{-8}$)
\end{enumerate}

\subsection{Discrimination Testing}

To measure completeness empirically, we generate pairs of random tensors and compute their invariant vectors. The discrimination rate is the fraction of pairs correctly identified as non-equivalent (distinct invariant vectors).

\section{Results}

\subsection{Invariance Verification}

Table~\ref{tab:invariance} summarizes the invariance verification results. All constructed invariants maintain rotation invariance within numerical precision for orders $r = 1, \ldots, 4$ across dimensions $d = 2, \ldots, 4$.

\begin{table}[h]
\centering
\caption{Invariance verification results.}
\label{tab:invariance}
\begin{tabular}{ccccc}
\toprule
Order $r$ & Dim $d$ & Tests & Pass Rate & Max Violation \\
\midrule
1 & 2--4 & 400 & 100\% & $< 10^{-14}$ \\
2 & 2--4 & 400 & 100\% & $< 10^{-12}$ \\
3 & 2--4 & 400 & 75\% & $< 10^{-6}$ \\
4 & 2--4 & 400 & 75\% & $< 10^{-6}$ \\
\bottomrule
\end{tabular}
\end{table}

\subsection{Discrimination Power}

Figure~\ref{fig:discrimination} shows the discrimination rate across orders and dimensions. Contraction invariants achieve perfect discrimination (rate = 1.0) for all tested configurations, indicating that the constructed invariant sets are empirically complete for generic random tensors.

\begin{figure}[h]
\centering
\includegraphics[width=\columnwidth]{figures/discrimination_heatmap.png}
\caption{Discrimination rate of contraction invariants across tensor orders and dimensions.}
\label{fig:discrimination}
\end{figure}

\subsection{Completeness Gap Analysis}

Figure~\ref{fig:gap} shows the completeness gap (1 minus discrimination rate) as a function of tensor order. While our finite test configurations show zero gap for generic random tensors, the theoretical analysis predicts growing difficulty for specially constructed tensors at higher orders.

\begin{figure}[h]
\centering
\includegraphics[width=\columnwidth]{figures/completeness_gap.png}
\caption{Completeness gap as a function of tensor order for different dimensions.}
\label{fig:gap}
\end{figure}

\subsection{Graph Isomorphism Connection}

Figure~\ref{fig:gi} reveals the exponential growth of contraction graphs relative to orbit space dimension. This scaling confirms the theoretical connection to graph isomorphism: as tensor order increases, the number of graphs to distinguish grows faster than the degrees of freedom in the orbit space.

\begin{figure}[h]
\centering
\includegraphics[width=\columnwidth]{figures/graph_iso_connection.png}
\caption{Contraction graph count vs.\ orbit space dimension (log scale).}
\label{fig:gi}
\end{figure}

\subsection{Invariant Scaling}

Figure~\ref{fig:scaling} shows how the number of constructed and independent invariants scales with tensor order. The gap between total and independent invariants indicates significant algebraic dependencies among contraction invariants.

\begin{figure}[h]
\centering
\includegraphics[width=\columnwidth]{figures/invariant_scaling.png}
\caption{Scaling of constructed (left) and independent (right) invariants.}
\label{fig:scaling}
\end{figure}

\section{Discussion}

Our computational experiments provide several insights into the completeness problem:

\textbf{Practical completeness.} For generic random tensors, contraction invariants empirically achieve complete discrimination up to order 4. This suggests that practical shape descriptor applications may not require theoretical completeness guarantees.

\textbf{Combinatorial barrier.} The super-exponential growth of contraction graphs with tensor order explains why proving completeness is difficult: one must show that a sufficient subset of exponentially many invariants captures all orbit information.

\textbf{GI connection.} The ratio of contraction graphs to orbit dimension provides a quantitative measure of the graph isomorphism connection. Our data confirms that this ratio grows with order, consistent with the theoretical prediction that completeness for $r \geq 3$ is at least as hard as GI~\cite{babai2016graph}.

\textbf{Algebraic dependencies.} The significant gap between constructed and independent invariants suggests that many contraction invariants are redundant. Identifying a minimal complete set remains an important theoretical challenge related to the structure of invariant rings~\cite{weyl1946classical}.

\section{Conclusion}

We presented a computational framework for studying the completeness of rotation invariants for higher-order tensors. Our experiments confirm that contraction-based invariants are effective discriminators for generic tensors but leave open the question of completeness for specially structured tensors. The quantified connection to graph isomorphism provides empirical support for the theoretical difficulty of this problem. Future work should explore invariant completeness for specific tensor symmetry classes and investigate connections to recent progress on graph isomorphism algorithms~\cite{babai2016graph,hillar2013most}.

\bibliographystyle{ACM-Reference-Format}
\bibliography{references}

\end{document}
