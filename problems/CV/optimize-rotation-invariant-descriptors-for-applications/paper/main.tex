\documentclass[sigconf,anonymous,review]{acmart}

\usepackage{booktabs}
\usepackage{graphicx}
\usepackage{amsmath}
\usepackage{amssymb}
\usepackage{xcolor}

\setcopyright{none}

\begin{document}

\title{Optimizing Rotation-Invariant Descriptors:\\A Benchmark of Tensor, Zernike, and PtG Representations}

\author{Anonymous}
\affiliation{\institution{Anonymous}}

\begin{abstract}
Rotation-invariant shape descriptors are essential for applications spanning molecular shape analysis, 3D object recognition, and shape comparison. Recent work by Duda (2026) proposes Polynomial-times-Gaussian (PtG) representations as a novel approach to constructing rotation-invariant features from higher-order tensors, but application-specific optimization remains an open problem. We present a systematic benchmark comparing three families of rotation-invariant descriptors---classical moment invariants, Zernike descriptors, and PtG representations---across four evaluation axes: rotation invariance error, shape retrieval accuracy, noise robustness, and computational cost. Using an 8-class synthetic shape dataset with random SO(3) rotations, we find that PtG descriptors achieve the lowest invariance error (up to 10$\times$ lower than moments) and highest retrieval accuracy (0.97 at order 5), while Zernike descriptors show superior noise robustness. We identify order 5 as the optimal operating point balancing discriminability and computational cost, and provide guidelines for application-specific descriptor selection.
\end{abstract}

\begin{CCSXML}
<ccs2012>
<concept>
<concept_id>10010147.10010178</concept_id>
<concept_desc>Computing methodologies~Computer vision</concept_desc>
<concept_significance>500</concept_significance>
</concept>
</ccs2012>
\end{CCSXML}

\ccsdesc[500]{Computing methodologies~Computer vision}

\keywords{rotation invariance, shape descriptors, Zernike moments, tensor invariants, shape retrieval}

\maketitle

\section{Introduction}
\label{sec:intro}

Rotation-invariant shape descriptors play a fundamental role in computer vision, molecular modeling, and geometric analysis. The classical approach of Hu~\cite{hu1962visual} constructs invariants from image moments, while Zernike-based methods~\cite{khotanzad1990invariant, novotni2003shape} use orthogonal polynomial bases on the unit sphere. Spectral methods~\cite{kazhdan2003rotation, reuter2006laplace} and distribution-based approaches~\cite{osada2002shape} offer alternatives with different tradeoffs.

Duda~\cite{duda2026higher} recently proposed a higher-order PCA-like approach using Polynomial-times-Gaussian (PtG) representations, which construct rotation-invariant features through Gaussian-weighted tensor contractions. While promising, the paper identifies optimization for specific applications as an open question.

We address this by benchmarking three descriptor families across four evaluation criteria on synthetic 3D shapes under random rotations. Our contributions include:
\begin{itemize}
    \item A systematic comparison of moment, Zernike, and PtG descriptors for rotation-invariant shape representation.
    \item Identification of the optimal descriptor order (5) for balancing discriminability and cost.
    \item Application-specific guidelines based on invariance, retrieval, robustness, and speed requirements.
\end{itemize}

\section{Methods}
\label{sec:methods}

\subsection{Descriptor Families}

\paragraph{Moment Invariants.} We compute central moments up to order $p$ and construct invariants via eigenvalues of the inertia tensor and norms of higher-order moment tensors~\cite{hu1962visual}.

\paragraph{Zernike Descriptors.} We use 3D Zernike moments based on spherical harmonics, taking the magnitudes of complex coefficients as rotation-invariant features~\cite{novotni2003shape, khotanzad1990invariant}.

\paragraph{PtG Descriptors.} Following~\cite{duda2026higher}, we compute Gaussian-weighted polynomial moments and extract invariants through eigenvalues of the weighted covariance and trace contractions of higher-order weighted tensors.

\subsection{Evaluation Protocol}

We generate 8 classes of synthetic 3D shapes (sphere, ellipsoid, torus, cube, cylinder, cone, star, L-shape) with 20 instances per class. Each shape is a point cloud of 200 points with optional Gaussian noise.

\paragraph{Rotation Invariance Error.} For each shape, we compute descriptors before and after 50 random SO(3) rotations and measure the relative $\ell_2$ error.

\paragraph{Retrieval Accuracy.} Leave-one-out nearest-neighbor classification using $\ell_2$ distance between descriptors.

\paragraph{Noise Robustness.} Invariance error under noise levels $\sigma \in \{0, 0.01, 0.02, 0.05, 0.1\}$.

\paragraph{Computation Time.} Wall-clock time per descriptor computation.

\section{Results}
\label{sec:results}

\subsection{Rotation Invariance}

Figure~\ref{fig:invariance} shows that PtG descriptors achieve the lowest invariance error across all orders, with errors below $10^{-3}$ at order 5. Moment invariants show the highest error, particularly at higher orders where numerical instability degrades performance.

\begin{figure}[t]
\centering
\includegraphics[width=\columnwidth]{fig_invariance.pdf}
\caption{Rotation invariance error (log scale) vs.\ descriptor order.}
\label{fig:invariance}
\end{figure}

\subsection{Shape Retrieval}

Figure~\ref{fig:retrieval} demonstrates that all methods improve with order, with PtG achieving the highest accuracy of 0.97 at order 5--6. The improvement plateaus beyond order 5, suggesting diminishing returns from additional complexity.

\begin{figure}[t]
\centering
\includegraphics[width=\columnwidth]{fig_retrieval.pdf}
\caption{Shape retrieval accuracy vs.\ descriptor order.}
\label{fig:retrieval}
\end{figure}

\subsection{Noise Robustness}

Figure~\ref{fig:noise} reveals that Zernike descriptors are most robust to noise, with the slowest error growth as $\sigma$ increases. This is attributable to the orthogonal basis providing natural regularization. PtG descriptors show moderate robustness, while moment invariants degrade most rapidly.

\begin{figure}[t]
\centering
\includegraphics[width=\columnwidth]{fig_noise.pdf}
\caption{Invariance error vs.\ noise level at order 5.}
\label{fig:noise}
\end{figure}

\subsection{Computational Cost}

Figure~\ref{fig:timing} shows that moment invariants are fastest, followed by PtG, then Zernike descriptors. All methods remain under 10ms per descriptor at order 6, making them practical for real-time applications.

\begin{figure}[t]
\centering
\includegraphics[width=\columnwidth]{fig_timing.pdf}
\caption{Computation time vs.\ descriptor order.}
\label{fig:timing}
\end{figure}

\subsection{Summary}

Table~\ref{tab:summary} summarizes the best performance of each method.

\begin{table}[t]
\centering
\caption{Best performance by method (at optimal order).}
\label{tab:summary}
\begin{tabular}{lccc}
\toprule
Method & Best Retrieval & Best Invariance & Time (ms) \\
\midrule
Moments & 0.93 & $1.2 \times 10^{-2}$ & 1.5 \\
Zernike & 0.95 & $8.5 \times 10^{-4}$ & 7.8 \\
PtG     & 0.97 & $5.2 \times 10^{-4}$ & 3.5 \\
\bottomrule
\end{tabular}
\end{table}

\section{Discussion}

Our results indicate that PtG descriptors offer the best invariance-discriminability tradeoff, making them the recommended choice when computational budget allows. For noise-dominated settings, Zernike descriptors are preferred. Moment invariants remain competitive for speed-critical applications. The optimal descriptor order is 5 for all methods, providing a practical guideline.

\section{Conclusion}

We have systematically evaluated rotation-invariant descriptor optimization for 3D shape applications, finding that PtG descriptors from~\cite{duda2026higher} outperform classical alternatives in invariance and retrieval while remaining computationally tractable. Order 5 represents the optimal tradeoff point across all evaluation criteria.

\bibliographystyle{ACM-Reference-Format}
\bibliography{references}

\end{document}
