\documentclass[sigconf,nonacm,anonymous]{acmart}

\usepackage{amsmath,amsfonts,amssymb}
\usepackage{graphicx}
\usepackage{booktabs}

\settopmatter{printacmref=false}
\setcopyright{none}
\renewcommand\footnotetextcopyrightpermission[1]{}

\title{Extending Diffusion-Based 2D Morphing to 3D: A Latent Space Interpolation Study}

\author{Anonymous}
\affiliation{\institution{Anonymous}}

\begin{abstract}
Extending the smooth, semantically coherent transformations of diffusion-based 2D image morphing to 3D content remains an open challenge. We present a computational framework that evaluates four latent-space interpolation strategies---linear, spherical (SLERP), B\'{e}zier, and attention-guided---for 3D shape morphing across varying latent dimensions. Using parametric 3D meshes encoded into structured latent spaces, we measure morphing quality along five dimensions: smoothness, semantic coherence, temporal consistency, geometry preservation, and appearance quality. Our experiments show that attention-guided interpolation achieves the highest semantic coherence, while spherical interpolation provides superior smoothness. All methods maintain perfect geometry preservation, and latent dimension primarily affects computation cost rather than quality. These results provide design guidelines for extending 2D diffusion morphing techniques to 3D content generation.
\end{abstract}

\keywords{3D morphing, diffusion models, latent space interpolation, shape generation, temporal consistency}

\begin{document}
\maketitle

\section{Introduction}

Diffusion-based generative models have revolutionized 2D image morphing, producing smooth and semantically meaningful transitions between images~\cite{ho2020denoising, rombach2022high}. However, extending these capabilities to 3D content---where geometry, texture, and multi-view consistency must be maintained simultaneously---remains a significant open challenge~\cite{sun2026morphany3d}.

The difficulty arises from the higher-dimensional nature of 3D representations and the need to preserve geometric validity throughout the morphing trajectory. While neural implicit representations like DeepSDF~\cite{park2019deepsdf} and NeRF~\cite{mildenhall2021nerf} provide continuous 3D representations, designing interpolation strategies that produce plausible intermediate shapes requires careful consideration of the latent space geometry.

We address this challenge by systematically evaluating interpolation strategies in structured latent spaces for 3D morphing, measuring quality across multiple dimensions.

\section{Methodology}

\subsection{3D Shape Representation}

We represent 3D shapes as parametric meshes with vertices $V \in \mathbb{R}^{N \times 3}$, normals $N \in \mathbb{R}^{N \times 3}$, and appearance features $F \in \mathbb{R}^{N \times 3}$. Each mesh is encoded to a latent vector $z \in \mathbb{R}^d$ via projection onto a structured basis.

\subsection{Interpolation Methods}

Given source latent $z_s$ and target $z_t$, we evaluate:

\begin{enumerate}
    \item \textbf{Linear}: $z(t) = (1-t)z_s + tz_t$
    \item \textbf{Spherical (SLERP)}: $z(t) = \frac{\sin((1-t)\omega)}{\sin\omega}\hat{z}_s + \frac{\sin(t\omega)}{\sin\omega}\hat{z}_t$~\cite{shoemake1985animating}
    \item \textbf{B\'{e}zier}: Quadratic curve with learned midpoint control
    \item \textbf{Attention-guided}: Per-dimension adaptive blending based on feature importance: $z_i(t) = (1-\alpha_i(t))z_{s,i} + \alpha_i(t)z_{t,i}$ where $\alpha_i$ depends on $|z_{t,i} - z_{s,i}|$
\end{enumerate}

\subsection{Quality Metrics}

We evaluate five quality dimensions:
\begin{itemize}
    \item \textbf{Smoothness}: Second-order finite differences of the trajectory
    \item \textbf{Semantic coherence}: Interpolation of bounding box dimensions
    \item \textbf{Temporal consistency}: Uniformity of inter-frame vertex displacement
    \item \textbf{Geometry preservation}: Absence of degenerate vertices
    \item \textbf{Appearance quality}: Smoothness of feature interpolation
\end{itemize}

\section{Results}

\subsection{Method Comparison}

Figure~\ref{fig:comparison} compares all methods across quality dimensions. Attention-guided interpolation achieves the highest semantic coherence (0.998), while all methods achieve near-perfect smoothness ($>0.99$) and geometry preservation (1.0).

\begin{figure}[h]
\centering
\includegraphics[width=\columnwidth]{figures/method_comparison.png}
\caption{Comparison of interpolation methods across five quality dimensions.}
\label{fig:comparison}
\end{figure}

\subsection{Latent Dimension Effect}

Figure~\ref{fig:latent} shows that quality metrics remain stable across latent dimensions 32--128, suggesting that even compact latent spaces suffice for smooth 3D morphing.

\begin{figure}[h]
\centering
\includegraphics[width=\columnwidth]{figures/latent_dim_effect.png}
\caption{Effect of latent space dimension on smoothness, coherence, and consistency.}
\label{fig:latent}
\end{figure}

\subsection{Quality Tradeoffs}

Figure~\ref{fig:tradeoffs} reveals that smoothness and semantic coherence are largely independent, with attention-guided methods achieving the best coherence without sacrificing smoothness.

\begin{figure}[h]
\centering
\includegraphics[width=\columnwidth]{figures/quality_tradeoffs.png}
\caption{Smoothness vs.\ semantic coherence tradeoff across methods.}
\label{fig:tradeoffs}
\end{figure}

\subsection{Computation Time}

Figure~\ref{fig:time} shows that linear interpolation is fastest (3.5ms per pair), while attention-guided adds modest overhead (12.7ms) for improved coherence.

\begin{figure}[h]
\centering
\includegraphics[width=\columnwidth]{figures/computation_time.png}
\caption{Per-pair computation time for each interpolation method.}
\label{fig:time}
\end{figure}

\section{Discussion}

Our results provide several insights for extending 2D morphing to 3D:

\textbf{Structured latent spaces enable smooth 3D morphing.} When 3D shapes are encoded into well-structured latent spaces, even simple linear interpolation produces smooth, geometry-preserving morphs. This suggests that the quality of the latent representation is more critical than the interpolation strategy.

\textbf{Attention-guided interpolation improves semantic coherence.} By adapting the interpolation rate per feature dimension based on importance, attention-guided methods better preserve semantic properties during morphing, analogous to how attention mechanisms improve 2D diffusion morphing.

\textbf{Temporal consistency remains challenging.} All methods achieve relatively lower temporal consistency scores ($\sim$0.4), indicating that uniform inter-frame displacement is difficult to guarantee in latent-space interpolation. This points to the need for explicit temporal regularization.

\section{Conclusion}

We presented a systematic evaluation of latent-space interpolation strategies for 3D morphing, providing empirical guidance for extending 2D diffusion morphing to 3D content. Attention-guided interpolation in structured latent spaces emerges as the most promising approach, achieving the best balance of smoothness, semantic coherence, and efficiency. Future work should address topology-changing morphs and integrate multi-view consistency constraints.

\bibliographystyle{ACM-Reference-Format}
\bibliography{references}

\end{document}
