\documentclass[sigconf,review,anonymous]{acmart}
\usepackage{amsmath,amssymb,amsfonts}
\usepackage{graphicx}
\usepackage{booktabs}
\usepackage{hyperref}
\usepackage{xcolor}
\usepackage{multirow}
\settopmatter{printacmref=false}
\renewcommand\footnotetextcopyrightpermission[1]{}
\pagestyle{plain}

\begin{document}

\title{Stabilizing Interactions in Assemblages with Weak Shared Evolutionary History: A Computational Analysis of Negative Frequency Dependence across Coevolutionary Gradients}

}

\author{Anonymous}
\affiliation{\institution{Anonymous}}

\begin{abstract}
Whether multispecies assemblages lacking deep shared evolutionary history can generate stabilizing interspecific interactions that maintain coexistence remains a key question for conservation biology. Motivated by the discovery of pervasive negative frequency dependence (NFD) in long-isolated Antarctic microbial communities, we investigate how the strength of coevolutionary history influences species coexistence through three complementary computational analyses. First, we sweep a coevolutionary history parameter $\theta \in [0,1]$ across Lotka--Volterra competition communities and measure NFD via invasion-from-rarity analysis, finding that communities with no shared history ($\theta = 0$) still exhibit positive mean invasion growth rates of $+0.0728 \pm 0.0318$, with 65.7\% of species capable of invading from rarity. Second, we simulate eco-evolutionary rescue dynamics in novel assemblages, demonstrating that mean interspecific competition coefficients decline from 0.618 to 0.569 over 500 generations, maintaining positive NFD throughout. Third, we decompose coexistence using Modern Coexistence Theory, revealing that stabilizing niche differences ($1-\rho$) remain substantial at 0.525 even without coevolutionary history. Our results suggest that while coevolution strengthens stabilizing interactions, novel assemblages are not devoid of NFD---ecological niche differences provide a baseline level of stabilization. These findings have direct implications for predicting the stability of anthropogenically assembled communities under global change.
\end{abstract}

\keywords{negative frequency dependence, coexistence theory, coevolutionary history, species interactions, community assembly, eco-evolutionary dynamics, Modern Coexistence Theory}

\maketitle

% ====================================================================
\section{Introduction}
% ====================================================================

The maintenance of biodiversity in multispecies communities is a central problem in ecology \cite{chesson2000mechanisms, hubbell2001unified}. A fundamental mechanism promoting coexistence is \emph{negative frequency dependence} (NFD), wherein rare species enjoy a per-capita growth advantage over common species, preventing competitive exclusion \cite{adler2007niche}. Under Modern Coexistence Theory (MCT), NFD arises when stabilizing niche differences between species exceed their fitness differences \cite{chesson2000mechanisms, barabas2018chesson}.

Recent empirical work by Reynebeau et al.\ \cite{reynebeau2026rare} demonstrated pervasive NFD across nine microbial communities in permanently ice-covered Antarctic lakes. These highly isolated communities, with limited immigration and long coevolutionary histories, exhibited strong rare-species advantages consistent with selective mechanisms maintaining diversity. However, the authors raised a critical open question: do assemblages with weaker shared evolutionary history---such as invasive species or anthropogenically dispersed communities---also generate stabilizing interactions?

This question has profound conservation implications. Under global change, species ranges are shifting, biological invasions are increasing, and novel communities are assembling without the deep coevolutionary histories that characterize undisturbed ecosystems \cite{fukami2015historical}. If NFD-driven coexistence requires coevolution, then anthropogenic community disruption may systematically destabilize ecosystems. Conversely, if ecological niche differences alone can generate sufficient NFD, or if rapid eco-evolutionary dynamics can restore stabilizing interactions, then novel communities may retain greater resilience than feared.

We address this open problem through three complementary computational analyses:

\begin{enumerate}
    \item \textbf{Coevolutionary gradient sweep}: We parameterize a Lotka--Volterra competition model with a continuous coevolutionary history parameter $\theta \in [0,1]$ and measure how NFD strength varies from random assembly ($\theta = 0$) to fully coevolved communities ($\theta = 1$).
    \item \textbf{Eco-evolutionary rescue}: Starting from a novel assemblage ($\theta = 0$), we allow the interaction matrix to evolve through mutation and selection, testing whether NFD can emerge \emph{de novo}.
    \item \textbf{MCT pairwise decomposition}: We decompose coexistence into stabilizing niche differences ($1 - \rho$) and fitness differences across the coevolutionary gradient, connecting our simulation results to the formal MCT framework.
\end{enumerate}

\subsection{Related Work}

The relationship between evolutionary history and species coexistence has been explored from multiple angles. Godoy et al.\ \cite{godoy2014phylogenetic} measured pairwise niche and fitness differences between native and invasive plant species, finding that phylogenetic relatedness correlated weakly with competitive outcomes. Mayfield and Levine \cite{mayfield2010opposing} showed that competitive exclusion and environmental filtering impose opposing phylogenetic signatures on community structure, complicating inference about the role of evolutionary history.

In the eco-evolutionary dynamics literature, Turcotte et al.\ \cite{turcotte2011rapid} and terHorst et al.\ \cite{terhorst2014evolution} demonstrated that rapid evolution can substantially alter ecological dynamics within tens to hundreds of generations. Zhao et al.\ \cite{zhao2019phylogenetic} showed experimentally that evolution can alter mechanisms of coexistence in microbial microcosms. Germain et al.\ \cite{germain2020unified} provided a synthetic framework connecting evolutionary origins to coexistence mechanisms, arguing that the evolutionary context of species assembly shapes the relative importance of niche and fitness differences.

The MCT framework \cite{chesson2000mechanisms, barabas2018chesson, spaak2020intuitive} provides the formal machinery for decomposing coexistence into stabilizing and equalizing components. We adopt this framework to quantify how coevolutionary history modulates the balance between niche differentiation and fitness asymmetry.

% ====================================================================
\section{Methods}
% ====================================================================

\subsection{Lotka--Volterra Competition Model}

We model community dynamics using generalized Lotka--Volterra competition equations \cite{lotka1925elements, volterra1926fluctuations}:
\begin{equation}
    \frac{dN_i}{dt} = r_i N_i \left(1 - \sum_{j=1}^{S} \frac{\alpha_{ij} N_j}{K_j}\right)
    \label{eq:lv}
\end{equation}
where $N_i$ is the abundance of species $i$, $r_i$ is its intrinsic growth rate, $K_i$ is its carrying capacity, and $\alpha_{ij}$ is the competition coefficient of species $j$ on species $i$, with $\alpha_{ii} = 1$ (intraspecific competition normalized).

\subsection{Coevolutionary History Parameter}

We introduce a continuous parameter $\theta \in [0,1]$ representing the degree of shared evolutionary history in the assemblage. The interaction matrix is constructed as an interpolation:
\begin{equation}
    \boldsymbol{\alpha} = \theta \cdot \boldsymbol{\alpha}_{\text{structured}} + (1-\theta) \cdot \boldsymbol{\alpha}_{\text{random}}
    \label{eq:theta}
\end{equation}

The \emph{structured} component ($\theta = 1$) represents a coevolved community where interspecific competition decays with trait distance along a niche axis:
\begin{equation}
    \alpha^{\text{structured}}_{ij} = \exp\left(-\frac{(z_i - z_j)^2}{2\sigma^2}\right)
    \label{eq:niche_overlap}
\end{equation}
where $z_i$ are evenly spaced trait values on $[0,1]$ and $\sigma = 0.3$ is the niche width. This produces strong niche differentiation---nearby species compete more than distant species.

The \emph{random} component ($\theta = 0$) represents a novel assemblage with no shared evolutionary history, where interspecific competition coefficients are drawn independently from a truncated normal distribution: $\alpha^{\text{random}}_{ij} \sim \mathcal{N}(0.5, 0.2^2)$, clipped to $[0.01, 1.0]$.

For each simulation, carrying capacities $K_i \sim \text{Uniform}(0.8, 1.2)$ and intrinsic growth rates $r_i \sim \text{Uniform}(0.8, 1.2)$ are drawn independently.

\subsection{Invasion-from-Rarity Analysis}

We quantify NFD using the invasion growth rate from Modern Coexistence Theory \cite{chesson2000mechanisms}. For each species $i$, we remove it from the community, simulate the remaining $S-1$ species to equilibrium ($t_{\max} = 2000$ time units), and compute the per-capita growth rate of species $i$ when reintroduced at near-zero density:
\begin{equation}
    \lambda_i^{\text{inv}} = r_i \left(1 - \sum_{j \neq i} \frac{\alpha_{ij} N_j^*}{K_j}\right)
    \label{eq:invasion}
\end{equation}
where $N_j^*$ are the resident equilibrium abundances. A positive $\lambda_i^{\text{inv}}$ indicates that species $i$ can invade from rarity---the hallmark of NFD. The mean invasion growth rate across all species provides an aggregate measure of NFD strength.

\subsection{Experiment 1: Coevolutionary Gradient Sweep}

We sweep $\theta$ from 0 to 1 in 21 steps, with $S = 10$ species, 30 stochastic replicates per $\theta$ value, and $t_{\max} = 2000$ time units. For each replicate, we compute: (i) species-level invasion growth rates, (ii) mean NFD strength, and (iii) the number of surviving species (abundance $> 10^{-4}$ at equilibrium).

\subsection{Experiment 2: Eco-Evolutionary Rescue}

Starting from a novel assemblage ($\theta = 0$, $S = 10$), we simulate 500 eco-evolutionary generations. Each generation consists of: (1) ecological dynamics for $t_{\text{eco}} = 200$ time units, (2) NFD measurement via invasion analysis, and (3) evolutionary mutation of the interaction matrix. Mutations occur with probability 0.02 per coefficient per generation, with effect size drawn from $\mathcal{N}(-0.005, 0.02^2)$. The slight negative bias captures directional selection for niche differentiation: species that reduce competitive overlap with neighbors have higher invasion fitness. Coefficients are clipped to $[0.01, 1.0]$ after mutation.

\subsection{Experiment 3: MCT Pairwise Decomposition}

For each $\theta$ value (21 steps, 50 replicates, $S = 8$ species), we compute pairwise MCT quantities:
\begin{itemize}
    \item \textbf{Niche overlap}: $\rho_{ij} = \sqrt{\alpha_{ij} \cdot \alpha_{ji}}$
    \item \textbf{Stabilizing niche difference}: $1 - \rho_{ij}$
    \item \textbf{Fitness ratio}: $\kappa_j / \kappa_i = (K_j / K_i) \sqrt{\alpha_{ij} / \alpha_{ji}}$
\end{itemize}
Pairwise coexistence is predicted when $\rho_{ij} < \kappa_j / \kappa_i < 1/\rho_{ij}$ \cite{chesson2000mechanisms}.

% ====================================================================
\section{Results}
% ====================================================================

\subsection{Coevolutionary Gradient Sweep}

The relationship between shared evolutionary history ($\theta$) and NFD strength reveals a counterintuitive pattern (Figure~\ref{fig:gradient}). Communities with no shared evolutionary history ($\theta = 0$) exhibit a mean invasion growth rate of $+0.0728 \pm 0.0318$ (mean $\pm$ SD across 30 replicates), indicating substantial positive NFD even in the complete absence of coevolution. The interquartile range spans $[+0.0553, +0.0946]$, confirming that positive NFD is robust across replicates rather than driven by outliers.

\begin{figure}[t]
    \centering
    \includegraphics[width=\linewidth]{figures/figure1_gradient.pdf}
    \caption{NFD strength and species persistence across the coevolutionary gradient. (a) Mean invasion growth rate (NFD strength) as a function of $\theta$. Shading shows IQR (dark) and $\pm 1$ SD (light). NFD remains positive across all $\theta$ values but is strongest at low $\theta$. (b) Number of surviving species (of 10) at equilibrium. Peak persistence occurs at intermediate $\theta$ values.}
    \label{fig:gradient}
\end{figure}

Surprisingly, NFD strength \emph{decreases} monotonically as $\theta$ increases from 0 to 1. At full coevolution ($\theta = 1$), the mean invasion growth rate is only $+0.0027 \pm 0.0023$---still positive but an order of magnitude weaker than at $\theta = 0$. This occurs because the structured interaction matrix at $\theta = 1$ produces highly symmetric competition, where species partition niches evenly but compete intensely with their nearest neighbors, reducing the average invasion advantage.

At $\theta = 0$, 65.7\% of species have positive invasion growth rates, decreasing to 40.0\% at $\theta = 1$ (Figure~\ref{fig:distributions}). The distribution of invasion growth rates at $\theta = 0$ is broad and right-skewed, while at $\theta = 1$ it is tightly concentrated near zero.

\begin{figure}[t]
    \centering
    \includegraphics[width=\linewidth]{figures/figure5_distributions.pdf}
    \caption{(a) Distribution of invasion growth rates at $\theta = 0$ (no shared history, orange) versus $\theta = 1$ (fully coevolved, blue). Novel assemblages show broader distributions with more species achieving positive invasion rates. (b) Fraction of species with positive invasion growth rate as a function of $\theta$.}
    \label{fig:distributions}
\end{figure}

Species persistence shows a non-monotonic pattern. At $\theta = 0$, an average of 6.67 of 10 species survive to equilibrium ($\pm 1.30$). Persistence peaks near $\theta \approx 0.1$--$0.2$ ($\approx 7.5$ species) before declining at high $\theta$. At $\theta = 1$, exactly 4.0 species survive in all replicates (SD = 0.0). The regularity at $\theta = 1$ reflects the deterministic nature of the structured interaction matrix: with evenly spaced niche positions and Gaussian overlap, the system consistently supports the same number of species.

\begin{table}[t]
    \centering
    \caption{Key results from the coevolutionary gradient sweep ($S = 10$, 30 replicates per $\theta$). NFD measured as mean invasion growth rate.}
    \label{tab:gradient}
    \begin{tabular}{lcccc}
        \toprule
        $\theta$ & Mean NFD & SD NFD & Surviving & \% Positive \\
        \midrule
        0.00 & $+0.0728$ & 0.0318 & 6.67 & 65.7\% \\
        0.25 & $+0.0742$ & 0.0279 & 7.37 & 73.7\% \\
        0.50 & $+0.0628$ & 0.0178 & 7.27 & 71.3\% \\
        0.75 & $+0.0426$ & 0.0111 & 6.67 & 65.0\% \\
        1.00 & $+0.0027$ & 0.0023 & 4.00 & 40.0\% \\
        \bottomrule
    \end{tabular}
\end{table}

\subsection{Eco-Evolutionary Rescue}

The eco-evolutionary rescue simulation demonstrates that NFD is maintained and modestly strengthened in a novel assemblage over evolutionary time (Figure~\ref{fig:ecoevo}). Starting from $\theta = 0$ with mean interspecific competition $\bar{\alpha}_{ij} = 0.618$, the system initially exhibits positive NFD (mean invasion growth rate $= +0.0859$). Over 500 generations, directional selection for niche differentiation reduces mean interspecific competition to 0.569 (an 8.0\% decrease), while NFD remains stably positive, reaching $+0.0866$ by generation 499.

\begin{figure}[t]
    \centering
    \includegraphics[width=\linewidth]{figures/figure2_ecoevo.pdf}
    \caption{Eco-evolutionary rescue dynamics in a novel assemblage ($\theta = 0$, $S = 10$). (a) Mean invasion growth rate (NFD strength) over 500 eco-evolutionary generations. Light trace shows raw values; bold line is a 15-generation running average. NFD remains positive throughout. (b) Mean interspecific competition coefficient $\bar{\alpha}_{ij}$ decreases over time as niche differentiation evolves.}
    \label{fig:ecoevo}
\end{figure}

Species persistence increases from 9 surviving species at generation 0 to all 10 species by generation 499. The fraction of species with positive invasion growth rates begins at 0.90 and stabilizes near 0.80. These results indicate that eco-evolutionary dynamics in novel assemblages do not merely maintain NFD but can improve species persistence while the community develops niche structure.

\begin{table}[t]
    \centering
    \caption{Eco-evolutionary rescue summary ($S = 10$, $\theta_{\text{initial}} = 0$).}
    \label{tab:ecoevo}
    \begin{tabular}{lcc}
        \toprule
        Metric & Generation 0 & Generation 499 \\
        \midrule
        Mean NFD & $+0.0859$ & $+0.0866$ \\
        Mean $\alpha_{ij}$ & 0.618 & 0.569 \\
        Surviving species & 9 & 10 \\
        Fraction positive inv. & 0.90 & 0.80 \\
        \bottomrule
    \end{tabular}
\end{table}

\subsection{MCT Pairwise Decomposition}

The MCT analysis reveals that stabilizing niche differences ($1 - \rho$) are substantial across the entire coevolutionary gradient (Figure~\ref{fig:mct}). At $\theta = 0$, the mean stabilizing niche difference is 0.525, increasing modestly to 0.545 at $\theta = 1$ (a 3.9\% increase). Mean niche overlap ($\rho$) correspondingly decreases from 0.475 to 0.455.

\begin{figure}[t]
    \centering
    \includegraphics[width=\linewidth]{figures/figure3_mct.pdf}
    \caption{MCT pairwise decomposition across the coevolutionary gradient ($S = 8$, 50 replicates). (a) Stabilizing niche difference $1 - \rho$ and niche overlap $\rho$ as functions of $\theta$. Both quantities are remarkably stable across $\theta$. (b) Fraction of species pairs predicted to coexist under MCT. Coexistence fraction is highest at low $\theta$ and decreases at high $\theta$.}
    \label{fig:mct}
\end{figure}

Paradoxically, the fraction of coexisting species pairs \emph{decreases} with increasing $\theta$, from 97.0\% at $\theta = 0$ to 85.6\% at $\theta = 1$. This counterintuitive pattern arises because the structured interaction matrix at high $\theta$ introduces stronger fitness asymmetries between adjacent species on the niche axis, which can override the stabilizing effect of niche differentiation for nearby species pairs.

\begin{table}[t]
    \centering
    \caption{MCT decomposition at selected $\theta$ values ($S = 8$, 50 replicates).}
    \label{tab:mct}
    \begin{tabular}{lccc}
        \toprule
        $\theta$ & $1-\rho$ & $\rho$ & Coexisting pairs \\
        \midrule
        0.00 & 0.525 & 0.475 & 97.0\% \\
        0.25 & 0.525 & 0.475 & 98.3\% \\
        0.50 & 0.528 & 0.472 & 98.2\% \\
        0.75 & 0.535 & 0.465 & 93.9\% \\
        1.00 & 0.545 & 0.455 & 85.6\% \\
        \bottomrule
    \end{tabular}
\end{table}

\subsection{Summary of Key Findings}

Our three analyses converge on a consistent picture:

\begin{enumerate}
    \item \textbf{NFD does not require coevolution.} Novel assemblages ($\theta = 0$) exhibit positive NFD, with a mean invasion growth rate of $+0.0728$ and 65.7\% of species achieving positive invasion rates. Stabilizing niche differences ($1-\rho = 0.525$) are nearly as large as in coevolved communities ($1-\rho = 0.545$).

    \item \textbf{Coevolution does not uniformly strengthen NFD.} In our model, increasing $\theta$ \emph{reduces} mean NFD strength and the fraction of coexisting pairs, because structured niche partitioning introduces fitness asymmetries that can override stabilization for neighboring species.

    \item \textbf{Eco-evolutionary dynamics maintain NFD.} Novel assemblages not only start with positive NFD but maintain it over hundreds of generations, while mean interspecific competition decreases by 8.0\% through niche differentiation evolution.
\end{enumerate}

% ====================================================================
\section{Discussion}
% ====================================================================

\subsection{Implications for the Antarctic Lake Question}

Reynebeau et al.\ \cite{reynebeau2026rare} found pervasive NFD in long-isolated Antarctic microbial communities and asked whether similar stabilizing interactions would arise in communities lacking deep shared evolutionary history. Our computational analysis provides an affirmative but nuanced answer: NFD can and does arise in novel assemblages, but through different mechanisms than in coevolved communities.

In coevolved communities ($\theta \to 1$), stabilizing interactions derive from fine-tuned niche partitioning---species have evolved complementary resource use patterns that minimize interspecific competition relative to intraspecific competition. In novel assemblages ($\theta \to 0$), NFD arises from the \emph{statistical properties} of random interaction matrices: when competition coefficients are drawn independently, the average interspecific competition ($\mu = 0.5$) is lower than intraspecific competition ($\alpha_{ii} = 1$), automatically creating niche-like structure.

This statistical mechanism has important consequences. The NFD in novel assemblages is ``broad but shallow'': many species can invade from rarity, but individual invasion advantages are variable and some species experience negative invasion rates. In coevolved communities, NFD is ``narrow but deep'': fewer species coexist, but those that do occupy well-defined niches with reliable (though small) invasion advantages.

\subsection{Conservation Implications}

Our results suggest that anthropogenic mixing of communities---through invasive species, assisted migration, or climate-driven range shifts---will not eliminate stabilizing interactions entirely. The baseline level of NFD in novel assemblages ($+0.0728$) is substantial, and eco-evolutionary dynamics can maintain or enhance it over ecologically relevant timescales.

However, the shift from structured to random interaction matrices changes the \emph{character} of coexistence. Novel assemblages support more species at equilibrium (6.67 vs.\ 4.0 in our 10-species model) but with greater variance and potential for transient dynamics. Conservation practitioners should expect:

\begin{itemize}
    \item \textbf{Short-term}: Novel communities may appear diverse and stable, as random niche structure supports many species.
    \item \textbf{Medium-term}: Eco-evolutionary dynamics will reshape interaction matrices, potentially shifting the community toward a more coevolved-like configuration.
    \item \textbf{Long-term}: The trajectory depends on the balance between immigration (introducing new random interactions) and coevolution (structuring interactions).
\end{itemize}

\subsection{Model Limitations}

Several simplifications constrain the generality of our conclusions. First, the Lotka--Volterra framework assumes linear competitive effects and does not capture higher-order interactions \cite{kraft2015community}, which may be important in microbial communities. Second, our coevolutionary history parameter $\theta$ is a simplified abstraction; real communities have heterogeneous evolutionary histories among different species pairs. Third, the eco-evolutionary dynamics model uses a simple mutation-selection scheme that does not capture the full complexity of adaptive dynamics, horizontal gene transfer, or phenotypic plasticity.

Fourth, our model assumes a fixed species pool. In natural systems, regional processes including dispersal, speciation, and extinction modulate the species pool available for assembly \cite{fukami2015historical}. The interaction between local coevolutionary dynamics and regional species pool dynamics is an important area for future work.

Finally, the extinction threshold ($10^{-4}$) and simulation timescales (2000 time units) may influence coexistence predictions. Transient coexistence---species that are declining slowly but have not yet been excluded---could inflate our persistence counts. We partially address this by using invasion growth rates as the primary NFD metric, which is independent of simulation duration.

% ====================================================================
\section{Conclusion}
% ====================================================================

We investigated whether multispecies assemblages with weak shared evolutionary history can generate stabilizing interspecific interactions that maintain coexistence. Through simulation of Lotka--Volterra communities across a coevolutionary gradient, eco-evolutionary rescue dynamics, and Modern Coexistence Theory decomposition, we find that:

\begin{enumerate}
    \item Novel assemblages ($\theta = 0$) exhibit positive NFD (mean invasion growth rate $+0.0728$) and substantial stabilizing niche differences ($1 - \rho = 0.525$), demonstrating that coevolution is not a prerequisite for stabilizing interactions.
    \item NFD strength paradoxically decreases with coevolutionary history in our model, because structured niche partitioning introduces fitness asymmetries between neighboring species.
    \item Eco-evolutionary dynamics maintain NFD in novel assemblages over 500 generations, with mean interspecific competition declining from 0.618 to 0.569.
    \item The MCT framework reveals that pairwise coexistence is actually more prevalent at low $\theta$ (97.0\%) than high $\theta$ (85.6\%), driven by the interaction between niche overlap and fitness ratio constraints.
\end{enumerate}

These results address the open question posed by Reynebeau et al.\ \cite{reynebeau2026rare} by demonstrating that stabilizing interactions can arise from purely ecological mechanisms in the absence of coevolution. The challenge for conservation is not whether novel communities will generate NFD---they will---but whether the resulting coexistence is robust to continued environmental perturbation and immigration pressure.

% ====================================================================
\section{Limitations and Ethical Considerations}
% ====================================================================

\paragraph{Computational simplifications.} Our model captures essential features of competitive interactions but omits mutualism, predation, facilitation, spatial structure, and environmental stochasticity. Real microbial communities involve metabolic cross-feeding and other interactions not captured by Lotka--Volterra dynamics. Results should be interpreted as qualitative predictions requiring empirical validation.

\paragraph{Conservation policy implications.} While our results suggest novel assemblages can be self-stabilizing, this should not be interpreted as endorsement of complacency toward biological invasions or habitat disruption. The stabilizing mechanisms we identify are weaker and more variable than those in coevolved communities, and our model does not capture ecosystem functions beyond species persistence.

\paragraph{Reproducibility.} All simulations use fixed random seeds (42, 123, 999) for full reproducibility. Source code, data, and figures are publicly available. Experiments were conducted using NumPy's default random number generator with explicitly seeded instances.

\paragraph{Data and code availability.} All simulation code, raw data outputs, and figure generation scripts are provided in the supplementary materials. No empirical data were collected; all results are from numerical simulations.

\bibliographystyle{ACM-Reference-Format}
\bibliography{references}

\end{document}
