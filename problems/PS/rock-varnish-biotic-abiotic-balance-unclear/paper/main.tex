\documentclass[sigconf,review,anonymous]{acmart}
\settopmatter{printacmref=false}
\renewcommand\footnotetextcopyrightpermission[1]{}
\pagestyle{plain}
\usepackage{graphicx}\usepackage{booktabs}\usepackage{amsmath}

\begin{document}
\title{Quantifying Biotic versus Abiotic Contributions to Rock Varnish Formation Across Diverse Environments}

\author{Anonymous}
\affiliation{\institution{Anonymous}}

\begin{abstract}
The relative contributions of biotic and abiotic processes to rock varnish formation remain unclear. We develop process-based rate models for six formation pathways---three abiotic (dust leaching, photo-oxidation, silica gelation) and three biotic (microbial Mn oxidation, EPS templating, Fe oxidation)---and evaluate their balance across seven environmental scenarios. Under fiducial desert conditions (30$^\circ$C, 20\% RH, 100 mm/yr rainfall), abiotic processes dominate with a biotic fraction of 4.9\%, primarily driven by dust leaching. However, the biotic fraction increases dramatically with moisture availability: from 1.3\% in hot deserts to 48.4\% in coastal environments. Climate sensitivity analysis reveals rainfall as the primary control, with the biotic fraction rising from $<$1\% at 20 mm/yr to 30\% at 800 mm/yr. The Mn/Fe ratio ($\sim$1.6) in our simulations reflects the biotic enhancement of Mn over Fe. These results support a polygenetic model where the biotic/abiotic balance is fundamentally climate-dependent.
\end{abstract}

\maketitle

\section{Introduction}
Rock varnish is a thin ($\sim$1--500 $\mu$m) dark coating rich in Mn and Fe oxides found on rock surfaces in arid environments worldwide \cite{dorn1982}. Despite study dating back to Darwin, the formation mechanism remains debated, with evidence for both abiotic \cite{perry2006} and biotic \cite{liu2000, goldsmith2014} pathways. The balance between these processes likely varies with environmental conditions \cite{cartwright2026}, but quantitative assessments are lacking.

\section{Methods}
\subsection{Rate Models}
We model three abiotic processes: (1) dust leaching, releasing Mn/Fe from aeolian deposits; (2) photo-oxidation of dissolved Mn$^{2+}$ under UV; (3) silica gelation as a cementing agent. Three biotic processes are modeled: (1) enzymatic Mn$^{2+}$ oxidation by microorganisms; (2) extracellular polymeric substance (EPS) templating; (3) microbial Fe oxidation. Each rate depends on temperature, humidity, and substrate availability.

\subsection{Environmental Scenarios}
Seven environments are tested: hot desert, temperate arid, Mediterranean, cold desert, coastal, alpine, and tropical semi-arid, each with characteristic temperature, humidity, rainfall, UV, and dust flux.

\section{Results}

\subsection{Process Decomposition}
Under fiducial desert conditions, dust leaching dominates at 42.3 $\mu$g/cm$^2$/yr, while biotic Mn oxidation contributes 1.83 $\mu$g/cm$^2$/yr. The total biotic fraction is 4.9\% (Figure~\ref{fig:rates}).

\begin{figure}[h]
\centering
\includegraphics[width=\columnwidth]{figures/rate_decomposition.png}
\caption{Abiotic and biotic process rates under fiducial conditions.}
\label{fig:rates}
\end{figure}

\subsection{Environmental Variation}
The biotic fraction varies from 0.3\% (cold desert) to 48.4\% (coastal), driven by moisture availability and microbial density (Figure~\ref{fig:env}).

\begin{table}[h]
\centering
\caption{Biotic fraction across environments.}
\label{tab:env}
\begin{tabular}{lc}
\toprule
Environment & Biotic Fraction (\%) \\
\midrule
Hot Desert & 1.3 \\
Cold Desert & 0.3 \\
Temperate Arid & 8.5 \\
Alpine & 1.9 \\
Tropical Semi-Arid & 6.5 \\
Mediterranean & 24.2 \\
Coastal & 48.4 \\
\bottomrule
\end{tabular}
\end{table}

\begin{figure}[h]
\centering
\includegraphics[width=\columnwidth]{figures/environment_balance.png}
\caption{Biotic vs.\ abiotic balance across environments.}
\label{fig:env}
\end{figure}

\subsection{Climate Sensitivity}
Rainfall is the primary control on the biotic fraction, with temperature having a secondary effect (Figure~\ref{fig:clim}). The Mn/Fe ratio of $\sim$1.6 reflects biotic enhancement of Mn oxidation over Fe.

\begin{figure}[h]
\centering
\includegraphics[width=\columnwidth]{figures/climate_sensitivity.png}
\caption{Climate sensitivity of biotic fraction.}
\label{fig:clim}
\end{figure}

\section{Conclusion}
Our results support a polygenetic model of rock varnish formation where the biotic/abiotic balance is strongly climate-dependent. In classic hot desert settings, abiotic dust leaching dominates ($>$95\%), while in wetter environments, biotic contributions can reach nearly 50\%. This variability explains conflicting findings in the literature and suggests that universal biotic or abiotic explanations are inadequate.

\section{Limitations and Ethical Considerations}
Limitations include simplified rate parameterizations, absence of substrate mineralogy effects, steady-state assumptions, and difficulty validating model predictions against ancient varnish. This computational work poses no ethical concerns, but we note that rock varnish interpretation has cultural significance for dating rock art, where biased models could affect archaeological conclusions.

\bibliographystyle{ACM-Reference-Format}
\bibliography{references}
\end{document}
