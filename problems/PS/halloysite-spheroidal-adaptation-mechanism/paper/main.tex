\documentclass[sigconf,review,anonymous]{acmart}
\settopmatter{printacmref=false}
\renewcommand\footnotetextcopyrightpermission[1]{}
\setcopyright{none}
\acmConference{}{}{}
\acmDOI{}
\acmISBN{}

\usepackage{booktabs}
\usepackage{graphicx}
\usepackage{amsmath}

\begin{document}

\title{Structural Adaptation of Halloysite to Spheroidal Morphology: An Elastic Energy and Phase Diagram Analysis}

\author{Anonymous}
\affiliation{\institution{Anonymous}}

\begin{abstract}
How halloysite, a kaolin-group 1:1 aluminosilicate with rolled-layer morphology, adapts to form spheroidal particles remains an open question. We develop an elastic energy framework that computes bending energies for tubular and spheroidal morphologies as functions of lattice mismatch and interlayer hydration. A phase diagram across 25 mismatch values and 20 hydration levels shows tubes dominate at low hydration (63.0\%) while spheroids emerge at high hydration with confinement (36.4\%). The natural curvature from tetrahedral-octahedral mismatch is 0.0131~\AA$^{-1}$ (natural radius 76.5~\AA). Transition pathway analysis identifies critical hydration of 0.571 for the tube-to-spheroid transition, with aspect ratio decreasing from 5.0 to 1.0. Monte Carlo simulation (200 samples) predicts 84.0\% tubes and 16.0\% spheroids, with spheroid formation requiring mean hydration 0.597 and confinement 0.846 versus tube hydration 0.367 and confinement 0.438. These results suggest that spheroidal halloysite forms through hydration-driven reduction of effective lattice mismatch combined with spatial confinement in volcanic weathering environments.
\end{abstract}

\maketitle

\section{Introduction}

Halloysite is a kaolin-group 1:1 aluminosilicate that commonly forms tubular and prismatic particles due to lattice mismatch between its tetrahedral and octahedral sheets~\cite{joussein2005halloysite}. In many volcanic weathering environments, halloysite also occurs as spheroids, yet the mineral's layered structure appears difficult to reconcile with a closed spherical geometry~\cite{cartwright2026selfassembled}.

The mismatch between the larger tetrahedral sheet ($a = 5.14$~\AA) and the smaller octahedral sheet ($a = 5.06$~\AA) induces natural curvature that favors tubular rolling~\cite{singh1996transformation}. Dehydration can convert tubes to prisms, and spheroids are frequently observed in highly saturated, confined precipitation spaces~\cite{churchman1995formation, yuan2015properties}. However, the specific structural pathway from layered halloysite to spheroidal particles has not been resolved.

We address this open problem by developing an elastic energy framework that: (1) computes bending energies for tube and sphere morphologies; (2) maps the morphology phase diagram across mismatch and hydration; (3) analyzes the transition pathway; and (4) predicts morphology distributions via Monte Carlo simulation.

\section{Methods}

\subsection{Elastic Energy Model}
Layer curvature arises from the mismatch $\epsilon = (a_{\text{tet}} - a_{\text{oct}})/\bar{a}$. The natural curvature is $\kappa_0 = 6\epsilon/t$ where $t$ is the layer thickness. For tubes, the bending energy is $E_{\text{tube}} = \frac{1}{2}D(\kappa - \kappa_0)^2 A$ where $D = Et^3/[12(1-\nu^2)]$ is the bending stiffness ($E = 170$~GPa, $\nu = 0.25$). For spheres, the energy includes Gaussian curvature contributions. Hydration modifies the effective mismatch: $\epsilon_{\text{eff}} = \epsilon(1 - 0.3w)$ where $w$ is the water content, and swells the layer thickness to $t(1 + 0.4w)$.

\subsection{Phase Diagram}
We compute energies across 25 mismatch values (0.005--0.030) and 20 hydration levels (0.0--1.0), classifying each point as tube, spheroid, or prismatic based on relative total energies including hydration and confinement contributions.

\subsection{Monte Carlo Simulation}
We sample 200 random environments with beta-distributed hydration, normally-distributed mismatch ($\mu = 0.0156$, $\sigma = 0.003$), and uniform confinement.

\section{Results}

\subsection{Natural Curvature}
The tetrahedral-octahedral mismatch of halloysite produces natural curvature $\kappa_0 = 0.0131$~\AA$^{-1}$, corresponding to a natural tube radius of 76.5~\AA. The optimal tube bending energy is 4.573~eV at radius 82.0~\AA.

\subsection{Phase Diagram}

\begin{table}[h]
\centering
\caption{Morphology phase fractions across 500 parameter combinations (25 mismatch $\times$ 20 hydration values).}
\label{tab:phases}
\begin{tabular}{lcc}
\toprule
Morphology & Count & Fraction \\
\midrule
Tube & 315 & 0.630 \\
Spheroid & 182 & 0.364 \\
Prismatic & 3 & 0.006 \\
\bottomrule
\end{tabular}
\end{table}

Tubes dominate at low hydration levels, while spheroids emerge predominantly at hydration $> 0.6$ where confinement energy overcomes the Gaussian curvature penalty. The prismatic phase occupies a narrow transitional region.

\subsection{Transition Pathway}

The tube-to-spheroid transition occurs at critical hydration 0.571. Along the pathway, the aspect ratio decreases from 5.0 (tubular) to 1.0 (spheroidal), the effective mismatch decreases from 0.0157 to 0.0110, and the natural radius increases from 75.9~\AA~to 109.5~\AA~as hydration swells the interlayer.

\subsection{Monte Carlo Morphology Distribution}

\begin{table}[h]
\centering
\caption{Monte Carlo morphology distribution ($n = 200$).}
\label{tab:mc}
\begin{tabular}{lccc}
\toprule
Morphology & Count & Fraction & Mean Hydration \\
\midrule
Tube & 168 & 0.840 & 0.367 \\
Spheroid & 32 & 0.160 & 0.597 \\
\bottomrule
\end{tabular}
\end{table}

Spheroid formation is associated with significantly higher hydration (0.597 vs.\ 0.367) and confinement (0.846 vs.\ 0.438), consistent with observations that spheroids form in highly saturated, confined precipitation spaces in volcanic weathering environments.

\begin{table}[h]
\centering
\caption{Environmental conditions for each morphology from MC simulation.}
\label{tab:mcstats}
\begin{tabular}{lccc}
\toprule
Parameter & Tube & Spheroid \\
\midrule
Mean hydration & 0.367 & 0.597 \\
Mean mismatch & 0.01572 & 0.01542 \\
Mean confinement & 0.438 & 0.846 \\
\bottomrule
\end{tabular}
\end{table}

\section{Conclusion}

Our elastic energy framework provides a mechanistic explanation for the tube-to-spheroid transition in halloysite. The key findings are: (1) the natural curvature from tetrahedral-octahedral mismatch ($\kappa_0 = 0.0131$~\AA$^{-1}$) strongly favors tubes at tube radius 76.5~\AA; (2) hydration reduces effective mismatch by up to 30\%, lowering the energetic preference for tubular curvature; (3) confinement energy is the critical factor enabling spheroid formation, overcoming the Gaussian curvature penalty; and (4) spheroids require both high hydration ($> 0.57$) and high confinement ($> 0.85$ on average). This explains why spheroidal halloysite is observed specifically in volcanic weathering environments with saturated, confined pore spaces~\cite{cartwright2026selfassembled}.

\subsection{Limitations}
The model uses continuum elasticity rather than atomistic simulation, which may miss discrete layer effects. The confinement term is phenomenological. The hydration model simplifies the complex chemistry of interlayer water in halloysite. Experimental validation of the predicted critical hydration threshold and its dependence on confinement geometry is needed.

\bibliographystyle{ACM-Reference-Format}
\bibliography{references}

\end{document}
