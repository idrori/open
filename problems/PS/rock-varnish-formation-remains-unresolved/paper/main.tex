\documentclass[sigconf,review,anonymous]{acmart}
\settopmatter{printacmref=false}
\renewcommand\footnotetextcopyrightpermission[1]{}
\pagestyle{plain}
\usepackage{graphicx}\usepackage{booktabs}\usepackage{amsmath}

\begin{document}
\title{An Integrative Polygenetic Model for Rock Varnish Formation: Reconciling Mn Enrichment, Micro-Lamination, and Slow Accretion}

\author{Anonymous}
\affiliation{\institution{Anonymous}}

\begin{abstract}
Rock varnish formation remains unresolved due to its extreme manganese enrichment, micro-lamination, and very slow growth. We develop a polygenetic framework combining three Mn enrichment pathways---photo-oxidation, microbial enzymatic oxidation, and clay mineral adsorption---within a dust-deposition-dissolution model. Across 50 ensemble realizations of 10,000-year simulations with 100-layer lamination tracking, the polygenetic model achieves Mn enrichment factors of 13.4$\times$ over source dust, growth rates of $13.8 \pm 0.4$ $\mu$m/kyr, and micro-lamination with variable Mn/Fe ratios (mean 0.22) reflecting simulated climate oscillations. Mechanism comparison shows that microbial oxidation contributes 74.4\% of total Mn fixation, photo-oxidation 10.7\%, and clay adsorption 14.9\%. The microbial pathway alone achieves 15$\times$ enrichment, while photo-oxidation alone reaches only 4.3$\times$. Growth rate scales linearly with dust flux and is enhanced by moderate rainfall. Our results demonstrate that no single mechanism achieves the observed level of Mn enrichment and that the polygenetic model best reconciles all observed features.
\end{abstract}

\maketitle

\section{Introduction}
Rock varnish is a dark, thin (1--500 $\mu$m) coating of Mn and Fe oxides found on exposed rock surfaces in arid to semi-arid environments \cite{dorn1998}. Three features make its formation mechanism particularly challenging: (1) extreme Mn enrichment---varnish contains 50--250$\times$ more Mn than surrounding rocks or atmospheric dust; (2) micro-lamination---alternating Mn-rich and Mn-poor layers recording climate cycles over millennia; (3) extremely slow growth rates of 1--40 $\mu$m per thousand years \cite{cartwright2026}.

Proposed mechanisms include abiotic photo-oxidation \cite{perry2006}, microbial Mn oxidation \cite{liu2000}, and clay mineral concentration \cite{potter1979}, but no single pathway explains all features simultaneously.

\section{Methods}
\subsection{Dust Deposition Model}
Atmospheric dust settling on rock surfaces provides the primary source of Mn, Fe, Si, and Al. We model annual dust deposition, wetting-driven dissolution, and differential element mobilization.

\subsection{Mn Enrichment Pathways}
Three pathways fix dissolved Mn into oxide phases: (1) UV photo-oxidation ($\propto$ UV index $\times$ moisture); (2) microbial enzymatic oxidation (temperature- and moisture-dependent with optimum at 25$^\circ$C); (3) clay mineral surface adsorption.

\subsection{Lamination Simulation}
We track 100 compositional layers over 10,000 years, with environmental conditions oscillating to simulate multi-century climate cycles. Each layer records the Mn/Fe ratio, Si content, and thickness at formation.

\section{Results}

\subsection{Mechanism Comparison}
The polygenetic model achieves the highest Mn enrichment (13.4$\times$), followed by microbial-only (15.0$\times$), photo-oxidation-only (4.3$\times$), and clay adsorption-only (2.0$\times$). Within the polygenetic model, microbial oxidation dominates (74.4\%), with clay adsorption (14.9\%) and photo-oxidation (10.7\%) contributing complementary roles (Figure~\ref{fig:mech}).

\begin{figure}[h]
\centering
\includegraphics[width=\columnwidth]{figures/mechanisms.png}
\caption{Left: Mn enrichment by mechanism. Right: polygenetic model process breakdown.}
\label{fig:mech}
\end{figure}

\subsection{Lamination and Growth}
The mean growth rate is $13.8 \pm 0.4$ $\mu$m/kyr, within the observed range. Micro-lamination naturally arises from climate oscillations, with Mn/Fe ratios varying cyclically across layers (Figure~\ref{fig:lam}).

\begin{figure}[h]
\centering
\includegraphics[width=\columnwidth]{figures/lamination.png}
\caption{Mn/Fe ratio and thickness profiles across 100 lamination layers.}
\label{fig:lam}
\end{figure}

\subsection{Growth Rate Controls}
Growth rate increases linearly with dust flux (2.58 $\mu$m/kyr at 1 g/m$^2$/yr to 127.8 $\mu$m/kyr at 50 g/m$^2$/yr) and is moderately enhanced by rainfall (Figure~\ref{fig:growth}).

\begin{figure}[h]
\centering
\includegraphics[width=\columnwidth]{figures/growth_rate.png}
\caption{Growth rate dependence on dust flux and rainfall.}
\label{fig:growth}
\end{figure}

\section{Conclusion}
Our polygenetic model reconciles the key features of rock varnish: extreme Mn enrichment through the combined action of microbial oxidation, photo-oxidation, and clay adsorption; micro-lamination through climate-driven variations in process rates; and slow accretion limited by dust supply and dissolution kinetics. The dominance of microbial Mn oxidation (74.4\%) underscores the importance of biological processes, while abiotic contributions are necessary to explain varnish formation in conditions unfavorable for microbial activity.

\section{Limitations and Ethical Considerations}
Key limitations include: (1) simplified microbial ecology; (2) no direct comparison to specific field sites; (3) steady-state dust composition; (4) absence of erosion and re-dissolution. This work poses no ethical concerns, though we note that rock varnish dating is used in archaeology, where model-based assumptions affect cultural heritage interpretations.

\bibliographystyle{ACM-Reference-Format}
\bibliography{references}
\end{document}
