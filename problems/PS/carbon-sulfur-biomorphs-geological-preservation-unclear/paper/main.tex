\documentclass[sigconf,review,anonymous]{acmart}
\settopmatter{printacmref=false}
\renewcommand\footnotetextcopyrightpermission[1]{}
\setcopyright{none}
\acmConference{}{}{}
\acmDOI{}
\acmISBN{}

\usepackage{booktabs}
\usepackage{graphicx}
\usepackage{amsmath}

\begin{document}

\title{Preservation Potential of Carbon--Sulfur Biomorphs in the Geological Record: A Computational Diagenetic Framework}

\author{Anonymous}
\affiliation{\institution{Anonymous}}

\begin{abstract}
Carbon--sulfur biomorphs are self-assembled structures with elemental sulfur cores and organic macromolecular shells whose geological preservation potential remains uncertain. We develop an integrated computational framework modeling sulfur core diagenesis, organic shell degradation, and silicification to quantify preservation across 200 simulated geological environments. Both sulfur cores and organic shells exhibit short half-lives (0.02~Myr each), but silicification provides rapid preservation when dissolved silica exceeds saturation thresholds. Across environments, 70.5\% achieve ``good'' preservation through silicification, with hydrothermal (100.0\%) and deep marine (100.0\%) settings most favorable. Sensitivity analysis identifies dissolved silica concentration as the dominant control (importance 1.0). Monte Carlo uncertainty quantification yields a preservation rate of 79.5\% (95\% CI: [73.4\%, 84.5\%]). These results demonstrate that carbon--sulfur biomorphs can persist in the geological record under silica-rich conditions.
\end{abstract}

\maketitle

\section{Introduction}

Carbon--sulfur biomorphs are self-assembled structures formed in sulfidic, organic-containing solutions that produce lifelike morphologies~\cite{cartwright2026selfassembled, garciaruiz2003selfassembled}. They consist of elemental sulfur cores encapsulated by organic macromolecular shells formed via sulfurization reactions. Their morphological similarity to microfossils raises important questions for interpreting the early geological record~\cite{rouillard2018biomorphs}.

The key open problem is whether these biomorphs can persist in the geological record despite the diagenetic instability of elemental sulfur~\cite{cartwright2026selfassembled}. Sulfur is susceptible to oxidation, dissolution, and phase transformation during burial, while organic shells degrade through thermal and microbial processes~\cite{cady2003preservation}. However, silicification of organic envelopes may provide a preservation pathway~\cite{knoll2003fossils}.

We present an integrated computational framework that models: (1) sulfur core diagenesis kinetics including dissolution, oxidation, and phase transformations; (2) organic shell degradation with crosslinking protection from sulfurization; (3) silicification rates as a function of dissolved silica, pH, and temperature; and (4) preservation outcomes across diverse geological environments.

\section{Methods}

\subsection{Sulfur Core Diagenesis}
Sulfur core dissolution follows Arrhenius kinetics with rate $k_{\text{diss}} = k_0 \exp(-E_a/RT)$ where $k_0 = 1.5 \times 10^{-9}$~mol/m$^2$/s and $E_a = 50$~kJ/mol. Oxidation (abiotic: $2 \times 10^{-8}$~mol/m$^2$/s; microbial: $5 \times 10^{-7}$~mol/m$^2$/s) is enhanced in the oxic zone. Core radius evolution follows a shrinking-sphere model: $dr/dt = -V_m k_{\text{total}}$.

\subsection{Organic Shell Degradation}
Shell degradation combines thermal ($k_{\text{thermal}} = 10^{-11}$~s$^{-1}$, $E_a = 60$~kJ/mol) and microbial ($k_{\text{microbial}} = 5 \times 10^{-10}$~s$^{-1}$) components. Sulfurization crosslinking reduces the effective rate by a factor of up to 3.0, with initial crosslink degree of 0.5.

\subsection{Silicification Model}
Silicification rate follows $k_{\text{sil}} = 5 \times 10^{-12}$~mol/m$^2$/s ($E_a = 45$~kJ/mol), modulated by pH (Gaussian around optimum 8.5) and dissolved silica supersaturation. Critical silica coating thickness for preservation is 0.5~$\mu$m.

\subsection{Environment Survey}
We simulate 200 environments across six depositional settings (shallow marine, deep marine, lacustrine, hydrothermal, evaporite, deltaic) with log-normally distributed burial rates, dissolved silica concentrations, and normally distributed pH values.

\subsection{Sensitivity and Uncertainty}
Sobol-type sensitivity analysis uses Latin Hypercube Sampling across six parameters. Monte Carlo uncertainty quantification employs 200 random environments with Wilson score confidence intervals.

\section{Results}

\subsection{Baseline Diagenesis}

Under default conditions (burial rate 1.0~mm/yr, 1.0~mM dissolved silica, pH~8.5), the sulfur core half-life is 0.02~Myr and the organic shell half-life is 0.02~Myr. However, silicification achieves critical preservation thickness essentially immediately (time $<$0.01~Myr), yielding a final replacement fraction of 0.95.

\subsection{Environment Survey}

Table~\ref{tab:envsurvey} shows preservation outcomes across 200 simulated environments.

\begin{table}[h]
\centering
\caption{Preservation class distribution across 200 environments.}
\label{tab:envsurvey}
\begin{tabular}{lcc}
\toprule
Class & Count & Fraction \\
\midrule
Excellent & 0 & 0.000 \\
Good & 141 & 0.705 \\
Moderate & 0 & 0.000 \\
Poor & 0 & 0.000 \\
None & 59 & 0.295 \\
\bottomrule
\end{tabular}
\end{table}

The overall preservation rate is 70.5\%, with silicification winning the race against degradation in 70.5\% of environments.

\subsection{Preservation by Environment Type}

\begin{table}[h]
\centering
\caption{Preservation rate by depositional environment.}
\label{tab:envtype}
\begin{tabular}{lcc}
\toprule
Environment & $n$ & Preservation Rate \\
\midrule
Hydrothermal & 28 & 1.000 \\
Deep marine & 29 & 1.000 \\
Lacustrine & 36 & 0.972 \\
Deltaic & 36 & 0.778 \\
Shallow marine & 36 & 0.389 \\
Evaporite & 35 & 0.200 \\
\bottomrule
\end{tabular}
\end{table}

Hydrothermal and deep marine environments achieve 100.0\% preservation, driven by high dissolved silica concentrations. Evaporite settings show only 20.0\% preservation due to low silica availability.

\subsection{Timescale Scenarios}

\begin{table}[h]
\centering
\caption{Preservation outcomes for specific geological scenarios.}
\label{tab:timescales}
\begin{tabular}{lcccc}
\toprule
Scenario & Burial & Si (mM) & pH & Class \\
\midrule
Rapid burial + high Si & 10.0 & 5.0 & 8.5 & Good \\
Rapid burial + low Si & 10.0 & 0.3 & 7.5 & None \\
Slow burial + high Si & 0.5 & 5.0 & 8.5 & Good \\
Slow burial + low Si & 0.5 & 0.3 & 7.5 & None \\
Hydrothermal & 2.0 & 8.0 & 8.0 & Good \\
Lacustrine alkaline & 3.0 & 2.0 & 9.0 & Good \\
\bottomrule
\end{tabular}
\end{table}

The critical factor is dissolved silica rather than burial rate: high-silica environments achieve preservation regardless of burial speed, while low-silica environments fail regardless.

\subsection{Sensitivity Analysis}
Dissolved silica concentration emerges as the sole significant parameter (importance 1.0), with burial rate, pH, geothermal gradient, initial radius, and shell thickness all showing zero sensitivity. This confirms that silicification availability is the dominant control on preservation.

\subsection{Uncertainty Quantification}
Monte Carlo analysis (200 simulations) yields a preservation rate of 79.5\% with 95\% Wilson score confidence interval [73.4\%, 84.5\%].

\begin{figure}[h]
\centering
\includegraphics[width=\columnwidth]{figures/preservation_by_env.png}
\caption{Preservation rates across geological environments.}
\label{fig:envpres}
\end{figure}

\begin{figure}[h]
\centering
\includegraphics[width=\columnwidth]{figures/sensitivity.png}
\caption{Sensitivity analysis showing dissolved silica as the dominant control.}
\label{fig:sensitivity}
\end{figure}

\section{Conclusion}

Our computational framework demonstrates that carbon--sulfur biomorphs can persist in the geological record under silica-rich conditions, with an overall preservation rate of 70.5\% across diverse environments. The key findings are: (1) both sulfur cores and organic shells degrade rapidly (half-life 0.02~Myr), making rapid silicification essential; (2) dissolved silica concentration is the sole significant control on preservation (sensitivity importance 1.0); (3) hydrothermal and deep marine environments achieve 100\% preservation due to high dissolved silica; and (4) Monte Carlo uncertainty quantification gives a 79.5\% preservation rate (95\% CI: [73.4\%, 84.5\%]). These results directly address the open problem posed by Cartwright et al.~\cite{cartwright2026selfassembled}, confirming that silicification of organic envelopes is the critical preservation pathway.

\subsection{Limitations}
The model simplifies real diagenetic processes: sulfur phase transformations are reduced to kinetic rate expressions, microbial activity is parameterized rather than explicitly modeled, and porewater chemistry evolution during burial is not fully coupled. The sensitivity analysis shows dissolved silica as the sole control, which may reflect model structure rather than true geological complexity. Experimental validation of silicification rates for carbon--sulfur biomorphs is needed.

\bibliographystyle{ACM-Reference-Format}
\bibliography{references}

\end{document}
