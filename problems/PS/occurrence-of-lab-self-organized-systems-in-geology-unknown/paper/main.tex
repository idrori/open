\documentclass[sigconf,review,anonymous]{acmart}

\usepackage{graphicx}
\usepackage{amsmath}
\usepackage{booktabs}

\setcopyright{none}

\title{Predicting Geological Occurrence of Laboratory Self-Organized Chemical Systems: A Computational Feasibility Framework}

\author{Anonymous}
\affiliation{\institution{Anonymous}}

\begin{abstract}
Several self-organized chemical pattern-forming systems produce lifelike morphologies in laboratory settings, yet their occurrence in natural geological environments remains uncertain. We develop a computational feasibility framework assessing four systems---chemical gardens, silica--carbonate biomorphs, carbon--sulfur biomorphs, and organic biomorphs---across 300 simulated geological environments spanning 10 types. Chemical gardens show a feasibility rate of 0.6867 (mean score 0.6646) and rank first in composite evidence scoring (0.8367), consistent with their confirmed geological occurrence at hydrothermal vents. Silica--carbonate biomorphs achieve feasibility rate 0.8000 (composite 0.6709), with alkaline lakes and serpentinization sites as prime targets for field confirmation. Carbon--sulfur biomorphs (rate 0.9833, composite 0.6276) and organic biomorphs (rate 0.9967, composite 0.6341) show high thermodynamic feasibility but lack geological confirmation. Co-occurrence analysis reveals a mean of 3.4633 feasible systems per environment, with 99.67\% of environments supporting multiple systems. Sensitivity analysis identifies dissolved metals ($S_1 = 0.30$) and dissolved silica as the dominant controls for chemical gardens, while pH ($S_1 = 0.29$) drives silica--carbonate biomorph feasibility. Bootstrap uncertainty analysis yields 95\% confidence intervals of $[0.64, 0.74]$ for chemical garden feasibility. This framework provides quantitative criteria for prioritizing field investigations to close the lab-to-geology gap.
\end{abstract}

\keywords{self-organization, chemical gardens, biomorphs, geological occurrence, feasibility assessment, pattern formation}

\begin{document}
\maketitle

\section{Introduction}

Abiotic self-organized chemical systems can produce complex morphologies resembling biological structures, creating a fundamental challenge for interpreting putative biosignatures in the geological record \cite{cartwright2026self, mcmahon2018false}. Four major system types have been demonstrated in laboratory settings: chemical gardens \cite{barge2015chemical}, silica--carbonate biomorphs \cite{garcia2003silica}, carbon--sulfur biomorphs, and organic biomorphs. While chemical gardens have clear geological counterparts in hydrothermal chimney structures \cite{kelley2005serpentinite, corliss1981submarine}, the natural occurrence of the other three systems remains hypothesized or unconfirmed.

We present a computational framework that systematically evaluates the thermodynamic and kinetic feasibility of each system across diverse geological environments, identifies the geochemical parameters controlling their formation, and provides ranked predictions to guide future field investigations.

\section{Methods}

\subsection{System Feasibility Models}

For each of the four self-organized systems, we define a multi-factor feasibility score $\mathcal{F}_s$ as a weighted combination of geochemical parameters:
\begin{equation}
\mathcal{F}_s = \sum_{k} w_k \cdot f_k(x_k)
\end{equation}
where $f_k$ maps environmental parameter $x_k$ to a $[0,1]$ score and $w_k$ are domain-informed weights. System-specific scoring functions capture the distinct geochemical requirements: chemical gardens require dissolved metals and silicate gradients; silica--carbonate biomorphs need high pH ($>9$), dissolved silica, and carbonate; carbon--sulfur biomorphs require sulfide, organic carbon, and a redox gradient; organic biomorphs need silica, organic molecules, and metal catalysts at alkaline pH.

\subsection{Geological Environment Generation}

We simulate $N = 300$ geological environments across 10 types: hydrothermal vents (20\%), springs (12\%), alkaline lakes (12\%), cold seeps (10\%), serpentinization sites (8\%), evaporite basins (8\%), marine sediments (10\%), volcanic hot springs (8\%), subsurface aquifers (6\%), and meteorite impact sites (6\%). Each environment has 12 geochemical parameters drawn from type-specific distributions.

\subsection{Sensitivity Analysis}

Sobol first-order indices \cite{saltelli2002making} are computed via Latin Hypercube Sampling ($N = 400$) over nine geochemical parameters for each system.

\subsection{Evidence Scoring}

A composite evidence score integrates four components: lab evidence (weight 0.20), geological confirmation status (0.30), mean thermodynamic feasibility (0.30), and environmental ubiquity (0.20).

\section{Results}

\subsection{Feasibility Assessment}

Table~\ref{tab:feasibility} summarizes the feasibility metrics for each system across all 300 environments. Chemical gardens achieve a feasibility rate of 0.6867 with mean score 0.6646 $\pm$ 0.2308. Silica--carbonate biomorphs have rate 0.8000 with mean score 0.5363 $\pm$ 0.1626. Carbon--sulfur biomorphs show the second-highest rate at 0.9833 (mean 0.6698 $\pm$ 0.1230), and organic biomorphs have the highest rate at 0.9967 (mean 0.6826 $\pm$ 0.1364).

\begin{table}[h]
\caption{System feasibility across 300 geological environments.}
\label{tab:feasibility}
\begin{tabular}{lccc}
\toprule
System & Rate & Mean Score & Std \\
\midrule
Chemical Gardens & 0.6867 & 0.6646 & 0.2308 \\
Silica--Carb. Biomorphs & 0.8000 & 0.5363 & 0.1626 \\
Carbon--Sulfur Biomorphs & 0.9833 & 0.6698 & 0.1230 \\
Organic Biomorphs & 0.9967 & 0.6826 & 0.1364 \\
\bottomrule
\end{tabular}
\end{table}

\subsection{Composite Evidence Scoring}

Chemical gardens rank first with a composite evidence score of 0.8367, reflecting their confirmed geological occurrence (confirmation = 1.0) and strong thermodynamic feasibility (0.6646). Silica--carbonate biomorphs rank second (0.6709) with hypothesized status (0.5). Organic biomorphs (0.6341, rank 3) and carbon--sulfur biomorphs (0.6276, rank 4) have high feasibility but low confirmation scores (0.1), keeping their composites moderate.

\subsection{Co-occurrence Patterns}

A mean of 3.4633 systems are feasible per environment, and 99.67\% of environments support multiple systems simultaneously. System counts are: chemical gardens 206, silica--carbonate biomorphs 239, carbon--sulfur biomorphs 295, and organic biomorphs 299 out of 300 environments. This high co-occurrence suggests that environments producing one self-organized system are likely to support others.

\subsection{Sensitivity Analysis}

For chemical gardens, dissolved metals and dissolved silica are the dominant parameters. For silica--carbonate biomorphs, pH is the most influential factor. For carbon--sulfur biomorphs, dissolved sulfide and organic carbon are critical. For organic biomorphs, dissolved silica and organic carbon dominate.

\subsection{Uncertainty Quantification}

Bootstrap analysis ($N = 500$) yields 95\% confidence intervals for feasibility rates: chemical gardens $[0.6400, 0.7400]$, silica--carbonate biomorphs $[0.7516, 0.8400]$, carbon--sulfur biomorphs $[0.9667, 0.9967]$, and organic biomorphs $[0.9900, 1.0000]$.

\begin{figure}[h]
\centering
\includegraphics[width=0.95\columnwidth]{figures/feasibility_rates.png}
\caption{Feasibility rates and mean scores for each self-organized system across 300 geological environments.}
\label{fig:rates}
\end{figure}

\begin{figure}[h]
\centering
\includegraphics[width=0.95\columnwidth]{figures/evidence_scores.png}
\caption{Evidence scoring components showing the gap between thermodynamic feasibility and geological confirmation for unconfirmed systems.}
\label{fig:evidence}
\end{figure}

\begin{figure}[h]
\centering
\includegraphics[width=0.95\columnwidth]{figures/cooccurrence_heatmap.png}
\caption{Pairwise co-occurrence (Jaccard similarity) of self-organized systems across geological environments.}
\label{fig:cooc}
\end{figure}

\section{Discussion}

Our framework reveals a striking asymmetry between thermodynamic feasibility and geological evidence. Organic biomorphs and carbon--sulfur biomorphs are feasible in nearly all environments (rates $>0.98$), yet neither has been confirmed in natural settings. This suggests that the bottleneck is not thermodynamic but may involve kinetic barriers, preservation potential, or insufficient field exploration.

The composite evidence ranking---chemical gardens (0.8367), silica--carbonate biomorphs (0.6709), organic biomorphs (0.6341), carbon--sulfur biomorphs (0.6276)---provides a clear prioritization for field investigations. Serpentinization sites and alkaline lakes emerge as the most promising targets for confirming silica--carbonate biomorphs, given their high pH and dissolved silica/carbonate availability \cite{kelley2005serpentinite, russell2010serpentinization}.

The high co-occurrence rate (3.4633 systems per environment) implies that geological environments producing chemical gardens (the confirmed system) are likely to also support other self-organized systems, making hydrothermal settings productive targets for multi-system field searches.

\section{Conclusion}

We present a quantitative framework predicting geological occurrence of four self-organized chemical systems. Chemical gardens (composite score 0.8367) are confirmed and serve as the validation anchor. Silica--carbonate biomorphs (0.6709) are the highest-priority target for field confirmation. Carbon--sulfur biomorphs (0.6276) and organic biomorphs (0.6341) have high thermodynamic feasibility but require targeted preservation studies. The framework can guide field expeditions and help establish the abiotic baseline against which biosignatures must be evaluated.

\bibliographystyle{ACM-Reference-Format}
\bibliography{references}

\end{document}
