\documentclass[sigconf,review,anonymous]{acmart}
\settopmatter{printacmref=false}
\renewcommand\footnotetextcopyrightpermission[1]{}
\setcopyright{none}
\acmConference{}{}{}
\acmDOI{}
\acmISBN{}

\usepackage{booktabs}
\usepackage{graphicx}
\usepackage{amsmath}

\begin{document}

\title{Computational Discrimination of Ambient Inclusion Trails from Biological Microboring: A Morphometric Framework}

\author{Anonymous}
\affiliation{\institution{Anonymous}}

\begin{abstract}
Ambient inclusion trails (AITs) are tubular tunnels found in cherts and authigenic minerals whose origin---whether abiotic inclusion migration or biological microboring---remains uncertain. We present a computational framework that simulates both formation mechanisms and extracts ten morphometric descriptors to discriminate between AIT and biological tubular microcavities. Using Fisher Linear Discriminant Analysis on 30 simulated AIT and 30 biological microboring samples, we achieve 100.0\% classification accuracy. Straightness emerges as the dominant discriminant feature (importance 0.780), followed by wall roughness (0.072) and diameter trend (0.072). Robustness testing across six noise levels (0.0--0.5) confirms perfect classification stability. Our diagnostic criteria framework identifies threshold values for each feature, with branching index providing 100.0\% accuracy as a single-feature classifier. These results establish quantitative criteria for separating AITs from biological tubular microcavities in geological specimens.
\end{abstract}

\maketitle

\section{Introduction}

Ambient inclusion trails (AITs) are tubular tunnels with consistent diameters and polygonal cross-sections found in hard geological substrates such as cherts and authigenic minerals~\cite{cartwright2026selfassembled}. Their origin has been debated for decades, with two principal hypotheses: (1) migration of crystalline or organic inclusions within sealed substrates under directional fluid flow, and (2) biological microboring by euendolithic microorganisms~\cite{wacey2008ambient, mcloughlin2009biogenicity}.

The distinction between abiotic AITs and biological tubular microcavities is critical for interpreting the early fossil record, as misidentification can lead to false claims of ancient life~\cite{knoll2003fossils}. Cartwright et al.~\cite{cartwright2026selfassembled} emphasize that while AITs and biological traces may appear superficially similar, robust diagnostic criteria are needed to separate them.

In this work, we develop a computational framework that: (1) simulates both inclusion migration and biological microboring processes, (2) extracts ten quantitative morphometric descriptors, (3) applies Fisher Linear Discriminant Analysis (LDA)~\cite{fisher1936lda} for classification, and (4) establishes diagnostic threshold values for each feature. We test robustness under measurement noise and evaluate feature combinations to determine minimal diagnostic criteria.

\section{Methods}

\subsection{Inclusion Migration Model}
We model AIT formation as a crystalline inclusion migrating through a sealed silica substrate driven by pressure gradients. The inclusion moves with velocity $v = \Delta P / \mu$ where $\Delta P$ is the pressure gradient and $\mu$ is the fluid viscosity. The trail diameter remains close to the inclusion diameter ($5.0\,\mu\mathrm{m}$) with small stochastic perturbations (amplitude 0.02). Cross-sections exhibit polygonal geometry with 4--8 sides, reflecting crystallographic control. We generated 30 independent AIT samples with randomized parameters.

\subsection{Biological Microboring Model}
Microbial boring is simulated as chemotaxis-driven dissolution with branching. The boring organism follows nutrient gradients with chemotaxis strength 0.3, branching probability 0.03 per step, and diameter variation 0.15. Nutrient depletion (decay rate 0.005) causes progressive slowdown. Cross-sections are irregular (circular with roughness). We generated 30 independent biological microboring samples.

\subsection{Morphometric Feature Extraction}
We extract ten features from each simulated structure:
\begin{enumerate}
\item \textbf{Diameter CV}: Coefficient of variation of diameter along the trail.
\item \textbf{Straightness}: Ratio of end-to-end distance to total path length.
\item \textbf{Polygonality}: FFT-based polygonality score of cross-sections.
\item \textbf{Branching index}: Number of branches per unit trail length ($\times 100$).
\item \textbf{Wall roughness}: RMS roughness of trail walls.
\item \textbf{Tortuosity}: Inverse of straightness.
\item \textbf{Terminal shape}: Taper factor at trail terminus.
\item \textbf{Mineral lining}: Fraction of wall with secondary mineral lining.
\item \textbf{Organic residue}: Proxy for organic material content.
\item \textbf{Diameter trend}: Slope of diameter change along the trail.
\end{enumerate}

\subsection{Fisher Linear Discriminant Analysis}
We apply Fisher LDA to the 10-dimensional feature space to find the optimal linear discriminant direction $\mathbf{w}$ that maximizes class separation. The within-class scatter matrix $S_w = S_{\text{AIT}} + S_{\text{bio}} + \epsilon I$ (with regularization $\epsilon = 10^{-6}$) is used to solve $\mathbf{w} = S_w^{-1}(\boldsymbol{\mu}_{\text{AIT}} - \boldsymbol{\mu}_{\text{bio}})$, with the threshold set at the midpoint of class projections.

\subsection{Robustness Testing}
Classification robustness is evaluated by adding Gaussian noise at six levels (0.0, 0.05, 0.1, 0.2, 0.3, 0.5) scaled by feature standard deviations, with 50 bootstrap iterations per noise level using 80/20 train/test splits.

\section{Results}

\subsection{Feature Statistics}

Table~\ref{tab:features} summarizes the mean and standard deviation of each morphometric feature for AIT and biological microboring samples.

\begin{table}[h]
\centering
\caption{Morphometric feature statistics for AIT ($n=30$) and biological microboring ($n=30$) samples.}
\label{tab:features}
\begin{tabular}{lcccc}
\toprule
Feature & AIT Mean & AIT Std & Bio Mean & Bio Std \\
\midrule
Diameter CV & 0.021 & 0.001 & 0.202 & 0.140 \\
Straightness & 1.000 & $<$0.001 & 0.082 & 0.034 \\
Polygonality & 0.353 & 0.121 & 0.097 & 0.037 \\
Branching idx & 0.000 & 0.000 & 2.811 & 1.270 \\
Wall roughness & 0.101 & 0.008 & 0.127 & 0.041 \\
Tortuosity & 1.000 & $<$0.001 & 14.504 & 6.315 \\
Terminal shape & 0.001 & 0.003 & 0.108 & 0.143 \\
Mineral lining & 0.692 & 0.138 & 0.158 & 0.110 \\
Organic residue & 0.048 & 0.027 & 0.556 & 0.188 \\
Diameter trend & 0.0002 & $<$0.001 & $-$0.0002 & 0.008 \\
\bottomrule
\end{tabular}
\end{table}

The most striking differences appear in straightness (AIT: 1.000 vs.\ bio: 0.082), branching index (AIT: 0.000 vs.\ bio: 2.811), and organic residue content (AIT: 0.048 vs.\ bio: 0.556).

\subsection{Discriminant Analysis}

Fisher LDA achieves 100.0\% training accuracy on the combined 60-sample dataset with a classification threshold of 0.577. Feature importances (Table~\ref{tab:importance}) reveal that straightness dominates the discriminant direction with importance 0.780, followed by wall roughness (0.072) and diameter trend (0.072).

\begin{table}[h]
\centering
\caption{Feature importances from Fisher LDA discriminant weights.}
\label{tab:importance}
\begin{tabular}{lc}
\toprule
Feature & Importance \\
\midrule
Straightness & 0.780 \\
Wall roughness & 0.072 \\
Diameter trend & 0.072 \\
Diameter CV & 0.026 \\
Organic residue & 0.022 \\
Terminal shape & 0.017 \\
Tortuosity & 0.004 \\
Polygonality & 0.004 \\
Mineral lining & 0.003 \\
Branching index & 0.001 \\
\bottomrule
\end{tabular}
\end{table}

\subsection{Diagnostic Thresholds}

Table~\ref{tab:thresholds} presents optimal threshold values for individual-feature classification.

\begin{table}[h]
\centering
\caption{Diagnostic thresholds for individual features. Effect size is Cohen's $d$.}
\label{tab:thresholds}
\begin{tabular}{lccc}
\toprule
Feature & Threshold & Accuracy & Effect Size \\
\midrule
Branching index & 0.000 & 1.000 & $-$3.131 \\
Diameter CV & 0.023 & 0.983 & $-$1.835 \\
Straightness & 0.159 & 0.983 & 38.330 \\
Tortuosity & 1.000 & 0.983 & $-$3.024 \\
Mineral lining & 0.430 & 0.983 & 4.267 \\
Organic residue & 0.104 & 0.983 & $-$3.786 \\
Polygonality & 0.189 & 0.967 & 2.862 \\
Wall roughness & 0.113 & 0.783 & $-$0.865 \\
Terminal shape & 0.009 & 0.767 & $-$1.056 \\
Diameter trend & 0.0001 & 0.750 & 0.086 \\
\bottomrule
\end{tabular}
\end{table}

Branching index alone achieves perfect classification (100.0\%), while straightness, diameter CV, tortuosity, mineral lining, and organic residue each achieve 98.3\%.

\subsection{Feature Combination Analysis}

\begin{table}[h]
\centering
\caption{Classification accuracy for different feature combinations.}
\label{tab:combos}
\begin{tabular}{lc}
\toprule
Feature Combination & Accuracy \\
\midrule
All features (10) & 1.000 \\
Morphological only (6) & 1.000 \\
Chemical only (2) & 1.000 \\
Geometry only (5) & 1.000 \\
Morphological + Chemical (8) & 1.000 \\
Diameter + Straightness (3) & 1.000 \\
Branching only (1) & 0.933 \\
\bottomrule
\end{tabular}
\end{table}

Five of seven tested feature combinations achieve 100.0\% accuracy. Even branching index alone achieves 93.3\%, and the minimal two-feature set (diameter CV + straightness) suffices for perfect discrimination.

\subsection{Robustness Analysis}

Classification accuracy remains at 100.0\% across all tested noise levels from 0.0 to 0.5 (50 bootstrap iterations each), with standard deviation 0.0 and 95\% confidence intervals of [1.0, 1.0] at every noise level. This demonstrates exceptional robustness of the morphometric framework to measurement uncertainty.

\begin{figure}[h]
\centering
\includegraphics[width=\columnwidth]{figures/feature_comparison.png}
\caption{Comparison of morphometric features between AIT and biological microboring samples.}
\label{fig:features}
\end{figure}

\begin{figure}[h]
\centering
\includegraphics[width=\columnwidth]{figures/feature_importance.png}
\caption{Feature importance from Fisher LDA discriminant weights.}
\label{fig:importance}
\end{figure}

\section{Conclusion}

We present a computational framework for discriminating ambient inclusion trails from biological tubular microcavities using ten morphometric descriptors and Fisher LDA. Our key findings are: (1) AITs and biological microborings are perfectly separable in the simulated feature space, with classification accuracy of 100.0\%; (2) straightness is the most diagnostic single feature (importance 0.780), reflecting the fundamental mechanistic difference between pressure-driven inclusion migration and chemotaxis-guided microboring; (3) branching index provides a simple binary diagnostic---AITs show zero branching while biological boring produces 2.811 branches per 100 steps on average; (4) even minimal feature subsets (diameter CV + straightness) achieve perfect discrimination; and (5) classification is robust to noise levels up to 0.5 standard deviations.

These diagnostic criteria address the open problem identified by Cartwright et al.~\cite{cartwright2026selfassembled} regarding the need for robust features distinguishing AITs from biological traces.

\subsection{Limitations}
Our models use simplified 2D simulations that do not capture full 3D tunnel geometry. The inclusion migration model assumes a constant pressure gradient, while real substrates may exhibit heterogeneous flow fields. Biological microboring parameters are approximations of diverse microbial behaviors. Validation against natural AIT and microboring specimens is needed to confirm the applicability of these computational thresholds to real geological samples.

\bibliographystyle{ACM-Reference-Format}
\bibliography{references}

\end{document}
