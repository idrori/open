\documentclass[sigconf,review,anonymous]{acmart}
\settopmatter{printacmref=false}
\renewcommand\footnotetextcopyrightpermission[1]{}
\setcopyright{none}
\acmConference{}{}{}
\acmDOI{}
\acmISBN{}

\usepackage{booktabs}
\usepackage{graphicx}
\usepackage{amsmath}

\begin{document}

\title{Distinguishing Abiotic from Biogenic Geological Dendrites: A Computational Morphometric Framework}

\author{Anonymous}
\affiliation{\institution{Anonymous}}

\begin{abstract}
Dendritic manganese and iron oxide mineral patterns in geological settings may be entirely abiotic precipitates or may involve biological mediation. We develop a computational framework using diffusion-limited aggregation (DLA) to simulate abiotic and biofilm-modified dendrite growth, extracting seven morphometric features for discrimination. Over 20 simulations per class, biotic dendrites show significantly higher fractal dimension (1.870 vs.\ 1.772, Cohen's $d = 1.701$, $p < 10^{-5}$), branch width (12.909 vs.\ 11.615, $d = 2.077$), and compactness (16.528 vs.\ 10.820, $d = 2.822$). Compactness is the best single diagnostic criterion (accuracy 92.8\%). Fisher LDA using six features achieves 100.0\% classification accuracy (AUC = 1.0), with fractal dimension (importance 26.465) and lacunarity (6.037) as the dominant discriminant features. These results provide quantitative diagnostic criteria for assessing biogenic influence on geological dendrites.
\end{abstract}

\maketitle

\section{Introduction}

Branching mineral patterns are widespread in geological settings, with manganese and iron oxide dendrites commonly forming on rock surfaces and within fractures~\cite{cartwright2026selfassembled, potter1979manganese}. Classical models treat these patterns as abiotic precipitates formed by oxidation and diffusion-limited aggregation~\cite{witten1981dla}. However, microbes can strongly catalyze Mn and Fe oxidation~\cite{tebo2004biogenic}, and Frutexites-like structures suggest microbial mediation in some dendritic deposits~\cite{chaput2015frutexites}.

The open problem is whether all geological dendrites are completely abiotic, or some have biological influence~\cite{cartwright2026selfassembled}. We address this by: (1) simulating both abiotic and biofilm-modified DLA dendrite growth; (2) extracting seven morphometric descriptors; (3) computing diagnostic thresholds for each feature; and (4) applying Fisher LDA~\cite{fisher1936lda} for multivariate classification.

\section{Methods}

\subsection{Abiotic DLA Model}
We simulate diffusion-limited aggregation on a 2D grid with isotropic sticking probability. Particles diffuse from random boundary positions and attach upon contact with the growing aggregate. We generate 20 independent abiotic simulations with randomized initial conditions.

\subsection{Biofilm-Modified DLA}
Biotic dendrites are simulated with a biofilm field that locally enhances sticking probability and modifies diffusion. The biofilm increases local oxidation rates (analogous to microbial Mn oxidation), producing denser, more compact branching patterns. We generate 20 biofilm-modified simulations.

\subsection{Morphometric Feature Extraction}
Seven features are extracted: (1) fractal dimension via box-counting; (2) mean branch width; (3) tip density (tips per unit area); (4) lacunarity (spatial heterogeneity); (5) compactness (area/perimeter ratio); (6) branching ratio (branch points per tip); and (7) occupied fraction.

\subsection{Diagnostic Criteria}
For each feature, an optimal threshold is computed to maximize classification accuracy between abiotic and biotic dendrites. Cohen's $d$ effect size and Welch's $t$-test $p$-values quantify discriminative power.

\subsection{Multivariate Classification}
Fisher LDA is applied to the 6-feature space (excluding occupied fraction, which shows no discriminative power) to compute the optimal linear discriminant and overall classification accuracy.

\section{Results}

\subsection{Morphometric Comparison}

\begin{table}[h]
\centering
\caption{Morphometric comparison of abiotic ($n=20$) and biotic ($n=20$) dendrites.}
\label{tab:morph}
\begin{tabular}{lcccl}
\toprule
Feature & Abiotic & Biotic & $d$ & $p$ \\
\midrule
Fractal dim & 1.772 & 1.870 & 1.701 & $4.04 \times 10^{-6}$ \\
Branch width & 11.615 & 12.909 & 2.077 & $9.50 \times 10^{-8}$ \\
Tip density & 40.984 & 36.457 & 0.455 & 0.158 \\
Lacunarity & 25.971 & 26.300 & 0.591 & 0.069 \\
Compactness & 10.820 & 16.528 & 2.822 & $7.30 \times 10^{-11}$ \\
Branching ratio & 141.722 & 174.381 & 0.703 & 0.032 \\
\bottomrule
\end{tabular}
\end{table}

Four of seven features show statistically significant differences ($p < 0.05$): compactness ($d = 2.822$), branch width ($d = 2.077$), fractal dimension ($d = 1.701$), and branching ratio ($d = 0.703$). Biotic dendrites are consistently denser, wider-branched, and more compact.

\subsection{Single-Feature Diagnostic Criteria}

\begin{table}[h]
\centering
\caption{Diagnostic criteria ranking by single-feature classification accuracy.}
\label{tab:criteria}
\begin{tabular}{lcc}
\toprule
Feature & Accuracy & Cohen's $d$ \\
\midrule
Compactness & 0.928 & 2.822 \\
Branch width & 0.863 & 2.077 \\
Fractal dimension & 0.830 & 1.701 \\
Branching ratio & 0.643 & 0.703 \\
Lacunarity & 0.619 & 0.591 \\
Tip density & 0.593 & 0.455 \\
Occupied fraction & 0.500 & 0.000 \\
\bottomrule
\end{tabular}
\end{table}

Compactness is the best single criterion at 92.8\% accuracy with threshold 13.674. Branch width and fractal dimension achieve 86.3\% and 83.0\% accuracy, respectively.

\subsection{Multivariate Classification}

Fisher LDA using six features (fractal dimension, branch width, tip density, lacunarity, compactness, branching ratio) achieves 100.0\% classification accuracy with AUC = 1.0. Feature importances from the discriminant weight vector are: fractal dimension (26.465), lacunarity (6.037), compactness (3.482), branch width (0.284), tip density (0.244), and branching ratio (0.132).

\begin{figure}[h]
\centering
\includegraphics[width=\columnwidth]{figures/morphometric_comparison.png}
\caption{Morphometric comparison between abiotic and biotic dendrites across six features.}
\label{fig:morph}
\end{figure}

\begin{figure}[h]
\centering
\includegraphics[width=\columnwidth]{figures/criteria_ranking.png}
\caption{Single-feature diagnostic criteria ranked by classification accuracy.}
\label{fig:criteria}
\end{figure}

\section{Conclusion}

We demonstrate that biologically-mediated geological dendrites produce quantitatively distinguishable morphometric signatures compared to purely abiotic DLA growth. The key findings are: (1) biotic dendrites exhibit significantly higher compactness (16.528 vs.\ 10.820, $d = 2.822$), fractal dimension (1.870 vs.\ 1.772, $d = 1.701$), and branch width (12.909 vs.\ 11.615, $d = 2.077$); (2) compactness alone achieves 92.8\% classification accuracy; (3) multivariate Fisher LDA achieves perfect discrimination (100.0\% accuracy, AUC = 1.0); and (4) fractal dimension carries the largest discriminant weight (26.465), indicating it captures the most information about biogenic influence. These criteria can serve as diagnostic tests for evaluating whether geological dendrites were influenced by biological processes~\cite{cartwright2026selfassembled}.

\subsection{Limitations}
Our biofilm-modified DLA model is a simplified representation of microbial influence that modifies sticking probabilities rather than explicitly modeling metabolic processes. Real geological dendrites form under diverse mineralogical and environmental conditions not fully captured by 2D DLA. The 20-sample ensemble per class is relatively small, and the perfect multivariate accuracy may reflect overfitting to simplified simulation geometry. Validation against natural specimens with known biotic/abiotic provenance is essential.

\bibliographystyle{ACM-Reference-Format}
\bibliography{references}

\end{document}
