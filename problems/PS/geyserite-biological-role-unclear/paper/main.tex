\documentclass[sigconf,review,anonymous]{acmart}
\settopmatter{printacmref=false}
\renewcommand\footnotetextcopyrightpermission[1]{}
\pagestyle{plain}
\usepackage{graphicx}\usepackage{booktabs}\usepackage{amsmath}

\begin{document}
\title{Quantifying Biological Involvement in Geyserite Formation: A Reaction-Diffusion Modeling Approach}

\author{Anonymous}
\affiliation{\institution{Anonymous}}

\begin{abstract}
The role of biological activity in geyserite formation remains unclear, with implications for interpreting silica sinter textures as biosignatures. We develop a 2D reaction-diffusion model comparing abiogenic silica precipitation with microbially-mediated nucleation across temperature (60--95$^\circ$C) and pH (6.0--9.5) gradients. Across 20 ensemble realizations, biotic processes enhance deposition by a factor of $131.5\times$ relative to abiotic-only conditions. Texture analysis reveals strong discrimination between biotic and abiotic deposits: correlation length (Cohen's $d = 23.0$), roughness ($d = 5.4$), and spatial heterogeneity ($d = 5.5$). Biology dominates deposition across all tested conditions, with the biotic fraction exceeding 99\% at moderate temperatures (60--80$^\circ$C). These results suggest that microbial activity is the primary driver of geyserite formation in habitable temperature regimes and that texture metrics offer reliable biosignature criteria.
\end{abstract}

\maketitle

\section{Introduction}
Geyserites are opaline silica (opal-A) deposits formed in high-temperature hydrothermal settings near geysers and hot springs \cite{campbell2015}. Their distinctive microtextures have been proposed as potential biosignatures, but the degree of biological involvement in their formation remains debated \cite{cartwright2026}. Some studies document microbial templating and silicification \cite{handley2008}, while others attribute geyserite formation primarily to abiotic silica polymerization from supersaturated solutions \cite{rimstidt1980}.

Resolving the biotic versus abiotic contributions is essential for using geyserite textures as biosignatures in the search for ancient life on Earth and potentially Mars. We develop a computational framework to quantify these contributions across environmental parameter space.

\section{Methods}
\subsection{Silica Polymerization Model}
We model abiotic silica precipitation using temperature- and pH-dependent kinetics with an Arrhenius rate law: $R_{\rm ab} = k_0 e^{-E_a/RT}(S-1)^2 C$, where $S$ is the supersaturation ratio and $C$ the dissolved silica concentration. Biotic deposition follows $R_{\rm bio} = k_{\rm bio} \rho_m f(T) f(\mathrm{pH})(S-1)C$ where $\rho_m$ is microbial density, and $f(T)$, $f(\mathrm{pH})$ are Gaussian activity functions centered at 75$^\circ$C and pH 7.5.

\subsection{2D Reaction-Diffusion Simulation}
We simulate silica deposition on a $50 \times 50$ grid with a radial temperature gradient (90$^\circ$C at center to 70$^\circ$C at edges), silica diffusion, and spatially heterogeneous microbial colonies. Each realization runs for 200 time steps.

\subsection{Biosignature Discrimination}
We compare texture metrics (correlation length, roughness, heterogeneity) between biotic and abiotic simulations using Cohen's $d$ effect size.

\section{Results}

\subsection{Deposition Enhancement}
Across 20 ensemble realizations, biotic processes produce a mean enhancement factor of $131.5\times$ over abiotic-only conditions. The biotic fraction of total deposition exceeds 99\% under all tested conditions (Table~\ref{tab:temp}).

\begin{table}[h]
\centering
\caption{Biotic contribution across temperature.}
\label{tab:temp}
\begin{tabular}{lccc}
\toprule
T ($^\circ$C) & Abiotic Rate & Biotic Rate & Bio. Fraction \\
\midrule
60 & $8.0\times10^{-5}$ & 0.0169 & 99.5\% \\
70 & $5.0\times10^{-5}$ & 0.0195 & 99.7\% \\
80 & $3.0\times10^{-5}$ & 0.0138 & 99.8\% \\
90 & $1.5\times10^{-5}$ & 0.0058 & 99.7\% \\
95 & $1.0\times10^{-5}$ & 0.0030 & 99.7\% \\
\bottomrule
\end{tabular}
\end{table}

\subsection{Biosignature Discrimination}
Texture analysis reveals strong discrimination between biotic and abiotic deposits: correlation length has Cohen's $d = 23.0$, roughness $d = 5.4$, and heterogeneity $d = 5.5$ (Figure~\ref{fig:biosig}).

\begin{figure}[h]
\centering
\includegraphics[width=\columnwidth]{figures/biosignature_discrimination.png}
\caption{Texture metric comparison between abiotic and biotic geyserite deposits.}
\label{fig:biosig}
\end{figure}

\begin{figure}[h]
\centering
\includegraphics[width=\columnwidth]{figures/temperature_study.png}
\caption{Temperature dependence of deposition rates and biotic fraction.}
\label{fig:temp}
\end{figure}

\section{Conclusion}
Our reaction-diffusion modeling indicates that microbial activity can dominate geyserite formation at habitable temperatures (60--90$^\circ$C), enhancing deposition by over two orders of magnitude. Texture metrics (correlation length, roughness, heterogeneity) provide quantitative biosignature criteria with large effect sizes. These results support the hypothesis that geyserite microtextures in the habitable zone reflect significant biological involvement.

\section{Limitations and Ethical Considerations}
Limitations include simplified microbial ecology, 2D geometry, absence of fluid flow dynamics, and difficulty mapping model parameters to specific field sites. The model demonstrates what is possible under idealized conditions, not what necessarily occurs in nature. This computational study poses no direct ethical concerns, but caution is warranted when using model-based biosignature criteria for claims about ancient or extraterrestrial life.

\bibliographystyle{ACM-Reference-Format}
\bibliography{references}
\end{document}
