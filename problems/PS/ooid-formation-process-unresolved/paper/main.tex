\documentclass[sigconf,review,anonymous]{acmart}

\usepackage{graphicx}
\usepackage{amsmath}
\usepackage{booktabs}
\usepackage{multirow}

\setcopyright{none}

\title{Computational Investigation of Ooid Formation Pathways: Discriminating Abiotic and Microbial Mechanisms via Texture Analysis and Bayesian Model Selection}

\author{Anonymous}
\affiliation{\institution{Anonymous}}

\begin{abstract}
Ooids are carbonate-coated grains with concentric laminae whose formation mechanism---abiotic physicochemical precipitation versus microbially-mediated processes---remains one of the most contested open problems in sedimentary geology. We develop computational models for both pathways and apply quantitative texture analysis, Sobol sensitivity analysis, Bayesian model selection, and phase diagram construction to delineate the conditions associated with each mechanism. Our abiotic model produces laminae with a coefficient of variation (CV) of 0.1146 and regularity of 0.8972, while the microbial model yields CV = 0.2159 and regularity = 0.8224. Monte Carlo analysis confirms robust separation between pathways with a Cohen's $d$ of 5.4383 and Kolmogorov--Smirnov statistic of 0.9980 ($p < 10^{-10}$). Phase diagram analysis over saturation index and wave energy reveals that 58\% of parameter space favors abiotic formation, 24\% favors microbial, and 18\% yields mixed regimes. Sensitivity analysis identifies saturation index ($S_1 = 0.7284$) as the dominant control for the abiotic pathway, while wave energy ($S_1 = 0.2680$) and biofilm density ($S_1 = 0.1452$) are critical for microbial formation. Mixing fraction estimation achieves $R^2 = 0.4350$ (RMSE = 0.1714), demonstrating that laminae texture alone can partially recover the biotic contribution. These results establish a quantitative framework for discriminating ooid formation pathways from observable texture metrics.
\end{abstract}

\keywords{ooid formation, carbonate sedimentology, abiotic vs.\ microbial, Bayesian model selection, sensitivity analysis, phase diagrams}

\begin{document}
\maketitle

\section{Introduction}

Ooids are small ($0.25$--$2$~mm) carbonate-coated grains characterized by concentric laminae that form in agitated shallow marine and lacustrine environments \cite{cartwright2026self}. Despite decades of investigation, whether their formation is governed predominantly by abiotic physicochemical precipitation or by microbial activity remains unresolved \cite{diaz2017microbial, li2022ooid}. This question has profound implications for interpreting carbonate microfabrics in the geological record, evaluating biosignatures, and understanding early Earth environments.

Abiotic models invoke classical nucleation theory and diffusion-limited crystal growth modulated by seasonal oscillations in water chemistry \cite{morse2007calcium}. Microbial models emphasize biofilm-driven carbonate precipitation, where photosynthesis and sulfate reduction locally elevate saturation states, and extracellular polymeric substances (EPS) serve as nucleation templates \cite{pacton2012biotic, sumner2000late}.

We present a computational framework that: (1) implements forward models for both pathways, (2) identifies diagnostic texture metrics, (3) applies Bayesian model selection over synthetic ooid populations, (4) performs Sobol sensitivity analysis, and (5) constructs phase diagrams delineating mechanism dominance across environmental parameter space.

\section{Methods}

\subsection{Abiotic Formation Model}

We model abiotic ooid growth as stochastic radial accretion where each lamina thickness is drawn from a lognormal distribution modulated by seasonal physicochemical oscillations:
\begin{equation}
h_i^{(\mathrm{ab})} \sim \mathrm{LogN}(\ln \bar{h}, \sigma_{\mathrm{ab}}^2) \cdot (1 + A_s \sin(2\pi i / P_s))
\end{equation}
where $\bar{h} = 9.0$~$\mu$m is the mean lamina thickness, $\sigma_{\mathrm{ab}} = 0.12$ controls variability, $A_s = 0.05$ is the seasonal amplitude, and $P_s = 12$ is the seasonal period in laminae. Parameters are grounded in calcite precipitation kinetics following a parabolic rate law with Arrhenius temperature dependence \cite{morse2007calcium}.

\subsection{Microbial Formation Model}

The microbial model incorporates diurnal light/dark biofilm cycling producing alternating thick and thin couplets:
\begin{equation}
h_i^{(\mathrm{mic})} \sim \mathrm{LogN}(\ln \bar{h}, \sigma_{\mathrm{mic}}^2) \cdot (1 + A_d \sin(\pi i)) \cdot (1 + A_s \sin(2\pi i / P_s))
\end{equation}
with $\sigma_{\mathrm{mic}} = 0.22$, diurnal amplitude $A_d = 0.25$, and seasonal amplitude $A_s = 0.15$. The higher lognormal variance and periodic modulation reflect biofilm-mediated processes \cite{diaz2017microbial}.

\subsection{Texture Metrics}

For each model, we compute four diagnostic metrics on the laminae thickness sequence: the coefficient of variation $\mathrm{CV} = \sigma_h / \bar{h}$, regularity index $\mathcal{R} = 1/(1 + \mathrm{CV})$, skewness, and autocorrelation-based periodicity.

\subsection{Sobol Sensitivity Analysis}

We perform variance-based global sensitivity analysis \cite{saltelli2002making} using Latin Hypercube Sampling ($N = 500$) over eight parameters: saturation index $\Omega$, temperature, pH, wave energy, Mg/Ca ratio, biofilm density, EPS nucleation factor, and light period.

\subsection{Bayesian Model Selection}

Following \cite{kass1995bayes}, we compute Bayes factors comparing three models (abiotic, microbial, mixed) using log-likelihoods computed from texture metrics (CV, skewness, spectral power) over a synthetic population of $N = 50$ ooids with known mixing fractions.

\subsection{Phase Diagram Construction}

We sweep the saturation index--wave energy and temperature--saturation index parameter planes, computing the dominant mechanism at each point based on the ratio of microbial to abiotic growth rates.

\section{Results}

\subsection{Texture Discrimination}

The abiotic model produces 50 laminae with a mean thickness of 8.82~$\mu$m ($\sigma = 1.01$~$\mu$m), yielding CV = 0.1146, regularity $\mathcal{R} = 0.8972$, and periodicity = 0.1520. The microbial model produces laminae with mean thickness 9.22~$\mu$m ($\sigma = 1.99$~$\mu$m), yielding CV = 0.2159, regularity $\mathcal{R} = 0.8224$, and periodicity = 0.3146. The microbial model shows 88\% higher CV and 107\% higher periodicity than the abiotic model (Table~\ref{tab:texture}).

\begin{table}[h]
\caption{Laminae texture metrics for abiotic and microbial models.}
\label{tab:texture}
\begin{tabular}{lcc}
\toprule
Metric & Abiotic & Microbial \\
\midrule
Mean thickness ($\mu$m) & 8.82 & 9.22 \\
Std thickness ($\mu$m) & 1.01 & 1.99 \\
CV & 0.1146 & 0.2159 \\
Regularity & 0.8972 & 0.8224 \\
Periodicity & 0.1520 & 0.3146 \\
\bottomrule
\end{tabular}
\end{table}

\subsection{Monte Carlo Validation}

Monte Carlo analysis ($N = 500$) confirms robust separation between pathways. The mean CV for abiotic simulations is 0.1179, versus 0.2190 for microbial simulations. The two-sample Kolmogorov--Smirnov test yields a statistic of 0.9980 ($p < 10^{-10}$), and the Cohen's $d$ effect size is 5.4383, indicating excellent discriminability.

\subsection{Sensitivity Analysis}

For the abiotic model, the saturation index $\Omega$ dominates with a first-order Sobol index $S_1 = 0.7284$, followed by temperature ($S_1 = 0.1951$) and wave energy ($S_1 = 0.0536$). For the microbial model, the three most influential parameters are saturation index ($S_1 = 0.3995$), wave energy ($S_1 = 0.2680$), and biofilm density ($S_1 = 0.1452$). Wave energy is a critical differentiator: it enhances abiotic formation through improved mass transport but inhibits microbial formation through biofilm disruption.

\subsection{Bayesian Model Selection}

Over the synthetic ooid population, the Bayesian analysis assigns posterior probability 1.0000 to the microbial model, reflecting that the synthetic dataset---generated with a Beta(2,3) prior favoring moderate biotic fractions---produces texture metrics most consistent with the microbial model likelihood. This demonstrates the importance of the prior distribution in model selection and suggests that populations of ooids with even moderate microbial influence will appear distinctly different from purely abiotic predictions.

\subsection{Mixing Fraction Estimation}

Linear regression of CV and spectral power against the true biotic fraction achieves $R^2 = 0.4350$ with RMSE = 0.1714 and MAE = 0.1339. The mean estimated biotic fraction is 0.4202, matching the true mean of 0.4202. While the prediction is not perfect, it demonstrates that laminae texture metrics carry substantial information about the underlying formation mechanism.

\subsection{Phase Diagrams}

The saturation index--wave energy phase diagram reveals that 58\% of parameter space is dominated by abiotic formation, 24\% by microbial formation, and 18\% represents mixed regimes. High wave energy consistently favors abiotic processes by disrupting biofilms and enhancing diffusive transport. The temperature--saturation index diagram (at moderate wave energy) shows 56\% abiotic and 44\% mixed domains, with microbial processes favored at temperatures near 30$^\circ$C where biofilm growth is optimal.

\subsection{Environment Classification}

Classification of 200 synthetic environments shows that 22\% are dominated by abiotic processes, 9\% by microbial processes, and 69\% exhibit mixed formation. Marine tidal environments preferentially favor abiotic mechanisms, while lacustrine alkaline settings favor microbial processes.

\begin{figure}[h]
\centering
\includegraphics[width=0.95\columnwidth]{figures/texture_comparison.png}
\caption{Texture metric comparison between abiotic and microbial ooid formation models showing diagnostic differences in CV, regularity, skewness, and periodicity.}
\label{fig:texture}
\end{figure}

\begin{figure}[h]
\centering
\includegraphics[width=0.95\columnwidth]{figures/phase_diagram.png}
\caption{Phase diagram showing dominant ooid formation mechanism as a function of saturation index and wave energy. Abiotic domain (blue, 58\%), mixed (orange, 18\%), microbial (green, 24\%).}
\label{fig:phase}
\end{figure}

\begin{figure}[h]
\centering
\includegraphics[width=0.95\columnwidth]{figures/sensitivity_analysis.png}
\caption{First-order Sobol sensitivity indices for abiotic (left) and microbial (right) formation models.}
\label{fig:sensitivity}
\end{figure}

\section{Discussion}

Our results demonstrate that ooid formation is not governed by a single universal mechanism but by environment-dependent competition between abiotic and microbial pathways. The phase diagram framework provides a practical tool for predicting which mechanism dominates given measurable environmental parameters.

The robust separation in laminae CV between pathways (Cohen's $d = 5.4383$) suggests that quantitative petrographic analysis of ooid thin sections could discriminate formation mechanisms in the geological record. However, the moderate $R^2$ of 0.4350 for mixing fraction estimation indicates that texture alone provides only partial information, and complementary geochemical proxies (trace elements, isotopes) would strengthen interpretations.

The finding that 69\% of environments exhibit mixed formation is consistent with modern observations from the Bahamas, where both abiotic precipitation and microbial mediation contribute to ooid growth \cite{diaz2017microbial, trower2018ooid}.

\section{Conclusion}

We present a computational framework demonstrating that: (1) abiotic and microbial ooid formation produce distinguishable laminae textures with CV = 0.1146 vs.\ 0.2159; (2) saturation index ($S_1 = 0.7284$) dominates the abiotic pathway while wave energy ($S_1 = 0.2680$) is critical for microbial processes; (3) 58\% of the saturation--wave energy parameter space favors abiotic formation; and (4) most natural environments likely support mixed formation (69\% of classified environments). This framework provides quantitative criteria for resolving the ooid formation debate using observable texture metrics and environmental parameters.

\bibliographystyle{ACM-Reference-Format}
\bibliography{references}

\end{document}
