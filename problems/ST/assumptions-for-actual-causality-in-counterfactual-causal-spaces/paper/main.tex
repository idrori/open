\documentclass[sigconf,review,anonymous]{acmart}

\usepackage{amsmath,amssymb,amsthm}
\usepackage{graphicx}
\usepackage{booktabs}

\newtheorem{definition}{Definition}
\newtheorem{proposition}{Proposition}

\setcopyright{none}

\title{Evaluating Assumptions for Actual Causality in Counterfactual Causal Spaces}

\author{Anonymous}
\affiliation{\institution{Anonymous}}

\begin{abstract}
We computationally evaluate candidate assumption sets for defining actual causality within counterfactual causal spaces. Testing five assumptions---consistency, monotonicity, faithfulness, modularity, and their combinations---across 120 random DAGs of sizes 3--8, we measure sufficiency as the ability to produce well-defined actual causality judgments under both Halpern--Pearl and Beckers definitions. Without assumptions, 61.7\% of DAGs yield sufficient results. Adding individual assumptions yields rates from 48.3\% (monotonicity alone) to 61.7\% (faithfulness alone). Combined sets show that consistency+monotonicity achieves 44.2\% with the strictest assumption checking, while the full assumption set reaches 44.2\%. HP-Beckers agreement averages 27.5--28.5\% across all sets, revealing fundamental definitional differences that assumptions alone cannot resolve. DAG size significantly affects sufficiency, with 4-variable DAGs achieving the highest rates (66.7\%) and 8-variable DAGs the lowest (20.8\%).
\end{abstract}

\keywords{actual causality, counterfactual, causal spaces, assumptions, structural causal models}

\begin{document}
\maketitle

\section{Introduction}

Distinguishing between general (type) causality and actual (token) causality is a fundamental challenge in causal reasoning~\cite{pearl2009causality, woodward2003making}. While type causality asks whether $X$ generally causes $Y$, actual causality asks whether a specific event $X=x$ caused $Y=y$ in a particular context.

Park et al.~\cite{park2026counterfactual} recently proposed counterfactual causal spaces as a framework for causal reasoning without relying on structural causal models (SCMs). While type causality is definable in this framework, they note that actual causality requires additional assumptions connecting observational and interventional distributions, leaving the identification of such assumptions as an open problem.

We address this computationally by evaluating candidate assumption sets on synthetic causal DAGs, testing whether each set is sufficient for the Halpern--Pearl~\cite{halpern2005causes} and Beckers~\cite{beckers2021causal} definitions of actual causality.

\section{Candidate Assumptions}

We formalize five candidate assumptions:

\begin{definition}[Consistency]
If $X = x$ is observed, then $Y$ under $\mathrm{do}(X=x)$ equals the observed $Y$: $Y_{x} = Y_{\mathrm{obs}}$ when $X_{\mathrm{obs}} = x$.
\end{definition}

\begin{definition}[Monotonicity]
For all contexts $u$: if $x > x'$, then $Y_x(u) \geq Y_{x'}(u)$.
\end{definition}

\begin{definition}[Faithfulness]
Every conditional independence in the data corresponds to a d-separation in the causal graph~\cite{spirtes2000causation}.
\end{definition}

\begin{definition}[Modularity]
Intervening on $X_i$ only changes the mechanism for $X_i$, leaving all other mechanisms invariant~\cite{peters2017elements}.
\end{definition}

\section{Methodology}

We generate 120 random DAGs across sizes 3, 4, 5, 6, and 8 variables with edge probability $2/n$. For each DAG, we construct a linear SCM with random coefficients and evaluate 10 assumption sets. Sufficiency requires that at least one actual cause is found and assumption scores exceed threshold 0.3.

\section{Results}

\subsection{Sufficiency Rates}

Table~\ref{tab:suff} shows sufficiency rates for each assumption set.

\begin{table}[h]
\centering
\caption{Sufficiency rates across assumption sets (120 DAGs).}
\label{tab:suff}
\begin{tabular}{lcc}
\toprule
Assumption Set & Sufficiency & Agreement \\
\midrule
None & 0.617 & 0.278 \\
Consistency & 0.525 & 0.277 \\
Monotonicity & 0.483 & 0.277 \\
Consistency + Monotonicity & 0.442 & 0.279 \\
Faithfulness & 0.617 & 0.275 \\
Consistency + Faithfulness & 0.550 & 0.279 \\
Cons. + Mono. + Faithful. & 0.442 & 0.279 \\
Modularity & 0.583 & 0.283 \\
Consistency + Modularity & 0.492 & 0.285 \\
Full & 0.442 & 0.281 \\
\bottomrule
\end{tabular}
\end{table}

\begin{figure}[h]
    \centering
    \includegraphics[width=0.85\columnwidth]{figures/sufficiency.png}
    \caption{Sufficiency rates across assumption sets.}
    \label{fig:suff}
\end{figure}

A key finding is that more assumptions do not always increase sufficiency on finite samples---stricter checking can exclude valid DAGs. The ``none'' baseline at 61.7\% reflects that many DAGs produce well-defined causality even without explicit assumption verification, but this comes without guarantees.

\subsection{DAG Size Effect}

\begin{figure}[h]
    \centering
    \includegraphics[width=0.85\columnwidth]{figures/dag_size.png}
    \caption{DAG size effect on sufficiency rate.}
    \label{fig:dag}
\end{figure}

DAG size strongly affects sufficiency: 4-variable DAGs achieve 66.7\% (full set), while 8-variable DAGs drop to 20.8\%. This reflects the increasing difficulty of satisfying all assumptions simultaneously as causal structure complexity grows.

\subsection{HP-Beckers Agreement}

Agreement between the Halpern--Pearl and Beckers definitions averages only 27.5--28.5\% across all assumption sets (Table~\ref{tab:suff}). This low agreement reflects fundamental definitional differences: HP uses a contingency-based counterfactual criterion, while Beckers uses a production-based approach. No assumption set substantially increases agreement, suggesting that the choice of actual causality definition matters more than the assumptions.

\begin{figure}[h]
    \centering
    \includegraphics[width=0.85\columnwidth]{figures/agreement.png}
    \caption{HP-Beckers agreement across assumption sets.}
    \label{fig:agree}
\end{figure}

\section{Discussion}

Our results reveal a tension in defining actual causality within counterfactual causal spaces:

\textbf{Assumption sufficiency is necessary but not sufficient.} While assumptions like consistency and monotonicity provide guarantees, they also restrict the class of applicable models. On finite DAGs, the restriction can reduce rather than increase the rate of well-defined causality judgments.

\textbf{Definitional choice dominates.} The low HP-Beckers agreement (27.5--28.5\%) shows that the choice between actual causality definitions has a larger effect than any assumption set. This suggests that counterfactual causal spaces need to specify not just assumptions but also a preferred definition.

\textbf{Scale matters.} The dramatic drop from 66.7\% at $n=4$ to 20.8\% at $n=8$ indicates that assumption verification becomes increasingly difficult with causal complexity.

\section{Conclusion}

Our computational evaluation identifies consistency, faithfulness, and modularity as the most promising individual assumptions for actual causality in counterfactual causal spaces, with faithfulness matching the no-assumption baseline (61.7\%). Combined sets with monotonicity achieve 44.2\% sufficiency with stronger guarantees. The persistent gap in HP-Beckers agreement highlights that assumptions alone cannot resolve definitional disagreements, suggesting that counterfactual causal spaces require both assumptions and a canonical definition.

\begin{acks}
Supported by computational resources at Anonymous Institution.
\end{acks}

\bibliographystyle{ACM-Reference-Format}
\bibliography{references}

\end{document}
