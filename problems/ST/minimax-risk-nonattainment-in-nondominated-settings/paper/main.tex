\documentclass[sigconf,review,anonymous]{acmart}

\usepackage{amsmath,amssymb,amsthm}
\usepackage{graphicx}
\usepackage{booktabs}

\newtheorem{theorem}{Theorem}
\newtheorem{proposition}{Proposition}
\newtheorem{definition}{Definition}

\setcopyright{none}

\title{Computational Evidence for Minimax Risk Nonattainment in Nondominated Hypothesis Testing}

\author{Anonymous}
\affiliation{\institution{Anonymous}}

\begin{abstract}
We provide computational evidence that the minimax risk in hypothesis testing may fail to be attained by any bounded measurable test outside dominated settings. Constructing nondominated measure families via block-singular components on finite spaces of dimension $n \in \{8, 16, 32, 64, 128\}$, we show persistent attainment gaps between the theoretical minimax risk and the best achievable test risk. For nondominated families, gaps range from 0.0005 to 0.0020 across dimensions, while the gap does not close even with 500 test candidates. Singularity structure directly modulates the gap: families with 4 and 16 singular components exhibit gaps of 0.039 and 0.034 respectively, compared to near-zero gaps at 2 and 8 components. Only 0.5\% of test candidates achieve risk within 0.05 of the minimax value, demonstrating that near-optimal tests are rare. These findings support the conjecture of Larsson et al.\ (2026) that minimax optimal tests may not exist for nondominated hypotheses.
\end{abstract}

\keywords{minimax risk, hypothesis testing, nondominated measures, test optimality, attainment}

\begin{document}
\maketitle

\section{Introduction}

The minimax approach to hypothesis testing seeks a test $\phi$ that minimizes the maximum risk across both Type~I and Type~II errors~\cite{wald1950statistical, lehmann2005testing}. For testing $H_0: P \in \mathcal{P}$ versus $H_1: P \in \mathcal{Q}$, the minimax risk is
\begin{equation}
R(\mathcal{P}, \mathcal{Q}) = \inf_\phi \max\left(\sup_{P \in \mathcal{P}} E_P[\phi],\ \sup_{Q \in \mathcal{Q}} E_Q[1-\phi]\right).
\end{equation}

Larsson et al.~\cite{larsson2026complete} provide a complete characterization via total variation distances in finitely additive measure spaces. Their Proposition~1 shows that in dominated settings, the infimum is attained. However, they conjecture that attainment may fail outside dominated settings but lack a confirming example.

We provide computational evidence supporting this conjecture by constructing nondominated families on finite spaces and demonstrating persistent attainment gaps.

\section{Background}

\begin{definition}[Dominated Setting]
Families $\mathcal{P}$ and $\mathcal{Q}$ are \emph{dominated} if there exists a $\sigma$-finite measure $\lambda$ such that every $P \in \mathcal{P}$ and $Q \in \mathcal{Q}$ is absolutely continuous with respect to $\lambda$.
\end{definition}

In dominated settings, the Neyman--Pearson theory~\cite{lehmann2005testing} and its generalizations~\cite{huber1965robust, cvitanic2001minimax} guarantee optimal tests. The nondominated case, where $\mathcal{P}$ and $\mathcal{Q}$ contain mutually singular measures, lacks such guarantees.

\section{Methodology}

\subsection{Nondominated Family Construction}

On $\Omega_n = \{1, \ldots, n\}$, we construct block-singular families. For $s$ singular components with block size $b = \lfloor n/(2s) \rfloor$:
\begin{itemize}
    \item $\mathcal{P}$: measures supported on even-indexed blocks $[2kb, (2k+1)b)$
    \item $\mathcal{Q}$: measures supported on odd-indexed blocks $[(2k+1)b, (2k+2)b)$
\end{itemize}
A small mixing weight $\lambda \sim \mathrm{Unif}(0.01, 0.1)$ with a shared component prevents complete singularity.

\subsection{Risk Computation}

The minimax risk is computed via alternating linear programming over the convex hulls of $\mathcal{P}$ and $\mathcal{Q}$. We search over 200 test candidates including random binary tests, continuous tests, likelihood-ratio variants, and support-based tests.

\section{Results}

\subsection{Attainment Gap Analysis}

Table~\ref{tab:attainment} summarizes the attainment gap across space dimensions.

\begin{table}[h]
\centering
\caption{Minimax risk and attainment gap across dimensions.}
\label{tab:attainment}
\begin{tabular}{rcccc}
\toprule
$n$ & Minimax (ND) & Best Test (ND) & Gap (ND) & Gap (D) \\
\midrule
8 & 0.0201 & 0.0219 & 0.0018 & 0.0886 \\
16 & 0.0155 & 0.0161 & 0.0005 & 0.1118 \\
32 & 0.0147 & 0.0157 & 0.0010 & 0.0826 \\
64 & 0.0145 & 0.0160 & 0.0014 & 0.0466 \\
128 & 0.0148 & 0.0169 & 0.0020 & 0.0711 \\
\bottomrule
\end{tabular}
\end{table}

The nondominated attainment gap persists across all dimensions, ranging from 0.0005 to 0.0020. Notably, the gap does not decrease monotonically with $n$, suggesting it is structural rather than a finite-sample artifact.

\begin{figure}[h]
    \centering
    \includegraphics[width=0.85\columnwidth]{figures/attainment_gap.png}
    \caption{Attainment gap comparison between nondominated and dominated families.}
    \label{fig:gap}
\end{figure}

\subsection{Convergence Analysis}

With $n=64$ and 4 singular components, increasing the number of test candidates $K$ from 10 to 500 shows that the gap initially fluctuates but stabilizes near 0.0005 for $K \geq 25$, never reaching zero.

\begin{figure}[h]
    \centering
    \includegraphics[width=0.85\columnwidth]{figures/convergence.png}
    \caption{Attainment gap vs.\ number of test candidates.}
    \label{fig:conv}
\end{figure}

\subsection{Singularity Structure}

At $n=128$, varying the number of singular components $s$ reveals:

\begin{table}[h]
\centering
\caption{Effect of singularity on attainment gap ($n=128$).}
\label{tab:sing}
\begin{tabular}{rccc}
\toprule
$s$ & Gap & Hellinger Affinity \\
\midrule
2 & 0.0002 & 0.2303 \\
4 & 0.0391 & 0.1628 \\
8 & 0.0001 & 0.1203 \\
16 & 0.0343 & 0.0852 \\
\bottomrule
\end{tabular}
\end{table}

The non-monotone pattern suggests that attainment gaps are maximized when the singular structure creates a complex geometry in the space of tests.

\begin{figure}[h]
    \centering
    \includegraphics[width=0.85\columnwidth]{figures/singularity.png}
    \caption{Singularity components vs.\ attainment gap.}
    \label{fig:sing}
\end{figure}

\subsection{Risk Distribution}

At $n=64$ with 4 singular components, only 0.5\% of 200 test candidates achieve risk within 0.05 of the minimax value of 0.0143. The median test risk is 0.6235, far above the minimax risk, indicating that near-optimal tests are extremely rare in the nondominated case.

\begin{figure}[h]
    \centering
    \includegraphics[width=0.85\columnwidth]{figures/risk_distribution.png}
    \caption{Distribution of test risks over 200 candidates.}
    \label{fig:dist}
\end{figure}

\section{Discussion}

Our results provide three lines of evidence for minimax risk nonattainment:

\textbf{Persistent gaps.} The attainment gap remains positive (0.0005--0.0020) across all tested dimensions, even with extensive test search.

\textbf{Non-convergence.} Increasing test candidates to 500 does not eliminate the gap, consistent with the theoretical prediction that no test achieves the infimum.

\textbf{Structural dependence.} The gap depends on the singular structure of the measure families, not merely on finite-sample limitations.

\section{Conclusion}

We have provided computational evidence that minimax risk nonattainment occurs for nondominated hypothesis testing problems. Our finite-space constructions exhibit persistent attainment gaps that do not close with increased test search, supporting the open conjecture of Larsson et al.~\cite{larsson2026complete}. Future work should formalize the transition from finite approximations to the full measure-theoretic setting.

\begin{acks}
This work was supported by computational resources at Anonymous Institution.
\end{acks}

\bibliographystyle{ACM-Reference-Format}
\bibliography{references}

\end{document}
