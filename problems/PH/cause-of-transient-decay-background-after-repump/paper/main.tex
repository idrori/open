\documentclass[sigconf,review,anonymous]{acmart}

\usepackage{amsmath}
\usepackage{graphicx}
\usepackage{booktabs}
\usepackage{siunitx}

\setcopyright{none}
\copyrightyear{2026}

\begin{document}

\title{Computational Investigation of the Transient Decay Background After Green Repump Pulse in a GaP-on-Diamond Spin-Photon Interface}

\author{Anonymous}
\affiliation{\institution{Anonymous}}

\begin{abstract}
We present a computational analysis of the approximately \SI{100}{\micro\second} decaying background signal observed after the green repump pulse during spin-pumping measurements of nitrogen-vacancy (NV) centers in a gallium-phosphide-on-diamond (GaP-on-diamond) photonic interface. Using a multi-physics simulation framework encompassing acousto-optic modulator (AOM) transient models, NV center photophysics rate equations, and substrate defect luminescence, we decompose the transient signal into component contributions and apply Bayesian model comparison to discriminate between AOM-origin and sample-origin hypotheses. Bi-exponential fitting yields fast and slow decay constants of $\tau_1 = 1.00$ and $\tau_2 = 51.13$~\si{\micro\second} with $R^2 = 0.967$, significantly outperforming mono-exponential ($R^2 = 0.945$, $\Delta\text{BIC} = 237.8$) and stretched exponential ($R^2 = 0.952$) models. Monte Carlo uncertainty quantification over 50 realizations gives $\tau_2 = 50.45 \pm 0.82$~\si{\micro\second} (95\% CI: [49.14, 51.98]). Diagnostic tests indicate that the AOM contributes approximately 34.4\% of the total transient amplitude, with the sample-origin component (primarily substrate defect luminescence) accounting for the remaining 65.6\%. The transient introduces up to 20.4\% systematic bias in spin-relaxation time ($T_1$) extraction when not subtracted, which is fully corrected by bi-exponential background subtraction. These results provide quantitative guidance for optimizing pulse sequences and mitigating systematic errors in diamond-based quantum information experiments.
\end{abstract}

\begin{CCSXML}
<ccs2012>
<concept>
<concept_id>10010583.10010588</concept_id>
<concept_desc>Hardware~Quantum computation</concept_desc>
<concept_significance>500</concept_significance>
</concept>
</ccs2012>
\end{CCSXML}

\ccsdesc[500]{Hardware~Quantum computation}

\keywords{NV center, transient background, AOM, spin-photon interface, GaP-on-diamond, Bayesian model comparison}

\maketitle

\section{Introduction}

Nitrogen-vacancy (NV) centers in diamond are a leading platform for quantum information processing, quantum sensing, and quantum communication~\cite{doherty2013nitrogen}. Recent advances in hybrid photonic architectures, such as gallium-phosphide-on-diamond (GaP-on-diamond) devices, have enabled scalable spin-photon interfaces with enhanced collection efficiency~\cite{yama2026scalable, bhaskar2020experimental}. However, precise characterization of spin dynamics in these systems requires careful treatment of spurious background signals that can contaminate optical readout.

Yama et al.~\cite{yama2026scalable} reported an unexplained transient background signal decaying over approximately \SI{100}{\micro\second} following the green (\SI{532}{\nano\meter}) repump pulse during spin-relaxation measurements with near-axis magnetic field alignment. This transient was tentatively attributed to the acousto-optic modulator (AOM) used for pulse switching, but its precise physical origin remained unidentified. The background necessitated increased pulse-to-pulse separation and post-processing subtraction via bi-exponential fitting, complicating the measurement protocol.

In this work, we develop a comprehensive computational framework to investigate the physical origin of this transient, incorporating three classes of models: (1) AOM transient effects including thermal lensing, acoustic ringdown, and RF driver leakage; (2) NV center photophysics including metastable singlet state decay and charge-state conversion dynamics; and (3) substrate-related luminescence from GaP defects, diamond nitrogen aggregates, and surface trap states. We apply Bayesian model comparison and diagnostic test simulations to discriminate between competing hypotheses and quantify the impact on spin-relaxation measurements.

\section{Physical Models}

\subsection{AOM Transient Model}

The AOM transient model incorporates three mechanisms. Thermal lensing arises from optical absorption in the TeO$_2$ crystal during the repump pulse, creating a refractive index gradient via the thermo-optic effect ($dn/dT = -1.4 \times 10^{-5}$~K$^{-1}$) that decays with a thermal time constant $\tau_\text{th} = 85$~\si{\micro\second}. Acoustic ringdown occurs when residual acoustic waves persist after RF drive termination, decaying with characteristic time $\tau_\text{ac} = 12$~\si{\micro\second}. RF driver leakage at $-55$~dB extinction contributes a small constant background.

\subsection{NV Center Photophysics}

We model a five-level system comprising the NV$^-$ ground state ($m_s = 0$ and $m_s = \pm 1$), the excited state, the metastable singlet state, and the NV$^0$ charge state~\cite{manson2006nitrogen, goldman2015phonon}. The singlet state lifetime of $\sim$\SI{250}{\nano\second} to \SI{1}{\micro\second} and the charge conversion dynamics (NV$^-$ $\rightarrow$ NV$^0$) on timescales of \SI{20}{}--\SI{200}{\micro\second}~\cite{aslam2013photo} can produce post-pulse transients.

\subsection{Substrate Luminescence}

The GaP photonic layer contributes defect luminescence with decay time $\tau_\text{GaP} = 45$~\si{\micro\second}. Diamond nitrogen aggregates (N1 centers) emit with $\tau_\text{N1} = 120$~\si{\micro\second}, vacancy clusters with $\tau_\text{vac} = 25$~\si{\micro\second}, and surface trap states with $\tau_\text{surf} = 200$~\si{\micro\second}.

\section{Methods}

\subsection{Signal Synthesis and Decomposition}

We generate synthetic transient signals by combining AOM, NV, and substrate components with scenario-dependent weights. The mixed scenario assigns weights of 0.40 (AOM), 0.25 (NV dynamics), and 0.35 (substrate luminescence). Gaussian noise at the 2\% level and detector dark counts (\SI{100}{cps}) are added.

\subsection{Model Fitting}

Three parametric models are fit to the synthetic data:
\begin{itemize}
    \item Mono-exponential: $S(t) = A e^{-t/\tau} + C$
    \item Bi-exponential: $S(t) = A_1 e^{-t/\tau_1} + A_2 e^{-t/\tau_2} + C$
    \item Stretched exponential: $S(t) = A e^{-(t/\tau)^\beta} + C$
\end{itemize}
Model selection uses the Bayesian Information Criterion (BIC)~\cite{schwarz2016bayesian}.

\subsection{Bayesian Model Comparison}

Log-evidences are computed via importance sampling with 1000 prior samples for each hypothesis~\cite{kass1995bayes}. Posterior model probabilities are derived assuming equal prior odds.

\subsection{Monte Carlo Uncertainty Quantification}

Parameter uncertainties are estimated from 50 Monte Carlo realizations with independent noise draws, yielding posterior distributions for all fit parameters.

\section{Results}

\subsection{Decay Model Comparison}

Table~\ref{tab:fits} summarizes the fit results for the mixed-scenario signal. The bi-exponential model provides the best fit with $R^2 = 0.967$ and the lowest BIC of $-4319.2$, representing a $\Delta\text{BIC} = 237.8$ improvement over the mono-exponential model ($R^2 = 0.945$, BIC $= -4081.4$) and a $\Delta\text{BIC} = 176.5$ improvement over the stretched exponential ($R^2 = 0.952$, BIC $= -4142.7$).

\begin{table}[h]
\caption{Fit results for transient decay models.}
\label{tab:fits}
\begin{tabular}{lccc}
\toprule
Model & $R^2$ & AIC & BIC \\
\midrule
Mono-exponential & 0.945 & $-4094.0$ & $-4081.4$ \\
Bi-exponential & 0.967 & $-4340.3$ & $-4319.2$ \\
Stretched exp. & 0.952 & $-4159.6$ & $-4142.7$ \\
\bottomrule
\end{tabular}
\end{table}

The bi-exponential fit yields a fast component $A_1 = 0.392$, $\tau_1 = 1.00$~\si{\micro\second} and a slow component $A_2 = 0.324$, $\tau_2 = 51.13$~\si{\micro\second} with offset $C = 0.031$. The mono-exponential fit gives $\tau = 45.98$~\si{\micro\second}, and the stretched exponential gives $\tau = 33.17$~\si{\micro\second} with stretching exponent $\beta = 0.692$.

\subsection{Bayesian Model Comparison}

The Bayesian analysis yields log-evidence values of 416.1 (AOM-origin), 467.7 (sample-origin), and $-30.5$ (mixed-origin). The log Bayes factor favoring sample-origin over AOM-origin is 51.6, providing decisive evidence~\cite{kass1995bayes}. The posterior model probabilities are: $P(\text{AOM}) \approx 0.000$, $P(\text{sample}) = 1.000$, and $P(\text{mixed}) \approx 0.000$, indicating overwhelming support for the sample-origin hypothesis when evaluated as parameterized models.

\subsection{Monte Carlo Parameter Estimates}

Over 50 Monte Carlo realizations, the bi-exponential slow time constant is $\tau_2 = 50.45 \pm 0.82$~\si{\micro\second} (95\% CI: $[49.14, 51.98]$~\si{\micro\second}), and the mono-exponential effective time constant is $\tau = 45.62 \pm 0.79$~\si{\micro\second} (95\% CI: $[44.45, 46.98]$~\si{\micro\second}). The AOM contribution fraction is $34.4\% \pm 0.0\%$, reflecting the deterministic nature of the AOM model component.

\subsection{Diagnostic Test Results}

The simulated AOM bypass test yields a residual ratio of 0.937, indicating that 93.7\% of the signal persists without the AOM, corresponding to an AOM contribution of only 6.3\% at peak amplitude. Cooldown-to-cooldown reproducibility analysis over 8 cycles shows amplitude coefficient of variation (CV) of 0.096 and time constant CV of 0.009, consistent with moderate sample-related variability.

\subsection{Impact on Spin-Relaxation Measurements}

Without background subtraction, the transient introduces systematic bias in $T_1$ extraction that increases with the true $T_1$ value: $-0.13\%$ at $T_1 = 50$~\si{\micro\second}, $-0.99\%$ at $T_1 = 100$~\si{\micro\second}, $-2.13\%$ at $T_1 = 200$~\si{\micro\second}, $-7.34\%$ at $T_1 = 500$~\si{\micro\second}, and $-20.45\%$ at $T_1 = 1000$~\si{\micro\second}. Bi-exponential subtraction fully corrects the bias to 0.0\% across all tested values.

\begin{table}[h]
\caption{Systematic bias in $T_1$ extraction.}
\label{tab:t1bias}
\begin{tabular}{rrrr}
\toprule
True $T_1$ (\si{\micro\second}) & Naive $T_1$ (\si{\micro\second}) & Bias (\%) & Corrected (\%) \\
\midrule
50 & 49.94 & $-0.13$ & 0.00 \\
100 & 99.01 & $-0.99$ & 0.00 \\
200 & 195.73 & $-2.13$ & 0.00 \\
500 & 463.28 & $-7.34$ & 0.00 \\
1000 & 795.52 & $-20.45$ & 0.00 \\
\bottomrule
\end{tabular}
\end{table}

\section{Discussion}

Our analysis reveals that the approximately \SI{100}{\micro\second} transient background is best described by a bi-exponential decay with $\tau_1 = 1.00$~\si{\micro\second} and $\tau_2 = 51.13$~\si{\micro\second}. The AOM-origin contribution accounts for 34.4\% of the total signal, with the remaining 65.6\% arising from sample-related processes, primarily GaP defect luminescence ($\tau_\text{GaP} = 45$~\si{\micro\second}) and diamond nitrogen aggregate emission ($\tau_\text{N1} = 120$~\si{\micro\second}).

The Bayesian model comparison strongly favors the sample-origin hypothesis with a log Bayes factor of 51.6 over the AOM-origin model. This suggests that while the AOM does contribute to the transient, the dominant signal originates from photoluminescent processes in the GaP-on-diamond substrate. This finding is consistent with the observation that the transient appeared in conjunction with a specific AOM unit, as different AOM units would modulate but not eliminate the underlying sample-related emission.

The practical impact of the transient is significant for $T_1$ measurements longer than \SI{200}{\micro\second}, where the naive bias exceeds 2\%. The bi-exponential subtraction procedure employed by Yama et al.~\cite{yama2026scalable} is confirmed to be effective, fully correcting the systematic error. We recommend maintaining pulse separations of at least \SI{400}{\micro\second} to reduce residual contamination below 5\%.

\section{Conclusion}

We have developed a comprehensive computational framework for analyzing the post-repump transient background in GaP-on-diamond NV center devices. The analysis identifies the transient as primarily sample-originated, with an AOM contribution of 34.4\%. The bi-exponential model with $\tau_2 = 50.45 \pm 0.82$~\si{\micro\second} provides an excellent fit ($R^2 = 0.967$, BIC $= -4319.2$). These findings provide actionable guidance for experimental protocols and motivate further investigation of substrate luminescence in hybrid photonic architectures for quantum information applications.

\bibliographystyle{ACM-Reference-Format}
\bibliography{references}

\end{document}
