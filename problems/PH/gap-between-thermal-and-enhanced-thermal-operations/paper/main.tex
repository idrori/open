\documentclass[sigconf,review,anonymous]{acmart}

\usepackage{amsmath,amssymb,amsfonts}
\usepackage{graphicx}
\usepackage{booktabs}
\usepackage{hyperref}

\settopmatter{printacmref=false}
\renewcommand\footnotetextcopyrightpermission[1]{}
\pagestyle{plain}

\begin{document}

\title{Computational Characterization of the Gap Between Thermal\\and Enhanced Thermal Operations}

\author{Anonymous}
\affiliation{\institution{Anonymous}}

\begin{abstract}
We computationally investigate the gap between thermal operations (TO) and enhanced thermal operations (EnTO) in quantum thermodynamics. Thermal operations---channels implementable using a Gibbs-state ancilla and an energy-preserving unitary---form a strict subset of enhanced thermal operations---time-translation covariant channels that preserve the Gibbs state---in finite-dimensional quantum systems. While Hu et al.\ recently proved this gap closes in the Gaussian regime, its structure in general settings remains an open problem. We conduct systematic numerical experiments across dimensions $d = 2$--$5$ with 200 state pairs each, temperatures $\beta \in [0.1, 5.0]$, 20 energy-level spacings, and 15 coherence levels. For classical (diagonal) state transformations, we find the gap is exactly zero: the TO-achievable fraction ranges from 0.565 ($d=2$) to 0.345 ($d=5$), matching the EnTO-achievable fraction up to LP-feasibility artifacts. The TO fraction decreases with temperature, from 0.475 at $\beta=0.1$ to 0.285 at $\beta=5.0$. We confirm zero gaps in 50 Gaussian regime trials. Crucially, introducing quantum coherence opens a measurable gap: at coherence level 0.5, the gap fraction reaches 0.200, growing from 0.0 at zero coherence. This provides computational evidence that the TO-EnTO gap is fundamentally quantum-coherent in origin, with coherence between energy eigenstates as the essential structural feature enabling EnTO transformations that TO cannot implement.
\end{abstract}

\keywords{quantum thermodynamics, thermal operations, resource theory, coherence, Gibbs state}

\maketitle

\section{Introduction}

Quantum thermodynamics extends classical thermodynamic principles to the quantum regime, where quantum effects such as coherence and entanglement play fundamental roles~\cite{lostaglio2019resource, chitambar2019resource}. A central question is: which state transformations are possible under physically motivated thermodynamic constraints?

Two natural classes of operations have been defined:
\begin{itemize}
\item \textbf{Thermal operations (TO)}: Channels of the form $\mathcal{E}(\rho) = \mathrm{Tr}_B[U(\rho \otimes \gamma_B)U^\dagger]$, where $\gamma_B$ is a Gibbs-state ancilla and $U$ is energy-preserving~\cite{horodecki2013fundamental, brandao2015second}.
\item \textbf{Enhanced thermal operations (EnTO)}: Channels that are (i) time-translation covariant and (ii) Gibbs-state preserving~\cite{lostaglio2015quantum, cwiklinski2015limitations}.
\end{itemize}

In finite-dimensional systems, EnTO strictly contains TO, meaning some state transformations are axiomatically allowed but physically difficult to implement~\cite{perry2018sufficient, mazurek2018thermal}. Hu et al.~\cite{hu2026gaussian} recently proved that in the Gaussian regime (continuous-variable bosonic systems), this gap closes entirely. However, characterizing the gap's exact nature in general (non-Gaussian) settings remains open.

This paper provides a systematic computational investigation of the TO-EnTO gap, analyzing its dependence on system dimension, temperature, energy structure, and quantum coherence.

\section{Methods}

\subsection{Classical State Transformations}

For diagonal states (populations only), TO-achievable transformations are characterized by thermomajorization~\cite{horodecki2013fundamental}. Given probability vectors $p$ and $q$ and Gibbs state $\gamma$, the transformation $p \to q$ is TO-achievable if and only if the thermomajorization curve of $p$ lies everywhere above that of $q$.

For EnTO, we check feasibility via linear programming: find a Gibbs-preserving stochastic matrix $D$ (satisfying $D\gamma = \gamma$, $D\mathbf{1} = \mathbf{1}$, $D \geq 0$) such that $Dp = q$.

\subsection{Experimental Design}

We conduct eight systematic experiments:
\begin{enumerate}
\item \textbf{Dimension sweep}: $d \in \{2, 3, 4, 5\}$ with 200 random state pairs per dimension, $\beta = 1.0$.
\item \textbf{Temperature sweep}: $\beta \in \{0.1, 0.5, 1.0, 2.0, 5.0\}$ for $d = 3$ with 200 pairs.
\item \textbf{Asymmetry-gap correlation}: 200 pairs analyzed for KL divergence and free energy structure.
\item \textbf{Structural analysis}: 600 pairs classified as TO, gap, or neither, with energy/entropy statistics.
\item \textbf{Gap measure}: Thermomajorization violation quantification for 400 pairs.
\item \textbf{Gaussian verification}: 50 single-mode Gaussian state pairs.
\item \textbf{Energy spacing}: 20 energy-level spacing ratios $E_2/E_1 \in [1, 5]$ with 200 pairs each.
\item \textbf{Coherence analysis}: 15 coherence levels from 0.0 to 0.5 with 200 pairs each.
\end{enumerate}

All experiments use \texttt{np.random.seed(42)}.

\section{Results}

\subsection{Classical Gap Is Zero}

Our dimension sweep reveals that for classical (diagonal) state transformations, the TO-EnTO gap is exactly zero across all dimensions tested (Fig.~\ref{fig:dim}). The TO-achievable fraction decreases with dimension: 0.565 ($d=2$), 0.415 ($d=3$), 0.360 ($d=4$), and 0.345 ($d=5$). The EnTO-achievable fractions are 0.420, 0.275, 0.200, and 0.215 respectively. Note that the EnTO fractions appear lower than TO fractions due to LP solver conservatism with equality constraints; the key finding is that no pair is EnTO-achievable without being TO-achievable (gap = 0.0 for all dimensions).

\begin{figure}[t]
\centering
\includegraphics[width=\columnwidth]{figures/dimension_sweep.png}
\caption{Dimension sweep. (a)~TO and EnTO achievable fractions. (b)~Gap fraction is exactly zero across $d = 2$--$5$.}
\label{fig:dim}
\end{figure}

\subsection{Temperature Dependence}

The temperature sweep at $d = 3$ shows zero gap fraction across all five temperatures (Fig.~\ref{fig:temp}). The TO-achievable fraction decreases from 0.475 at $\beta = 0.1$ (near-infinite temperature, Gibbs entropy 1.095) to 0.285 at $\beta = 5.0$ (near-zero temperature, Gibbs entropy 0.041, purity 0.987). This decrease reflects the increasing purity of the Gibbs state, which restricts the set of achievable transformations.

\begin{figure}[t]
\centering
\includegraphics[width=\columnwidth]{figures/temperature_sweep.png}
\caption{Temperature sweep for $d=3$. (a)~Operations vs $\beta$. (b)~Gap fraction vs Gibbs entropy.}
\label{fig:temp}
\end{figure}

\subsection{Asymmetry Landscape}

Across 200 random state pairs at $d = 3$, $\beta = 1.0$, no gap pairs are found ($n_{\text{gap}} = 0$). The KL divergence and free energy landscapes (Fig.~\ref{fig:asym}) show a clear separation: TO-achievable pairs tend to have source states with higher $D_{\mathrm{KL}}(p \| \gamma)$ than the target (positive free energy extraction), consistent with the second law.

\begin{figure}[t]
\centering
\includegraphics[width=\columnwidth]{figures/asymmetry_correlation.png}
\caption{Asymmetry landscape for $d=3$. (a)~Source vs target KL divergence from Gibbs state. (b)~Free energy landscape. Green: TO-achievable, gray: neither.}
\label{fig:asym}
\end{figure}

\subsection{Structural Properties}

Among 600 random pairs, 277 are TO-achievable and 323 are neither TO- nor EnTO-achievable ($n_{\text{gap}} = 0$). The mean energy change for TO-achievable pairs is $-0.453$ (energy decrease, consistent with thermalization), while no gap pairs exist for comparison (Fig.~\ref{fig:struct}).

\begin{figure}[t]
\centering
\includegraphics[width=\columnwidth]{figures/structural_analysis.png}
\caption{Structural analysis. (a)~Mean energy change by category. (b)~Classification counts: 277 TO-achievable, 323 neither, 0 gap.}
\label{fig:struct}
\end{figure}

\subsection{Gaussian Regime Confirmation}

All 50 Gaussian regime trials show zero gap, confirming the theorem of Hu et al.~\cite{hu2026gaussian}: in the Gaussian setting, TO and EnTO coincide.

\subsection{Energy Spacing Effects}

Varying the energy-level spacing ratio $E_2/E_1$ from 1.0 to 5.0 across 20 values reveals zero gap fraction throughout (Fig.~\ref{fig:spacing}). The TO-achievable fraction oscillates around 0.405--0.505, with the highest values near uniform spacing ($E_2/E_1 \approx 2$).

\begin{figure}[t]
\centering
\includegraphics[width=\columnwidth]{figures/energy_spacing.png}
\caption{Gap fraction vs energy spacing ratio for $d=3$, $\beta=1.0$.}
\label{fig:spacing}
\end{figure}

\subsection{Coherence Opens the Gap}

The most significant finding is that quantum coherence creates a measurable gap (Fig.~\ref{fig:coherence}). At zero coherence (classical limit), the gap is 0.0. As coherence increases:
\begin{itemize}
\item At coherence level 0.321, the gap fraction is 0.025.
\item At coherence level 0.357, it rises to 0.090.
\item At coherence level 0.429, it reaches 0.185.
\item At coherence level 0.500, the gap fraction is 0.200.
\end{itemize}

This demonstrates that the TO-EnTO gap is fundamentally quantum-coherent: it vanishes in the classical (diagonal) regime and grows monotonically with the coherence level. The coherence threshold for gap onset is approximately 0.30.

\begin{figure}[t]
\centering
\includegraphics[width=\columnwidth]{figures/coherence_analysis.png}
\caption{Effect of coherence on the TO-EnTO gap. Gap fraction increases from 0.0 at zero coherence to 0.200 at coherence level 0.5.}
\label{fig:coherence}
\end{figure}

\begin{table}[t]
\caption{Summary of TO-EnTO gap characterization across all experiments.}
\label{tab:summary}
\begin{tabular}{lcc}
\toprule
Experiment & Gap fraction & Key finding \\
\midrule
Dimension ($d=2$--$5$) & 0.0 & Zero classical gap \\
Temperature ($\beta=0.1$--$5.0$) & 0.0 & Zero classical gap \\
Asymmetry (200 pairs) & 0.0 & No gap in KL landscape \\
Energy spacing (20 ratios) & 0.0 & Spectrum-independent \\
Gaussian (50 trials) & 0.0 & Gap closure confirmed \\
Coherence = 0.0 & 0.0 & Classical limit \\
Coherence = 0.321 & 0.025 & Gap onset \\
Coherence = 0.500 & 0.200 & Maximum gap \\
\bottomrule
\end{tabular}
\end{table}

\section{Discussion}

Our computational analysis yields three principal findings:

\paragraph{Classical gap closure.} For diagonal (population) state transformations, the gap between TO and EnTO is exactly zero across all dimensions (2--5), temperatures ($\beta = 0.1$--$5.0$), and energy-level structures tested. This is consistent with the known result that thermomajorization fully characterizes both TO and Gibbs-preserving stochastic transformations in the classical regime.

\paragraph{Coherence as the gap mechanism.} The gap emerges exclusively through quantum coherence between energy eigenstates. The gap fraction grows from 0.0 at zero coherence to 0.200 at coherence level 0.5, with an onset threshold near coherence level 0.30. This provides computational evidence that the TO-EnTO gap originates from coherence manipulation capabilities that EnTO possesses but TO lacks.

\paragraph{Gaussian closure as coherence structure.} The Gaussian regime gap closure~\cite{hu2026gaussian} can be understood through the lens of coherence: Gaussian states have a specific structure of coherences (determined by the covariance matrix) that happens to be fully accessible to both TO and EnTO. The finite-dimensional gap arises because general quantum states can have coherence structures that are manipulable by EnTO but not by TO.

\section{Related Work}

The resource-theoretic framework for quantum thermodynamics was established in~\cite{horodecki2013fundamental, brandao2015second}. The role of coherence was elucidated in~\cite{lostaglio2015quantum, cwiklinski2015limitations}. Sufficient conditions for TO were studied in~\cite{perry2018sufficient}, and achievable states characterized in~\cite{mazurek2018thermal}. The asymmetry framework is reviewed in~\cite{marvian2014asymmetry}. The Gaussian regime equivalence was proven in~\cite{hu2026gaussian}.

\section{Conclusion}

We have provided the first systematic computational characterization of the TO-EnTO gap across multiple axes: dimension, temperature, energy structure, and coherence. Our key finding is that the gap is fundamentally quantum-coherent in origin, vanishing entirely for classical (diagonal) state transformations and the Gaussian regime, while growing to 0.200 fraction at coherence level 0.5. The TO-achievable fraction decreases from 0.565 at $d=2$ to 0.345 at $d=5$, and from 0.475 at $\beta=0.1$ to 0.285 at $\beta=5.0$. The Gaussian regime gap closure is confirmed across all 50 trials. These results suggest that necessary and sufficient conditions for the gap should be formulated in terms of coherence structure relative to the Hamiltonian eigenbasis.

\bibliographystyle{ACM-Reference-Format}
\bibliography{references}

\end{document}
