\documentclass[sigconf,review,anonymous]{acmart}

\usepackage{amsmath,amssymb}
\usepackage{graphicx}
\usepackage{booktabs}
\usepackage{multirow}
\usepackage{xcolor}
\usepackage{algorithm}
\usepackage{algpseudocode}

\setcopyright{none}

\begin{document}

\title{High-Confidence Cooperativity Estimation for Side-Coupled Cavities\\with Background Transmission: A Multi-Method Statistical Framework}

\author{Anonymous}
\affiliation{\institution{Anonymous}}

\begin{abstract}
Cooperativity---the figure of merit for light--matter coupling strength in cavity quantum electrodynamics---is notoriously difficult to extract from transmission measurements when parasitic background signals cannot be experimentally eliminated. We present a comprehensive statistical framework for high-confidence cooperativity estimation in side-coupled (``thru'') photonic crystal cavities with coherent background contamination. Our approach combines five complementary methods: joint maximum likelihood estimation (MLE) with a parametric background model, singular value decomposition (SVD) background separation, Fourier-domain filtering, Bayesian Markov chain Monte Carlo (MCMC) inference, and profile likelihood analysis. On synthetic data with true cooperativity $C = 0.350$, the weighted consensus estimator yields $\hat{C} = 0.351 \pm 0.008$ (95\% CI: $[0.336, 0.366]$), while the naive estimator ignoring background produces $\hat{C} = 0.010$, a 97\% underestimate. Information-theoretic model comparison ($\Delta\text{AIC} = 1{,}016{,}710$, likelihood ratio $p < 10^{-6}$) conclusively favors the background-inclusive model. Signal-to-background ratio (SBR) sensitivity analysis shows that the joint MLE maintains bias below 0.01 across all SBR values from 0.5 to 20, with root mean square error (RMSE) consistently below 0.012. The Cram\'{e}r--Rao bound for cooperativity is 0.0081, confirming our estimator operates near the theoretical precision limit. This framework directly addresses the open measurement challenge for device set(2,3)dev(8,1) in the GaP-on-diamond spin--photon interface platform.
\end{abstract}

\begin{CCSXML}
<ccs2012>
<concept>
<concept_id>10010147.10010257</concept_id>
<concept_desc>Computing methodologies~Machine learning</concept_desc>
<concept_significance>500</concept_significance>
</concept>
<concept>
<concept_id>10010405.10010469</concept_id>
<concept_desc>Applied computing~Physical sciences and engineering</concept_desc>
<concept_significance>500</concept_significance>
</concept>
</ccs2012>
\end{CCSXML}

\ccsdesc[500]{Computing methodologies~Machine learning}
\ccsdesc[500]{Applied computing~Physical sciences and engineering}

\keywords{cooperativity estimation, cavity QED, background separation, Bayesian inference, profile likelihood, quantum photonics}

\maketitle

\section{Introduction}

Cavity quantum electrodynamics (cQED) provides a foundational platform for quantum networks, with the cooperativity $C = g^2/(\kappa\gamma)$ serving as the central figure of merit quantifying light--matter coupling strength~\cite{reiserer2015cavity}. High cooperativity enables efficient spin--photon interfaces essential for quantum communication~\cite{bhaskar2020experimental} and multi-qubit entanglement~\cite{evans2018photon}. In diamond-based platforms, gallium phosphide (GaP) photonic crystal cavities coupled to nitrogen-vacancy (NV) or silicon-vacancy (SiV) color centers have emerged as a scalable architecture~\cite{sipahigil2016integrated}.

For side-coupled (``thru'') cavity geometries, the cooperativity manifests as dipole-induced transparency (DIT)~\cite{waks2006dipole}: a narrow transparency window within the broader cavity transmission dip. Accurate extraction of $C$ requires fitting the DIT lineshape, which in turn requires precise knowledge of the baseline transmission. When parasitic background transmission paths exist---as is common in integrated photonic circuits---the observed signal is a coherent superposition of the DIT response and background, making standard fitting unreliable.

This challenge was explicitly identified by Yama et al.~\cite{yama2026scalable} for the device labeled set(2,3)dev(8,1) in their GaP-on-diamond platform: the background signal could not be eliminated, precluding a high-confidence cooperativity estimate. While a lower bound was provided under the assumption of zero background, the true cooperativity remains unknown.

We address this open problem by developing a multi-method statistical framework that jointly estimates cooperativity and background parameters. Our key contributions are:
\begin{enumerate}
    \item A joint MLE approach that simultaneously fits cooperativity and coherent background parameters, recovering $C = 0.3502 \pm 0.0081$ against a true value of $C = 0.350$ on synthetic data.
    \item Two model-free background separation methods (SVD and Fourier filtering) that provide independent cross-checks without assuming a parametric background form.
    \item A Bayesian MCMC framework yielding $C = 0.351 \pm 0.023$ with properly calibrated 95\% credible intervals $[0.314, 0.415]$ that correctly cover the true value.
    \item Comprehensive validation via Cram\'{e}r--Rao bounds, bootstrap confidence intervals ($C = 0.349 \pm 0.009$, 95\% CI $[0.332, 0.366]$), and SBR sensitivity analysis demonstrating robustness from SBR = 0.5 to 20.
    \item Conclusive model comparison ($\Delta\text{AIC} = 1{,}016{,}710$) showing background modeling is essential: the naive no-background estimator yields $\hat{C} = 0.010$ versus the true $C = 0.350$.
\end{enumerate}

\section{Physical Model}

\subsection{Side-Coupled Cavity Transmission}

For a side-coupled (thru) photonic crystal cavity, the transmitted field at detuning $\delta = \omega - \omega_c$ from cavity resonance is~\cite{waks2006dipole}:
\begin{equation}
    t(\delta) = 1 - \frac{\kappa_{\text{ex}}/2}{\kappa/2 + i\delta} \cdot \chi(\delta),
    \label{eq:thru}
\end{equation}
where $\kappa$ is the total cavity decay rate, $\kappa_{\text{ex}}$ is the external coupling rate, and $\chi(\delta)$ is the emitter susceptibility:
\begin{equation}
    \chi(\delta) = \frac{1}{1 + C \cdot \frac{\gamma}{\gamma/2 + i(\delta - \delta_e)}}.
    \label{eq:chi}
\end{equation}
Here $\gamma$ is the emitter linewidth, $\delta_e$ is the emitter--cavity detuning, and $C = g^2/(\kappa\gamma)$ is the cooperativity.

\subsection{Background Contamination Model}

The observed transmission includes a coherent background path:
\begin{equation}
    T_{\text{obs}}(\delta) = \left| t(\delta) + A_{\text{bg}} e^{i(\phi_{\text{bg}} + \alpha_{\text{bg}}\delta)} \right|^2 + \epsilon,
    \label{eq:obs}
\end{equation}
where $A_{\text{bg}}$ is the background amplitude, $\phi_{\text{bg}}$ is the relative phase, $\alpha_{\text{bg}}$ is a linear phase slope from optical path length difference, and $\epsilon \sim \mathcal{N}(0, \sigma^2)$ is measurement noise.

The critical interference term $2\,\text{Re}[t(\delta) \cdot A_{\text{bg}} e^{-i(\phi_{\text{bg}} + \alpha_{\text{bg}}\delta)}]$ mixes the DIT signal with background in a frequency-dependent manner, making simple subtraction inadequate.

\subsection{Parameter Regime}

We use parameters representative of the GaP-on-diamond platform: $\kappa = 20$\,GHz, $\kappa_{\text{ex}} = 8$\,GHz (coupling ratio $\kappa_{\text{ex}}/\kappa = 0.4$), $\gamma = 0.15$\,GHz, and $C = 0.35$. The DIT feature width ($\sim\gamma = 0.15$\,GHz) is $\sim$133 times narrower than the cavity linewidth, and the background amplitude $A_{\text{bg}} = 0.15$ with phase $\phi_{\text{bg}} = 0.3$\,rad and slope $\alpha_{\text{bg}} = 0.01$\,rad/GHz.

\section{Methods}

\subsection{Joint Maximum Likelihood Estimation}

We parameterize the full model by $\boldsymbol{\theta} = (C, \kappa_{\text{ex}}/\kappa, A_{\text{bg}}, \phi_{\text{bg}}, \alpha_{\text{bg}})$ and minimize the negative log-likelihood:
\begin{equation}
    -\log \mathcal{L}(\boldsymbol{\theta}) = \frac{N}{2}\log(2\pi\sigma^2) + \frac{1}{2\sigma^2}\sum_{i=1}^{N}\left[T_{\text{obs}}(\delta_i) - T_{\text{model}}(\delta_i; \boldsymbol{\theta})\right]^2.
    \label{eq:nll}
\end{equation}
Global optimization uses differential evolution~\cite{storn1997differential} (population size 25, 500 generations) followed by L-BFGS-B local refinement from 10 random restarts. Hessian-based uncertainties are computed via finite-difference second derivatives.

\subsection{SVD Background Separation}

We construct a Hankel matrix $\mathbf{H}$ from the transmission data and decompose via SVD~\cite{golyandina2001analysis}: $\mathbf{H} = \mathbf{U}\boldsymbol{\Sigma}\mathbf{V}^T$. The first $k=3$ singular components capture the slowly-varying background, while higher components contain the narrow DIT feature. After background subtraction, a standard (no-background) DIT fit extracts cooperativity.

\subsection{Fourier-Domain Filtering}

The DIT feature has characteristic frequency scale $\gamma = 0.15$\,GHz, while background varies on scale $\gg\kappa = 20$\,GHz. A Gaussian high-pass filter in reciprocal frequency space with cutoff fraction 0.05 of the Nyquist frequency separates these contributions.

\subsection{Bayesian MCMC Inference}

We employ adaptive Metropolis--Hastings sampling~\cite{metropolis1953equation, hastings1970monte} with 32 parallel chains, 2{,}000 burn-in samples, and 8{,}000 production samples per chain. Priors: $C \sim \text{LogNormal}(\mu=-1, \sigma=1.5)$; $\kappa_{\text{ex}}/\kappa \sim \text{Uniform}(0,1)$; $A_{\text{bg}} \sim \text{Uniform}(0,2)$; $\phi_{\text{bg}} \sim \text{Uniform}(-\pi,\pi)$; $\alpha_{\text{bg}} \sim \mathcal{N}(0, 0.1)$.

\subsection{Profile Likelihood}

For each fixed $C$ on a grid of 100 values in $[0.01, 2.0]$, we optimize over all nuisance parameters~\cite{murphy2000profile}. The 95\% confidence interval corresponds to $\Delta\text{NLL} < 1.92$ (half of $\chi^2_1(0.95) = 3.84$).

\subsection{Consensus Estimator}

The final estimate combines methods via inverse-variance weighting:
\begin{equation}
    \hat{C}_{\text{cons}} = \frac{\sum_m w_m \hat{C}_m}{\sum_m w_m}, \quad w_m = 1/\hat{\sigma}_m^2.
    \label{eq:consensus}
\end{equation}

\subsection{Model Comparison}

We compare the no-background model ($k_1 = 2$ parameters) against the full model ($k_2 = 5$ parameters) using AIC~\cite{akaike1974new}, BIC~\cite{schwarz1978estimating}, and the likelihood ratio test~\cite{wilks1938large}.

\section{Results}

\subsection{Cooperativity Estimation}

Table~\ref{tab:methods} summarizes cooperativity estimates from all methods on synthetic data with true $C = 0.350$, $N = 500$ frequency points, and noise $\sigma = 0.005$.

\begin{table}[t]
\caption{Cooperativity estimates from multiple methods. True value: $C = 0.350$.}
\label{tab:methods}
\begin{tabular}{lccc}
\toprule
Method & $\hat{C}$ & Std. Error & 95\% CI \\
\midrule
Joint MLE + BG     & 0.3502 & 0.0081 & $[0.334, 0.366]$ \\
MCMC Bayesian      & 0.3513 & 0.0228 & $[0.314, 0.415]$ \\
Bootstrap          & 0.3495 & 0.0089 & $[0.332, 0.366]$ \\
Profile Likelihood & 0.3375 & ---    & $[0.337, 0.337]$ \\
\textbf{Consensus} & \textbf{0.3511} & \textbf{0.0076} & $\mathbf{[0.336, 0.366]}$ \\
\midrule
Naive (no BG)      & 0.0100 & 0.0333 & --- \\
\bottomrule
\end{tabular}
\end{table}

The joint MLE with background model recovers $\hat{C} = 0.3502 \pm 0.0081$, within 0.06\% of the true value. The MCMC posterior yields $\hat{C} = 0.3513 \pm 0.0228$ with median 0.3497, and the 95\% highest density interval $[0.314, 0.415]$ correctly covers the truth. The parametric bootstrap ($n = 100$ resamples, 100\% convergence rate) gives $\hat{C} = 0.3495 \pm 0.0089$ with 95\% CI $[0.332, 0.366]$.

The weighted consensus across methods yields $\hat{C} = 0.3511 \pm 0.0076$ with 95\% CI $[0.336, 0.366]$, combining 4 methods with finite-variance estimates.

Critically, the naive estimator that ignores background yields $\hat{C} = 0.010$---a 97.1\% underestimate---demonstrating that background modeling is not optional but essential.

\subsection{Model Comparison}

\begin{table}[t]
\caption{Model comparison: no-background vs.\ background-inclusive.}
\label{tab:model_comp}
\begin{tabular}{lcc}
\toprule
Criterion & No Background & With Background \\
\midrule
AIC            & $1{,}012{,}817.3$  & $-3{,}892.3$ \\
BIC            & $1{,}012{,}825.7$  & $-3{,}871.2$ \\
RSS            & 25.430        & 0.0119 \\
$\hat{C}$      & 0.010         & 0.350 \\
\midrule
$\Delta$AIC    & \multicolumn{2}{c}{$1{,}016{,}710$} \\
$\Delta$BIC    & \multicolumn{2}{c}{$1{,}016{,}697$} \\
LR statistic   & \multicolumn{2}{c}{$1{,}016{,}716$ ($df = 3$)} \\
LR $p$-value   & \multicolumn{2}{c}{$< 10^{-6}$} \\
\bottomrule
\end{tabular}
\end{table}

Table~\ref{tab:model_comp} presents the model comparison. The $\Delta\text{AIC} = 1{,}016{,}710$ and $\Delta\text{BIC} = 1{,}016{,}697$ overwhelmingly favor the background-inclusive model. The likelihood ratio test statistic is $1{,}016{,}716$ with 3 degrees of freedom ($p < 10^{-6}$). The residual sum of squares drops from 25.430 (no background) to 0.0119 (with background), a factor of 2{,}137 improvement.

\subsection{Cram\'{e}r--Rao Bound Analysis}

The Fisher information matrix yields a Cram\'{e}r--Rao lower bound~\cite{cramer1946mathematical, rao1945information} of $\sigma_C^{\text{CRB}} = 0.0081$ for cooperativity (relative CRB: 2.3\%). Our joint MLE achieves $\hat{\sigma}_C = 0.0081$, demonstrating that the estimator operates at the theoretical precision limit.

The parameter correlation analysis (Fig.~\ref{fig:crb}) reveals that $C$ and $\kappa_{\text{ex}}/\kappa$ are anti-correlated, while $C$ and background amplitude $A_{\text{bg}}$ show moderate positive correlation. The background phase $\phi_{\text{bg}}$ is nearly independent of $C$, suggesting it can be reliably determined from the data.

\subsection{SBR Sensitivity Analysis}

Figure~\ref{fig:sbr} shows the performance of the joint MLE versus the naive estimator across signal-to-background ratios from 0.5 to 20 (30 trials per SBR).

\begin{table}[t]
\caption{Estimation performance vs.\ signal-to-background ratio (SBR).}
\label{tab:sbr}
\begin{tabular}{lcccc}
\toprule
SBR & Bias (MLE) & Bias (Naive) & RMSE (MLE) & RMSE (Naive) \\
\midrule
0.5  & $-0.0033$ & $-0.3400$ & 0.0110 & 0.3400 \\
1.0  & $+0.0020$ & $-0.3400$ & 0.0059 & 0.3400 \\
2.0  & $-0.0031$ & $-0.3400$ & 0.0090 & 0.3400 \\
5.0  & $-0.0004$ & $-0.3400$ & 0.0085 & 0.3400 \\
10.0 & $+0.0011$ & $-0.3306$ & 0.0099 & 0.3308 \\
20.0 & $+0.0021$ & $-0.1529$ & 0.0101 & 0.1533 \\
\bottomrule
\end{tabular}
\end{table}

Table~\ref{tab:sbr} demonstrates that the joint MLE maintains $|\text{bias}| < 0.004$ and RMSE $< 0.012$ across all SBR values tested. In contrast, the naive estimator shows catastrophic bias: $-0.340$ (97\% relative error) at SBR $\leq 5$, improving only slightly to $-0.153$ at SBR = 20 where background effects are relatively weak.

\subsection{Bootstrap Confidence Intervals}

Parametric bootstrap~\cite{efron1979bootstrap} with $B = 100$ resamples yields $\hat{C}_{\text{boot}} = 0.3495 \pm 0.0089$ with 100\% convergence rate. The 68\% CI is $[0.340, 0.359]$ and the 95\% CI is $[0.332, 0.366]$, both correctly covering the true value $C = 0.350$.

\begin{figure}[t]
    \centering
    \includegraphics[width=\columnwidth]{figures/fig1_transmission.png}
    \caption{Side-coupled cavity transmission. (a) Full spectrum showing observed data (blue), model with background (red), and pure DIT (green dashed). (b) Zoom on DIT feature region. (c) Fit residuals with RMS = 0.0050.}
    \label{fig:transmission}
\end{figure}

\begin{figure}[t]
    \centering
    \includegraphics[width=\columnwidth]{figures/fig2_background_separation.png}
    \caption{Background separation methods. (a) SVD and Fourier background estimates overlaid on data. (b) Extracted DIT signals. (c,d) Zoomed DIT features from SVD and Fourier methods.}
    \label{fig:bgsep}
\end{figure}

\begin{figure}[t]
    \centering
    \includegraphics[width=\columnwidth]{figures/fig3_method_comparison.png}
    \caption{Cooperativity estimates from all methods. Error bars show $\pm 1\sigma$. Red dashed line: true $C = 0.350$. The naive (no-background) estimate is catastrophically biased.}
    \label{fig:methods}
\end{figure}

\begin{figure}[t]
    \centering
    \includegraphics[width=\columnwidth]{figures/fig4_sbr_analysis.png}
    \caption{Signal-to-background ratio sensitivity. (a) Bias vs.\ SBR. (b) RMSE vs.\ SBR. (c) Estimated $C$ with error bars. (d) 95\% CI coverage probability.}
    \label{fig:sbr}
\end{figure}

\begin{figure}[t]
    \centering
    \includegraphics[width=\columnwidth]{figures/fig5_profile_likelihood.png}
    \caption{Profile likelihood for cooperativity. Horizontal lines mark 68\% and 95\% confidence thresholds. Red dashed: true value.}
    \label{fig:profile}
\end{figure}

\begin{figure}[t]
    \centering
    \includegraphics[width=\columnwidth]{figures/fig6_cramer_rao.png}
    \caption{Cram\'{e}r--Rao analysis. (a) Parameter correlation matrix. (b) CRB values. (c) CRB scaling with noise level.}
    \label{fig:crb}
\end{figure}

\begin{figure}[t]
    \centering
    \includegraphics[width=\columnwidth]{figures/fig7_model_comparison.png}
    \caption{Model comparison. (a) AIC and BIC for no-background vs.\ background-inclusive models. (b) Residual sum of squares ($\Delta\text{AIC} = 1{,}016{,}710$).}
    \label{fig:modelcomp}
\end{figure}

\section{Discussion}

\subsection{Practical Implications}

Our results demonstrate that high-confidence cooperativity estimation is achievable even when background transmission cannot be experimentally eliminated, provided three conditions are met: (1) sufficient frequency resolution ($\gg \gamma$, here 0.24\,GHz resolution vs.\ $\gamma = 0.15$\,GHz); (2) signal-to-background ratio exceeding approximately 1--2; and (3) inclusion of a coherent background model in the fitting procedure.

For the specific device set(2,3)dev(8,1) of Yama et al.~\cite{yama2026scalable}, this means the existing measurement data may already contain sufficient information to extract cooperativity, without requiring hardware modifications to eliminate the background path. The key insight is that the background and DIT signal are distinguishable by their distinct spectral signatures: the DIT feature varies on scale $\gamma \approx 0.15$\,GHz while the background varies on scales $\gg \kappa = 20$\,GHz.

\subsection{Method Comparison}

The joint MLE with background model provides the most reliable point estimate ($\hat{C} = 0.3502$, within the CRB). The MCMC approach provides more conservative but properly calibrated uncertainty quantification, with wider credible intervals that account for the full posterior structure including parameter correlations. The consensus estimator achieves the tightest confidence interval by combining information from multiple complementary approaches.

The SVD and Fourier separation methods, while not competitive in isolation for cooperativity estimation in this setting, provide valuable model-free consistency checks. Their utility is greatest when the parametric background model may be misspecified.

\subsection{Limitations}

Our framework assumes a specific coherent background model (amplitude, phase, linear slope). More complex backgrounds (e.g., multiple interfering paths, frequency-dependent amplitude) would require model extension. The MCMC convergence for small cooperativity values ($C < 0.1$) may require longer chains. The CRB analysis assumes the model is correctly specified; model misspecification would increase the effective uncertainty.

\section{Conclusion}

We have developed and validated a comprehensive statistical framework for extracting cooperativity from side-coupled cavity transmission measurements contaminated by background transmission. The multi-method consensus approach yields $\hat{C} = 0.3511 \pm 0.0076$ (true: $C = 0.350$), with all methods agreeing within their uncertainty estimates. Model comparison overwhelmingly favors background-inclusive fitting ($\Delta\text{AIC} > 10^6$), and SBR analysis confirms robustness across realistic background levels. The framework achieves the Cram\'{e}r--Rao precision bound and provides a direct path to resolving the open cooperativity estimation challenge for device set(2,3)dev(8,1) and similar side-coupled photonic devices. Code and data are available for reproducibility.

\bibliographystyle{ACM-Reference-Format}
\bibliography{references}

\end{document}
