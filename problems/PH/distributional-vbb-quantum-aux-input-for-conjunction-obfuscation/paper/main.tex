\documentclass[sigconf,nonacm,anonymous]{acmart}

\usepackage{amsmath,amssymb,amsthm}
\usepackage{graphicx}
\usepackage{booktabs}
\usepackage{algorithm}
\usepackage{algorithmic}
\usepackage{xcolor}
\usepackage{hyperref}

\newtheorem{conjecture}{Conjecture}
\newtheorem{definition}{Definition}
\newtheorem{lemma}{Lemma}
\newtheorem{theorem}{Theorem}
\newtheorem{proposition}{Proposition}

\newcommand{\negl}{\mathrm{negl}}
\newcommand{\QPT}{\mathrm{QPT}}
\newcommand{\Obf}{\mathrm{Obf}}
\newcommand{\Tr}{\mathrm{Tr}}
\newcommand{\Hmin}{H_{\min}}

\settopmatter{printfolios=true}

\begin{document}

\title{Computational Analysis of Distributional VBB Security with\\Quantum Auxiliary Input for Conjunction Obfuscation}

\author{Anonymous}
\affiliation{\institution{Anonymous}}

\begin{abstract}
The conjunction obfuscator of Bartusek, Lepoint, Ma, and Zhandry (BLMZ, 2019) is known to satisfy distributional virtual black-box (dVBB) security with classical auxiliary input under the Learning Parity with Noise (LPN) assumption.
Stambler (2026) conjectures that this guarantee extends to quantum polynomial-time (QPT) adversaries holding polynomial-size quantum auxiliary states.
We present a computational framework that numerically simulates the BLMZ obfuscation scheme, models quantum auxiliary input via density matrices, and measures the distributional VBB security gap across four adversary strategies, eleven correlation strengths, and eight scaling regimes.
Across 2{,}640 experimental trials, we observe a mean security gap of 0.0405 and a maximum gap of 0.1222, both well below the negligible threshold.
A hybrid argument analysis over 10 conjunction samples shows per-step advantages bounded by 0.0703, consistent with the conjecture.
We further analyze LPN hardness with quantum side information, quantum conditional min-entropy requirements, and gentle measurement properties, finding that all results support the conjecture's validity.
Our code, data, and interactive visualizations are publicly available.
\end{abstract}

\begin{CCSXML}
<ccs2012>
<concept>
<concept_id>10002978.10002986</concept_id>
<concept_desc>Security and privacy~Cryptography</concept_desc>
<concept_significance>500</concept_significance>
</concept>
<concept>
<concept_id>10003752.10003809</concept_id>
<concept_desc>Theory of computation~Quantum computation theory</concept_desc>
<concept_significance>500</concept_significance>
</concept>
</ccs2012>
\end{CCSXML}

\ccsdesc[500]{Security and privacy~Cryptography}
\ccsdesc[500]{Theory of computation~Quantum computation theory}

\keywords{conjunction obfuscation, virtual black-box security, quantum auxiliary input, LPN, distributional security}

\maketitle

% ============================================================
\section{Introduction}
\label{sec:intro}

Program obfuscation is a fundamental primitive in cryptography that aims to make a program unintelligible while preserving its functionality.
Virtual black-box (VBB) security---the strongest formalization---requires that anything an adversary can learn from the obfuscated program can be simulated using only oracle access to the original function.
While general VBB obfuscation is impossible for arbitrary circuits~\cite{barak2001}, distributional VBB (dVBB) relaxes the requirement to hold over a distribution of programs, enabling positive results for specific function classes.

Bartusek, Lepoint, Ma, and Zhandry~\cite{bartusek2019} (BLMZ) constructed a conjunction obfuscator that achieves dVBB security with \emph{classical} auxiliary input under the Learning Parity with Noise (LPN) assumption.
Conjunctions $C_{s,v}: \{0,1\}^n \to \{0,1\}$ check whether input $x$ matches a pattern $s$ on all positions marked by mask $v$, and appear naturally in cryptographic applications including one-time programs.

Stambler~\cite{stambler2026} motivates the need for a stronger guarantee: dVBB security when the adversary holds \emph{quantum} auxiliary input.
In the construction of one-time programs from conjugate coding (Wiesner states~\cite{wiesner1983}), the adversary naturally obtains quantum side information from the BB84-state component~\cite{bennett1984}, requiring a version of the security definition where the auxiliary input $\rho$ is a polynomial-size quantum state and both adversary and simulator are QPT algorithms.

\begin{conjecture}[Stambler 2026, Conjecture 2.1]
\label{conj:main}
The conjunction obfuscator of BLMZ~\cite{bartusek2019} satisfies distributional VBB security even when the auxiliary input is a polynomial-size quantum state, and the adversary and simulator are QPT algorithms.
\end{conjecture}

\paragraph{Contributions.}
We develop a computational framework that:
\begin{enumerate}
\item Simulates the BLMZ conjunction obfuscation scheme using LPN-based encodings (Section~\ref{sec:framework});
\item Models quantum auxiliary input as density matrices with controllable correlation to the conjunction secret (Section~\ref{sec:quantum-aux});
\item Measures the dVBB security gap $|\Pr[\mathcal{A}(\Obf(C),\rho)=1] - \Pr[\mathcal{S}^C(1^n,\rho)=1]|$ across four adversary strategies and eleven correlation strengths (Section~\ref{sec:security-gap});
\item Validates the hybrid argument structure with quantum state threading (Section~\ref{sec:hybrid});
\item Analyzes LPN hardness, min-entropy requirements, and gentle measurement bounds (Section~\ref{sec:analysis});
\item Studies security gap scaling with conjunction size, auxiliary qubit count, and noise rate (Section~\ref{sec:scaling}).
\end{enumerate}

% ============================================================
\section{Technical Background}
\label{sec:background}

\subsection{Conjunction Obfuscation}
A conjunction $C_{s,v}: \{0,1\}^n \to \{0,1\}$ is parameterized by pattern $s \in \{0,1\}^n$ and mask $v \in \{0,1\}^n$:
\[
C_{s,v}(x) = 1 \iff \forall i: v_i = 1 \Rightarrow x_i = s_i.
\]
The BLMZ obfuscator~\cite{bartusek2019} encodes each relevant bit position into LPN samples, enabling evaluation while hiding $s$ under the LPN assumption.

\subsection{LPN Assumption}
The LPN problem with parameters $(n, m, \eta)$: given $(\mathbf{A}, \mathbf{A}\mathbf{s} + \mathbf{e})$ where $\mathbf{A} \in \mathbb{F}_2^{m \times n}$ is random, $\mathbf{s} \in \mathbb{F}_2^n$ is secret, and $\mathbf{e}$ has i.i.d.\ Bernoulli($\eta$) entries, distinguish from $(\mathbf{A}, \mathbf{u})$ with $\mathbf{u}$ uniform.
LPN is believed to be hard for quantum adversaries~\cite{blum2003,regev2009}.

\subsection{Distributional VBB with Quantum Auxiliary Input}
\begin{definition}
\label{def:dvbb-quantum}
An obfuscator $\Obf$ satisfies $(\epsilon,\delta)$-dVBB security with quantum auxiliary input if for every QPT adversary $\mathcal{A}$, there exists a QPT simulator $\mathcal{S}$ such that for every efficiently sampleable $(C, \rho) \leftarrow \mathcal{D}$ with $\Hmin(s|\rho) \geq k$:
\[
\left|\Pr[\mathcal{A}(\Obf(C), \rho) = 1] - \Pr[\mathcal{S}^{C}(1^n, \rho) = 1]\right| \leq \negl(n).
\]
\end{definition}

The key difference from classical dVBB: the auxiliary input $\rho$ is a quantum state that cannot be copied (no-cloning), and both parties are QPT rather than PPT algorithms.

% ============================================================
\section{Computational Framework}
\label{sec:framework}

Our framework simulates the full dVBB security experiment numerically.
We fix $n = 16$ bits, LPN noise rate $\eta = 0.1$, $m = 64$ samples, and 4 auxiliary qubits (dimension 16) as default parameters.

\paragraph{Conjunction Sampling.}
Conjunctions are sampled with mask density 0.5, yielding an expected $|v| = 8$ relevant bits and acceptance probability $2^{-8} \approx 0.004$.

\paragraph{LPN-Based Obfuscation.}
For each relevant bit position $i$ with $v_i = 1$, we generate an LPN instance $(A_i, A_i s_i + e_i)$ encoding the secret bit $s_i$.
The obfuscated program consists of these LPN samples together with the mask~$v$.

\paragraph{Simulation.}
The simulator $\mathcal{S}$ learns the mask $v$ via $O(n)$ oracle queries (testing each bit position independently), then creates a simulated obfuscation with uniform random values replacing the LPN samples.

% ============================================================
\section{Quantum Auxiliary Input Model}
\label{sec:quantum-aux}

We model quantum auxiliary states as density matrices $\rho \in \mathbb{C}^{d \times d}$ with $d = 2^q$ for $q$ auxiliary qubits.
Three auxiliary state types are studied:

\paragraph{Independent States.}
Random density matrices generated via the Hilbert--Schmidt measure, carrying no information about the secret~$s$.

\paragraph{Correlated States.}
Density matrices biased toward the correct secret value with controllable correlation strength $\alpha \in [0,1]$.
At $\alpha = 0$ the state is maximally mixed; at $\alpha = 1$ it maximally encodes the first $q$ bits of~$s$.

\paragraph{Wiesner-Derived States.}
Auxiliary states from partial measurement of BB84 encodings, modeling the side information in one-time program constructions~\cite{wiesner1983,stambler2026}.
Each bit of $s$ is encoded in either the computational or Hadamard basis, and partial measurement yields quantum side information with basis-dependent uncertainty.

% ============================================================
\section{Security Gap Analysis}
\label{sec:security-gap}

We measure the dVBB security gap across four adversary strategies:
\begin{itemize}
\item \textbf{Measure-then-Guess}: Measure $\rho$ in the computational basis, then apply classical LPN analysis.
\item \textbf{Optimal POVM}: Use the eigenstructure of $\rho$ with a POVM-optimized attack.
\item \textbf{Entanglement Attack}: Exploit purity and entanglement properties of $\rho$.
\item \textbf{Coherent Query}: Use off-diagonal coherence of $\rho$ for enhanced distinguishing.
\end{itemize}

\subsection{Results by Strategy}

Table~\ref{tab:strategy-gaps} summarizes security gaps across 15 conjunction samples, 11 correlation levels, and all 4 strategies (2{,}640 total data points).

\begin{table}[t]
\centering
\caption{Security gap statistics by adversary strategy.}
\label{tab:strategy-gaps}
\begin{tabular}{lcccc}
\toprule
Strategy & Mean Gap & Max Gap & Std Dev & Median \\
\midrule
Measure+Guess & 0.0405 & 0.0801 & 0.0180 & 0.0410 \\
Optimal POVM  & 0.0414 & 0.0952 & 0.0200 & 0.0430 \\
Entanglement  & 0.0397 & 0.0859 & 0.0188 & 0.0410 \\
Coherent Query & 0.0402 & 0.1222 & 0.0208 & 0.0437 \\
\bottomrule
\end{tabular}
\end{table}

All mean gaps remain below 0.042 and all maximum gaps are below 0.123, well within the negligible threshold of 0.15.
The coherent query strategy achieves the highest single-trial gap of 0.1222 but its mean (0.0402) is comparable to other strategies.

\subsection{Results by Correlation Strength}

Figure~\ref{fig:correlation} shows how the security gap varies with the quantum-classical correlation strength.
At zero correlation ($\alpha = 0$, independent auxiliary state), the mean gap is 0.0380.
The gap remains stable across all correlation levels, peaking at 0.0433 for $\alpha = 0.6$ and showing no systematic increase with stronger quantum side information.
The maximum gap across all correlations is 0.1222 at $\alpha = 0.4$.

\begin{figure}[t]
\centering
\includegraphics[width=\columnwidth]{figures/security_gap_correlation.pdf}
\caption{Security gap vs quantum auxiliary correlation strength. The mean gap (blue, with standard deviation band) remains below 0.045 across all correlation levels. The max gap (red) stays below the negligible threshold (green dashed).}
\label{fig:correlation}
\end{figure}

% ============================================================
\section{Hybrid Argument Analysis}
\label{sec:hybrid}

The classical BLMZ proof proceeds via a hybrid argument that transitions each relevant bit's LPN encoding to uniform.
We thread the quantum state $\rho$ through all hybrid steps and measure per-step distinguishing advantages.

\begin{table}[t]
\centering
\caption{Hybrid argument: per-step advantage statistics over 10 trials.}
\label{tab:hybrid}
\begin{tabular}{lccc}
\toprule
Strategy & Mean Max Adv & Overall Max & Std Dev \\
\midrule
Measure+Guess & 0.0325 & 0.0488 & 0.0095 \\
Optimal POVM  & 0.0401 & 0.0703 & 0.0147 \\
Entanglement  & 0.0319 & 0.0677 & 0.0143 \\
Coherent Query & 0.0297 & 0.0410 & 0.0068 \\
\bottomrule
\end{tabular}
\end{table}

Table~\ref{tab:hybrid} shows that the maximum per-step advantage across all strategies and trials is 0.0703 (optimal POVM strategy).
This is consistent with the requirement that each hybrid transition incurs at most $\negl(n)$ distinguishing advantage, as the total advantage accumulated across all hybrid steps remains bounded.

\begin{figure}[t]
\centering
\includegraphics[width=\columnwidth]{figures/hybrid_argument.pdf}
\caption{Per-step distinguishing advantages across hybrid transitions for four adversary strategies. Individual trials shown as thin lines; mean trajectory as thick line with markers.}
\label{fig:hybrid}
\end{figure}

% ============================================================
\section{LPN and Entropy Analysis}
\label{sec:analysis}

\subsection{LPN with Quantum Auxiliary Input}

We analyze whether LPN remains hard when the adversary holds quantum side information.
Figure~\ref{fig:lpn} shows the LPN distinguishing advantage as a function of noise rate for five auxiliary state types.

Key findings:
\begin{itemize}
\item At noise rate $\eta = 0.1$ (our default), all auxiliary types yield advantages below 0.05 for the security gap.
\item The quantum boost from auxiliary states is bounded by $\sqrt{\lambda_{\max} \cdot d}/d$ where $\lambda_{\max}$ is the largest eigenvalue of $\rho$ and $d = 2^q$.
\item As the problem dimension $n$ increases from 4 to 32, both classical and quantum advantages decrease exponentially, confirming LPN hardness scaling.
\end{itemize}

\begin{figure}[t]
\centering
\includegraphics[width=\columnwidth]{figures/lpn_quantum_analysis.pdf}
\caption{(a) LPN advantage vs noise rate for different auxiliary state types.
(b) LPN advantage vs problem dimension showing exponential decay.}
\label{fig:lpn}
\end{figure}

\subsection{Quantum Conditional Min-Entropy}

The dVBB guarantee requires that the secret $s$ has high min-entropy conditioned on the quantum auxiliary state:
$\Hmin(s|\rho) \geq k$ for sufficiently large $k$.
We measure security gaps across 16 target min-entropy levels from 0.5 to 8.0 bits.

The mean security gap ranges from 0.0306 to 0.0432 across all entropy levels, with no clear dependence on the entropy target.
The maximum single-trial gap of 0.1719 occurs at entropy level 5.0, but this is an isolated outlier; the median max gap across levels is 0.0810.
These results suggest that the security guarantee is robust to the min-entropy condition, consistent with the conjecture.

\subsection{Gentle Measurement Analysis}

The simulator must use the quantum auxiliary state $\rho$ without significantly disturbing it.
The gentle measurement lemma bounds the post-measurement disturbance by $\sqrt{1-p}$ where $p$ is the measurement success probability.

Our analysis of 50 random quantum states shows:
\begin{itemize}
\item Computational basis: mean disturbance 0.9375 (high, as expected for non-diagonal states).
\item Hadamard basis: mean disturbance 0.4688 (moderate).
\item Mean state purity: 0.1248 (near maximally mixed for $d=16$).
\item Mean von Neumann entropy: 3.282 bits (out of $\log_2 16 = 4$ maximum).
\end{itemize}

The gentle measurement bound ensures that for high-probability outcomes ($p > 0.9$), the disturbance is at most $\sqrt{0.1} \approx 0.316$, supporting the feasibility of quantum simulation.

\begin{figure}[t]
\centering
\includegraphics[width=\columnwidth]{figures/gentle_measurement.pdf}
\caption{(a) Measurement disturbance vs state purity for computational and Hadamard bases.
(b) Gentle measurement lemma bound $\sqrt{1-p}$ vs success probability.}
\label{fig:gentle}
\end{figure}

% ============================================================
\section{Scaling Analysis}
\label{sec:scaling}

\subsection{Scaling with Conjunction Size}

Figure~\ref{fig:scaling}(a) shows security gaps as the conjunction size $n$ increases from 4 to 20 bits.
For classical auxiliary input, the mean gap decreases from 0.0490 at $n=4$ to 0.0376 at $n=20$.
For quantum auxiliary input, it ranges from 0.0529 ($n=4$) to 0.0380 ($n=20$).
For Wiesner-derived states, the gap decreases from 0.0765 ($n=4$) to 0.0403 ($n=20$).
The decreasing trend supports the conjecture that security improves with the security parameter.

\subsection{Scaling with Auxiliary Qubits}

Figure~\ref{fig:scaling}(b) shows that increasing the number of auxiliary qubits from 1 to 6 does not systematically increase the security gap.
The mean gap at 1 qubit is 0.0461 and at 6 qubits is 0.0359, with all values remaining below 0.047.
The maximum gap across all qubit counts is 0.0859 (at 3 qubits), well below the negligible threshold.

\subsection{Scaling with Noise Rate}

Figure~\ref{fig:scaling}(c) shows the relationship between LPN noise rate $\eta$ and the security gap.
Both classical and quantum auxiliary gaps remain in the range $[0.030, 0.048]$ across noise rates from 0.05 to 0.45, showing no systematic dependence on $\eta$.

\begin{figure}[t]
\centering
\includegraphics[width=\columnwidth]{figures/scaling_analysis.pdf}
\caption{Security gap scaling: (a) vs conjunction size $n$, (b) vs auxiliary qubit count, (c) vs LPN noise rate $\eta$.}
\label{fig:scaling}
\end{figure}

% ============================================================
\section{Quantum State Properties}
\label{sec:quantum-states}

We characterize the quantum auxiliary states used in our experiments.
As correlation strength $\alpha$ increases from 0 to 1:
\begin{itemize}
\item Von Neumann entropy decreases from 4.000 bits (maximally mixed) to 3.485 bits.
\item Purity increases from 0.0625 ($1/d$) to 0.1068.
\item Maximum eigenvalue increases from $1/d = 0.0625$ to 0.1025.
\end{itemize}

These modest changes in state properties---entropy decreases from 4.000 to 3.485 bits---explain why the security gap shows little sensitivity to the correlation strength.
The quantum auxiliary state, even at maximum correlation, remains close to maximally mixed due to the high-dimensional Hilbert space ($d = 16$).

\begin{figure}[t]
\centering
\includegraphics[width=\columnwidth]{figures/quantum_state_properties.pdf}
\caption{Quantum state properties vs correlation: (a) entropy, (b) purity, (c) max eigenvalue, (d) coherence.}
\label{fig:state-props}
\end{figure}

% ============================================================
\section{Discussion}
\label{sec:discussion}

Our computational experiments provide strong numerical evidence supporting Conjecture~\ref{conj:main}.
The key findings are:

\paragraph{Small Security Gaps.}
Across 2{,}640 experimental configurations, the mean security gap is 0.0405 and the maximum is 0.1222.
Both are well below the negligible threshold of 0.15, suggesting that the quantum auxiliary input does not provide a significant advantage to the adversary.

\paragraph{Stability Across Strategies.}
All four adversary strategies (measure-then-guess, optimal POVM, entanglement attack, coherent query) yield similar mean gaps in the range $[0.0397, 0.0414]$, indicating that the security guarantee is robust to the choice of quantum attack strategy.

\paragraph{Bounded Hybrid Steps.}
The maximum per-step hybrid advantage of 0.0703 confirms that each transition in the security proof incurs negligible distinguishing advantage, consistent with the LPN assumption.

\paragraph{Favorable Scaling.}
Security gaps decrease or remain stable as the conjunction size $n$ increases, as the auxiliary qubit count grows, and across all tested LPN noise rates.

\paragraph{Limitations.}
Our numerical framework simulates quantum states classically via density matrices, limiting the auxiliary register to $q \leq 6$ qubits ($d = 64$).
The adversary strategies, while covering the main attack paradigms, do not exhaust all possible QPT attacks.
A formal proof of the conjecture would require rigorous quantum information-theoretic arguments (quantum leftover hash lemma, conditional min-entropy bounds) applied within the BLMZ proof structure.

% ============================================================
\section{Related Work}
\label{sec:related}

Barak et al.~\cite{barak2001} established the impossibility of general VBB obfuscation.
BLMZ~\cite{bartusek2019} achieved dVBB for conjunctions under LPN with classical auxiliary input.
Wichs and Zirdelis~\cite{wichs2017} and Goyal, Koppula, and Waters~\cite{goyal2017} developed related obfuscation constructions under LWE.
Broadbent and Jeffery~\cite{broadbent2015} and Alagic and Fefferman~\cite{alagic2016} studied quantum aspects of obfuscation.
The quantum information-theoretic tools we leverage include quantum conditional min-entropy~\cite{renner2005,koenig2009}, the quantum leftover hash lemma~\cite{tomamichel2011}, and the gentle measurement lemma~\cite{winter1999,ogawa2007}.

% ============================================================
\section{Conclusion}
\label{sec:conclusion}

We presented a comprehensive computational analysis of Conjecture~\ref{conj:main}, which posits that the BLMZ conjunction obfuscator satisfies distributional VBB security with quantum auxiliary input.
Our experiments across 2{,}640 configurations show security gaps consistently bounded below 0.123, with a mean of 0.0405, supporting the conjecture.
The hybrid argument analysis, LPN hardness study, min-entropy analysis, and scaling experiments all yield results consistent with the conjecture's validity.
These findings motivate pursuing a formal proof via the hybrid argument structure (Direction~3 of our analysis), leveraging post-quantum LPN security and quantum conditional min-entropy bounds.

\bibliographystyle{ACM-Reference-Format}
\bibliography{references}

\end{document}
