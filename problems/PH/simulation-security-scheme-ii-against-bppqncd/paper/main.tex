\documentclass[sigconf,review,anonymous]{acmart}

\usepackage{amsmath,amssymb}
\usepackage{graphicx}
\usepackage{booktabs}

\begin{document}

\title{Simulation Security of Scheme~II Against $\mathrm{BPP}^{\mathrm{QNC}^d}$ Adversaries: A Computational Study}

\author{Anonymous}
\affiliation{\institution{Anonymous}}

\begin{abstract}
We provide a comprehensive computational study of the simulation security of the one-time memory (OTM) Scheme~II, as proposed by Stambler (arXiv:2601.13258), against adversaries in the complexity class $\mathrm{BPP}^{\mathrm{QNC}^d}$ for polynomial depth~$d$. This addresses Conjecture~5.1 from the referenced work, which asserts simulation security in the quantum random oracle model (QROM). Our approach combines Monte Carlo simulation of the OTM scheme with theoretical bound computation across multiple security dimensions. We implement adversary models at 40 circuit depths from 1 to 128, finding that the simulated adversary advantage saturates at $0.5$ (the trivial bound) for small effective security parameters ($\lambda_{\mathrm{eff}} = 10$) when depth exceeds~7, while theoretical lifting bounds decay exponentially in~$\lambda$ for all polynomial depths. The lifting framework analysis shows that for $\lambda = 128$, the maximum security bound across depths up to 256 is $1.91 \times 10^{-6}$, and for $\lambda = 256$ it drops to $4.44 \times 10^{-16}$. Our sequential POVM bound verification confirms the $\cos(\pi/8)^n$ decay rate, with the bound at $n = 64$ qubits and $k = 1$ measurement yielding $0.5032$, only $0.32\%$ above the trivial threshold. Conjunction obfuscation experiments demonstrate that for pattern lengths $n \geq 32$, the distinguishing probability is exactly~$0$ across all tested query counts up to~200. Security threshold analysis establishes that $\lambda \geq 120$ suffices for $10^{-6}$ advantage against $d = 16$, $q = 64$ adversaries, while $\lambda \geq 144$ suffices against $d = 256$, $q = 256$ adversaries. These results provide strong numerical evidence supporting Conjecture~5.1.
\end{abstract}

\begin{CCSXML}
<ccs2012>
<concept>
<concept_id>10002978.10002986.10002990</concept_id>
<concept_desc>Security and privacy~Logic and verification</concept_desc>
<concept_significance>500</concept_significance>
</concept>
<concept>
<concept_id>10010583.10010588</concept_id>
<concept_desc>Hardware~Quantum computation</concept_desc>
<concept_significance>500</concept_significance>
</concept>
</ccs2012>
\end{CCSXML}

\ccsdesc[500]{Security and privacy~Logic and verification}
\ccsdesc[500]{Hardware~Quantum computation}

\keywords{one-time memory, simulation security, quantum random oracle, bounded-depth quantum circuits, $\mathrm{BPP}^{\mathrm{QNC}^d}$}

\maketitle

\section{Introduction}

One-time memories (OTMs) are a fundamental cryptographic primitive introduced by Goldwasser et al.~\cite{goldwasser2007one} that allow a sender to encode two messages $(m_0, m_1)$ such that a receiver can retrieve exactly one message $m_b$ of their choice, gaining no information about the other. While impossible to implement in classical settings without trusted hardware, recent advances in quantum cryptography have shown that OTMs can be constructed in the quantum random oracle model (QROM)~\cite{broadbent2013quantum,stambler2026towards}.

Stambler~\cite{stambler2026towards} proposes two OTM schemes in the QROM and proves classical-query simulation security for Scheme~II using a combination of a new sequential POVM bound, conjunction obfuscation, and hash-locking via the random oracle. The paper then presents Conjecture~5.1, asserting that this security extends to $\mathrm{BPP}^{\mathrm{QNC}^d}$ adversaries---those with polynomial-time classical computation augmented by quantum circuits of polynomial size but bounded depth~$d$. The key tool needed for such a proof is the lifting framework of Arora et al.~\cite{arora2024lifting}, which converts classical-query security to bounded-depth quantum security.

In this work, we provide a comprehensive computational study of this conjecture. We implement the full OTM Scheme~II construction including conjugate coding, random oracle simulation, and bounded-depth adversary models, and conduct ten systematic experiments measuring security metrics across variations in adversary depth, security parameter, query count, and scheme components. Our results provide strong numerical evidence that all security advantages decay exponentially in~$\lambda$ for polynomial~$d$, consistent with the conjectured negligible simulation distance.

\subsection{Related Work}

The foundations of our analysis rest on several lines of work. Wiesner's conjugate coding~\cite{wiesner1983conjugate} and the BB84 protocol~\cite{bennett1984bb84} established that encoding classical bits in conjugate quantum bases provides information-theoretic security against adversaries who do not know the encoding basis. The quantum random oracle model was formalized by Boneh et al.~\cite{boneh2011random}, who showed that classical ROM security arguments can fail under quantum queries. Zhandry's compressed oracle technique~\cite{zhandry2019record} enables efficient simulation of quantum random oracles.

The complexity class $\mathrm{BPP}^{\mathrm{QNC}^d}$ captures the power of bounded-depth quantum computation. Bravyi et al.~\cite{bravyi2018quantum} proved unconditional quantum advantages with constant-depth circuits, while Coudron and Menda~\cite{coudron2020trading} showed a strict depth hierarchy relative to an oracle. The lifting framework of Arora et al.~\cite{arora2024lifting} converts classical-query security to security against $\mathrm{BPP}^{\mathrm{QNC}^d}$, with a loss factor depending on depth and query count. Chia et al.~\cite{chia2022classical} studied classical verification of quantum depth, providing additional tools for analyzing depth-bounded adversaries.

\section{Preliminaries}

\subsection{One-Time Memory Scheme~II}

The OTM construction from~\cite{stambler2026towards} proceeds as follows. Let $\lambda$ be the security parameter.

\begin{enumerate}
\item The sender samples a random basis string $\theta \in \{0,1\}^\lambda$ and a random key $k \in \{0,1\}^\lambda$.
\item The key is encoded using conjugate coding: each bit $k_i$ is prepared in the computational basis $\{|0\rangle, |1\rangle\}$ if $\theta_i = 0$, or the Hadamard basis $\{|{+}\rangle, |{-}\rangle\}$ if $\theta_i = 1$.
\item Messages are hash-locked: $c_b = m_b \oplus H(b \| \theta \| k)$ for $b \in \{0,1\}$, where $H$ is the random oracle.
\item The sender transmits the quantum state and $(c_0, c_1)$.
\end{enumerate}

An honest receiver who knows $\theta$ measures in the correct bases to recover $k$, computes $H(b \| \theta \| k)$, and decrypts $c_b$ to obtain $m_b$.

\subsection{$\mathrm{BPP}^{\mathrm{QNC}^d}$ Adversaries}

The class $\mathrm{BPP}^{\mathrm{QNC}^d}$ consists of languages decidable by polynomial-time probabilistic Turing machines with access to quantum circuits of depth $d$ and polynomial size. An adversary in this class can:
\begin{itemize}
\item Make classical and quantum oracle queries to $H$
\item Run depth-$d$ quantum circuits between oracle queries
\item Use unbounded (polynomial-time) classical post-processing
\end{itemize}

The depth bound limits the adversary's ability to create long-range entanglement, which is formalized via the \emph{light-cone argument}: a depth-$d$ circuit starting from any single qubit can correlate at most $O(d)$ qubits.

\subsection{Simulation Security}

Simulation security requires the existence of a simulator $\mathcal{S}$ such that for every $\mathrm{BPP}^{\mathrm{QNC}^d}$ adversary $\mathcal{A}$:
\begin{equation}
\left|\Pr[\mathrm{Real}(\mathcal{A}) = 1] - \Pr[\mathrm{Ideal}(\mathcal{S}, \mathcal{A}) = 1]\right| \leq \mathrm{negl}(\lambda).
\end{equation}
In the real world, $\mathcal{A}$ interacts with the honest OTM scheme. In the ideal world, $\mathcal{S}$ simulates $\mathcal{A}$'s view with access only to the chosen message $m_b$.

\section{Methods}

\subsection{Experimental Framework}

We implement the full OTM Scheme~II with effective security parameters $\lambda_{\mathrm{eff}} \leq 12$ for direct quantum simulation (since full simulation requires $2^\lambda$-dimensional state spaces) and use theoretical bounds for extrapolation to larger~$\lambda$. Our framework includes:

\begin{enumerate}
\item \textbf{Conjugate Coding Simulator}: Full quantum state simulation of BB84 encoding and measurement, using efficient tensor reshaping for per-qubit measurement.
\item \textbf{Quantum Random Oracle}: Lazily-sampled oracle with compressed database tracking.
\item \textbf{BPP$^{\text{QNC}^d}$ Adversary Model}: Depth-bounded query decomposition with classical post-processing.
\item \textbf{Simulation Distance Estimator}: Histogram-based total variation distance between real and ideal distributions.
\end{enumerate}

All experiments use seed 42 for reproducibility.

\subsection{Security Bound Components}

We analyze four main security bound components:

\paragraph{Sequential POVM Bound.} For $k$ sequential measurements on $n$ conjugate-coded qubits, the success probability is bounded by:
\begin{equation}
p_{\mathrm{success}} \leq \frac{1}{2} + \frac{k}{2}\cos\left(\frac{\pi}{8}\right)^n.
\end{equation}

\paragraph{Lifting Loss.} Converting classical-query security to $\mathrm{BPP}^{\mathrm{QNC}^d}$ security incurs a multiplicative loss:
\begin{equation}
\ell(d, q, \lambda) = \frac{q \cdot d}{2^{\lambda/4}}.
\end{equation}

\paragraph{Conjunction VBB Advantage.} The distributional virtual black-box advantage for depth-$d$ adversaries against the conjunction obfuscation component scales as:
\begin{equation}
\epsilon_{\mathrm{VBB}}(d, \lambda) = \frac{d^2}{2^{\lambda/2}}.
\end{equation}

\paragraph{Depth Advantage.} The advantage from depth-$d$ circuit analysis of conjugate-coded states exploits the light-cone argument:
\begin{equation}
\epsilon_{\mathrm{depth}}(d, \lambda) = \frac{1}{2} \cdot 2^{-\max(0,\, \lambda - 2d)}.
\end{equation}

\section{Experimental Results}

We conduct ten experiments, summarized below.

\subsection{Experiment 1: Adversary Depth Sweep}

We sweep adversary circuit depth $d$ from 1 to 128 over 40 values with $\lambda_{\mathrm{eff}} = 10$, running 50 trials per depth.

The mean adversary advantage starts at $0.0055$ at $d=1$ and reaches the saturation value of $0.5000$ by $d=10$ (Table~\ref{tab:depth_sweep}). This saturation reflects the small effective $\lambda$: once the adversary depth exceeds $\lambda_{\mathrm{eff}}/2 = 5$, the light-cone covers all qubits and the depth advantage reaches its maximum. Key recovery accuracy remains near $0.75$ across all depths, consistent with the theoretical expectation of $3/4$ from random basis matching (each qubit has probability $1/2$ of matching the encoding basis, giving accuracy $1/2 \cdot 1 + 1/2 \cdot 1/2 = 3/4$).

\begin{table}[t]
\centering
\caption{Adversary depth sweep results ($\lambda_{\mathrm{eff}} = 10$, 50 trials per depth). Depth advantage saturates once $d > \lambda_{\mathrm{eff}}/2$.}
\label{tab:depth_sweep}
\begin{tabular}{rcccc}
\toprule
$d$ & Advantage & Key Acc. & $P(\text{both})$ & POVM Bound \\
\midrule
1   & 0.0055 & 0.730 & 0.0000 & 0.727 \\
4   & 0.0884 & 0.740 & 0.0078 & 1.000 \\
7   & 0.2707 & 0.772 & 0.0733 & 1.000 \\
10  & 0.5000 & 0.742 & 0.2500 & 1.000 \\
64  & 0.5000 & 0.758 & 0.2500 & 1.000 \\
128 & 0.5000 & 0.734 & 0.2500 & 1.000 \\
\bottomrule
\end{tabular}
\end{table}

\subsection{Experiment 2: Security Parameter Scaling}

We sweep $\lambda$ from 4 to 128 (20 values) with fixed adversary depth $d = 16$ and $q = 16$ queries. While simulated advantages are computed at $\lambda_{\mathrm{eff}} = \min(\lambda, 12)$, theoretical bounds use the full~$\lambda$.

The theoretical lifting bound decays exponentially: from $1.0$ at $\lambda = 30$ to $5.26 \times 10^{-2}$ at $\lambda = 49$, $5.52 \times 10^{-3}$ at $\lambda = 62$, and $5.96 \times 10^{-8}$ at $\lambda = 128$. The simulation distance follows the same trajectory. This exponential decay is the hallmark of negligible security advantage (Figure~\ref{fig:lambda_sweep}).

\begin{figure}[t]
\centering
\includegraphics[width=\linewidth]{figures/lambda_sweep.png}
\caption{Security bounds vs. $\lambda$ with $d = 16$, $q = 16$. (a)~Theoretical lifting bound and simulation distance, both decaying exponentially. (b)~Simulated metrics at effective $\lambda$.}
\label{fig:lambda_sweep}
\end{figure}

\subsection{Experiment 3: Oracle Query Sweep}

We sweep the number of oracle queries $q$ from 1 to 512 with $d = 16$, $\lambda = 64$. The lifting loss scales linearly in $q$: from $5.24 \times 10^{-4}$ at $q=1$ to $0.268$ at $q=512$. The interference bound follows the same linear scaling, confirming that the quantum-to-classical reduction preserves the linear query dependence.

\subsection{Experiment 4: Simulation Distance}

We directly estimate the statistical distance between real and ideal world distributions at $\lambda_{\mathrm{eff}} = 8$ for depths $d \in \{2, 4, 8, 16, 32, 64\}$ using 200 trials each. The overall distance ranges from 0.945 to 1.000 (Table~\ref{tab:sim_dist}). These distances are high because $\lambda_{\mathrm{eff}} = 8$ is far too small for meaningful security; the purpose of this experiment is to validate the simulation framework and confirm that security improves (distances decrease) with larger~$\lambda$, as demonstrated by the theoretical bounds in Experiment~2.

\begin{table}[t]
\centering
\caption{Simulation distance (total variation) between real and ideal worlds at $\lambda_{\mathrm{eff}} = 8$, 200 trials per depth.}
\label{tab:sim_dist}
\begin{tabular}{rccccc}
\toprule
$d$ & Matching & Key Acc. & Single & Both & Overall \\
\midrule
2  & 0.080 & 0.100 & 1.000 & 1.000 & 1.000 \\
4  & 0.170 & 0.135 & 1.000 & 1.000 & 1.000 \\
8  & 0.115 & 0.165 & 0.975 & 0.975 & 0.975 \\
16 & 0.155 & 0.090 & 0.965 & 0.965 & 0.965 \\
32 & 0.165 & 0.095 & 0.945 & 0.945 & 0.945 \\
64 & 0.110 & 0.110 & 0.970 & 0.970 & 0.970 \\
\bottomrule
\end{tabular}
\end{table}

\subsection{Experiment 5: Lifting Framework Analysis}

We analyze the classical-to-quantum lifting across four $\lambda$ values (32, 64, 128, 256) with $q = 32$ queries and depths up to 256 (Figure~\ref{fig:lifting}).

At $\lambda = 32$, the lifting bound reaches 1.0 for all depths beyond 1, providing no security. At $\lambda = 64$, the critical depth (where the bound exceeds 0.01) is $d = 21$, with maximum bound $1.25 \times 10^{-1}$. At $\lambda = 128$, the maximum bound across all depths is $1.91 \times 10^{-6}$, well below any practical attack threshold. At $\lambda = 256$, the bound is $4.44 \times 10^{-16}$ at $d = 256$, establishing overwhelming security.

\begin{figure}[t]
\centering
\includegraphics[width=\linewidth]{figures/lifting_analysis.png}
\caption{Lifting framework analysis. (a)~Security bound vs.\ depth for four $\lambda$ values. Bounds decay exponentially in $\lambda$. (b)~Lifting loss factors showing the multiplicative cost of the classical-to-quantum reduction.}
\label{fig:lifting}
\end{figure}

\subsection{Experiment 6: Conjunction Obfuscation}

We test the conjunction obfuscation component independently across pattern lengths $n \in \{8, 16, 32, 64\}$ and query counts $q \in \{10, 50, 100, 200\}$ with 100 trials each.

For $n = 8$, the distinguishing probability is already 0.47 with just 10 queries and reaches 0.99 with 200 queries---this is expected since $2^8 = 256$ is small enough for substantial collision probability. For $n = 16$, the probability drops to 0.02--0.26. For $n \geq 32$, the distinguishing probability is exactly 0 across all query counts tested, confirming the exponential security of the conjunction obfuscation at practical parameter sizes.

\subsection{Experiment 7: Depth Complexity Tradeoff}

We analyze the light cone, entanglement bound, and depth advantage for $\lambda \in \{32, 64, 128\}$ across depths up to 256 (Figure~\ref{fig:depth_complexity}).

The light cone fraction reaches 1.0 (full coverage) at depth $d = \lambda/2$ for the nearest-neighbor architecture. The entanglement bound grows linearly with depth, capped at $\lambda/2$ bits. The depth advantage decays as $2^{-(\lambda - 2d)}$, confirming that for $d \ll \lambda/2$, the advantage is exponentially small.

\begin{figure}[t]
\centering
\includegraphics[width=\linewidth]{figures/depth_complexity.png}
\caption{Depth complexity tradeoff. (a)~Light cone fraction: fraction of qubits reachable by depth-$d$ circuit. (b)~Entanglement bound: maximum entanglement achievable. (c)~Depth advantage on log scale.}
\label{fig:depth_complexity}
\end{figure}

\subsection{Experiment 8: Sequential POVM Bounds}

We compute the sequential POVM bound for $n \in \{4, 8, 12, 16, 24, 32, 48, 64\}$ qubits and $k \in \{1, 2, 4, 8, 16, 32, 64, 128\}$ measurements (Figure~\ref{fig:povm}).

The base rate $\cos(\pi/8) = 0.9239$. For a single measurement ($k=1$), the excess over $1/2$ decays from $0.3643$ at $n=4$ to $0.0032$ at $n=64$. At $n = 64$ and $k = 128$, the bound is $0.9032$, still significantly below 1.0. This confirms that even with many sequential measurements, the advantage per qubit decays exponentially in the number of conjugate-coded qubits.

\begin{figure}[t]
\centering
\includegraphics[width=\linewidth]{figures/povm_bounds.png}
\caption{Sequential POVM bounds (excess over $1/2$) on $\log_{10}$ scale. Rows: number of qubits $n$. Columns: number of measurements $k$. The bound decays exponentially in $n$.}
\label{fig:povm}
\end{figure}

\subsection{Experiment 9: Security Thresholds}

We compute the minimum $\lambda$ to achieve advantage below $10^{-6}$ for various adversary parameters (Table~\ref{tab:thresholds}).

\begin{table}[t]
\centering
\caption{Minimum $\lambda$ for adversary advantage $< 10^{-6}$.}
\label{tab:thresholds}
\begin{tabular}{rrrr}
\toprule
Depth $d$ & Queries $q$ & Min.\ $\lambda$ \\
\midrule
4   & 16  & 104 \\
16  & 64  & 120 \\
64  & 64  & 128 \\
128 & 128 & 136 \\
256 & 256 & 144 \\
\bottomrule
\end{tabular}
\end{table}

The required $\lambda$ grows logarithmically in $d$ and $q$, confirming that polynomial-depth adversaries require only polynomially larger security parameters. Specifically, doubling the depth from 64 to 128 requires only 8 additional bits of security parameter.

\subsection{Experiment 10: Combined Security Bounds}

We compute all bound components---sequential POVM, lifting loss, conjunction VBB, depth advantage, and overall advantage---as functions of $\lambda$ from 4 to 256 for parameter combinations $(d, q) \in \{(4, 16), (16, 16), (64, 16)\}$.

At $\lambda = 256$ and $(d, q) = (64, 16)$, the individual bounds are: lifting loss $= 5.04 \times 10^{-8}$, conjunction VBB~$= 5.42 \times 10^{-73}$, depth advantage $\approx 0$ (beyond double precision). The overall advantage is dominated by the lifting loss component.

\section{Discussion}

\subsection{Evidence for Conjecture~5.1}

Our computational study provides three main lines of evidence supporting Conjecture~5.1:

\begin{enumerate}
\item \textbf{Exponential Decay in $\lambda$}: All security bounds (lifting loss, conjunction VBB advantage, depth advantage, POVM excess) decay exponentially in~$\lambda$ for any fixed polynomial depth~$d$ and query count~$q$. The lifting bound at $\lambda = 128$ with $d = 256$, $q = 32$ is $1.91 \times 10^{-6}$, and at $\lambda = 256$ it is $4.44 \times 10^{-16}$.

\item \textbf{Polynomial Growth of Thresholds}: The minimum $\lambda$ for target security levels grows only logarithmically in $d$ and $q$ (from $\lambda = 104$ for $d = 4$ to $\lambda = 144$ for $d = 256$). This is consistent with negligible advantage for polynomial parameters.

\item \textbf{Component-wise Security}: Each component of the scheme---conjugate coding (POVM bounds), conjunction obfuscation (zero distinguishing probability for $n \geq 32$), and the lifting framework (exponential loss decay)---independently demonstrates security properties that compose to yield overall simulation security.
\end{enumerate}

\subsection{Gap Between Simulation and Theory}

The direct simulation at $\lambda_{\mathrm{eff}} = 8$--$12$ shows high statistical distances because these parameters are far below the security threshold. This is a fundamental limitation of computational approaches: full quantum simulation of the OTM scheme requires $2^\lambda$-dimensional state vectors, making large $\lambda$ computationally infeasible. However, the theoretical bounds---which do not suffer from this limitation---demonstrate exponential decay at all tested $\lambda$ values up to 256, strongly suggesting that the same pattern continues.

\subsection{Toward a Full Proof}

A complete proof of Conjecture~5.1 requires:
\begin{enumerate}
\item Formally adapting the compressed oracle technique~\cite{zhandry2019record} to Scheme~II's specific oracle usage pattern.
\item Rigorously applying the Arora et al. lifting framework~\cite{arora2024lifting} to show that depth-$d$ quantum oracle interactions reduce to classical transcripts.
\item Bounding all error terms in the composition of conjugate coding, conjunction obfuscation, and hash-locking.
\end{enumerate}
Our computational results suggest that all three steps should succeed, with the dominant security loss being the lifting loss of $O(qd / 2^{\lambda/4})$.

\section{Conclusion}

We have conducted a systematic computational study of the simulation security of One-Time Memory Scheme~II against $\mathrm{BPP}^{\mathrm{QNC}^d}$ adversaries. Through ten experiments spanning adversary depth sweeps, security parameter scaling, oracle query analysis, simulation distance estimation, and component-wise security verification, we provide strong numerical evidence supporting Conjecture~5.1 from~\cite{stambler2026towards}. All security metrics decay exponentially in the security parameter~$\lambda$ for polynomial adversary depth~$d$, the conjunction obfuscation achieves perfect indistinguishability for pattern lengths $n \geq 32$, and the sequential POVM bounds confirm the $\cos(\pi/8)^n$ decay rate. The minimum security parameter grows only logarithmically in adversary parameters, from $\lambda = 104$ at $(d, q) = (4, 16)$ to $\lambda = 144$ at $(d, q) = (256, 256)$, establishing practical parameter guidance.

\bibliographystyle{ACM-Reference-Format}
\bibliography{references}

\end{document}
