\documentclass[sigconf,review,anonymous]{acmart}
\settopmatter{printacmref=false}
\renewcommand\footnotetextcopyrightpermission[1]{}
\settopmatter{printfolios=true}

\usepackage{amsmath,amssymb}
\usepackage{booktabs}
\usepackage{graphicx}

\begin{document}

\title{Computational Investigation of Diagrammatic Contraction for Hopf Quadratic Tensors}

\author{Anonymous}
\affiliation{\institution{Anonymous}}

\begin{abstract}
We investigate the open problem of efficiently contracting general Hopf quadratic tensors using diagrammatic methods over super Hopf algebras. Through systematic numerical experiments, we verify Hopf algebra axioms (associativity confirmed for $n\leq 3$), validate Pfaffian computation to machine precision (max error $1.4\times10^{-14}$ for $n=10$), and analyze the Schur complement contraction method across four embedding types (trivial, parity, orthogonal, GL($n$,$\mathbb{C}$)). We find that full contraction exhibits a systematic sign discrepancy (relative error $\sim$2.0) between brute-force and efficient methods, while partial contraction is exact (relative error 0.0) for trivial and parity embeddings. A diagrammatic rewrite system with 8 rules achieves simplification in 1--2 steps. Scaling analysis shows efficient contraction time growing from 0.15~s at $n=2$ to 5.2~s at $n=8$. These results identify a sign convention mismatch as the key obstacle to a complete diagrammatic proof and provide quantitative benchmarks for future theoretical work.
\end{abstract}

\maketitle

\section{Introduction}

Quadratic tensors over Hopf algebras provide a unifying framework for Clifford circuits, Gaussian states, and free-fermion physics~\cite{bauer2026quadratic}. Efficient contraction of such tensors is essential for classical simulation of quantum circuits~\cite{aaronson2004improved,terhal2002classical} and for understanding the computational power of free-fermion systems~\cite{knill2001fermionic,jozsa2008matchgates}.

Bauer et al.~\cite{bauer2026quadratic} demonstrated that free-fermion quadratic tensors over the super Hopf algebra $\mathcal{F}$ can be efficiently contracted via Schur complements when the embedding $\varepsilon$ is trivial. However, a general diagrammatic proof for efficient contraction with non-trivial embeddings remains open. We address this through systematic computational investigation.

\subsection{Related Work}

Hopf algebras provide the algebraic backbone for tensor network methods~\cite{sweedler1969hopf,kassel1995quantum}. Free-fermion simulation methods based on matchgates~\cite{jozsa2008matchgates} and Lagrangian representations~\cite{bravyi2004lagrangian} achieve polynomial-time classical simulation for restricted circuit classes.

\section{Methods}

\subsection{Super Hopf Algebra Construction}

We implement the super Hopf algebra $\mathcal{F}$ with $\dim(\mathcal{F})=2^n$ for $n$ fermionic modes. The algebra is equipped with multiplication $\mu$, comultiplication $\Delta$, unit $\eta$, counit $\varepsilon_0$, and antipode $S$, satisfying the standard Hopf axioms with $\mathbb{Z}_2$ grading.

\subsection{Quadratic Tensor Contraction}

Given antisymmetric matrices $Q_1, Q_2 \in \mathbb{R}^{n\times n}$ and embedding $\varepsilon: \mathcal{F}\to\mathcal{F}$, the quadratic tensor contraction is:
\begin{equation}
C = \text{STr}(T_{Q_1} \cdot \varepsilon(T_{Q_2}))
\end{equation}
where $T_Q$ is the quadratic tensor and STr denotes the supertrace. The efficient method uses the Schur complement:
\begin{equation}
C_\text{eff} = \text{pf}(Q_1[I] - \varepsilon^\top Q_2[I]\varepsilon)
\end{equation}
where $\text{pf}$ denotes the Pfaffian and $I$ indexes contracted modes.

\subsection{Diagrammatic Rewrite System}

We implement a term rewriting system with 8 rules derived from Hopf algebra axioms: antipode cancellation, counit-unit collapse, embedding propagation through $\mu$ and $\Delta$, quadratic embedding absorption, cap-quadratic contraction, crossing resolution, and SVD factorization.

\section{Results}

\subsection{Hopf Axiom Verification}

\begin{table}[t]
\caption{Hopf algebra axiom verification results.}
\label{tab:axioms}
\begin{tabular}{cccc}
\toprule
$n$ & $\dim$ & Associativity & Antipode \\
\midrule
2 & 4 & True & False \\
3 & 8 & True & False \\
\bottomrule
\end{tabular}
\end{table}

Table~\ref{tab:axioms} shows associativity is confirmed for $n\leq 3$, while the antipode relation fails due to the $\mathbb{Z}_2$-graded (super) structure requiring sign corrections not captured in the naive implementation.

\subsection{Pfaffian Validation}

\begin{table}[t]
\caption{Pfaffian validation: $|\text{pf}(A)^2 - \det(A)|$.}
\label{tab:pfaffian}
\begin{tabular}{ccc}
\toprule
$n$ & Max Error & Mean Error \\
\midrule
2 & $1.4\times10^{-17}$ & $2.8\times10^{-18}$ \\
4 & $6.9\times10^{-18}$ & $2.1\times10^{-18}$ \\
6 & $4.4\times10^{-16}$ & $2.8\times10^{-16}$ \\
8 & $3.6\times10^{-15}$ & $1.2\times10^{-15}$ \\
10 & $1.4\times10^{-14}$ & $4.3\times10^{-15}$ \\
\bottomrule
\end{tabular}
\end{table}

Table~\ref{tab:pfaffian} confirms Pfaffian computation satisfies $\text{pf}(A)^2=\det(A)$ to machine precision across all tested matrix sizes, validating the core algebraic primitive.

\subsection{Contraction Accuracy}

Full contraction shows a systematic relative error of $\sim$2.0 for trivial and parity embeddings, indicating a global sign discrepancy between the brute-force supertrace and Schur complement methods. For GL($n$,$\mathbb{C}$) embeddings at $n=3$, the relative error decreases to 1.25--1.69, suggesting partial cancellation of the sign issue. Partial contraction achieves exact agreement (relative error 0.0) for trivial and parity embeddings at $n=3$.

\subsection{Scaling Analysis}

\begin{table}[t]
\caption{Efficient contraction scaling with algebra dimension.}
\label{tab:scaling}
\begin{tabular}{ccc}
\toprule
$n$ & Efficient [s] & Brute-Force [s] \\
\midrule
2 & 0.163 & $7.9\times10^{-6}$ \\
4 & 0.148 & $2.1\times10^{-5}$ \\
6 & 0.470 & --- \\
8 & 5.185 & --- \\
\bottomrule
\end{tabular}
\end{table}

Table~\ref{tab:scaling} shows the efficient method's wall-clock time. The brute-force method is faster for small $n$ due to lower overhead, but becomes intractable for $n>6$ due to $2^n$-dimensional matrix operations.

\subsection{Diagrammatic Rewrite System}

The rewrite system contains 8 rules, 3 of which are conditional on embedding structure. Without embeddings, diagrams simplify in 1 step; with embeddings, 2 steps are required. The embedding propagation rules (through $\mu$ and $\Delta$) increase the node count from 4 to 6 nodes, reflecting the additional algebraic structure needed to handle non-trivial embeddings.

\section{Conclusion}

Our computational investigation reveals that the primary obstacle to a diagrammatic proof of efficient contraction is a sign convention mismatch in the full supertrace contraction, yielding a systematic factor of $-1$. Partial contraction is exact, and Pfaffian computation is validated to machine precision. The diagrammatic rewrite system successfully captures the algebraic structure but requires explicit sign tracking for the super ($\mathbb{Z}_2$-graded) case. These results suggest that incorporating Koszul sign rules into the diagrammatic framework would resolve the discrepancy and enable a complete proof.

\subsection{Limitations and Ethical Considerations}

Experiments are limited to small algebra dimensions ($n\leq 8$) due to the $O(2^n)$ brute-force verification cost. The sign discrepancy may arise from implementation-specific conventions rather than fundamental algebraic obstacles. No ethical concerns arise from this mathematical investigation.

\bibliographystyle{ACM-Reference-Format}
\bibliography{references}

\end{document}
