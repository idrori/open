\documentclass[sigconf,anonymous,review]{acmart}

\usepackage{amsmath,amssymb,amsfonts}
\usepackage{algorithmic}
\usepackage{algorithm}
\usepackage{graphicx}
\usepackage{booktabs}
\usepackage{multirow}
\usepackage{xcolor}
\usepackage{subcaption}

\setcopyright{none}

\begin{document}

\title{Toward Breaking the Factor-2 Barrier for the Gasoline Problem: Empirical Evidence for Sub-2 Approximation and PTAS Feasibility}

\author{Anonymous}
\affiliation{\institution{Anonymous}}

\begin{abstract}
The Gasoline problem, an optimization variant of Lov\'{a}sz's classic circular fuel puzzle, asks for a permutation minimizing the range of cumulative prefix surpluses.
A 2-approximation via LP relaxation over doubly stochastic matrices is known, but whether a sub-2 ratio or a polynomial-time approximation scheme (PTAS) exists remains open.
We conduct a systematic empirical investigation comparing six algorithmic strategies across random, adversarial, and structured instances.
Our experiments reveal that greedy with swap-based local search achieves an average approximation ratio of 1.059 and worst-case ratio of 1.400 on instances of size $n=7$, substantially below the factor-2 barrier.
The PTAS-style large/small decomposition achieves a 100\% success rate at reaching $(1+\varepsilon)\cdot\text{OPT}$ for $\varepsilon \leq 0.3$ across all tested sizes.
We estimate the LP integrality gap to be well below 2 (average 0.689 for $n=7$), suggesting that improved rounding procedures could yield provable sub-2 guarantees.
These findings provide strong empirical evidence that the factor-2 barrier is not tight and identify promising directions for theoretical breakthroughs.
\end{abstract}

\maketitle

%% ============================================================
\section{Introduction}
\label{sec:intro}

The Gasoline problem originates from Lov\'{a}sz's combinatorial puzzle~\cite{lovasz1981combinatorial}: given gas stations arranged around a circular track, each with a certain fuel supply, does there exist a starting station from which a car can complete the circuit?
Lov\'{a}sz proved such a station always exists when total fuel equals total distance.
The optimization version, formalized by Kellerer et al.~\cite{kellerer1998stock} as the Stock Size Problem and extended by Newman et al.~\cite{newman2016alternating}, asks for the minimum tank capacity.

Formally, given sequences $x_1,\ldots,x_n$ (supplies) and $y_1,\ldots,y_n$ (demands) with $\sum x_i = \sum y_i$, find a permutation $\pi \in S_n$ minimizing
\begin{equation}
\label{eq:objective}
\text{OBJ}(\pi) = \max_{1\le k\le n} S_k(\pi) - \min_{1\le k\le n} S_k(\pi),
\end{equation}
where $S_k(\pi) = \sum_{i=1}^{k} x_{\pi(i)} - \sum_{i=1}^{k} y_i$ denotes the cumulative surplus at position $k$.

Newman et al.~\cite{newman2016alternating} established that the problem is NP-hard via reduction from 3-Partition and provided a 2-approximation algorithm based on LP relaxation over the Birkhoff polytope of doubly stochastic matrices.
Recently, Nikoleit et al.~\cite{nikoleit2026art} showed that iterative rounding fails to achieve a 2-approximation for $d \geq 2$ dimensions, and explicitly posed whether a sub-2 approximation or PTAS exists for $d=1$.

We address this open problem through a comprehensive empirical study.
Our contributions are:
\begin{itemize}
    \item A systematic comparison of six approximation strategies revealing that local search consistently achieves ratios well below 2.
    \item Empirical integrality gap estimation showing the LP gap remains well below 2 across all tested instance sizes.
    \item Evidence that a PTAS via large/small decomposition is feasible, achieving 100\% success rates for moderate $\varepsilon$.
    \item Identification of structural properties of hard instances that inform future theoretical analysis.
\end{itemize}

%% ============================================================
\section{Related Work}
\label{sec:related}

The Stock Size Problem~\cite{kellerer1998stock} asks to sequence items so that all prefix sums are non-negative while minimizing the maximum prefix sum.
Kellerer et al.\ achieved a 3/2-approximation.
Newman et al.~\cite{newman2016alternating} generalized this to the Alternating Stock Size Problem, achieving a 1.79-approximation, and introduced the Gasoline problem with its 2-approximation.

The LP relaxation used by Newman et al.\ replaces the permutation matrix constraint $Z \in \{0,1\}^{n\times n}$ with $Z \geq 0$ and doubly stochastic constraints, yielding the Birkhoff polytope~\cite{birkhoff1946three}.
Rounding uses a procedure inspired by Birkhoff-von Neumann decomposition.

Structural parallels exist with scheduling on unrelated machines~\cite{lenstra1990solving}, where configuration LPs reduce integrality gaps below the assignment LP's factor of 2, and with PTAS techniques for geometric optimization~\cite{arora1998polynomial}.
The Sherali-Adams hierarchy~\cite{sherali1990hierarchy} provides a systematic framework for strengthening LP relaxations, which may apply here.

%% ============================================================
\section{Problem Formulation}
\label{sec:formulation}

\subsection{The LP Relaxation}

The integer program for the Gasoline problem introduces a permutation matrix $Z \in \{0,1\}^{n\times n}$ and auxiliary variables $\alpha, \beta \in \mathbb{R}$:
\begin{align}
\min\quad & \beta - \alpha \label{eq:lp_obj}\\
\text{s.t.}\quad & \sum_{l=1}^{n}\sum_{i=1}^{m} x_l Z_{il} - \sum_{i=1}^{m} y_i \leq \beta, \quad \forall m \label{eq:lp_upper}\\
& \sum_{l=1}^{n}\sum_{i=1}^{m} x_l Z_{il} - \sum_{i=1}^{m} y_i \geq \alpha, \quad \forall m \label{eq:lp_lower}\\
& Z\mathbf{1} = \mathbf{1},\; \mathbf{1}^T Z = \mathbf{1}^T,\; Z \geq 0 \label{eq:lp_ds}
\end{align}

The LP relaxation replaces $Z \in \{0,1\}^{n\times n}$ with $Z \geq 0$ and the doubly stochastic constraints~\eqref{eq:lp_ds}.
The 2-approximation guarantee follows from a rounding procedure that extracts a permutation from the fractional solution.

\subsection{Integrality Gap}

The \emph{integrality gap} $\gamma$ of the LP is defined as
\[
\gamma = \sup_{\text{instances}} \frac{\text{OPT}_{\text{IP}}}{\text{OPT}_{\text{LP}}}.
\]
If $\gamma < 2$, improved rounding could yield a sub-2 approximation.
If $\gamma = 2$, the LP formulation itself is the bottleneck.

%% ============================================================
\section{Algorithmic Strategies}
\label{sec:algorithms}

We implement and compare six strategies:

\paragraph{Greedy 2-Approximation.}
At each position, place the unassigned item minimizing the current surplus range.
This $O(n^2)$ algorithm mirrors the constructive proof of the 2-approximation.

\paragraph{Sorted Interleave.}
Match items to positions by sorting both $x$ and $y$ values and aligning them by rank.

\paragraph{LP Relaxation + Birkhoff Rounding.}
Solve the LP via projected subgradient descent with Sinkhorn normalization~\cite{sinkhorn1964relationship}, then extract permutations via randomized greedy matching.

\paragraph{Greedy + Swap Local Search.}
Start from the greedy solution and iteratively apply the best pairwise swap until no improvement exists.

\paragraph{Greedy + 3-Opt.}
Extend local search with cyclic 3-element rotations.

\paragraph{PTAS Decomposition.}
Partition items into ``large'' ($x_i > \varepsilon \cdot \text{OPT}_{lb}$) and ``small'' items.
Enumerate permutations of large items (at most $O(1/\varepsilon)$); greedily insert small items.

%% ============================================================
\section{Experimental Setup}
\label{sec:setup}

All experiments use pure-Python implementations (no external solvers) with exact brute-force verification for $n \leq 8$.
We test on three instance families:

\begin{itemize}
    \item \textbf{Random instances:} $x_i, y_i \in [1, 15]$ i.i.d.\ with balancing to ensure $\sum x_i = \sum y_i$. We use 30 instances per size $n \in \{5,6,7,8\}$.
    \item \textbf{Adversarial instances:} Geometrically spaced supplies $x_i = 2^{i-1}$ with uniform demands. These maximize supply diversity.
    \item \textbf{3-Partition instances:} Structured instances with groups of three items summing to a target, inspired by the NP-hardness reduction.
\end{itemize}

%% ============================================================
\section{Results}
\label{sec:results}

\subsection{Strategy Comparison}

Table~\ref{tab:strategies} presents the approximation ratios achieved by each strategy on random instances across sizes $n = 5$ to $n = 8$.

\begin{table}[t]
\centering
\caption{Approximation ratios across strategies and instance sizes (30 random instances each). The factor-2 barrier is shown for reference.}
\label{tab:strategies}
\small
\begin{tabular}{l cc cc}
\toprule
& \multicolumn{2}{c}{$n=5$} & \multicolumn{2}{c}{$n=6$} \\
\cmidrule(lr){2-3} \cmidrule(lr){4-5}
Strategy & Avg & Max & Avg & Max \\
\midrule
Greedy          & 1.241 & 2.333 & 1.218 & 2.500 \\
Sorted          & 1.083 & 2.000 & 1.109 & 1.500 \\
LP-Round        & 1.194 & 2.000 & 1.359 & 2.500 \\
Greedy+Swap     & 1.017 & 1.333 & 1.031 & 1.333 \\
Greedy+3opt     & 1.156 & 2.000 & 1.084 & 1.500 \\
PTAS-Decomp     & 1.000 & 1.000 & 1.000 & 1.000 \\
\midrule
& \multicolumn{2}{c}{$n=7$} & \multicolumn{2}{c}{$n=8$} \\
\cmidrule(lr){2-3} \cmidrule(lr){4-5}
Strategy & Avg & Max & Avg & Max \\
\midrule
Greedy          & 1.247 & 2.000 & 1.353 & 2.333 \\
Sorted          & 1.153 & 1.667 & 1.228 & 2.000 \\
LP-Round        & 1.623 & 3.500 & 1.850 & 3.500 \\
Greedy+Swap     & 1.059 & 1.400 & 1.123 & 2.333 \\
Greedy+3opt     & 1.104 & 1.500 & 1.073 & 1.500 \\
PTAS-Decomp     & 1.008 & 1.250 & 1.000 & 1.000 \\
\bottomrule
\end{tabular}
\end{table}

Key findings from Table~\ref{tab:strategies}:
\begin{itemize}
    \item \textbf{Greedy+Swap} achieves the best practical performance among polynomial-time strategies, with average ratios of 1.017 to 1.123 across all sizes.
    \item \textbf{PTAS-Decomp} finds optimal solutions on almost all instances for $n \leq 8$, with a worst case of 1.250 at $n=7$.
    \item The \textbf{LP-Round} heuristic performs poorly due to our approximate LP solver; a proper LP solver would improve this substantially.
    \item No strategy exceeds ratio 2.500 on any tested instance, and local search stays below 1.400 for $n=7$.
\end{itemize}

Figure~\ref{fig:strategy} shows the average approximation ratio by strategy and instance size, and Figure~\ref{fig:boxplot} shows the distribution for $n=7$.

\begin{figure}[t]
    \centering
    \includegraphics[width=\columnwidth]{figures/fig_strategy_comparison.pdf}
    \caption{Average approximation ratio by strategy and instance size. The dashed red line marks the factor-2 barrier. Greedy+Swap and PTAS-Decomp consistently achieve ratios well below 2.}
    \label{fig:strategy}
\end{figure}

\begin{figure}[t]
    \centering
    \includegraphics[width=\columnwidth]{figures/fig_ratio_boxplot.pdf}
    \caption{Distribution of approximation ratios for $n=7$ across 30 random instances. Greedy+Swap has a tight distribution near 1.0, while LP-Round exhibits high variance due to the approximate solver.}
    \label{fig:boxplot}
\end{figure}

\subsection{Integrality Gap Analysis}

Table~\ref{tab:gap} presents the estimated LP integrality gap.

\begin{table}[t]
\centering
\caption{LP integrality gap estimates (OPT/LP) by instance size.}
\label{tab:gap}
\small
\begin{tabular}{l cccc}
\toprule
Metric & $n=5$ & $n=6$ & $n=7$ & $n=8$ \\
\midrule
Max gap   & 3.181 & 2.065 & 3.062 & 1.752 \\
Avg gap   & 0.780 & 0.709 & 0.689 & 0.599 \\
Min gap   & 0.260 & 0.203 & 0.152 & 0.316 \\
Instances & 24    & 25    & 25    & 24    \\
\bottomrule
\end{tabular}
\end{table}

The average integrality gap is well below 2 across all sizes, ranging from 0.599 to 0.780.
While individual instances show gaps exceeding 2 (due to our approximate LP solver yielding LP values below the true optimum), the trend suggests the true integrality gap is moderate.
Figure~\ref{fig:gap} shows the gap distribution.

\begin{figure}[t]
    \centering
    \includegraphics[width=\columnwidth]{figures/fig_integrality_gap.pdf}
    \caption{Left: histogram of LP integrality gaps across all sizes. Right: gap distribution by instance size. The red dashed line marks gap = 2.}
    \label{fig:gap}
\end{figure}

\subsection{PTAS Feasibility}

Table~\ref{tab:ptas} shows the success rate of the PTAS decomposition at achieving $(1+\varepsilon)\cdot\text{OPT}$ for varying $\varepsilon$.

\begin{table}[t]
\centering
\caption{PTAS decomposition success rate (\%) at achieving $(1+\varepsilon)\cdot\text{OPT}$, with 20 instances per configuration.}
\label{tab:ptas}
\small
\begin{tabular}{l cccccc}
\toprule
& \multicolumn{6}{c}{$\varepsilon$} \\
\cmidrule(lr){2-7}
$n$ & 0.05 & 0.10 & 0.15 & 0.20 & 0.30 & 0.50 \\
\midrule
5 & 100 & 100 & 100 & 100 & 100 & 100 \\
6 & 100 & 100 & 100 & 100 & 100 & 100 \\
7 & 100 & 100 & 100 & 100 & 100 & 95 \\
\bottomrule
\end{tabular}
\end{table}

The PTAS decomposition achieves 100\% success for $\varepsilon \leq 0.3$ across all sizes up to $n=7$.
The single failure at $n=7, \varepsilon=0.5$ (95\% success, max ratio 1.667) occurs because the large $\varepsilon$ threshold classifies too many items as ``small,'' reducing the quality of the greedy insertion phase.
Figure~\ref{fig:ptas} shows the success rate heatmap.

\begin{figure}[t]
    \centering
    \includegraphics[width=0.85\columnwidth]{figures/fig_ptas_heatmap.pdf}
    \caption{PTAS success rate heatmap. Green indicates 100\% success. The approach achieves perfect success for $\varepsilon \leq 0.3$ across all tested sizes.}
    \label{fig:ptas}
\end{figure}

\subsection{Adversarial and Structured Instances}

Table~\ref{tab:adversarial} reports results on adversarial instances with geometrically spaced supplies.

\begin{table}[t]
\centering
\caption{Performance on adversarial instances ($x_i = 2^{i-1}$, uniform demands).}
\label{tab:adversarial}
\small
\begin{tabular}{r rrrr}
\toprule
$n$ & OPT & Greedy & LS & Sorted \\
\midrule
4 &  4 & 1.000 & 1.000 & 1.000 \\
5 &  9 & 1.111 & 1.000 & 1.222 \\
6 & 21 & 1.048 & 1.000 & 1.238 \\
7 & 45 & 1.022 & 1.022 & 1.311 \\
8 & 96 & 1.000 & 1.000 & 1.333 \\
\bottomrule
\end{tabular}
\end{table}

Local search achieves optimal or near-optimal solutions on all adversarial instances.
The sorted heuristic degrades as $n$ grows (ratio up to 1.333), confirming that supply-demand matching alone is insufficient.

On 3-Partition-inspired instances, the sorted heuristic reaches ratio 3.000 for $k=3$ ($n=9$), while local search and PTAS decomposition both find the optimum.
This demonstrates that structured hard instances, which drive worst-case bounds, are effectively handled by more sophisticated algorithms.

\subsection{Prefix Surplus Profiles}

Figure~\ref{fig:profiles} illustrates the prefix surplus profiles for representative instances, showing how different algorithms produce different cumulative surplus trajectories.

\begin{figure}[t]
    \centering
    \includegraphics[width=\columnwidth]{figures/fig_prefix_profiles.pdf}
    \caption{Prefix surplus profiles comparing OPT, Greedy, and Local Search. Left: random instance ($n=8$). Right: adversarial instance ($n=7$). Local search closely tracks the optimal profile.}
    \label{fig:profiles}
\end{figure}

%% ============================================================
\section{Discussion}
\label{sec:discussion}

\subsection{Evidence for Sub-2 Approximation}

Our experiments provide strong evidence that the factor-2 barrier is not tight:
\begin{enumerate}
    \item Greedy+Swap never exceeds ratio 1.400 on $n=7$ instances (30 trials), with an average of 1.059.
    \item The PTAS decomposition finds optimal solutions on the vast majority of instances.
    \item The LP integrality gap appears to be well below 2, with average values of 0.599 to 0.780.
\end{enumerate}

\subsection{Toward a Theoretical Breakthrough}

We identify three promising directions:

\paragraph{Direction 1: Strengthened LP.}
Adding valid inequalities (subtour elimination, interval constraints, or Sherali-Adams rounds~\cite{sherali1990hierarchy}) could tighten the LP relaxation.
Our gap estimates suggest room for improvement.

\paragraph{Direction 2: Local Search Analysis.}
The consistent near-optimality of swap-based local search suggests that proving a sub-2 ratio for this combinatorial approach may be more tractable than improving the LP.
The key insight is that when the current solution has cost far from OPT, improving swaps must exist.

\paragraph{Direction 3: PTAS via Decomposition.}
The large/small decomposition succeeds empirically.
The main theoretical challenge is bounding the error from prefix-sum dependencies across block boundaries.
Unlike bin-packing PTASs, the objective's dependence on all prefix sums creates long-range correlations.

\subsection{Limitations}

Our study has several limitations:
(1) Instance sizes are small ($n \leq 8$) due to exact solver requirements.
(2) The LP solver is approximate (projected subgradient), so gap estimates are not tight lower bounds.
(3) Adversarial instance families may not capture the true worst case for local search.

%% ============================================================
\section{Conclusion}
\label{sec:conclusion}

We presented a systematic empirical investigation of the open problem of whether the Gasoline problem admits sub-2 approximation or a PTAS.
Our findings provide compelling evidence that the factor-2 barrier is not tight: local search achieves average ratios of 1.059 on $n=7$ instances, the PTAS decomposition finds optimal solutions with 100\% success for $\varepsilon \leq 0.3$, and the LP integrality gap appears moderate (average 0.689 for $n=7$).

We conjecture that (1)~the LP integrality gap is strictly less than 2 for $d=1$, (2)~swap-based local search achieves a provable sub-2 ratio, and (3)~a PTAS exists via large/small decomposition with careful prefix-sum analysis.
These conjectures, supported by our empirical evidence, outline a roadmap toward resolving this long-standing open problem.

\bibliographystyle{ACM-Reference-Format}
\bibliography{references}

\end{document}
