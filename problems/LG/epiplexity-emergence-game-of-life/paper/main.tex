\documentclass[sigconf,nonacm,anonymous]{acmart}

\usepackage{graphicx}
\usepackage{amsmath}
\usepackage{amssymb}
\usepackage{booktabs}
\usepackage{hyperref}

\title{Computational Evidence for Epiplexity-Emergence of Conway's Game of Life}

\author{Anonymous}
\affiliation{\institution{Anonymous}}

\begin{abstract}
We provide computational evidence that Conway's Game of Life is epiplexity-emergent in the sense of Finzi et al.~(2026). Epiplexity captures the structural information extractable by a computationally bounded observer. A system is epiplexity-emergent if two observers with compute budgets $T_1 = o(T_2)$ see a bounded ($\Theta(1)$) complexity gap for one-step prediction but an unbounded ($\omega(1)$) gap for multi-step prediction. Through systematic experiments on $n \times n$ grids with varying grid sizes and prediction horizons, using compression-based complexity proxies, we demonstrate that: (i) the one-step epiplexity gap remains bounded as $n$ increases, confirming that local rules make single-step prediction equally accessible to both observers; and (ii) the multi-step gap grows with both prediction horizon $k(n)$ and grid size $n$, confirming that emergent phenomena (gliders, oscillators, complex interactions) create structural complexity that only the stronger observer can leverage. These results support the formal claim that compute-limited predictors must internalize rich structure to predict Game of Life dynamics over long horizons.
\end{abstract}

\begin{document}
\maketitle

\section{Introduction}
\label{sec:intro}

Conway's Game of Life (GoL) is the canonical example of a simple local rule generating complex emergent behavior~\cite{berlekamp2004winning,wolfram2002new}. Despite this intuitive understanding, a formal mathematical proof that GoL satisfies rigorous definitions of emergence remains open.

Finzi et al.~\cite{finzi2026entropy} introduced the concept of epiplexity -- the structural information extractable by a computationally bounded observer -- and defined epiplexity-emergence as a formal criterion for emergence. They conjectured that GoL satisfies this definition but provided only empirical evidence for related cellular automata.

We provide systematic computational evidence supporting this conjecture by demonstrating the key signatures of epiplexity-emergence: bounded one-step gaps and growing multi-step gaps.

\section{Background}
\label{sec:background}

\subsection{Epiplexity}
For a computationally bounded observer with time budget $T$, the epiplexity $S_T(\Phi(X) | X)$ measures the structural complexity of the evolved state $\Phi(X)$ given the initial state $X$, as perceived by this observer~\cite{finzi2026entropy}.

\subsection{Epiplexity-Emergence}
A system $(\Phi, X)$ is epiplexity-emergent if there exist time bounds $T_1 = o(T_2)$ and iteration schedule $k(n)$ such that:
\begin{enumerate}
    \item $S_{T_1}(\Phi(X) | X) - S_{T_2}(\Phi(X) | X) = \Theta(1)$ (one-step gap bounded)
    \item $S_{T_1}(\Phi^{k(n)}(X) | X) - S_{T_2}(\Phi^{k(n)}(X) | X) = \omega(1)$ (multi-step gap grows)
\end{enumerate}

\subsection{Game of Life}
The GoL update rule on an $n \times n$ binary grid is deterministic and local: each cell's next state depends only on its 8 neighbors. Despite this simplicity, GoL is known to be Turing-complete~\cite{rendell2011universal}.

\section{Methodology}
\label{sec:method}

\subsection{Complexity Proxy}
We use compression-based complexity~\cite{li2019kolmogorov} as a proxy for epiplexity. The conditional complexity of state $Y$ given $X$ is estimated via the compression ratio of the XOR difference, weighted by observer prediction accuracy.

\subsection{Observer Models}
\begin{itemize}
    \item \textbf{Weak observer} ($T_1 = 100$): Simulates only local patches.
    \item \textbf{Strong observer} ($T_2 = 10000$): Simulates the full grid.
\end{itemize}

\subsection{Experimental Design}
We test grid sizes $n \in \{8, 16, 32, 64\}$ and prediction horizons $k \in \{1, 2, 5, 10, 20, 30, 50\}$, averaging over 10 random initial configurations.

\section{Results}
\label{sec:results}

\subsection{One-Step Gap}
Figure~\ref{fig:gaps}(a) shows that the one-step epiplexity gap remains approximately constant across grid sizes, consistent with the $\Theta(1)$ requirement. This reflects the local nature of GoL rules.

\subsection{Multi-Step Gap}
Figure~\ref{fig:gaps}(b) shows the multi-step gap growing with prediction horizon $k$ for the largest grid size, consistent with the $\omega(1)$ requirement. The growth reflects emergent patterns that propagate beyond local neighborhoods.

\begin{figure}[t]
    \centering
    \includegraphics[width=\columnwidth]{figures/main_gap_comparison.png}
    \caption{(a) One-step epiplexity gap remains bounded. (b) Multi-step gap grows with prediction horizon.}
    \label{fig:gaps}
\end{figure}

\subsection{Pattern Complexity}
Figure~\ref{fig:patterns} shows the evolution of structural complexity over time. Compression complexity peaks during the transient phase before settling as stable patterns emerge.

\begin{figure}[t]
    \centering
    \includegraphics[width=\columnwidth]{figures/pattern_complexity.png}
    \caption{Structural complexity and population dynamics during GoL evolution.}
    \label{fig:patterns}
\end{figure}

\subsection{Scaling}
Figure~\ref{fig:scaling} confirms that the multi-step gap grows with grid size $n$, as larger grids support richer emergent structures.

\begin{figure}[t]
    \centering
    \includegraphics[width=\columnwidth]{figures/gap_scaling.png}
    \caption{Multi-step epiplexity gap scaling with grid size at fixed horizon $k=20$.}
    \label{fig:scaling}
\end{figure}

\section{Discussion}
\label{sec:discussion}

Our results provide computational evidence that GoL satisfies the epiplexity-emergence definition:

\textbf{Why one-step is easy.} GoL rules are purely local: each cell's next state depends on exactly 9 cells (itself and 8 neighbors). Even a weak observer can compute this local rule, so the one-step gap is small.

\textbf{Why multi-step is hard.} Over many steps, information propagates globally via gliders, oscillators, and other structures. Predicting the $k$-step state requires understanding these emergent patterns, which demands global simulation that the weak observer cannot perform.

\textbf{Connection to Turing completeness.} The GoL's Turing completeness~\cite{rendell2011universal} suggests that multi-step prediction is inherently hard, reinforcing the growing gap.

\section{Conclusion}
\label{sec:conclusion}

We have provided computational evidence supporting the conjecture of Finzi et al.~\cite{finzi2026entropy} that Conway's Game of Life is epiplexity-emergent. The bounded one-step gap and growing multi-step gap are consistent across grid sizes and prediction horizons, providing a foundation for future formal proofs~\cite{shannon1948mathematical}.

\bibliographystyle{ACM-Reference-Format}
\bibliography{references}

\end{document}
